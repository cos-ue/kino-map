\documentclass[11pt,headsepline,a4paper]{scrreprt}
%onehalfspacing
\usepackage[utf8]{inputenc} 
\usepackage[german]{babel}
\usepackage[T1]{fontenc} %Schöner trennen
\usepackage{amsmath}
\usepackage{amsfonts}
\usepackage{amssymb}
\usepackage{makeidx}
\usepackage{graphicx}
\usepackage{lmodern}
\usepackage{float}
%\usepackage{kpfonts}
\usepackage{longtable}
\usepackage{svg}
\usepackage{listings,xcolor}
\usepackage{inconsolata}

\usepackage[colorlinks,
pdfpagelabels,
pdfstartview = FitH,
bookmarksopen = true,
bookmarksnumbered = true,
linkcolor = black,
plainpages = false,
hypertexnames = false,
citecolor = black] {hyperref}%Paket/Befehl um verlinkungen im Inhaltsverzeichnis zu bekommen

\usepackage{scrlayer-scrpage}
\automark{chapter}
\automark*{section}

\definecolor{dkgreen}{rgb}{0,.6,0}
\definecolor{dkblue}{rgb}{0,0,.6}
\definecolor{dkyellow}{cmyk}{0,0,.8,.3}

\lstdefinelanguage{JavaScript}{
	keywords={typeof, new, true, false, catch, function, return, null, catch, switch, var, if, in, while, do, else, case, break},
	keywordstyle    = \color{dkblue},
	ndkeywords={class, export, boolean, throw, implements, import, this},
	ndkeywordstyle=\color{darkgray},
	identifierstyle=\color{dkgreen},
	sensitive=false,
	comment=[l]{//},
	morecomment=[s]{/*}{*/},
	commentstyle    = \color{gray},
	stringstyle     = \color{red},
	morestring=[b]',
	morestring=[b]"
}

\lstset{
	language        = php,
	basicstyle      = \small\ttfamily,
	keywordstyle    = \color{dkblue},
	stringstyle     = \color{red},
	identifierstyle = \color{dkgreen},
	commentstyle    = \color{gray},
	emph            =[1]{php},
	emphstyle       =[1]\color{black},
	emph            =[2]{if,and,or,else},
	emphstyle       =[2]\color{dkyellow},
	numbers=left,
	stepnumber=1,  
	numberfirstline=true,
	numberstyle=\footnotesize,
	xleftmargin=4.0ex,
	upquote=true,
	showlines=true,
	extendedchars=true,
	literate={ä}{{\"a}}1 {ö}{{\"o}}1 {ü}{{\"u}}1 {Ä}{{\"A}}1 {Ö}{{\"O}}1 {Ü}{{\"U}}1,
	breaklines=true,
	postbreak=\mbox{\textcolor{red}{$\hookrightarrow$}\space},
}

\begin{document}
\setcounter{tocdepth}{3}
\title{Dokumentation  \\ \normalsize{Modul {\glqq Kino Karte\grqq}} für Citizen Science Platform {\glqq COSP\grqq}}
\date{\today}
\author{Projektteam \\
	\normalsize{
		\begin{tabular}[t]{ll}
			Version:  & \quad Beta 1.5.0 \\[1.2ex]
		\end{tabular}
	}
}
\maketitle

%Titelseite
%Inhaltsverzeichnis
\newpage
\tableofcontents
\thispagestyle{empty}
\newpage
\setcounter{page}{1}
\chapter{Rahmenbedingungen}


\section{Technische Rahmenbedingungen}
Die Website soll, aufgrund der verschiedenen Bedingungen der Anwender, mindestens auf folgenden Browsern fehlerfrei dargestellt werden:
\begin{itemize}
\item Mozilla Firefox (Version 66.0.3)
\item Edge (Version 44.17763.1.0)
\item Google Chrome (74.0.3729.108)
\item Apple Safari (12 (13606.2.11))
\end{itemize}
\vspace{\baselineskip}
Als Server-Betriebssystem wird Debian im Stable-Release verwendet. Desweiteren wird unter dem genannten Betriebssystem MariaDB als Datenbank und PHP7 als Plugin für einen aktuellen Apache2 Webserver verwendet.

\section{Entwicklungsumgebung}
Als Entwicklungsumgebung dient GitLab. Die Dokumentation wird automatisiert aus den Sources mittels der Gitlab-CI und zwei Gitlabrunnern erstellt.

\section{Verwendete Software}
Als Bibliothek für die Karte wirde Leaflet verwendet. Desweiteren wird Bootstrap 4 eingesetzt.

\chapter{Farbkonventionen}
Um die Nutzung unserer Website angenehm zu gestalten, haben wir uns für schlichte und gedeckte Farben entschieden. \\
Die Navigationsleiste ist schwarz mit weißer Schrift. \\ Um die Suchfunktion besser zu finden, aber trotzdem stimmig in das Gesamtkonzept einzupassen, ist sie in einem dunklem grau gehalten. Der Suchbutton ist in einem hellen Grau gehalten, damit er besser sichtbar ist, sich aber trotzdem ins Gesamtkonzept einfügt.\\
Auch schnell gefunden werden muss der Button zum Eintragen neuer POI. Deshalb ist er groß und auffällig unten links platziert, aber auch in schwarz und weiß, um ihn farblich den anderen anzupassen. \\
Für die Marker, die POIs auf der Karte markieren, haben wir uns auf zwei Farben festgelegt. Mit grünen Markern werden bekannte Spielstätten markiert, wohingegen ein blauer Marker die zu setzende Spielstätte anzeigt. \\
Das Formular zum Eintragen neuer POIs ist schwarz mit weißer Schrift um zur Navigationsleiste zu passen, sowie einem, im gleichen hellen Grau, wie der \glqq Suchen\grqq-Button, gehaltenen Sendeknopf.\\
An POIs gebundene Popups und die Fenster, mit weiteren Informationen haben ebenfalls einen schwarzen Hintergrund, mit einem hell Grauen Knopf und weißer Schrift.\\


\chapter{Konfiguration}
\label{chapter:config}
\section{Grundlegendes}
In der Konfigurationsdatei, welche im Ordner {\glqq bin\grqq} zu finden ist, können alle wichtigen Parameter der Website eingestellt werden. Hierzu ist im ersten Schritt die Datei mit dem Pfad {\glqq bin/config-sample.php\grqq} nach {\glqq bin/config.php\grqq} zu kopieren. Anschließend sind alle entsprechenden Einstellungen zur Kommunikation mit der Datenbank und zum Mailversand zu treffen. Hiernach sollten alle wichtigen Parameter bereits korrekt vordefiniert sein.
\section{Parameter}
\subsection{Auflistung}
In Nachfolgender Tabelle sind alle Konfigurationsparameter mit einer entsprechenden Kurzbeschreibung zusammengefasst. Im Anschluss an die Tabelle finden Sie eine ausführliche Beschreibung.
\begin{longtable}[H]{|c|p{8cm}|}
	\hline
	\textbf{Parameter}   & \textbf{Kurzbeschreibung}                                                                                                     \\ \hline
	\$SQL\_SERVER             & Adresse des SQL-Servers                                                                                                     \\ \hline	
	\$SQL\_USER               & Nutzername am SQL-Server                                                                                                     \\ \hline	
	\$SQL\_PASSWORD           & Passwort am SQL-Server                                                                                                     \\ \hline	
	\$SQL\_SCHEMA             & Schema auf dem SQL-Servers                                                                                                     \\ \hline	
	\$SQL\_PREFIX             & Tabellennamenprefix im Schema auf dem SQL-Servers                                                                                                     \\ \hline	
	\$SQL\_Connector          & Angabe des zu nutzenden SQL-Connectors                                                                                                     \\ \hline	
	\$PICTURE\_PATH           & Unterverzeichnis in das hochgeladene Bilder temporär gespeichert werden                                                                                                   \\ \hline	
	\$DEBUG                   & Versetzt Plattform in Debug-Modus                                                                                                     \\ \hline	
	\$DEBUG\_LEVEL            & Tiefe der Debug-Ausgaben                                                                                                     \\ \hline	
	\$PWD\_LENGTH             & Mindestpasswortlänge                                                                                                     \\ \hline	
	\$PWD\_ALGORITHM          & Algorithmus zur Verschlüsselung neuer Passwörter                                                                                                     \\ \hline	
	\$RANDOM\_STRING\_LENGTH  & Länge zufallsgenerierter Zeichenketten                                                                                                     \\ \hline	
	\$CSAPI                   & URI der {\glqq COSP\grqq}-Modul API                                                                                                     \\ \hline
	\$USAPI                   & URI der {\glqq COSP\grqq}-Nutzer API                                                                                                     \\ \hline
	\$CSTOKEN                 & Authentifizierungstoken für {\glqq COSP\grqq}-Modul API                                                                                                \\ \hline
	\$ENABLE\_STORIES         & Aktiviert oder Deakticiert das Geschichtenfeature                                                                                                \\ \hline	
	\$BETA                    & Versetzt Plattform in Beta-Modus                                                                                                     \\ \hline	
	\$MAINTENANCE             & Versetzt Plattform in Wartungsmodus                                                                                                     \\ \hline	
	\$ROLE\_GUEST             & Rollenwert den ein Gast benötigt                                                                                                    \\ \hline	
	\$ROLE\_UNAUTH\_USER      & Rollenwert den ein unauthentifizierter Nutzer benötigt                                                                                                    \\ \hline	
	\$ROLE\_AUTH\_USER        & Rollenwert den ein authentifizierter Nutzer benötigt                                                                                                    \\ \hline	
	\$ROLE\_EMPLOYEE          & Rollenwert den ein Mitarbeiter benötigt                                                                                                    \\ \hline	
	\$ROLE\_ADMIN             & Rollenwert den ein Administrator benötigt                                                                                                    \\ \hline	
	\$SPECIAL\_CHARS\_CAPTCHA & Schaltet Sonderzeichen für Captcha-Generator ein                                                                                                     \\ \hline	
	\$PUBLIC\_CONTACT         & Schaltet Kontaktseite für Gäste frei                                                                                                     \\ \hline	
	\$ZENTRAL\_MAIL			  & Mailadresse an die Administrationsmails versendet werden                                                                                                     \\ \hline		
	\$IMPRESSUM\_NAME         & Name des Inhaltsverantwortlichen                                                                                                     \\ \hline
	\$IMPRESSUM\_STREET       & Straße und Hausnummer der Adresse des Inhaltsverantwortlichen                                                                                                     \\ \hline	
	\$IMPRESSUM\_CITY         & Stadt und Postleitzahl der Adresse des Inhaltsverantwortlichen                                                                                                     \\ \hline	
	\$PRIVACY\_COMPANY\_NAME  & Name der Firma/Organisation                                                                                                     \\ \hline
	\$PRIVACY\_COMPANY\_STREET & Straße und Hausnummer der Adresse der Firma/Organisation                                                                                                     \\ \hline	
	\$PRIVACY\_COMPANY\_CITY   & Stadt und Postleitzahl der Adresse der Firma/Organisation                                                                                                     \\ \hline	
	\$PRIVACY\_COMPANY\_FON    & Telefonnummer der Firma/Organisation                                                                                                     \\ \hline	
	\$PRIVACY\_COMPANY\_FAX    & Faxnummer der Firma/Organisation                                                                                                     \\ \hline	
	\$PRIVACY\_COMPANY\_MAIL   & Mailadresse der Firma/Organisation                                                                                                     \\ \hline	
	\$PRIVACY\_REP\_NAME      & Name des Datenschutzbeauftragten           																							\\ \hline
	\$PRIVACY\_REP\_POS       & Position der Datenschutzbeauftragten                                                                                          \\ \hline
	\$PRIVACY\_REP\_STREET    & Straße und Hausnummer der Adresse des Datenschutzbeauftragten                                                                                                     \\ \hline	
	\$PRIVACY\_REP\_CITY      & Stadt und Postleitzahl der Adresse des Datenschutzbeauftragten                                                                                                     \\ \hline	
	\$PRIVACY\_REP\_FON       & Telefonnummer des Datenschutzbeauftragten                                                                                                     \\ \hline	
	\$PRIVACY\_REP\_FAX       & Faxnummer des Datenschutzbeauftragten                                                                                                     \\ \hline	
	\$PRIVACY\_REP\_MAIL      & Mailadresse des Datenschutzbeauftragten                                                                                                     \\ \hline	
	\$DIRECT\_DELETE          & Daten direkt löschen                                                                                                     \\ \hline
\end{longtable}
\subsection{Parameterbeschreibung}
\subsubsection{\$SQL\_SERVER}
\paragraph{Beschreibung}Dieser Parameter setzt die Adresse des zu nutzenden SQL-Servers. Hier sollte daher eine Valide URI oder IP zu einem MySQL- oder MariaDB-Server angegeben werden. Es wird ein MariaDB-Server bevorzugt.
\paragraph{Standardwert}Es gibt keinen Standardwert.

\subsubsection{\$SQL\_USER}
\paragraph{Beschreibung}Dieser Parameter setzt den am SQL-Server zu nutzenden Nutzernamen.
\paragraph{Standardwert}Es gibt keinen Standardwert.

\subsubsection{\$SQL\_PASSWORD}
\paragraph{Beschreibung}Dieser Parameter setzt das am SQL-Server zu nutzenden Passwort für den angegebenen Nutzernamen. 
\paragraph{Standardwert}Es gibt keinen Standardwert.

\subsubsection{\$SQL\_SCHEMA}
\paragraph{Beschreibung}Dieser Parameter setzt das am SQL-Server zu nutzende Schema.
\paragraph{Standardwert}Es gibt keinen Standardwert.

\subsubsection{\$SQL\_PREFIX}
\paragraph{Beschreibung}Dieser Parameter setzt das vor jedem Tabellennamen stehende Prefix. Es sollte üblicherweise mit einem {\glqq \_\grqq} enden.
\paragraph{Standardwert}Der Standartwert ist {\glqq dload\_\_\grqq}.

\subsubsection{\$SQL\_Connector}
\paragraph{Beschreibung}Dieser Parameter setzt den durch PHP zu nutzenden SQL-Connector. Aktuell ist nur PHP-PDO als Connector implementiert. 
\paragraph{Parameterwert}Es kann einer der folgenden Werte eingetragen werden: {\glqq pdo\grqq}.
\paragraph{Standardwert}Der Standartwert ist {\glqq pdo\grqq}.

\subsubsection{\$PICTURE\_PATH}
\paragraph{Beschreibung}Dieser Parameter setzt das Verzeichnis zum temporären Speichern von hochgeladenen Bildern.
\paragraph{Parameterwert}Es kann jeder beliebige Unterpfad im Webverzeichnis hier relativ zum Wurzelverzeichnis der Website angegeben werden. Der Pfad darf nicht absolut sein.
\paragraph{Standardwert}Der Standartwert ist {\glqq images/uploadMat\grqq}.

\subsubsection{\$DEBUG}
\paragraph{Beschreibung}Dieser Parameter kann den Debug-Modus der Plattform aktivieren. Im Produktiveinsatz sollte der Wert des Parameters stets {\glqq false\grqq} sein.
\paragraph{Parameterwert}Es kann einer der folgenden Werte eingetragen werden: {\glqq true\grqq}, {\glqq false\grqq}.
\paragraph{Standardwert}Der Standartwert ist {\glqq true\grqq}.

\subsubsection{\$DEBUG\_LEVEL}\label{config:debug-level}
\paragraph{Beschreibung}Dieser Parameter gibt die Anzahl und tiefe der Debug-Ausgaben an. Es ist ein Integerwert zu wählen. Dieser Parameter ist nur bei aktivierten Debug-Modus von Relevanz. Bei einem Wert über 3 können API-Anfragen möglicherweise Debug-Ausgaben beinhalten und daher nicht vom Anfragenden System verarbeitet werden.
\paragraph{Parameterwert}Es kann eine Werte zwischen 0 und 8 eingetragen werden.
\paragraph{Standardwert}Der Standartwert ist {\glqq 3\grqq}.

\subsubsection{\$PWD\_LENGTH}
\paragraph{Beschreibung}Dieser Parameter gibt die Mindestlänge des Passwortes an. Er sollte stets größer 3 sein.
\paragraph{Parameterwert}Es kann ein Wert der Natürlichen Zahlen, welcher größer 3 ist angegeben werden.
\paragraph{Standardwert}Der Standartwert ist {\glqq 8\grqq}.

\subsubsection{\$PWD\_ALGORITHM}
\paragraph{Beschreibung}Dieser Parameter gibt zum Passwort speichern zu verwenden Hashalgorithmus an. 
\paragraph{Parameterwert}Es kann einer der folgenden Werte eingetragen werden: {\glqq PASSWORD\_DEFAULT\grqq}, {\glqq PASSWORD\_BCRYPT\grqq}, {\glqq PASSWORD\_ARGON2I\grqq}, {\glqq PASSWORD\_ARGON2ID\grqq}.
\paragraph{Standardwert}Der Standartwert ist {\glqq PASSWORD\_ARGON2ID\grqq}. Unter PHP 7.2 sollte als Standartwert {\glqq PASSWORD\_ARGON2I\grqq} gewählt werden.

\subsubsection{\$RANDOM\_STRING\_LENGTH}
\paragraph{Beschreibung}Dieser Parameter gibt die Länge zufallsgenerierter Zeichenketten an.
\paragraph{Parameterwert}Es kann ein Wert der Natürlichen Zahlen, welcher größer 20 ist angegeben werden.
\paragraph{Standardwert}Der Standartwert ist {\glqq 170\grqq}.

\subsubsection{\$CSAPI}
\paragraph{Beschreibung}Dieser Parameter die URI der {\glqq COSP\grqq}-Modul-API an.
\paragraph{Parameterwert}Hier ist die {\glqq COSP\grqq}-URI anzugeben und ein {\glqq /api.php\grqq}anzuhängen.
\paragraph{Standardwert}Es gibt keinen Standardwert.

\subsubsection{\$USAPI}
\paragraph{Beschreibung}Dieser Parameter die URI der {\glqq COSP\grqq}-Nutzer-API an.
\paragraph{Parameterwert}Hier ist die {\glqq COSP\grqq}-URI anzugeben und ein {\glqq /api.php\grqq}anzuhängen.
\paragraph{Standardwert}Es gibt keinen Standardwert.

\subsubsection{\$CSTOKEN}
\paragraph{Beschreibung}Dieser Parameter gibt den Authentifizierungsschlüssel bei der in {\glqq \$CSAPI\grqq} angegeben API-Schnittstelle an.
\paragraph{Parameterwert}Hier muss eine vom {\glqq COSP\grqq}-Admin generierte alphanumerische Zeichenkette eingefügt werden.
\paragraph{Standardwert}Es gibt keinen Standardwert.

\subsubsection{\$ENABLE\_STORIES}
\paragraph{Beschreibung}Dieser Parameter gibt den Aktivierungsstatus des Geschichten-/Biographienmoduls an.
\paragraph{Parameterwert}Es kann einer der folgenden Werte eingetragen werden: {\glqq true\grqq}, {\glqq false\grqq}.
\paragraph{Standardwert}Der Standartwert ist {\glqq false\grqq}.

\subsubsection{\$BETA}
\paragraph{Beschreibung}Dieser Parameter versetzt die Plattform in den Beta-Modus. Der Zugriff auf die Plattform ist dann nur noch mit einem entsprechenden Link möglich. Grundsätzlich ist für einen Zugriff auf die Website das Suffix {\glqq ?b=0\grqq} an die im Parameter {\glqq \$DOMAIN\grqq} genannte URI anzuhängen.
\paragraph{Parameterwert}Es kann einer der folgenden Werte eingetragen werden: {\glqq true\grqq}, {\glqq false\grqq}.
\paragraph{Standardwert}Der Standartwert ist {\glqq false\grqq}.

\subsubsection{\$MAINTENANCE}
\paragraph{Beschreibung}Dieser Parameter versetzt die Plattform in den Wartungsmodus. Der Zugriff auf die Plattform ist dann nur noch mit einem entsprechenden Link möglich. Grundsätzlich ist für einen Zugriff auf die Website das Suffix {\glqq ?m=0\grqq} an die im Parameter {\glqq \$DOMAIN\grqq} genannte URI anzuhängen.
\paragraph{Parameterwert}Es kann einer der folgenden Werte eingetragen werden: {\glqq true\grqq}, {\glqq false\grqq}.
\paragraph{Standardwert}Der Standartwert ist {\glqq false\grqq}.

\subsubsection{\$ROLE\_GUEST}
\paragraph{Beschreibung}Dieser Parameter setzt den mindest Rollenwert für den Gastnutzer.
\paragraph{Standardwert}Der Standartwert ist {\glqq 0\grqq}.

\subsubsection{\$ROLE\_UNAUTH\_USER}
\paragraph{Beschreibung}Dieser Parameter setzt den mindest Rollenwert für den unautorisierten Nutzer.
\paragraph{Standardwert}Der Standartwert ist {\glqq 1\grqq}.

\subsubsection{\$ROLE\_AUTH\_USER}
\paragraph{Beschreibung}Dieser Parameter setzt den mindest Rollenwert für den autorisierten Nutzer.
\paragraph{Standardwert}Der Standartwert ist {\glqq 2\grqq}.

\subsubsection{\$ROLE\_EMPLOYEE}
\paragraph{Beschreibung}Dieser Parameter setzt den mindest Rollenwert für einen Mitarbeiter.
\paragraph{Standardwert}Der Standartwert ist {\glqq 10\grqq}.

\subsubsection{\$ROLE\_ADMIN}
\paragraph{Beschreibung}Dieser Parameter setzt den mindest Rollenwert für einen Administrator.
\paragraph{Standardwert}Der Standartwert ist {\glqq 20\grqq}.

\subsubsection{\$SPECIAL\_CHARS\_CAPTCHA}
\paragraph{Beschreibung}Dieser Parameter ermöglicht die Nutzung von Sonderzeichen in Captcha-Codes.
\paragraph{Parameterwert}Es kann einer der folgenden Werte eingetragen werden: {\glqq true\grqq}, {\glqq false\grqq}.
\paragraph{Standardwert}Der Standartwert ist {\glqq true\grqq}.

\subsubsection{\$ZENTRAL\_MAIL}
\paragraph{Beschreibung}Dieser Parameter Mailadresse an welche Mails Systemmails für administrative Anfragen gesendet werden. Diese Mailadresse erhält auch alle Anfragen, welche über das Kontaktformular eingesendet werden.
\paragraph{Standardwert}Es gibt keinen Standardwert.

\subsubsection{\$IMPRESSUM\_NAME}\label{config:impressum-name}
\paragraph{Beschreibung}Dieser Parameter setzt den Name des Inhaltsverantwortlichen der Plattform im Impressum.
\paragraph{Standardwert}Es gibt keinen Standardwert.

\subsubsection{\$IMPRESSUM\_STREET}\label{config:impressum-street}
\paragraph{Beschreibung}Dieser Parameter setzt den Straßenname und die Hausnummer der Adresse des Inhaltsverantwortlichen im Impressum.
\paragraph{Standardwert}Es gibt keinen Standardwert.

\subsubsection{\$IMPRESSUM\_CITY}\label{config:impressum-city}
\paragraph{Beschreibung}Dieser Parameter setzt den Ort und die Postleitzahl der Adresse des Inhaltsverantwortlichen im Impressum.
\paragraph{Standardwert}Es gibt keinen Standardwert.

\subsubsection{\$PRIVACY\_COMPANY\_NAME}\label{config:privacy-comp-name}
\paragraph{Beschreibung}Dieser Parameter setzt den Name der Firma oder Organisation der Plattform in der Datenschutzerklärung.
\paragraph{Standardwert}Es gibt keinen Standardwert.

\subsubsection{\$PRIVACY\_COMPANY\_STREET}\label{config:privacy-comp-street}
\paragraph{Beschreibung}Dieser Parameter setzt den Straßennamen und die Hausnummer der Adresse der Firma oder Organisation der Plattform in der Datenschutzerklärung.
\paragraph{Standardwert}Es gibt keinen Standardwert.

\subsubsection{\$PRIVACY\_COMPANY\_CITY}\label{config:privacy-comp-city}
\paragraph{Beschreibung}Dieser Parameter setzt den Ortsnamen und die Postleitzahl der Adresse der Firma oder Organisation der Plattform in der Datenschutzerklärung.
\paragraph{Standardwert}Es gibt keinen Standardwert.

\subsubsection{\$PRIVACY\_COMPANY\_FON}\label{config:privacy-comp-fon}
\paragraph{Beschreibung}Dieser Parameter setzt die Telefonnummer der Firma oder Organisation der Plattform in der Datenschutzerklärung.
\paragraph{Standardwert}Es gibt keinen Standardwert.

\subsubsection{\$PRIVACY\_COMPANY\_FAX}\label{config:privacy-comp-fax}
\paragraph{Beschreibung}Dieser Parameter setzt die Faxnummer der Firma oder Organisation der Plattform in der Datenschutzerklärung.
\paragraph{Standardwert}Es gibt keinen Standardwert.

\subsubsection{\$PRIVACY\_COMPANY\_MAIL}\label{config:privacy-comp-mail}
\paragraph{Beschreibung}Dieser Parameter setzt die Mailadresse der Firma oder Organisation der Plattform in der Datenschutzerklärung.
\paragraph{Standardwert}Es gibt keinen Standardwert.

\subsubsection{\$PRIVACY\_REP\_NAME}\label{config:privacy-rep-name}
\paragraph{Beschreibung}Dieser Parameter setzt den Name des Datenschutzverantwortlichen der Plattform in der Datenschutzerklärung.
\paragraph{Standardwert}Es gibt keinen Standardwert.

\subsubsection{\$PRIVACY\_REP\_POS}\label{config:privacy-rep-pos}
\paragraph{Beschreibung}Dieser Parameter setzt die Position des Datenschutzverantwortlichen der Plattform in der Datenschutzerklärung.
\paragraph{Standardwert}Es gibt keinen Standardwert.


\subsubsection{\$PRIVACY\_REP\_STREET}\label{config:privacy-rep-street}
\paragraph{Beschreibung}Dieser Parameter setzt den Straßennamen und die Hausnummer der Adresse des Datenschutzverantwortlichen der Plattform in der Datenschutzerklärung.
\paragraph{Standardwert}Es gibt keinen Standardwert.

\subsubsection{\$PRIVACY\_REP\_CITY}\label{config:privacy-rep-city}
\paragraph{Beschreibung}Dieser Parameter setzt den Ortsnamen und die Postleitzahl der Adresse des Datenschutzverantwortlichen der Plattform in der Datenschutzerklärung.
\paragraph{Standardwert}Es gibt keinen Standardwert.

\subsubsection{\$PRIVACY\_REP\_FON}\label{config:privacy-rep-fon}
\paragraph{Beschreibung}Dieser Parameter setzt die Telefonnummer des Datenschutzverantwortlichen der Plattform in der Datenschutzerklärung.
\paragraph{Standardwert}Es gibt keinen Standardwert.

\subsubsection{\$PRIVACY\_REP\_FAX}\label{config:privacy-rep-fax}
\paragraph{Beschreibung}Dieser Parameter setzt die Faxnummer des Datenschutzverantwortlichen der Plattform in der Datenschutzerklärung.
\paragraph{Standardwert}Es gibt keinen Standardwert.

\subsubsection{\$PRIVACY\_REP\_MAIL}\label{config:privacy-rep-mail}
\paragraph{Beschreibung}Dieser Parameter setzt die Mailadresse des Datenschutzverantwortlichen der Plattform in der Datenschutzerklärung.
\paragraph{Standardwert}Es gibt keinen Standardwert.

\subsubsection{\$DIRECT\_DELETE}
\paragraph{Beschreibung}Dieser Parameter ermöglicht legt fest, ob Daten direkt gelöscht werden oder nur als gelöscht markiert werden.
\paragraph{Parameterwert}Es kann einer der folgenden Werte eingetragen werden: {\glqq true\grqq} (direkt löschen), {\glqq false\grqq} (markieren).
\paragraph{Standardwert}Der Standartwert ist {\glqq false\grqq}.
\chapter{API-Spezifikation}\label{rapi}
\section{Verwaltungs-API}
\subsection{Befehlsübersicht}
\begin{longtable}[H]{|c|p{12cm}|}
		\hline
		\textbf{Api-Befehl} & \textbf{Kurzbeschreibung}              \\ \hline
		dau                 & Deaktiviere bestimmten Nutzer          \\ \hline
		rau                 & Aktiviere bestimmten Nutzer            \\ \hline
		rpt                 & Entferne Referenzen zu bestimmten Bild \\ \hline
		dus                 & Entferne Referenzen zu bestimmter Geschichte \\ \hline
		rst                 & Wiederherstellen von Referenzen zu bestimmter Geschichte \\ \hline
		rpc                 & Wiederherstellen von Referenzen zu bestimmtem Bild \\ \hline
\end{longtable}
\newpage
\subsection{Befehle}
\subsubsection{Nutzerdeaktivierung}
\paragraph{Kurzbeschreibung}Dieser API-Request wird dazu genutzt einen bereits aktivierten Nutzer wieder zu deaktivieren. Dies ist Notwendig, da das Modul {\glqq Kino Karte\grqq} keine eigene Nutzerverwaltung besitzt.
\paragraph{Anfrage}Folgende Daten werden zu Anfrage benötigt:
\begin{table}[H]
	\begin{tabular}{|c|c|c|p{6.5cm}|}
		\hline
		\textbf{Parametername} & \textbf{Datentyp} & \textbf{Konstante} & \textbf{Kurzbeschreibung}                                                                                               \\ \hline
		type                & string            & dau                & Deaktiviere Nutzer                                                                                                      \\ \hline
		token               & string            &                    & Eingehender API-Aufruf benötigt Token der in config::\$CSTOKEN zu finden ist \\ \hline
		username            & string            &                    & Nutzername des zu deaktivierenden Nutzers                                                                               \\ \hline
	\end{tabular}
\end{table}
\paragraph{Antwort}Die Antwort ist wie folgt aufgebaut:
\begin{table}[H]
	\begin{tabular}{|c|c|c|p{6.5cm}|}
		\hline
		\textbf{Parametername} & \textbf{Datentyp} & \textbf{Konstante} & \textbf{Kurzbeschreibung}                                                                                               \\ \hline
		result              & string            &                    & Bei erfolgreichen Request {\glqq ack\grqq}                                                                            \\ \hline
		code                & int               &                    & Bei erfolgreichen Request {\glqq 0\grqq} \\ \hline
		type                & string            & dau                & Deaktiviere Nutzer                                                                                                      \\ \hline
		token               & string            &                    & Eingehender API-Aufruf benötigt Token der in config::\$CSTOKEN zu finden ist \\ \hline
		username            & string            &                    & Nutzername des zu deaktivierenden Nutzers                                                                               \\ \hline
	\end{tabular}
\end{table}
\subsubsection{Nutzeraktivierung}
\paragraph{Kurzbeschreibung}Dieser API-Request wird dazu genutzt einen deaktivierten Nutzer wieder zu aktivieren. Dies ist Notwendig, da das Modul {\glqq Kino Karte\grqq} keine eigene Nutzerverwaltung besitzt.
\paragraph{Anfrage}Folgende Daten werden zu Anfrage benötigt:
\begin{table}[H]
	\begin{tabular}{|c|c|c|p{6.5cm}|}
		\hline
		\textbf{Paramtername} & \textbf{Datentyp} & \textbf{Konstante} & \textbf{Kurzbeschreibung}                                                                                               \\ \hline
		type                & string            & rau                & Aktiviere Nutzer                                                                                                      \\ \hline
		token               & string            &                    & Eingehender API-Aufruf benötigt Token der in config::\$CSTOKEN zu finden ist \\ \hline
		username            & string            &                    & Nutzername des zu aktivierenden Nutzers                                                                               \\ \hline
	\end{tabular}
\end{table}
\paragraph{Antwort}Die Antwort ist wie folgt aufgebaut:
\begin{table}[H]
	\begin{tabular}{|c|c|c|p{6.5cm}|}
		\hline
		\textbf{Parametername} & \textbf{Datentyp} & \textbf{Konstante} & \textbf{Kurzbeschreibung}                                                                                               \\ \hline
		result              & string            &                    & Bei erfolgreichen Request {\glqq ack\grqq}                                                                            \\ \hline
		code                & int               &                    & Bei erfolgreichen Request {\glqq 0\grqq} \\ \hline
		type                & string            & rau                & Aktiviere Nutzer                                                                                                      \\ \hline
		token               & string            &                    & Eingehender API-Aufruf benötigt Token der in config::\$CSTOKEN zu finden ist \\ \hline
		username            & string            &                    & Nutzername des zu deaktivierenden Nutzers                                                                               \\ \hline
	\end{tabular}
\end{table}
\subsubsection{Entfernen von Bildreferenzen}
\paragraph{Kurzbeschreibung}Dieser API-Request wird dazu genutzt um Links auf gelöschte Bilder zu entfernen. Dies sichert die Konsistenz der Datenbanken.
\paragraph{Anfrage}Folgende Daten werden zu Anfrage benötigt:
\begin{table}[H]
	\begin{tabular}{|c|c|c|p{6.5cm}|}
		\hline
		\textbf{Paramtername} & \textbf{Datentyp} & \textbf{Konstante} & \textbf{Kurzbeschreibung}                                                                                               \\ \hline
		type                & string            & rpt                & Entferne Referenzen zu bestimmten Bild \\ \hline
		token               & string            &                    & Eingehender API-Aufruf benötigt Token der in config::\$CSTOKEN zu finden ist \\ \hline
		picToken            & string            &                    & Bildtoken der zu entfernenden Referenzen \\ \hline
		override			& bool              &                    & Optional: TRUE: Verweise werden direkt gelöscht; FALSE: Verweise werden als gelöscht markiert\\ \hline   
	\end{tabular}
\end{table}
\paragraph{Antwort}Die Antwort ist wie folgt aufgebaut:
\begin{table}[H]
	\begin{tabular}{|c|c|c|p{6.5cm}|}
		\hline
		\textbf{Parametername} & \textbf{Datentyp} & \textbf{Konstante} & \textbf{Kurzbeschreibung}                                                                                               \\ \hline
		result              & string            &                    & Bei erfolgreichen Request {\glqq ack\grqq}                                                                            \\ \hline
		code                & int               &                    & Bei erfolgreichen Request {\glqq 0\grqq} \\ \hline
		type                & string            & rpt                & Entferne Referenzen zu bestimmten Bild \\ \hline
		token               & string            &                    & Eingehender API-Aufruf benötigt Token der in config::\$CSTOKEN zu finden ist \\ \hline
		picToken            & string            &                    & Bildtoken der zu entfernenden Referenzen \\ \hline   
		override			& bool              &                    & Optional: TRUE: Verweise werden direkt gelöscht; FALSE: Verweise werden als gelöscht markiert\\ \hline   
	\end{tabular}
\end{table}
\subsubsection{Entfernen von Referenzen auf Geschichten}
\paragraph{Kurzbeschreibung}Dieser API-Request wird dazu genutzt um Links auf gelöschte Geschichten zu entfernen. Dies sichert die Konsistenz der Datenbanken.
\paragraph{Anfrage}Folgende Daten werden zu Anfrage benötigt:
\begin{table}[H]
	\begin{tabular}{|c|c|c|p{6.5cm}|}
		\hline
		\textbf{Paramtername} & \textbf{Datentyp} & \textbf{Konstante} & \textbf{Kurzbeschreibung}                                                                                               \\ \hline
		type                & string            & dus                & Entferne Referenzen zu bestimmter Geschichte \\ \hline
		token               & string            &                    & Eingehender API-Aufruf benötigt Token der in config::\$CSTOKEN zu finden ist \\ \hline
		StoryToken          & string            &                    & Identifikator der zu entfernenden Geschichte \\ \hline
		overwrite           & bool              &                    & Wahr, wenn Story final gelöscht wird \\ \hline
	\end{tabular} 
\end{table}
\paragraph{Antwort}Die Antwort ist wie folgt aufgebaut:
\begin{table}[H]
	\begin{tabular}{|c|c|c|p{6.5cm}|}
		\hline
		\textbf{Parametername} & \textbf{Datentyp} & \textbf{Konstante} & \textbf{Kurzbeschreibung}                                                                                               \\ \hline
		result              & string            &                    & Bei erfolgreichen Request {\glqq ack\grqq}                                                                            \\ \hline
		code                & int               &                    & Bei erfolgreichen Request {\glqq 0\grqq} \\ \hline
		type                & string            & dus                & Entferne Referenzen zu bestimmter Geschichte \\ \hline
		token               & string            &                    & Eingehender API-Aufruf benötigt Token der in config::\$CSTOKEN zu finden ist \\ \hline
		StoryToken          & string            &                    & Identifikator der zu entfernenden Geschichte \\ \hline       
		overwrite           & bool              &                    & Wahr, wenn Story final gelöscht wird \\ \hline
	\end{tabular}
\end{table}
\subsubsection{Wiederherstellen von Referenzen auf Geschichten}
\paragraph{Kurzbeschreibung}Dieser API-Request wird dazu genutzt um Links auf Geschichten wiederherzustellen. Dies sichert die Konsistenz der Datenbanken.
\paragraph{Anfrage}Folgende Daten werden zu Anfrage benötigt:
\begin{table}[H]
	\begin{tabular}{|c|c|c|p{6.5cm}|}
		\hline
		\textbf{Paramtername} & \textbf{Datentyp} & \textbf{Konstante} & \textbf{Kurzbeschreibung}                                                                                               \\ \hline
		type                & string            & rst                & Entferne Referenzen zu bestimmten Bild \\ \hline
		token               & string            &                    & Eingehender API-Aufruf benötigt Token der in config::\$CSTOKEN zu finden ist \\ \hline
		StoryToken          & string            &                    & Identifikator der wiederherzustellenden Geschichte \\ \hline
	\end{tabular} 
\end{table}
\paragraph{Antwort}Die Antwort ist wie folgt aufgebaut:
\begin{table}[H]
	\begin{tabular}{|c|c|c|p{6.5cm}|}
		\hline
		\textbf{Parametername} & \textbf{Datentyp} & \textbf{Konstante} & \textbf{Kurzbeschreibung}                                                                                               \\ \hline
		result              & string            &                    & Bei erfolgreichen Request {\glqq ack\grqq}                                                                            \\ \hline
		code                & int               &                    & Bei erfolgreichen Request {\glqq 0\grqq} \\ \hline
		type                & string            & rst                & Entferne Referenzen zu bestimmten Bild \\ \hline
		token               & string            &                    & Eingehender API-Aufruf benötigt Token der in config::\$CSTOKEN zu finden ist \\ \hline
		StoryToken          & string            &                    & Identifikator der wiederherzustellenden Geschichte \\ \hline       
	\end{tabular}
\end{table}
\subsubsection{Wiederherstellen von Referenzen auf Bild}
\paragraph{Kurzbeschreibung}Dieser API-Request wird dazu genutzt um Links auf Bilder wiederherzustellen. Dies sichert die Konsistenz der Datenbanken.
\paragraph{Anfrage}Folgende Daten werden zu Anfrage benötigt:
\begin{table}[H]
	\begin{tabular}{|c|c|c|p{6.5cm}|}
		\hline
		\textbf{Paramtername} & \textbf{Datentyp} & \textbf{Konstante} & \textbf{Kurzbeschreibung}                                                                                               \\ \hline
		type                & string            & rpc                & Stelle Referenzen zu bestimmten Bild wieder her \\ \hline
		token               & string            &                    & Eingehender API-Aufruf benötigt Token der in config::\$CSTOKEN zu finden ist \\ \hline
		picToken            & string            &                    & Identifikator des wiederherzustellenden Bildes \\ \hline
	\end{tabular} 
\end{table}
\paragraph{Antwort}Die Antwort ist wie folgt aufgebaut:
\begin{table}[H]
	\begin{tabular}{|c|c|c|p{6.5cm}|}
		\hline
		\textbf{Parametername} & \textbf{Datentyp} & \textbf{Konstante} & \textbf{Kurzbeschreibung}                                                                                               \\ \hline
		result              & string            &                    & Bei erfolgreichen Request {\glqq ack\grqq}                                                                            \\ \hline
		code                & int               &                    & Bei erfolgreichen Request {\glqq 0\grqq} \\ \hline
		type                & string            & rpc                & Stelle Referenzen zu bestimmten Bild wieder her \\ \hline
		token               & string            &                    & Eingehender API-Aufruf benötigt Token der in config::\$CSTOKEN zu finden ist \\ \hline
		picToken            & string            &                    & Identifikator des wiederherzustellenden Bildes \\ \hline       
	\end{tabular}
\end{table}
\newpage
\section{Frontend API}\label{api}
\subsection{Befehlsübersicht}
\begin{longtable}[H]{|c|p{12cm}|}
		\hline
		\textbf{Api-Befehl} & \textbf{Kurzbeschreibung}                                                                                                                          \\ \hline
		pac                 & Abfrage der Daten für den persönlichen Bereich                                                                                                     \\ \hline
		duc                 & Löschen eines Kommentares                                                                                                                          \\ \hline
		auc                 & Kommentar hinzufügen                                                                                                                               \\ \hline
		smd                 & Alle Daten für das "Mehr Anzeigen"-Modal abfragen                                                                                                  \\ \hline
		gpu                 & Liste aller Interessenpunkte für aktuellen Nutzer abfragen                                                                                         \\ \hline
		mmy                 & \begin{tabular}[c]{@{}l@{}}Minimum und Maximum für der Jahreszahlen \\ der Betriebszeit der Interessenpunkte abfragen\end{tabular}                 \\ \hline
		ccp                 & URI der Zentralplattform "COSP" abfragen                                                                                                           \\ \hline
		aus                 & Neue Geschichte anlegen                                                                                                                            \\ \hline
		gas                 & Daten zu allen für den Nutzer verfügbaren Geschichten laden                                                                                        \\ \hline
		dpi                 & Interessenpunkt löschen                                                                                                                            \\ \hline
		gcs                 & Einzelnen Kommentar laden                                                                                                                          \\ \hline
		sec                 & Speichern eines bereits vorhandenen Kommentares                                                                                                    \\ \hline
		dsm                 & Daten für ein Bildabruf abfragen                                                                                                                   \\ \hline
		ssm                 & Daten eines bereits vorhanden Bildes speichern                                                                                                     \\ \hline
		eus                 & Daten einer bereits vorhandenen Geschichte speichern                                                                                               \\ \hline
		aha                 & Historische Adresse zu einem Interessenpunkt hinzufügen                                                                                            \\ \hline
		ado                 & Betreiber zu einem Interessenpunkt hinzufügen                                                                                                      \\ \hline
		adn                 & Name zu einem Interessenpunkt hinzufügen                                                                                                           \\ \hline
		dha                 & Löschen einer historischen Adresse eines Interessenpunktes                                                                                         \\ \hline
		dop                 & Löschen eines Betreibers eines Interessenpunktes                                                                                                   \\ \hline
		dna                 & Löschen eines Namen eines Interessenpunktes                                                                                                        \\ \hline
		vha                 & Validieren einer historischen Adresse eines Interessenpunktes                                                                                      \\ \hline
		vop                 & Validieren eines Betreibers eines Interessenpunktes                                                                                                \\ \hline
		vna                 & Validieren eines Namen eines Interessenpunktes                                                                                                     \\ \hline
		vts                 & Geschichte eines Interessenpunktes validieren                                                                                                      \\ \hline
		vca                 & Validieren der aktuellen Geschichte eines Interessenpunktes                                                                                        \\ \hline
		vhi                 & Validieren des Betriebszeitraum eines Interessenpunktes                                                                                            \\ \hline
		uha                 & Speichern einer bereits vorhandenen historischen Adresse eines Interessenpunktes                                                                   \\ \hline
		uop                 & Speichern einer bereits vorhandenen Betreibers eines Interessenpunktes                                                                             \\ \hline
		una                 & Speichern einer bereits vorhandenen Namen eines Interessenpunktes                                                                                  \\ \hline
		dsp                 & Löschen eines einzelnen Bildes                                                                                                                     \\ \hline
		gpt                 & Fragt alle Titel der Interessenpunkte ab                                                                                                           \\ \hline
		aps                 & Stellt einen Link zwischen Interessenpunkt und Geschichte her                                                                                      \\ \hline
		gps                 & Fragt alle Links zwischen Interessenpunkt und Geschichte auf Basis des Identifikators der Geschichte ab                                            \\ \hline
		vps                 & Validiert Link zwischen Geschichte und Interessenpunkt                                                                                             \\ \hline
		dps                 & Löscht Link zwischen Geschichte und Interessenpunkt                                                                                                \\ \hline
		dus                 & Löscht eine Geschichte                                                                                                                             \\ \hline
		cha                 & Prüft eine eingegebene Adresse, ob diese bereits bekannt                                                                                           \\ \hline
		gpf                 & Gibt Daten zum abrufen eines Originalbildes und einer Vorschau zurück                                                                              \\ \hline
		app                 & Stellt Link zwischen Bild und Interessenpunkt her                                                                                                  \\ \hline
		vpp                 & Validiert Link zwischen Bild und Interessenpunkt                                                                                                   \\ \hline
		dpp                 & Löscht Link zwischen Bild und Interessenpunkt                                                                                                      \\ \hline
		lpp                 & Lädt Daten für Bildauswahl                                                                                                                         \\ \hline
		asc                 & Fügt Sitzplatzanzahl zu Interessenpunkt hinzu                                                                                                      \\ \hline
		vsc                 & Validiert eine Sitzplatzanzahl zu Interessenpunktes                                                                                                \\ \hline
		dsc                 & Löscht eine Sitzplatzanzahl zu Interessenpunktes                                                                                                   \\ \hline
		usc                 & Speichert eine geänderte Sitzplatzanzahl eines Interessenpunktes                                                                                   \\ \hline
		acc                 & Fügt Saalanzahl zu Interessenpunkt hinzu                                                                                                           \\ \hline
		vcc                 & Validiert eine Saalanzahl zu Interessenpunktes                                                                                                     \\ \hline
		dcc                 & Löscht eine Saalanzahl zu Interessenpunktes                                                                                                        \\ \hline
		ucc                 & Speichert eine geänderte Saalanzahl eines Interessenpunktes                                                                                        \\ \hline
		vty                 & Validiert Kinotyp eines Interessenpunktes                                                                                                          \\ \hline
		asg                 & Abfrage ob Nutzer Gast ist                                                                                                                         \\ \hline
		gsd                 & Abfrage von Statistikdaten                                                                                                                         \\ \hline
		asa                 & Freigeben einer Geschichte                                                                                                                         \\ \hline
		das                 & Sperren einer Geschichte                                                                                                                           \\ \hline
		snp                 & Abfrage aller Namen eines Interessenpunktes                                                                                                        \\ \hline
		sop                 & Abfrage aller Betreiber eines Interessenpunktes                                                                                                    \\ \hline
		shp                 & Abfrage aller historischen Adressen eines Interessenpunktes                                                                                        \\ \hline
		gue                 & Abfrage ob Nutzer Gast ist, basierend auf Nutzername                                                                                               \\ \hline
		scp                 & Abfrage aller Saalanzahlen eines Interessenpunktes                                                                                                 \\ \hline
		ssp                 & Abfrage aller Sitzplatzanzahlen eines Interessenpunktes                                                                                            \\ \hline
		slp                 & Abfrage aller Geschichten-Interessenpunkt-Links                                                                                                    \\ \hline
		gsp                 & Abfrage aller Title der Geschichten die noch keine Link zu dem gewählten Interessenpunkt haben                                                     \\ \hline
		plp                 & Abfrage der Daten zum Laden des Hauptbildes eines Interessenpunktes                                                                                \\ \hline
		apl                 & Abfragen der Daten zu den zusätzlichen Bildern eines Interessenpunktes                                                                             \\ \hline
		lcp                 & laden der Kommentare eines Interessenpunktes                                                                                                       \\ \hline
		cse                 & Abfrage des Aktivierungsstatus des Geschichtenmoduls                                                                                               \\ \hline
		cpa                 & Anfordern eines Captcha-Codes                                                                                                                      \\ \hline
		cmg                 & Absenden einer Kontaktnachricht                                                                                                                    \\ \hline
		fdp                 & Finales Löschen einer Verknüpfung zwischen Interessenpunkt und Bild                                                                                \\ \hline
		rdp                 & Wiederherstellen einer Verknüpfung zwischen Interessenpunkt und Bild                                                                               \\ \hline
		fna                 & Finales Löschen eines Namen eines Interessenpunktes                                                                                                \\ \hline
		rna                 & Wiederherstellen eines Namen eines Interessenpunktes                                                                                               \\ \hline
		fop                 & Finales Löschen eines Betreibers eines Interessenpunktes                                                                                           \\ \hline
		rop                 & Wiederherstellen eines Betreibers eines Interessenpunktes                                                                                          \\ \hline
		fsc                 & Finales Löschen einer Sitzplatzanzahl eines Interessenpunktes                                                                                      \\ \hline
		rsc                 & Wiederherstellen einer Sitzplatzanzahl eines Interessenpunktes                                                                                     \\ \hline
		fcc                 & Finales Löschen einer Saalanzahl eines Interessenpunktes                                                                                           \\ \hline
		rcc                 & Wiederherstellen einer Saalanzahl eines Interessenpunktes                                                                                          \\ \hline
		fha                 & Finales Löschen einer historischen Adresse eines Interessenpunktes                                                                                 \\ \hline
		rha                 & Wiederherstellen einer historischen Adresse eines Interessenpunktes                                                                                \\ \hline
		fsp                 & Finales Löschen einer Verknüpfung zwischen Interessenpunkt und Geschichte                                                                          \\ \hline
		rsp                 & Wiederherstellen einer Verknüpfung zwischen Interessenpunkt und Geschichte                                                                         \\ \hline
		fcp                 & Finales Löschen eines Kommentars eines Interessenpunktes                                                                                           \\ \hline
		rcp                 & Wiederherstellen eines Kommentars eines Interessenpunktes                                                                                          \\ \hline
		fpi                 & Finales Löschen eines Interessenpunktes                                                                                                            \\ \hline
		rpi                 & Wiederherstellen eines Interessenpunktes                                                                                                           \\ \hline
		fst                 & Finales Löschen einer Geschichte                                                                                                                   \\ \hline
		rst                 & Wiederherstellen einer Geschichte                                                                                                                  \\ \hline
		aan                 & Ankündigung hinzufügen                                                                                                                             \\ \hline
		gan                 & Ankündigung Abfragen                                                                                                                               \\ \hline
		uan                 & Ankündigung Ändern                                                                                                                                 \\ \hline
		dan                 & Ankündigung Löschen                                                                                                                                \\ \hline
		gca                 & Aktuelle Ankündigungen abfragen                                                                                                                    \\ \hline
		saa                 & Ankündigungen aktivieren/deaktivieren                                                                                                              \\ \hline
		asp                 & Quelle für Interessenpunkt hinzufügen                                                                                                              \\ \hline
		grp                 & Quellen von Interessenpunkt abrufen                                                                                                                \\ \hline
		grs                 & Alle Bezüge von Quellen abfragen                                                                                                                   \\ \hline
		gts                 & Alle Typen von Quellen abfragen                                                                                                                    \\ \hline
		usp                 & Quelleneintrag ändern                                                                                                                              \\ \hline
		des                 & Quelleneintrag löschen                                                                                                                             \\ \hline
		fds                 & Quelleneintrag endgültig löschen                                                                                                                   \\ \hline
		rso                 & Quelleneintrag wiederherstellen                                                                                                                    \\ \hline
		vpi                 & Interessenpunkt validieren                                                                                                                         \\ \hline
		ddl                 & Fragt ab, ob Daten Direkt gelöscht werden                                                                                                          \\ \hline
		emp					& Hauptbild eines Interessenpunktes ändern					                                                                                         \\ \hline
		cma					& Prüft, ob Mailadresse bereits verwendet wird				                                                                                         \\ \hline
\end{longtable}
\newpage
\subsection{Befehle}
\subsubsection{Abfrage der Daten für den Persönlichen Bereich}
\paragraph{Kurzbeschreibung}Dieser API-Request wird dazu genutzt um alle Daten zur Darstellung des persönlichen Bereiches zu laden.
\paragraph{Anfrage}Folgende Daten werden zu Anfrage benötigt:
\begin{table}[H]
	\begin{tabular}{|c|c|c|p{6.5cm}|}
		\hline
		\textbf{Paramtername} & \textbf{Datentyp} & \textbf{Konstante} & \textbf{Kurzbeschreibung}                                                                                               \\ \hline
		type                & string            & pac                & Anfragen Aller Daten für persönlichen Bereich \\ \hline
	\end{tabular}
\end{table}
\paragraph{Antwort}Die Antwort ist wie folgt aufgebaut:
\begin{table}[H]
	\begin{tabular}{|c|c|c|p{6.5cm}|}
		\hline
		\textbf{Paramtername} & \textbf{Datentyp} & \textbf{Konstante} & \textbf{Kurzbeschreibung}                                                                                               \\ \hline
		pois                & array            &                 & Alle Daten zu Interessenpunkten des Nutzers \\ \hline
		User                & array            &                 & Alle nutzerbezogenen Daten \\ \hline
		comments            & array            &                 & Alle Kommentare des Nutzers bei Interessenpunkten \\ \hline
		result              & string           &                 & Erfolgreich wenn Wert {\glqq ack\grqq} ist \\ \hline
		Code                & int              &                 & Erfolgreich wenn Wert {\glqq 0\grqq} ist \\ \hline
	\end{tabular}
\end{table}
\subparagraph{pois}Dieses Array enthält Elemente mit Einträgen, welche die nachstehend dargestellten Form haben:
\begin{table}[H]
	\begin{tabular}{|c|c|c|p{6.5cm}|}
		\hline
		\textbf{Paramtername} & \textbf{Datentyp} & \textbf{Konstante} & \textbf{Kurzbeschreibung}                                                                                               \\ \hline
		lat                & double            &                 & Geographischer Breitengrad \\ \hline
		lng                & double            &                 & Geographischer Längengrad \\ \hline
		name               & string            &                 & Name des Interessenpunktes \\ \hline
		poi\_id            & int               &                 & Identifikator des Interessenpunktes \\ \hline
		edit\_enable       & bool              &                 & Gibt an, ob Bearbeitungsfunktionen verfügbar sind\\ \hline
	\end{tabular}
\end{table}
\subparagraph{User}Dieses Array enthält Einträge in der nachstehend dargestellten Form haben:
\begin{table}[H]
	\begin{tabular}{|c|c|c|p{6.5cm}|}
		\hline
		\textbf{Paramtername} & \textbf{Datentyp} & \textbf{Konstante} & \textbf{Kurzbeschreibung}    \\ \hline
		username           & string            &                 & Nutzername \\ \hline
		firstname          & string            &                 & Angegebener Vorname \\ \hline
		lastname           & string            &                 & Angegebener Nachname \\ \hline
		email              & string            &                 & Angegebene E-Mailadresse \\ \hline
	\end{tabular}
\end{table}
\subparagraph{comments}Dieses Array enthält Elemente mit Einträgen, welche die nachstehend dargestellten Form haben:
\begin{table}[H]
	\begin{tabular}{|c|c|c|p{6.5cm}|}
		\hline
		\textbf{Paramtername} & \textbf{Datentyp} & \textbf{Konstante} & \textbf{Kurzbeschreibung}                                                                                               \\ \hline
		date               & timestamp         &                 & Datum und Uhrzeit der Erstellung \\ \hline
		content            & string            &                 & Inhalt des Kommentares \\ \hline
		poiname            & string            &                 & Name des Interessenpunktes des Kommentares \\ \hline
		poiid              & int               &                 & Identifikator des Interessenpunktes des Kommentares \\ \hline
		lat                & double            &                 & Geographischer Breitengrad des Interessenpunktes \\ \hline
		lng                & double            &                 & Geographischer Längengrad  des Interessenpunktes \\ \hline
	\end{tabular}
\end{table}
\subsubsection{Löschen eines Kommentares}
\paragraph{Kurzbeschreibung}Dieser API-Request wird dazu genutzt um einen bestimmten Kommentar zu löschen.
\paragraph{Anfrage}Folgende Daten werden zu Anfrage benötigt:
\begin{table}[H]
	\begin{tabular}{|c|c|c|p{6.5cm}|}
		\hline
		\textbf{Paramtername} & \textbf{Datentyp} & \textbf{Konstante} & \textbf{Kurzbeschreibung}                                                                                               \\ \hline
		type                & string            & duc                & Löschen eines bestimmten Kommentares \\ \hline
		commentid           & int               &                    & Identifikator des Kommentares \\ \hline
	\end{tabular}
\end{table}
\paragraph{Antwort}Die Antwort ist wie folgt aufgebaut:
\begin{table}[H]
	\begin{tabular}{|c|c|c|p{6.5cm}|}
		\hline
		\textbf{Paramtername} & \textbf{Datentyp} & \textbf{Konstante} & \textbf{Kurzbeschreibung}                                                                                               \\ \hline
		result              & string           &                 & Erfolgreich wenn Wert {\glqq ack\grqq} ist \\ \hline
		Code                & int              &                 & Erfolgreich wenn Wert {\glqq 0\grqq} ist \\ \hline
	\end{tabular}
\end{table}
\subsubsection{Nutzerkommentar zu Interessenpunkt hinzufügen}
\paragraph{Kurzbeschreibung}Dieser API-Request wird dazu genutzt um einen Kommentar zu einem bestimmten Interessenpunkt hinzuzufügen.
\paragraph{Anfrage}Folgende Daten werden zu Anfrage benötigt:
\begin{table}[H]
	\begin{tabular}{|c|c|c|p{6.5cm}|}
		\hline
		\textbf{Paramtername} & \textbf{Datentyp} & \textbf{Konstante} & \textbf{Kurzbeschreibung}                                                                                               \\ \hline
		type                & string            & auc                & Neuanlegen eines Kommentares \\ \hline
		comment             & string            &                    & Inhalt des Kommentares \\ \hline
		poi\_id             & int               &                    & Identifikator des Interessenpunktes \\ \hline
	\end{tabular}
\end{table}
\paragraph{Antwort}Die Antwort ist wie folgt aufgebaut:
\begin{table}[H]
	\begin{tabular}{|c|c|c|p{6.5cm}|}
		\hline
		\textbf{Paramtername} & \textbf{Datentyp} & \textbf{Konstante} & \textbf{Kurzbeschreibung}                                                                                               \\ \hline
		result              & string           &                 & Erfolgreich wenn Wert {\glqq ack\grqq} ist \\ \hline
		Code                & int              &                 & Erfolgreich wenn Wert {\glqq 0\grqq} ist \\ \hline
		comment             & string           &                 & Inhalt des Kommentares \\ \hline
		poi\_id             & int              &                 & Identifikator des Interessenpunktes \\ \hline
	\end{tabular}
\end{table}
\subsubsection{Datenabfrage für bestimmten Interessenpunkt}
\paragraph{Kurzbeschreibung}Dieser API-Request wird dazu genutzt um alle Daten zu einem bestimmten Interessenpunkt im Backend abzufragen.
\paragraph{Anfrage}Folgende Daten werden zu Anfrage benötigt:
\begin{table}[H]
	\begin{tabular}{|c|c|c|p{6.5cm}|}
		\hline
		\textbf{Paramtername} & \textbf{Datentyp} & \textbf{Konstante} & \textbf{Kurzbeschreibung}                                                                                               \\ \hline
		type                & string            & smd                & Daten eines Interessenpunktes laden \\ \hline
		poi\_id             & int               &                    & Identifikator des Interessenpunktes \\ \hline
	\end{tabular}
\end{table}
\paragraph{Antwort}Die Antwort ist wie folgt aufgebaut:
\begin{table}[H]
	\begin{tabular}{|c|c|c|p{6.5cm}|}
		\hline
		\textbf{Paramtername} & \textbf{Datentyp} & \textbf{Konstante} & \textbf{Kurzbeschreibung}                                                                                               \\ \hline
		result              & string           &                 & Erfolgreich wenn Wert {\glqq ack\grqq} ist \\ \hline
		Code                & int              &                 & Erfolgreich wenn Wert {\glqq 0\grqq} ist \\ \hline
		data                & array            &                 & Daten des Interessenpunktes \\ \hline
	\end{tabular}
\end{table}
\subparagraph{data}Dieses Array enthält Einträge in der nachstehend dargestellten Form haben:
\begin{table}[H]
	\begin{tabular}{|c|c|c|p{6.5cm}|}
		\hline
		\textbf{Paramtername} & \textbf{Datentyp} & \textbf{Konstante} & \textbf{Kurzbeschreibung}    \\ \hline
		City               & string            &                 & Stadt der aktuellen Adresse \\ \hline
		Housenumber        & string            &                 & Hausnummer der aktuellen Adresse \\ \hline
		Postalcode         & int               &                 & Postleitzahl der aktuellen Adresse \\ \hline
		Streetname         & string            &                 & Straßenname der aktuellen Adresse \\ \hline
		category           & int               &                 & Kategorie des Interessenpunktes \\ \hline
		currAddr\_validate & bool              &                 & Validierungsstatus der aktuellen Adresse \\ \hline
		end                & int               &                 & Ende der Betriebszeit \\ \hline
		history            & string            &                 & Geschichte des Interessenpunktes \\ \hline
		history\_validate  & bool              &                 & Validierungsstatus der Geschichte \\ \hline
		poi\_id            & int               &                 & Identifikator des Interessenpunktes \\ \hline
		poi\_name          & string            &                 & Titel des Interessenpunktes \\ \hline
		start              & int               &                 & Anfang der Betriebszeit \\ \hline
		timespan\_validate & bool              &                 & Validierungsstatus des Betriebszeitraumes \\ \hline
		user\_name         & string            &                 & Nutzername des Erstellers \\ \hline
		deleted            & bool              &                 & Gibt an, ob Interessenpunkt als gelöscht gilt \\ \hline
		duty               & bool              &                 & Gibt an, ob der Interessenpunkt aktuell noch in Betrieb ist \\ \hline
		validated          & bool              &                 & Gibt an, ob der Interessenpunkt bereits Validiert ist\\ \hline
		editable           & bool              &                 & Gibt an, ob der Interessenpunkt änderbar ist \\ \hline
		deletable          & bool              &                 & Gibt an, ob der Interessenpunkt löschbar ist \\ \hline
		validatable        & bool              &                 & Gibt an, ob der Interessenpunkt validierbar ist \\ \hline
		finalDelete        & bool              &                 & Gibt an, ob der Interessenpunkt final löschbar ist \\ \hline
		editLink           & string            &                 & Link zum ändern der Daten des Interessenpunktes \\ \hline
	\end{tabular}
\end{table}
\subsubsection{Datenabfrage für alle Interessenpunkte für Nutzer zur Kartendarstellung}
\paragraph{Kurzbeschreibung}Dieser API-Request wird dazu genutzt um Kartendarstellungsdaten spezifisch für diesen Nutzer zu laden.
\paragraph{Anfrage}Folgende Daten werden zu Anfrage benötigt:
\begin{table}[H]
	\begin{tabular}{|c|c|c|p{6.5cm}|}
		\hline
		\textbf{Paramtername} & \textbf{Datentyp} & \textbf{Konstante} & \textbf{Kurzbeschreibung}                                                                                               \\ \hline
		type                & string            & gpu                & Laden der Kartendarstellungsdaten \\ \hline
	\end{tabular}
\end{table}
\paragraph{Antwort}Die Antwort ist wie folgt aufgebaut:
\begin{table}[H]
	\begin{tabular}{|c|c|c|p{6.5cm}|}
		\hline
		\textbf{Paramtername} & \textbf{Datentyp} & \textbf{Konstante} & \textbf{Kurzbeschreibung}                                                                                               \\ \hline
		result              & string           &                 & Erfolgreich wenn Wert {\glqq ack\grqq} ist \\ \hline
		Code                & int              &                 & Erfolgreich wenn Wert {\glqq 0\grqq} ist \\ \hline
		data                & array            &                 & Daten des Interessenpunktes \\ \hline
	\end{tabular}
\end{table}
\subparagraph{data}Dieses Array enthält Elemente mit Einträgen, welche die nachstehend dargestellten Form haben:
\begin{table}[H]
	\begin{tabular}{|c|c|c|p{6.5cm}|}
		\hline
		\textbf{Paramtername} & \textbf{Datentyp} & \textbf{Konstante} & \textbf{Kurzbeschreibung}    \\ \hline
		City               & string            &                 & Stadt der aktuellen Adresse \\ \hline
		Housenumber        & string            &                 & Hausnummer der aktuellen Adresse \\ \hline
		Postalcode         & int               &                 & Postleitzahl der aktuellen Adresse \\ \hline
		Streetname         & string            &                 & Straßenname der aktuellen Adresse \\ \hline
		category           & int               &                 & Kategorie des Interessenpunktes \\ \hline
		end                & int               &                 & Ende der Betriebszeit \\ \hline
		history            & string            &                 & Geschichte des Interessenpunktes \\ \hline
		poi\_id            & int               &                 & Identifikator des Interessenpunktes \\ \hline
		name               & string            &                 & Titel des Interessenpunktes \\ \hline
		start              & int               &                 & Anfang der Betriebszeit \\ \hline
		username           & string            &                 & Nutzername des Erstellers \\ \hline
		validated          & bool              &                 & Validierungsstatus des Interessenpunktes \\ \hline
		validatedByUser    & bool              &                 & Wahr wenn Nutzer diesen Interessenpunkt validiert hat \\ \hline
		lat                & double            &                 & Geographischer Breitengrad des Interessenpunktes \\ \hline
		lng                & double            &                 & Geographischer Längengrad  des Interessenpunktes \\ \hline
		duty               & bool              &                 & Legt fest ob Interessenpunkt aktuell in Betrieb \\ \hline
	\end{tabular}
\end{table}
\subsubsection{Abfrage des kleinsten und größten Jahres der Betriebszeiten aller Kinos}
\paragraph{Kurzbeschreibung}Dieser API-Request wird dazu genutzt um das minimale und maximale Jahr des Sliders abzufragen.
\paragraph{Anfrage}Folgende Daten werden zu Anfrage benötigt:
\begin{table}[H]
	\begin{tabular}{|c|c|c|p{6.5cm}|}
		\hline
		\textbf{Paramtername} & \textbf{Datentyp} & \textbf{Konstante} & \textbf{Kurzbeschreibung}                                                                                               \\ \hline
		type                & string            & mmy                & Abfrage des minimalen und maximalen Jahres \\ \hline
	\end{tabular}
\end{table}
\paragraph{Antwort}Die Antwort ist wie folgt aufgebaut:
\begin{table}[H]
	\begin{tabular}{|c|c|c|p{6.5cm}|}
		\hline
		\textbf{Paramtername} & \textbf{Datentyp} & \textbf{Konstante} & \textbf{Kurzbeschreibung}                                                                                               \\ \hline
		result              & string           &                 & Erfolgreich wenn Wert {\glqq ack\grqq} ist \\ \hline
		Code                & int              &                 & Erfolgreich wenn Wert {\glqq 0\grqq} ist \\ \hline
		MinYear             & int              &                 & Minimales Jahr \\ \hline
		MaxYear             & int              &                 & Maximales Jahr \\ \hline
	\end{tabular}
\end{table}
\subsubsection{Abfrage der {\glqq COSP\grqq}-URI}
\paragraph{Kurzbeschreibung}Dieser API-Request wird dazu genutzt um die {\glqq COSP\grqq}-URI abzufragen.
\paragraph{Anfrage}Folgende Daten werden zu Anfrage benötigt:
\begin{table}[H]
	\begin{tabular}{|c|c|c|p{6.5cm}|}
		\hline
		\textbf{Paramtername} & \textbf{Datentyp} & \textbf{Konstante} & \textbf{Kurzbeschreibung}                                                                                               \\ \hline
		type                & string            & ccp                & Abfrage der {\glqq COSP\grqq}-URI \\ \hline
	\end{tabular}
\end{table}
\paragraph{Antwort}Die Antwort ist wie folgt aufgebaut:
\begin{table}[H]
	\begin{tabular}{|c|c|c|p{6.5cm}|}
		\hline
		\textbf{Paramtername} & \textbf{Datentyp} & \textbf{Konstante} & \textbf{Kurzbeschreibung}                                                                                               \\ \hline
		result              & string           &                 & Erfolgreich wenn Wert {\glqq ack\grqq} ist \\ \hline
		Code                & int              &                 & Erfolgreich wenn Wert {\glqq 0\grqq} ist \\ \hline
		data                & array            &                 & Daten des Interessenpunktes \\ \hline
	\end{tabular}
\end{table}
\subparagraph{data}Dieses Array enthält Einträge in der nachstehend dargestellten Form haben:
\begin{table}[H]
	\begin{tabular}{|c|c|c|p{6.5cm}|}
		\hline
		\textbf{Paramtername} & \textbf{Datentyp} & \textbf{Konstante} & \textbf{Kurzbeschreibung}    \\ \hline
		uapi               & string            &                 & {\glqq COSP\grqq}-URI \\ \hline
	\end{tabular}
\end{table}
\subsubsection{Anlegen einer neuen Geschichte}\label{api:NewStoryAdd}
\paragraph{Kurzbeschreibung}Dieser API-Request wird dazu genutzt um eine neue Geschichte im Biographien-Modul anzulegen.
\paragraph{Anfrage}Folgende Daten werden zu Anfrage benötigt:
\begin{table}[H]
	\begin{tabular}{|c|c|c|p{6.5cm}|}
		\hline
		\textbf{Paramtername} & \textbf{Datentyp} & \textbf{Konstante} & \textbf{Kurzbeschreibung}                                                                                               \\ \hline
		type                & string            & aus                & Hinzufügen einer neuen Geschichte \\ \hline
		story               & string            &                    & Inhalt der Geschichte \\ \hline
		title               & string            &                    & Titel der Geschichte \\ \hline
		rights              & bool              &                    & Wahrheitswert des Rechte-Checkbox \\ \hline
	\end{tabular}
\end{table}
\paragraph{Antwort}Die Antwort ist wie folgt aufgebaut:
\begin{table}[H]
	\begin{tabular}{|c|c|c|p{6.5cm}|}
		\hline
		\textbf{Paramtername} & \textbf{Datentyp} & \textbf{Konstante} & \textbf{Kurzbeschreibung}                                                                                               \\ \hline
		result              & string           &                 & Erfolgreich wenn Wert {\glqq ack\grqq} ist \\ \hline
		Code                & int              &                 & Erfolgreich wenn Wert {\glqq 0\grqq} ist \\ \hline
		story               & string           &                 & Inhalt der Geschichte \\ \hline
		title               & string           &                 & Titel der Geschichte \\ \hline
		rights              & bool             &                 & Wahrheitswert des Rechte-Checkbox \\ \hline
		token               & string           &                 & Identifikator der Geschichte \\ \hline
		username            & string           &                 & Nutzername des Erstellers \\ \hline
	\end{tabular}
\end{table}
\subsubsection{Abfrage von Geschichten}
\paragraph{Kurzbeschreibung}Dieser API-Request wird dazu genutzt um alle für den Nutzer verfügbaren Geschichten abzufragen.
\paragraph{Anfrage}Folgende Daten werden zu Anfrage benötigt:
\begin{table}[H]
	\begin{tabular}{|c|c|c|p{6.5cm}|}
		\hline
		\textbf{Paramtername} & \textbf{Datentyp} & \textbf{Konstante} & \textbf{Kurzbeschreibung}                                                                                               \\ \hline
		type                & string            & gas                & Abfragen aller Geschichten für Nutzer\\ \hline
	\end{tabular}
\end{table}
\paragraph{Antwort}Die Antwort ist wie folgt aufgebaut:
\begin{table}[H]
	\begin{tabular}{|c|c|c|p{6.5cm}|}
		\hline
		\textbf{Paramtername} & \textbf{Datentyp} & \textbf{Konstante} & \textbf{Kurzbeschreibung}                                                                                               \\ \hline
		result              & string           &                 & Erfolgreich wenn Wert {\glqq ack\grqq} ist \\ \hline
		Code                & int              &                 & Erfolgreich wenn Wert {\glqq 0\grqq} ist \\ \hline
		data                & array            &                 & Daten zur Darstellung\\ \hline
	\end{tabular}
\end{table}
\subparagraph{data}Dieses Array enthält Einträge in der nachstehend dargestellten Form haben:
\begin{table}[H]
	\begin{tabular}{|c|c|c|p{6.5cm}|}
		\hline
		\textbf{Paramtername} & \textbf{Datentyp} & \textbf{Konstante} & \textbf{Kurzbeschreibung}    \\ \hline
		approver               & bool            &                 & Wahr, wenn Nutzer Geschichten freischalten darf \\ \hline
		guest                  & bool            &                 & Wahr, wenn Nutzer Gast ist \\ \hline
		result                 & array           &                 & Daten zum abrufen der Geschichten \\ \hline
	\end{tabular}
\end{table}
\subparagraph{result}Dieses Array enthält Einträge in der nachstehend dargestellten Form haben:
\begin{table}[H]
	\begin{tabular}{|c|c|c|p{6.5cm}|}
		\hline
		\textbf{Paramtername} & \textbf{Datentyp} & \textbf{Konstante} & \textbf{Kurzbeschreibung}    \\ \hline
		data                   & string          &                 & Gesicherte und signierte Daten \\ \hline
		seccode                & string          &                 & Sicherheitscode \\ \hline
		time                   & int             &                 & Zeitstempel des Requests \\ \hline
		type                   & string          & gas             & Abfragen aller Geschichten für Nutzer\\ \hline
		url                    & string          &                 & URI der {\glqq COSP\grqq}-Nutzer-API \\ \hline
	\end{tabular}
\end{table}
\subsubsection{Interessenpunkt löschen}
\paragraph{Kurzbeschreibung}Dieser API-Request wird dazu genutzt um einen Interessenpunkt zu löschen.
\paragraph{Anfrage}Folgende Daten werden zu Anfrage benötigt:
\begin{table}[H]
	\begin{tabular}{|c|c|c|p{6.5cm}|}
		\hline
		\textbf{Paramtername} & \textbf{Datentyp} & \textbf{Konstante} & \textbf{Kurzbeschreibung}                                                                                               \\ \hline
		type                & string            & dpi                & Interessenpunkt löschen \\ \hline
		poiid               & int               &                    & Wahrheitswert des Rechte-Checkbox \\ \hline
	\end{tabular}
\end{table}
\paragraph{Antwort}Die Antwort ist wie folgt aufgebaut:
\begin{table}[H]
	\begin{tabular}{|c|c|c|p{6.5cm}|}
		\hline
		\textbf{Paramtername} & \textbf{Datentyp} & \textbf{Konstante} & \textbf{Kurzbeschreibung}                                                                                               \\ \hline
		result              & string           &                 & Erfolgreich wenn Wert {\glqq ack\grqq} ist \\ \hline
		Code                & int              &                 & Erfolgreich wenn Wert {\glqq 0\grqq} ist \\ \hline
	\end{tabular}
\end{table}
\subsubsection{Ändern eines Kommentares/Speichern eines bereits vorhandenen Kommentares}
\paragraph{Kurzbeschreibung}Dieser API-Request wird dazu genutzt um einen bereits vorhandenen Kommentar in eventuell geänderter Form ab zu speichern.
\paragraph{Anfrage}Folgende Daten werden zu Anfrage benötigt:
\begin{table}[H]
	\begin{tabular}{|c|c|c|p{6.5cm}|}
		\hline
		\textbf{Paramtername} & \textbf{Datentyp} & \textbf{Konstante} & \textbf{Kurzbeschreibung}                                                                                               \\ \hline
		type                & string            & sec                & Kommentar speichern \\ \hline
		commentid           & int               &                    & Identifikator des Kommentares \\ \hline
		commentContent      & string            &                    & Inhalt des Kommentares \\ \hline
	\end{tabular}
\end{table}
\paragraph{Antwort}Die Antwort ist wie folgt aufgebaut:
\begin{table}[H]
	\begin{tabular}{|c|c|c|p{6.5cm}|}
		\hline
		\textbf{Paramtername} & \textbf{Datentyp} & \textbf{Konstante} & \textbf{Kurzbeschreibung}                                                                                               \\ \hline
		result              & string           &                 & Erfolgreich wenn Wert {\glqq ack\grqq} ist \\ \hline
		Code                & int              &                 & Erfolgreich wenn Wert {\glqq 0\grqq} ist \\ \hline
	\end{tabular}
\end{table}
\subsubsection{Abfrage der Abrufdaten eines Bildes}
\paragraph{Kurzbeschreibung}Dieser API-Request wird dazu genutzt um Daten zum Abrufen eines bestimmten Bildes zu erhalten.
\paragraph{Anfrage}Folgende Daten werden zu Anfrage benötigt:
\begin{table}[H]
	\begin{tabular}{|c|c|c|p{6.5cm}|}
		\hline
		\textbf{Paramtername} & \textbf{Datentyp} & \textbf{Konstante} & \textbf{Kurzbeschreibung}                                                                                               \\ \hline
		type                & string            & dsm                & Bild-URI abfragen \\ \hline
		token               & string            &                    & Identifikator des Bildes \\ \hline
	\end{tabular}
\end{table}
\paragraph{Antwort}Die Antwort ist wie folgt aufgebaut:
\begin{table}[H]
	\begin{tabular}{|c|c|c|p{6.5cm}|}
		\hline
		\textbf{Paramtername} & \textbf{Datentyp} & \textbf{Konstante} & \textbf{Kurzbeschreibung}                                                                                               \\ \hline
		result              & string           &                 & Erfolgreich wenn Wert {\glqq ack\grqq} ist \\ \hline
		Code                & int              &                 & Erfolgreich wenn Wert {\glqq 0\grqq} ist \\ \hline
		token               & string           &                 & Identifikator des Bildes \\ \hline
		title               & string           &                 & Titel des Bildes \\ \hline
		description         & string           &                 & Beschreibung des Bildes \\ \hline
		picture             & string           &                 & Bild-URI \\ \hline
		sourcetype          & string           &                 & Name des Typs der Quelle \\ \hline
		source              & string           &                 & Quellenangabe \\ \hline
		sourcetypeid        & int              &                 & Identifikator des Typs der Quelle \\ \hline
	\end{tabular}
\end{table}
\subsubsection{Ändern der Metadaten eines Bildes}
\paragraph{Kurzbeschreibung}Dieser API-Request wird dazu genutzt um Daten eines Bildes zu ändern.
\paragraph{Anfrage}Folgende Daten werden zu Anfrage benötigt:
\begin{table}[H]
	\begin{tabular}{|c|c|c|p{6.5cm}|}
		\hline
		\textbf{Paramtername} & \textbf{Datentyp} & \textbf{Konstante} & \textbf{Kurzbeschreibung}                                                                                               \\ \hline
		type                & string            & ssm                & Ändern von Bildmetadaten \\ \hline
		token               & string            &                    & Identifikator des Bildes \\ \hline
		title               & string            &                    & Titel des Bildes \\ \hline
		description         & string            &                    & Beschreibung des Bildes \\ \hline
		source              & string            &                    & Quellenangabe (optional) \\ \hline
		sourcetype          & int               &                    & Identifikator des Typs der Quelle (optional) \\ \hline
	\end{tabular}
\end{table}
\paragraph{Antwort}Die Antwort ist wie folgt aufgebaut:
\begin{table}[H]
	\begin{tabular}{|c|c|c|p{6.5cm}|}
		\hline
		\textbf{Paramtername} & \textbf{Datentyp} & \textbf{Konstante} & \textbf{Kurzbeschreibung}                                                                                               \\ \hline
		result              & string           &                 & Erfolgreich wenn Wert {\glqq ack\grqq} ist \\ \hline
		Code                & int              &                 & Erfolgreich wenn Wert {\glqq 0\grqq} ist \\ \hline
	\end{tabular}
\end{table}
\subsubsection{Historische Adresse hinzufügen}
\paragraph{Kurzbeschreibung}Dieser API-Request wird dazu genutzt um eine historische Adresse zu einem Interessenpunkt hinzuzufügen.
\paragraph{Anfrage}Folgende Daten werden zu Anfrage benötigt:
\begin{table}[H]
	\begin{tabular}{|c|c|c|p{6.5cm}|}
		\hline
		\textbf{Paramtername} & \textbf{Datentyp} & \textbf{Konstante} & \textbf{Kurzbeschreibung}                                                                                               \\ \hline
		type                & string            & aha                & Historische Adresse hinzufügen \\ \hline
		poi\_id             & int               &                    & Identifikator des Interessenpunktes \\ \hline
		from                & int               &                    & Adresse benutzt ab \\ \hline
		till                & int               &                    & Adresse benutzt bis \\ \hline
		streetname          & string            &                    & Straßenname \\ \hline
		housenumber         & string            &                    & Hausnummer \\ \hline
		city                & string            &                    & Stadt \\ \hline
		postalcode          & int               &                    & Postleitzahl \\ \hline
	\end{tabular}
\end{table}
\paragraph{Antwort}Die Antwort ist wie folgt aufgebaut:
\begin{table}[H]
	\begin{tabular}{|c|c|c|p{6.5cm}|}
		\hline
		\textbf{Paramtername} & \textbf{Datentyp} & \textbf{Konstante} & \textbf{Kurzbeschreibung}                                                                                               \\ \hline
		result              & string           &                 & Erfolgreich wenn Wert {\glqq ack\grqq} ist \\ \hline
		Code                & int              &                 & Erfolgreich wenn Wert {\glqq 0\grqq} ist \\ \hline
		type                & string           & aha             & Historische Adresse hinzufügen \\ \hline
		poi\_id             & int              &                 & Identifikator des Interessenpunktes \\ \hline
		from                & int              &                 & Adresse benutzt ab \\ \hline
		till                & int              &                 & Adresse benutzt bis \\ \hline
		streetname          & string           &                 & Straßenname \\ \hline
		housenumber         & string           &                 & Hausnummer \\ \hline
		city                & string           &                 & Stadt \\ \hline
		postalcode          & int              &                 & Postleitzahl \\ \hline
	\end{tabular}
\end{table}
\subsubsection{Betreiber hinzufügen}
\paragraph{Kurzbeschreibung}Dieser API-Request wird dazu genutzt um einen Betreiber zu einem Interessenpunkt hinzuzufügen.
\paragraph{Anfrage}Folgende Daten werden zu Anfrage benötigt:
\begin{table}[H]
	\begin{tabular}{|c|c|c|p{6.5cm}|}
		\hline
		\textbf{Paramtername} & \textbf{Datentyp} & \textbf{Konstante} & \textbf{Kurzbeschreibung}                                                                                               \\ \hline
		type                & string            & ado                & Historische Adresse hinzufügen \\ \hline
		poi\_id             & int               &                    & Identifikator des Interessenpunktes \\ \hline
		from                & int               &                    & Betreiber ab \\ \hline
		till                & int               &                    & Betreiber bis \\ \hline
		operator            & string            &                    & Betreibername \\ \hline
	\end{tabular}
\end{table}
\paragraph{Antwort}Die Antwort ist wie folgt aufgebaut:
\begin{table}[H]
	\begin{tabular}{|c|c|c|p{6.5cm}|}
		\hline
		\textbf{Paramtername} & \textbf{Datentyp} & \textbf{Konstante} & \textbf{Kurzbeschreibung}                                                                                               \\ \hline
		result              & string           &                 & Erfolgreich wenn Wert {\glqq ack\grqq} ist \\ \hline
		Code                & int              &                 & Erfolgreich wenn Wert {\glqq 0\grqq} ist \\ \hline
		type                & string           & ado             & Historische Adresse hinzufügen \\ \hline
		poi\_id             & int              &                 & Identifikator des Interessenpunktes \\ \hline
		from                & int              &                 & Betreiber ab \\ \hline
		till                & int              &                 & Betreiber bis \\ \hline
		operator            & string           &                 & Betreibername \\ \hline
	\end{tabular}
\end{table}
\subsubsection{Name hinzufügen}
\paragraph{Kurzbeschreibung}Dieser API-Request wird dazu genutzt um einen Namen zu einem Interessenpunkt hinzuzufügen.
\paragraph{Anfrage}Folgende Daten werden zu Anfrage benötigt:
\begin{table}[H]
	\begin{tabular}{|c|c|c|p{6.5cm}|}
		\hline
		\textbf{Paramtername} & \textbf{Datentyp} & \textbf{Konstante} & \textbf{Kurzbeschreibung}                                                                                               \\ \hline
		type                & string            & adn                & Namen hinzufügen \\ \hline
		poi\_id             & int               &                    & Identifikator des Interessenpunktes \\ \hline
		from                & int               &                    & Name benutzt ab \\ \hline
		till                & int               &                    & Name benutzt bis \\ \hline
		name                & string            &                    & Name des Interessenpunkts \\ \hline
	\end{tabular}
\end{table}
\paragraph{Antwort}Die Antwort ist wie folgt aufgebaut:
\begin{table}[H]
	\begin{tabular}{|c|c|c|p{6.5cm}|}
		\hline
		\textbf{Paramtername} & \textbf{Datentyp} & \textbf{Konstante} & \textbf{Kurzbeschreibung}                                                                                               \\ \hline
		result              & string           &                 & Erfolgreich wenn Wert {\glqq ack\grqq} ist \\ \hline
		Code                & int              &                 & Erfolgreich wenn Wert {\glqq 0\grqq} ist \\ \hline
		type                & string           & adn             & Namen hinzufügen \\ \hline
		poi\_id             & int              &                 & Identifikator des Interessenpunktes \\ \hline
		from                & int              &                 & Name benutzt ab \\ \hline
		till                & int              &                 & Name benutzt bis \\ \hline
 		name                & string           &                 & Name des Interessenpunkts \\ \hline
	\end{tabular}
\end{table}
\subsubsection{Historische Adresse löschen}
\paragraph{Kurzbeschreibung}Dieser API-Request wird dazu genutzt um eine historische Adresse eines Interessenpunkts zu löschen.
\paragraph{Anfrage}Folgende Daten werden zu Anfrage benötigt:
\begin{table}[H]
	\begin{tabular}{|c|c|c|p{6.5cm}|}
		\hline
		\textbf{Paramtername} & \textbf{Datentyp} & \textbf{Konstante} & \textbf{Kurzbeschreibung}                                                                                               \\ \hline
		type                & string            & dha                & historische Adresse löschen \\ \hline
		IDent               & int               &                    & Identifikator der historischen Adresse \\ \hline
	\end{tabular}
\end{table}
\paragraph{Antwort}Die Antwort ist wie folgt aufgebaut:
\begin{table}[H]
	\begin{tabular}{|c|c|c|p{6.5cm}|}
		\hline
		\textbf{Paramtername} & \textbf{Datentyp} & \textbf{Konstante} & \textbf{Kurzbeschreibung}                                                                                               \\ \hline
		result              & string           &                 & Erfolgreich wenn Wert {\glqq ack\grqq} ist \\ \hline
		Code                & int              &                 & Erfolgreich wenn Wert {\glqq 0\grqq} ist \\ \hline
		type                & string           & dha             & historische Adresse löschen \\ \hline
		IDent               & int              &                 & Identifikator der historischen Adresse \\ \hline
	\end{tabular}
\end{table}
\subsubsection{Betreiber löschen}
\paragraph{Kurzbeschreibung}Dieser API-Request wird dazu genutzt um einen Betreiber eines Interessenpunkts zu löschen.
\paragraph{Anfrage}Folgende Daten werden zu Anfrage benötigt:
\begin{table}[H]
	\begin{tabular}{|c|c|c|p{6.5cm}|}
		\hline
		\textbf{Paramtername} & \textbf{Datentyp} & \textbf{Konstante} & \textbf{Kurzbeschreibung}                                                                                               \\ \hline
		type                & string            & dop                & Betreiber löschen \\ \hline
		IDent               & int               &                    & Identifikator des Betreibers \\ \hline
	\end{tabular}
\end{table}
\paragraph{Antwort}Die Antwort ist wie folgt aufgebaut:
\begin{table}[H]
	\begin{tabular}{|c|c|c|p{6.5cm}|}
		\hline
		\textbf{Paramtername} & \textbf{Datentyp} & \textbf{Konstante} & \textbf{Kurzbeschreibung}                                                                                               \\ \hline
		result              & string           &                 & Erfolgreich wenn Wert {\glqq ack\grqq} ist \\ \hline
		Code                & int              &                 & Erfolgreich wenn Wert {\glqq 0\grqq} ist \\ \hline
		type                & string           & dop             & Betreiber löschen \\ \hline
		IDent               & int              &                 & Identifikator des Betreibers \\ \hline
	\end{tabular}
\end{table}
\subsubsection{Name löschen}
\paragraph{Kurzbeschreibung}Dieser API-Request wird dazu genutzt um einen Namen eines Interessenpunkts zu löschen.
\paragraph{Anfrage}Folgende Daten werden zu Anfrage benötigt:
\begin{table}[H]
	\begin{tabular}{|c|c|c|p{6.5cm}|}
		\hline
		\textbf{Paramtername} & \textbf{Datentyp} & \textbf{Konstante} & \textbf{Kurzbeschreibung}                                                                                               \\ \hline
		type                & string            & dna                & Name löschen \\ \hline
		IDent               & int               &                    & Identifikator des Namens \\ \hline
	\end{tabular}
\end{table}
\paragraph{Antwort}Die Antwort ist wie folgt aufgebaut:
\begin{table}[H]
	\begin{tabular}{|c|c|c|p{6.5cm}|}
		\hline
		\textbf{Paramtername} & \textbf{Datentyp} & \textbf{Konstante} & \textbf{Kurzbeschreibung}                                                                                               \\ \hline
		result              & string           &                 & Erfolgreich wenn Wert {\glqq ack\grqq} ist \\ \hline
		Code                & int              &                 & Erfolgreich wenn Wert {\glqq 0\grqq} ist \\ \hline
		type                & string           & dna             & Name löschen \\ \hline
		IDent               & int              &                 & Identifikator des Namens \\ \hline
	\end{tabular}
\end{table}
\subsubsection{Historische Adresse validieren}
\paragraph{Kurzbeschreibung}Dieser API-Request wird dazu genutzt um eine historische Adresse eines Interessenpunkts zu validieren.
\paragraph{Anfrage}Folgende Daten werden zu Anfrage benötigt:
\begin{table}[H]
	\begin{tabular}{|c|c|c|p{6.5cm}|}
		\hline
		\textbf{Paramtername} & \textbf{Datentyp} & \textbf{Konstante} & \textbf{Kurzbeschreibung}                                                                                               \\ \hline
		type                & string            & vha                & historische Adresse validieren \\ \hline
		ADDRESSID           & int               &                    & Identifikator der historischen Adresse \\ \hline
	\end{tabular}
\end{table}
\paragraph{Antwort}Die Antwort ist wie folgt aufgebaut:
\begin{table}[H]
	\begin{tabular}{|c|c|c|p{6.5cm}|}
		\hline
		\textbf{Paramtername} & \textbf{Datentyp} & \textbf{Konstante} & \textbf{Kurzbeschreibung}                                                                                               \\ \hline
		result              & string           &                 & Erfolgreich wenn Wert {\glqq ack\grqq} ist \\ \hline
		Code                & int              &                 & Erfolgreich wenn Wert {\glqq 0\grqq} ist \\ \hline
		type                & string           & vha             & historische Adresse validieren \\ \hline
		ADDRESSID           & int              &                 & Identifikator der historischen Adresse \\ \hline
	\end{tabular}
\end{table}
\subsubsection{Betreiber validieren}
\paragraph{Kurzbeschreibung}Dieser API-Request wird dazu genutzt um einen Betreiber eines Interessenpunkts zu validieren.
\paragraph{Anfrage}Folgende Daten werden zu Anfrage benötigt:
\begin{table}[H]
	\begin{tabular}{|c|c|c|p{6.5cm}|}
		\hline
		\textbf{Paramtername} & \textbf{Datentyp} & \textbf{Konstante} & \textbf{Kurzbeschreibung}                                                                                               \\ \hline
		type                & string            & vop                & Betreiber validieren \\ \hline
		OPERATORID          & int               &                    & Identifikator des Betreibers \\ \hline
	\end{tabular}
\end{table}
\paragraph{Antwort}Die Antwort ist wie folgt aufgebaut:
\begin{table}[H]
	\begin{tabular}{|c|c|c|p{6.5cm}|}
		\hline
		\textbf{Paramtername} & \textbf{Datentyp} & \textbf{Konstante} & \textbf{Kurzbeschreibung}                                                                                               \\ \hline
		result              & string           &                 & Erfolgreich wenn Wert {\glqq ack\grqq} ist \\ \hline
		Code                & int              &                 & Erfolgreich wenn Wert {\glqq 0\grqq} ist \\ \hline
		type                & string           & vop             & Betreiber validieren \\ \hline
		OPERATORID          & int              &                 & Identifikator des Betreibers \\ \hline
	\end{tabular}
\end{table}
\subsubsection{Name validieren}
\paragraph{Kurzbeschreibung}Dieser API-Request wird dazu genutzt um einen Namen eines Interessenpunkts zu validieren.
\paragraph{Anfrage}Folgende Daten werden zu Anfrage benötigt:
\begin{table}[H]
	\begin{tabular}{|c|c|c|p{6.5cm}|}
		\hline
		\textbf{Paramtername} & \textbf{Datentyp} & \textbf{Konstante} & \textbf{Kurzbeschreibung}                                                                                               \\ \hline
		type                & string            & vna                & Name validieren \\ \hline
		NAMEID              & int               &                    & Identifikator des Namens \\ \hline
	\end{tabular}
\end{table}
\paragraph{Antwort}Die Antwort ist wie folgt aufgebaut:
\begin{table}[H]
	\begin{tabular}{|c|c|c|p{6.5cm}|}
		\hline
		\textbf{Paramtername} & \textbf{Datentyp} & \textbf{Konstante} & \textbf{Kurzbeschreibung}                                                                                               \\ \hline
		result              & string           &                 & Erfolgreich wenn Wert {\glqq ack\grqq} ist \\ \hline
		Code                & int              &                 & Erfolgreich wenn Wert {\glqq 0\grqq} ist \\ \hline
		type                & string           & vna             & Name validieren \\ \hline
		NAMEID              & int              &                 & Identifikator des Namens \\ \hline
	\end{tabular}
\end{table}
\subsubsection{Geschichte eines Interessenpunktes validieren}
\paragraph{Kurzbeschreibung}Dieser API-Request wird dazu genutzt um die Geschichte eines Interessenpunktes zu validieren.
\paragraph{Anfrage}Folgende Daten werden zu Anfrage benötigt:
\begin{table}[H]
	\begin{tabular}{|c|c|c|p{6.5cm}|}
		\hline
		\textbf{Paramtername} & \textbf{Datentyp} & \textbf{Konstante} & \textbf{Kurzbeschreibung}                                                                                               \\ \hline
		type                & string            & vts                & Geschichte eines Interessenpunktes validieren \\ \hline
		POIID               & int               &                    & Identifikator des Interessenpunktes \\ \hline
	\end{tabular}
\end{table}
\paragraph{Antwort}Die Antwort ist wie folgt aufgebaut:
\begin{table}[H]
	\begin{tabular}{|c|c|c|p{6.5cm}|}
		\hline
		\textbf{Paramtername} & \textbf{Datentyp} & \textbf{Konstante} & \textbf{Kurzbeschreibung}                                                                                               \\ \hline
		result              & string           &                 & Erfolgreich wenn Wert {\glqq ack\grqq} ist \\ \hline
		Code                & int              &                 & Erfolgreich wenn Wert {\glqq 0\grqq} ist \\ \hline
		type                & string           & vts             & Geschichte eines Interessenpunktes validieren \\ \hline
		POIID               & int              &                 & Identifikator des Interessenpunktes \\ \hline
	\end{tabular}
\end{table}
\subsubsection{Aktuelle Adresse eines Interessenpunktes validieren}
\paragraph{Kurzbeschreibung}Dieser API-Request wird dazu genutzt um die aktuelle Adresse eines Interessenpunktes zu validieren.
\paragraph{Anfrage}Folgende Daten werden zu Anfrage benötigt:
\begin{table}[H]
	\begin{tabular}{|c|c|c|p{6.5cm}|}
		\hline
		\textbf{Paramtername} & \textbf{Datentyp} & \textbf{Konstante} & \textbf{Kurzbeschreibung}                                                                                               \\ \hline
		type                & string            & vca                & Aktuelle Adresse eines Interessenpunktes validieren \\ \hline
		POIID               & int               &                    & Identifikator des Interessenpunktes \\ \hline
	\end{tabular}
\end{table}
\paragraph{Antwort}Die Antwort ist wie folgt aufgebaut:
\begin{table}[H]
	\begin{tabular}{|c|c|c|p{6.5cm}|}
		\hline
		\textbf{Paramtername} & \textbf{Datentyp} & \textbf{Konstante} & \textbf{Kurzbeschreibung}                                                                                               \\ \hline
		result              & string           &                 & Erfolgreich wenn Wert {\glqq ack\grqq} ist \\ \hline
		Code                & int              &                 & Erfolgreich wenn Wert {\glqq 0\grqq} ist \\ \hline
		type                & string           & vca             & Aktuelle Adresse eines Interessenpunktes validieren \\ \hline
		POIID               & int              &                 & Identifikator des Interessenpunktes \\ \hline
	\end{tabular}
\end{table}
\subsubsection{Betriebszeitraum eines Interessenpunktes validieren}
\paragraph{Kurzbeschreibung}Dieser API-Request wird dazu genutzt um den Betriebszeitraum eines Interessenpunktes zu validieren.
\paragraph{Anfrage}Folgende Daten werden zu Anfrage benötigt:
\begin{table}[H]
	\begin{tabular}{|c|c|c|p{6.5cm}|}
		\hline
		\textbf{Paramtername} & \textbf{Datentyp} & \textbf{Konstante} & \textbf{Kurzbeschreibung}                                                                                               \\ \hline
		type                & string            & vhi                & Betriebszeitraum eines Interessenpunktes validieren \\ \hline
		POIID               & int               &                    & Identifikator des Interessenpunktes \\ \hline
	\end{tabular}
\end{table}
\paragraph{Antwort}Die Antwort ist wie folgt aufgebaut:
\begin{table}[H]
	\begin{tabular}{|c|c|c|p{6.5cm}|}
		\hline
		\textbf{Paramtername} & \textbf{Datentyp} & \textbf{Konstante} & \textbf{Kurzbeschreibung}                                                                                               \\ \hline
		result              & string           &                 & Erfolgreich wenn Wert {\glqq ack\grqq} ist \\ \hline
		Code                & int              &                 & Erfolgreich wenn Wert {\glqq 0\grqq} ist \\ \hline
		type                & string           & vhi             & Betriebszeitraum eines Interessenpunktes validieren \\ \hline
		POIID               & int              &                 & Identifikator des Interessenpunktes \\ \hline
	\end{tabular}
\end{table}

\subsubsection{Historische Adresse ändern}
\paragraph{Kurzbeschreibung}Dieser API-Request wird dazu genutzt um eine historische Adresse zu einem Interessenpunkt ändern.
\paragraph{Anfrage}Folgende Daten werden zu Anfrage benötigt:
\begin{table}[H]
	\begin{tabular}{|c|c|c|p{6.5cm}|}
		\hline
		\textbf{Paramtername} & \textbf{Datentyp} & \textbf{Konstante} & \textbf{Kurzbeschreibung}                                                                                               \\ \hline
		type                & string            & uha                & Historische Adresse ändern \\ \hline
		id                  & int               &                    & Identifikator der historischen Adresse \\ \hline
		from                & int               &                    & Adresse benutzt ab \\ \hline
		till                & int               &                    & Adresse benutzt bis \\ \hline
		streetname          & string            &                    & Straßenname \\ \hline
		housenumber         & string            &                    & Hausnummer \\ \hline
		city                & string            &                    & Stadt \\ \hline
		postalcode          & int               &                    & Postleitzahl \\ \hline
	\end{tabular}
\end{table}
\paragraph{Antwort}Die Antwort ist wie folgt aufgebaut:
\begin{table}[H]
	\begin{tabular}{|c|c|c|p{6.5cm}|}
		\hline
		\textbf{Paramtername} & \textbf{Datentyp} & \textbf{Konstante} & \textbf{Kurzbeschreibung}                                                                                               \\ \hline
		result              & string           &                 & Erfolgreich wenn Wert {\glqq ack\grqq} ist \\ \hline
		Code                & int              &                 & Erfolgreich wenn Wert {\glqq 0\grqq} ist \\ \hline
		state               & bool             &                 & Wahr, wenn erfolgreich \\ \hline
	\end{tabular}
\end{table}
\subsubsection{Betreiber ändern}
\paragraph{Kurzbeschreibung}Dieser API-Request wird dazu genutzt um einen Betreiber zu einem Interessenpunkt ändern.
\paragraph{Anfrage}Folgende Daten werden zu Anfrage benötigt:
\begin{table}[H]
	\begin{tabular}{|c|c|c|p{6.5cm}|}
		\hline
		\textbf{Paramtername} & \textbf{Datentyp} & \textbf{Konstante} & \textbf{Kurzbeschreibung}                                                                                               \\ \hline
		type                & string            & uop                & Historische Adresse ändern \\ \hline
		id                  & int               &                    & Identifikator des Betreibers \\ \hline
		from                & int               &                    & Betreiber ab \\ \hline
		till                & int               &                    & Betreiber bis \\ \hline
		operator            & string            &                    & Betreibername \\ \hline
	\end{tabular}
\end{table}
\paragraph{Antwort}Die Antwort ist wie folgt aufgebaut:
\begin{table}[H]
	\begin{tabular}{|c|c|c|p{6.5cm}|}
		\hline
		\textbf{Paramtername} & \textbf{Datentyp} & \textbf{Konstante} & \textbf{Kurzbeschreibung}                                                                                               \\ \hline
		result              & string           &                 & Erfolgreich wenn Wert {\glqq ack\grqq} ist \\ \hline
		Code                & int              &                 & Erfolgreich wenn Wert {\glqq 0\grqq} ist \\ \hline
		state               & bool             &                 & Wahr, wenn erfolgreich \\ \hline
	\end{tabular}
\end{table}
\subsubsection{Name ändern}
\paragraph{Kurzbeschreibung}Dieser API-Request wird dazu genutzt um einen Namen zu einem Interessenpunkt ändern.
\paragraph{Anfrage}Folgende Daten werden zu Anfrage benötigt:
\begin{table}[H]
	\begin{tabular}{|c|c|c|p{6.5cm}|}
		\hline
		\textbf{Paramtername} & \textbf{Datentyp} & \textbf{Konstante} & \textbf{Kurzbeschreibung}                                                                                               \\ \hline
		type                & string            & una                & Namen ändern \\ \hline
		id                  & int               &                    & Identifikator des Namens \\ \hline
		from                & int               &                    & Name benutzt ab \\ \hline
		till                & int               &                    & Name benutzt bis \\ \hline
		name                & string            &                    & Name des Interessenpunkts \\ \hline
	\end{tabular}
\end{table}
\paragraph{Antwort}Die Antwort ist wie folgt aufgebaut:
\begin{table}[H]
	\begin{tabular}{|c|c|c|p{6.5cm}|}
		\hline
		\textbf{Paramtername} & \textbf{Datentyp} & \textbf{Konstante} & \textbf{Kurzbeschreibung}                                                                                               \\ \hline
		result              & string           &                 & Erfolgreich wenn Wert {\glqq ack\grqq} ist \\ \hline
		Code                & int              &                 & Erfolgreich wenn Wert {\glqq 0\grqq} ist \\ \hline
		state               & bool             &                 & Wahr, wenn erfolgreich \\ \hline
	\end{tabular}
\end{table}
\subsubsection{Bild löschen}
\paragraph{Kurzbeschreibung}Dieser API-Request wird dazu genutzt um ein Bild zu löschen.
\paragraph{Anfrage}Folgende Daten werden zu Anfrage benötigt:
\begin{table}[H]
	\begin{tabular}{|c|c|c|p{6.5cm}|}
		\hline
		\textbf{Paramtername} & \textbf{Datentyp} & \textbf{Konstante} & \textbf{Kurzbeschreibung}                                                                                               \\ \hline
		type                & string            & dsp                & Bild löschen\\ \hline
		token               & string            &                    & Identifikator des Bildes \\ \hline
	\end{tabular}
\end{table}
\paragraph{Antwort}Die Antwort ist wie folgt aufgebaut:
\begin{table}[H]
	\begin{tabular}{|c|c|c|p{6.5cm}|}
		\hline
		\textbf{Paramtername} & \textbf{Datentyp} & \textbf{Konstante} & \textbf{Kurzbeschreibung}                                                                                               \\ \hline
		result              & string           &                 & Erfolgreich wenn Wert {\glqq ack\grqq} ist \\ \hline
		Code                & int              &                 & Erfolgreich wenn Wert {\glqq 0\grqq} ist \\ \hline
	\end{tabular}
\end{table}
\subsubsection{Abfrage der Titel aller Interessenpunkte}
\paragraph{Kurzbeschreibung}Dieser API-Request wird dazu genutzt um die Titel aller Interessenpunkt abzufragen, welche noch nicht mit einer spezifischen Geschichte verlinkt sind.
\paragraph{Anfrage}Folgende Daten werden zu Anfrage benötigt:
\begin{table}[H]
	\begin{tabular}{|c|c|c|p{6.5cm}|}
		\hline
		\textbf{Paramtername} & \textbf{Datentyp} & \textbf{Konstante} & \textbf{Kurzbeschreibung}                                                                                               \\ \hline
		type                & string            & gpt                & Titel aller Interessenpunkte abfragen\\ \hline
		storytoken          & string            &                    & Identifikator der Geschichte \\ \hline
	\end{tabular}
\end{table}
\paragraph{Antwort}Die Antwort ist wie folgt aufgebaut:
\begin{table}[H]
	\begin{tabular}{|c|c|c|p{6.5cm}|}
		\hline
		\textbf{Paramtername} & \textbf{Datentyp} & \textbf{Konstante} & \textbf{Kurzbeschreibung}                                                                                               \\ \hline
		result              & string           &                 & Erfolgreich wenn Wert {\glqq ack\grqq} ist \\ \hline
		Code                & int              &                 & Erfolgreich wenn Wert {\glqq 0\grqq} ist \\ \hline
		data                & array            &                 & Liste der Titel \\ \hline
	\end{tabular}
\end{table}
\subsubsection{Verlinkung von Interessenpunkt und Geschichte}
\paragraph{Kurzbeschreibung}Dieser API-Request wird dazu genutzt um eine Geschichte und einen Interessenpunkt zu verlinken.
\paragraph{Anfrage}Folgende Daten werden zu Anfrage benötigt:
\begin{table}[H]
	\begin{tabular}{|c|c|c|p{6.5cm}|}
		\hline
		\textbf{Paramtername} & \textbf{Datentyp} & \textbf{Konstante} & \textbf{Kurzbeschreibung}                                                                                               \\ \hline
		type                & string            & aps                & Geschichte und Interessenpunkt verlinken\\ \hline
		storytoken          & string            &                    & Identifikator der Geschichte \\ \hline
		poiid               & int               &                    & Identifikator des Interessenpunktes \\ \hline
	\end{tabular}
\end{table}
\paragraph{Antwort}Die Antwort ist wie folgt aufgebaut:
\begin{table}[H]
	\begin{tabular}{|c|c|c|p{6.5cm}|}
		\hline
		\textbf{Paramtername} & \textbf{Datentyp} & \textbf{Konstante} & \textbf{Kurzbeschreibung}                                                                                               \\ \hline
		result              & string           &                 & Erfolgreich wenn Wert {\glqq ack\grqq} ist \\ \hline
		Code                & int              &                 & Erfolgreich wenn Wert {\glqq 0\grqq} ist \\ \hline
		data                & bool             &                 & Erfolgreich wenn Wert {\glqq true\grqq} ist \\ \hline
	\end{tabular}
\end{table}
\subsubsection{Abfrage von Verlinkungen zwischen Interessenpunkt und Geschichte}
\paragraph{Kurzbeschreibung}Dieser API-Request wird dazu genutzt um die Verlinkungen einer bestimmten Geschichte mit Interessenpunkten ab zu fragen.
\paragraph{Anfrage}Folgende Daten werden zu Anfrage benötigt:
\begin{table}[H]
	\begin{tabular}{|c|c|c|p{6.5cm}|}
		\hline
		\textbf{Paramtername} & \textbf{Datentyp} & \textbf{Konstante} & \textbf{Kurzbeschreibung}                                                                                               \\ \hline
		type                & string            & gps                & Abfrage Verknüpfung Geschichte und Interessenpunkt \\ \hline
		storytoken          & string            &                    & Identifikator der Geschichte \\ \hline
	\end{tabular}
\end{table}
\paragraph{Antwort}Die Antwort ist wie folgt aufgebaut:
\begin{table}[H]
	\begin{tabular}{|c|c|c|p{6.5cm}|}
		\hline
		\textbf{Paramtername} & \textbf{Datentyp} & \textbf{Konstante} & \textbf{Kurzbeschreibung}                                                                                               \\ \hline
		result              & string           &                 & Erfolgreich wenn Wert {\glqq ack\grqq} ist \\ \hline
		Code                & int              &                 & Erfolgreich wenn Wert {\glqq 0\grqq} ist \\ \hline
		data                & array            &                 & Strukturiertes Ergebnis \\ \hline
	\end{tabular}
\end{table}
\subparagraph{data}Dieses Array enthält Einträge in der nachstehend dargestellten Form haben:
\begin{table}[H]
	\begin{tabular}{|c|c|c|p{6.5cm}|}
		\hline
		\textbf{Paramtername} & \textbf{Datentyp} & \textbf{Konstante} & \textbf{Kurzbeschreibung}    \\ \hline
		guest                  & bool            &                 & Wahr, wenn Nutzer Gast \\ \hline
		valpos                 & int             &                 & Wahr, wenn Nutzer Validierungswert größer 0 hat \\ \hline
		pois                   & array           &                 & Liste der verknüpften Interessenpunkte \\ \hline
	\end{tabular}
\end{table}
\subparagraph{pois}Dieses Array enthält Elemente mit Einträge in der nachstehend dargestellten Form haben:
\begin{table}[H]
	\begin{tabular}{|c|c|c|p{6.5cm}|}
		\hline
		\textbf{Paramtername} & \textbf{Datentyp} & \textbf{Konstante} & \textbf{Kurzbeschreibung}    \\ \hline
		deletable              & bool            &                 & Wahr, wenn Verlinkung Löschen darf \\ \hline
		deleted                & bool            &                 & Wahr, wenn Link als gelöscht gilt \\ \hline
		validate               & int             &                 & Wahr, wenn Link validiert \\ \hline
		id                     & int             &                 & Identifikator des Links \\ \hline
		poi\_id                & int             &                 & Identifikator des Interessenpunktes \\ \hline
		name                   & string          &                 & Name des Interessenpunktes \\ \hline
		restrictions           & bool            &                 & Wahr, wenn Wiederherstellung gelöschte Abhängigkeiten hat \\ \hline
	\end{tabular}
\end{table}
\subsubsection{Validierung einer Verlinkungen zwischen Interessenpunkt und Geschichte}
\paragraph{Kurzbeschreibung}Dieser API-Request wird dazu genutzt um die Verlinkungen einer bestimmten Geschichte mit einem Interessenpunkt zu validieren.
\paragraph{Anfrage}Folgende Daten werden zu Anfrage benötigt:
\begin{table}[H]
	\begin{tabular}{|c|c|c|p{6.5cm}|}
		\hline
		\textbf{Paramtername} & \textbf{Datentyp} & \textbf{Konstante} & \textbf{Kurzbeschreibung}                                                                                               \\ \hline
		type                & string            & vps                & Validierung Verknüpfung Geschichte und Interessenpunkt \\ \hline
		poiStoryId          & int               &                    & Identifikator der Verlinkung \\ \hline
	\end{tabular}
\end{table}
\paragraph{Antwort}Die Antwort ist wie folgt aufgebaut:
\begin{table}[H]
	\begin{tabular}{|c|c|c|p{6.5cm}|}
		\hline
		\textbf{Paramtername} & \textbf{Datentyp} & \textbf{Konstante} & \textbf{Kurzbeschreibung}                                                                                               \\ \hline
		result              & string           &                 & Erfolgreich wenn Wert {\glqq ack\grqq} ist \\ \hline
		Code                & int              &                 & Erfolgreich wenn Wert {\glqq 0\grqq} ist \\ \hline
		type                & string           & vps             & Validierung Verknüpfung Geschichte und Interessenpunkt \\ \hline
		poiStoryId          & int              &                 & Identifikator der Verlinkung \\ \hline
	\end{tabular}
\end{table}
\subsubsection{Löschen einer Verlinkungen zwischen Interessenpunkt und Geschichte}
\paragraph{Kurzbeschreibung}Dieser API-Request wird dazu genutzt um die Verlinkungen einer bestimmten Geschichte mit einem Interessenpunkt zu löschen.
\paragraph{Anfrage}Folgende Daten werden zu Anfrage benötigt:
\begin{table}[H]
	\begin{tabular}{|c|c|c|p{6.5cm}|}
		\hline
		\textbf{Paramtername} & \textbf{Datentyp} & \textbf{Konstante} & \textbf{Kurzbeschreibung}                                                                                               \\ \hline
		type                & string            & dps                & Löschung Verknüpfung Geschichte und Interessenpunkt \\ \hline
		poiStoryId          & int               &                    & Identifikator der Verlinkung \\ \hline
	\end{tabular}
\end{table}
\paragraph{Antwort}Die Antwort ist wie folgt aufgebaut:
\begin{table}[H]
	\begin{tabular}{|c|c|c|p{6.5cm}|}
		\hline
		\textbf{Paramtername} & \textbf{Datentyp} & \textbf{Konstante} & \textbf{Kurzbeschreibung}                                                                                               \\ \hline
		result              & string           &                 & Erfolgreich wenn Wert {\glqq ack\grqq} ist \\ \hline
		Code                & int              &                 & Erfolgreich wenn Wert {\glqq 0\grqq} ist \\ \hline
		type                & string           & dps             & Löschung Verknüpfung Geschichte und Interessenpunkt \\ \hline
		poiStoryId          & int              &                 & Identifikator der Verlinkung \\ \hline
	\end{tabular}
\end{table}
\subsubsection{Löschen einer Geschichte}
\paragraph{Kurzbeschreibung}Dieser API-Request wird dazu genutzt um die Verlinkungen eine bestimmten Geschichte zu löschen.
\paragraph{Anfrage}Folgende Daten werden zu Anfrage benötigt:
\begin{table}[H]
	\begin{tabular}{|c|c|c|p{6.5cm}|}
		\hline
		\textbf{Paramtername} & \textbf{Datentyp} & \textbf{Konstante} & \textbf{Kurzbeschreibung}                                                                                               \\ \hline
		type                & string            & dus                & Löschung Verknüpfung Geschichte und Interessenpunkt \\ \hline
		story\_token        & string            &                    & Identifikator der Geschichte \\ \hline
	\end{tabular}
\end{table}
\paragraph{Antwort}Die Antwort ist wie folgt aufgebaut:
\begin{table}[H]
	\begin{tabular}{|c|c|c|p{6.5cm}|}
		\hline
		\textbf{Paramtername} & \textbf{Datentyp} & \textbf{Konstante} & \textbf{Kurzbeschreibung}                                                                                               \\ \hline
		result              & string           &                 & Erfolgreich wenn Wert {\glqq ack\grqq} ist \\ \hline
		Code                & int              &                 & Erfolgreich wenn Wert {\glqq 0\grqq} ist \\ \hline
		data                & array            &                 & Strukturiert wie die Anfrage \\ \hline
	\end{tabular}
\end{table}
\subparagraph{data}Dieses Array enthält Einträge in der nachstehend dargestellten Form haben:
\begin{table}[H]
	\begin{tabular}{|c|c|c|p{6.5cm}|}
		\hline
		\textbf{Paramtername} & \textbf{Datentyp} & \textbf{Konstante} & \textbf{Kurzbeschreibung}    \\ \hline
		type                   & string          & dus             & Löschung Verknüpfung Geschichte und Interessenpunkt \\ \hline
		story\_token           & string          &                 & Identifikator der Geschichte \\ \hline
	\end{tabular}
\end{table}
\subsubsection{Existenzprüfung einer Adresse}
\paragraph{Kurzbeschreibung}Dieser API-Request wird dazu genutzt um zu überprüfen, ob eine bestimmte Adresse bereits in der Datenbank vorhanden ist.
\paragraph{Anfrage}Folgende Daten werden zu Anfrage benötigt:
\begin{table}[H]
	\begin{tabular}{|c|c|c|p{6.5cm}|}
		\hline
		\textbf{Paramtername} & \textbf{Datentyp} & \textbf{Konstante} & \textbf{Kurzbeschreibung}                                                                                               \\ \hline
		type                & string            & cha                & Existenzprüfung einer Adresse \\ \hline
		st                  & string            &                    & Straßenname \\ \hline
		hn                  & string            &                    & Hausnummer \\ \hline
		ct                  & string            &                    & Stadt oder Ort \\ \hline
		pc                  & int               &                    & Postleitzahl \\ \hline
	\end{tabular}
\end{table}
\paragraph{Antwort}Die Antwort ist wie folgt aufgebaut:
\begin{table}[H]
	\begin{tabular}{|c|c|c|p{6.5cm}|}
		\hline
		\textbf{Paramtername} & \textbf{Datentyp} & \textbf{Konstante} & \textbf{Kurzbeschreibung}                                                                                               \\ \hline
		result              & string           &                 & Erfolgreich wenn Wert {\glqq ack\grqq} ist \\ \hline
		Code                & int              &                 & Erfolgreich wenn Wert {\glqq 0\grqq} ist \\ \hline
		data                & bool             &                 & Wahr, wenn Adresse bereits in Datenbank \\ \hline
		request             & array            &                 & Strukturiert wie Anfrage \\ \hline
	\end{tabular}
\end{table}
\subparagraph{request}Dieses Array enthält Einträge in der nachstehend dargestellten Form haben:
\begin{table}[H]
	\begin{tabular}{|c|c|c|p{6.5cm}|}
		\hline
		\textbf{Paramtername} & \textbf{Datentyp} & \textbf{Konstante} & \textbf{Kurzbeschreibung}    \\ \hline
		type                   & string          & cha             & Existenzprüfung einer Adresse \\ \hline
		st                     & string          &                 & Straßenname \\ \hline
		hn                     & string          &                 & Hausnummer \\ \hline
		ct                     & string          &                 & Stadt oder Ort \\ \hline
		pc                     & int             &                 & Postleitzahl \\ \hline
	\end{tabular}
\end{table}
\subsubsection{Abfrage der Ladedaten von Bildern}
\paragraph{Kurzbeschreibung}Dieser API-Request wird dazu genutzt um die Daten zum laden mehrerer Bilder zu bekommen.
\paragraph{Anfrage}Folgende Daten werden zu Anfrage benötigt:
\begin{table}[H]
	\begin{tabular}{|c|c|c|p{6.5cm}|}
		\hline
		\textbf{Paramtername} & \textbf{Datentyp} & \textbf{Konstante} & \textbf{Kurzbeschreibung}                                                                                               \\ \hline
		type                & string            & gpf                & Abfrage der Ladedaten von Bildern \\ \hline
	\end{tabular}
\end{table}
\paragraph{Antwort}Die Antwort ist wie folgt aufgebaut:
\begin{table}[H]
	\begin{tabular}{|c|c|c|p{6.5cm}|}
		\hline
		\textbf{Paramtername} & \textbf{Datentyp} & \textbf{Konstante} & \textbf{Kurzbeschreibung}                                                                                               \\ \hline
		result              & string           &                 & Erfolgreich wenn Wert {\glqq ack\grqq} ist \\ \hline
		Code                & int              &                 & Erfolgreich wenn Wert {\glqq 0\grqq} ist \\ \hline
		data                & array            &                 & Array mit allen benötigten Daten \\ \hline
	\end{tabular}
\end{table}
\subparagraph{data}Dieses Array enthält Einträge in der nachstehend dargestellten Form haben:
\begin{table}[H]
	\begin{tabular}{|c|c|c|p{6.5cm}|}
		\hline
		\textbf{Paramtername} & \textbf{Datentyp} & \textbf{Konstante} & \textbf{Kurzbeschreibung}    \\ \hline
		description            & string          &                 & Beschreibung des Bildes \\ \hline
		fullsize               & string          &                 & URI zum Laden des Bildes \\ \hline
		id                     & int             &                 & Identifikator des Bildes \\ \hline
		identifier             & string          &                 & zweiter Identifikator des Bildes \\ \hline
		preview                & string          &                 & URI zum Laden einer Vorschau \\ \hline
		title                  & string          &                 & Titel des Bildes \\ \hline
		token                  & array           &                 & Informationen zum Laden des Bildes \\ \hline
		username               & string          &                 & Nutzername des Erstellers \\ \hline
		valUsers               & array           &                 & Array mit allen Validatoren \\ \hline
		validationValue        & int             &                 & Validierungsstatus \\ \hline
	\end{tabular}
\end{table}
\subparagraph{token}Dieses Array enthält Einträge in der nachstehend dargestellten Form haben:
\begin{table}[H]
	\begin{tabular}{|c|c|c|p{6.5cm}|}
		\hline
		\textbf{Paramtername} & \textbf{Datentyp} & \textbf{Konstante} & \textbf{Kurzbeschreibung}    \\ \hline
		seccode                & string          &                 & Security Code \\ \hline
		token                  & string          &                 & Identifikator des Bildes \\ \hline
		time                   & int             &                 & Timestamp \\ \hline
	\end{tabular}
\end{table}
\subsubsection{Verknüpfung zwischen Interessenpunkt und Bild hinzufügen}
\paragraph{Kurzbeschreibung}Dieser API-Request wird dazu genutzt um eine Verlinkung zwischen einem Bild und einem Interessenpunkt herzustellen.
\paragraph{Anfrage}Folgende Daten werden zu Anfrage benötigt:
\begin{table}[H]
	\begin{tabular}{|c|c|c|p{6.5cm}|}
		\hline
		\textbf{Paramtername} & \textbf{Datentyp} & \textbf{Konstante} & \textbf{Kurzbeschreibung}                                                                                               \\ \hline
		type                & string            & app                & Anlegen eines neuen Links \\ \hline
		poi                 & int               &                    & Identifikator des Interessenpunktes \\ \hline
		data                & array             &                    & Array mit Bilderidentifikatoren \\ \hline
	\end{tabular}
\end{table}
\paragraph{Antwort}Die Antwort ist wie folgt aufgebaut:
\begin{table}[H]
	\begin{tabular}{|c|c|c|p{6.5cm}|}
		\hline
		\textbf{Paramtername} & \textbf{Datentyp} & \textbf{Konstante} & \textbf{Kurzbeschreibung}                                                                                               \\ \hline
		result              & string           &                 & Erfolgreich wenn Wert {\glqq ack\grqq} ist \\ \hline
		Code                & int              &                 & Erfolgreich wenn Wert {\glqq 0\grqq} ist \\ \hline
	\end{tabular}
\end{table}
\subsubsection{Validierung einer Verknüpfung zwischen Interessenpunkt und Bild hinzufügen}
\paragraph{Kurzbeschreibung}Dieser API-Request wird dazu genutzt um eine Verlinkung zwischen einem Bild und einem Interessenpunkt zu validieren.
\paragraph{Anfrage}Folgende Daten werden zu Anfrage benötigt:
\begin{table}[H]
	\begin{tabular}{|c|c|c|p{6.5cm}|}
		\hline
		\textbf{Paramtername} & \textbf{Datentyp} & \textbf{Konstante} & \textbf{Kurzbeschreibung}                                                                                               \\ \hline
		type                & string            & vpp                & Anlegen eines neuen Links \\ \hline
		id                  & int               &                    & Identifikator der Verknüpfung \\ \hline
	\end{tabular}
\end{table}
\paragraph{Antwort}Die Antwort ist wie folgt aufgebaut:
\begin{table}[H]
	\begin{tabular}{|c|c|c|p{6.5cm}|}
		\hline
		\textbf{Paramtername} & \textbf{Datentyp} & \textbf{Konstante} & \textbf{Kurzbeschreibung}                                                                                               \\ \hline
		result              & string           &                 & Erfolgreich wenn Wert {\glqq ack\grqq} ist \\ \hline
		Code                & int              &                 & Erfolgreich wenn Wert {\glqq 0\grqq} ist \\ \hline
	\end{tabular}
\end{table}
\subsubsection{Löschung einer Verknüpfung zwischen Interessenpunkt und Bild hinzufügen}
\paragraph{Kurzbeschreibung}Dieser API-Request wird dazu genutzt um eine Verlinkung zwischen einem Bild und einem Interessenpunkt zu löschen.
\paragraph{Anfrage}Folgende Daten werden zu Anfrage benötigt:
\begin{table}[H]
	\begin{tabular}{|c|c|c|p{6.5cm}|}
		\hline
		\textbf{Paramtername} & \textbf{Datentyp} & \textbf{Konstante} & \textbf{Kurzbeschreibung}                                                                                               \\ \hline
		type                & string            & vpp                & Löschen eines Links \\ \hline
		id                  & int               &                    & Identifikator der Verknüpfung \\ \hline
	\end{tabular}
\end{table}
\paragraph{Antwort}Die Antwort ist wie folgt aufgebaut:
\begin{table}[H]
	\begin{tabular}{|c|c|c|p{6.5cm}|}
		\hline
		\textbf{Paramtername} & \textbf{Datentyp} & \textbf{Konstante} & \textbf{Kurzbeschreibung}                                                                                               \\ \hline
		result              & string           &                 & Erfolgreich wenn Wert {\glqq ack\grqq} ist \\ \hline
		Code                & int              &                 & Erfolgreich wenn Wert {\glqq 0\grqq} ist \\ \hline
	\end{tabular}
\end{table}
\subsubsection{Abfrage alle Informationen zum Laden des Interessenpunkt-Bild-Link-Modals}
\paragraph{Kurzbeschreibung}Dieser API-Request wird dazu genutzt um alle Informationen zum Anzeigen des Bild-Interessenpunkt-Verknüpfers im Biographien Modul zu laden.
\paragraph{Anfrage}Folgende Daten werden zu Anfrage benötigt:
\begin{table}[H]
	\begin{tabular}{|c|c|c|p{6.5cm}|}
		\hline
		\textbf{Paramtername} & \textbf{Datentyp} & \textbf{Konstante} & \textbf{Kurzbeschreibung}                                                                                               \\ \hline
		type                & string            & lpp                & Abfrage aller verfügbaren Interessenpunktitel \\ \hline
		pictoken            & string            &                    & Identifikator eines Bildes \\ \hline
	\end{tabular}
\end{table}
\paragraph{Antwort}Die Antwort ist wie folgt aufgebaut:
\begin{table}[H]
	\begin{tabular}{|c|c|c|p{6.5cm}|}
		\hline
		\textbf{Paramtername} & \textbf{Datentyp} & \textbf{Konstante} & \textbf{Kurzbeschreibung}                                                                                               \\ \hline
		result              & string           &                 & Erfolgreich wenn Wert {\glqq ack\grqq} ist \\ \hline
		Code                & int              &                 & Erfolgreich wenn Wert {\glqq 0\grqq} ist \\ \hline
		data                & array            &                 & Array mit Verknüpfungsinformationen \\ \hline
	\end{tabular}
\end{table}
\subparagraph{token}Dieses Array enthält Einträge in der nachstehend dargestellten Form haben:
\begin{table}[H]
	\begin{tabular}{|c|c|c|p{6.5cm}|}
		\hline
		\textbf{Paramtername} & \textbf{Datentyp} & \textbf{Konstante} & \textbf{Kurzbeschreibung}    \\ \hline
		guest                  & bool            &                 & Wahr wenn Nutzer Gast ist \\ \hline
		linked                 & Array           &                 & Bestehende Verlinkungen \\ \hline
		options                & int             &                 & Nicht bestehende Verlinkungen \\ \hline
		valpos                 & bool            &                 & Wahr wenn Nutzer validieren darf \\ \hline
	\end{tabular}
\end{table}
\subparagraph{linked}Dieses Array enthält Elemente mit Einträgen in der nachstehend dargestellten Form haben:
\begin{table}[H]
	\begin{tabular}{|c|c|c|p{6.5cm}|}
		\hline
		\textbf{Paramtername} & \textbf{Datentyp} & \textbf{Konstante} & \textbf{Kurzbeschreibung}    \\ \hline
		deletable              & bool            &                 & Gibt an, ob Nutzer Link löschen darf \\ \hline
		lid                    & int             &                 & Identifikator des Links\\ \hline
		name                   & string          &                 & Name des Interessenpunktes \\ \hline
		poi\_id                & int             &                 & Identifikator des Interessenpunktes \\ \hline
		validated              & bool            &                 & Validierungsstatus des Links \\ \hline
		deleted                & bool            &                 & Wahr, wenn Link als gelöscht markiert \\ \hline
		restrictions           & bool            &                 & Wahr, wenn Abhängigkeiten als gelöscht gelten \\ \hline
	\end{tabular}
\end{table}
\subparagraph{options}Dieses Array enthält Elemente mit Einträgen in der nachstehend dargestellten Form haben:
\begin{table}[H]
	\begin{tabular}{|c|c|c|p{6.5cm}|}
		\hline
		\textbf{Paramtername} & \textbf{Datentyp} & \textbf{Konstante} & \textbf{Kurzbeschreibung}    \\ \hline
		name                   & string          &                 & Name des Interessenpunktes \\ \hline
		poi\_id                & int             &                 & Identifikator des Interessenpunktes \\ \hline
	\end{tabular}
\end{table}
\subsubsection{Sitzplatzanzahl zu Interessenpunkt hinzufügen}
\paragraph{Kurzbeschreibung}Dieser API-Request wird dazu genutzt um eine neue Sitzplatzanzahl zu einem Interessenpunkt hinzuzufügen.
\paragraph{Anfrage}Folgende Daten werden zu Anfrage benötigt:
\begin{table}[H]
	\begin{tabular}{|c|c|c|p{6.5cm}|}
		\hline
		\textbf{Paramtername} & \textbf{Datentyp} & \textbf{Konstante} & \textbf{Kurzbeschreibung}                                                                                               \\ \hline
		type                & string            & asc                & Sitzplatzanzahl hinzufügen \\ \hline
		poi\_id             & int               &                    & Identifikator eines Interessenpunktes \\ \hline
		from                & int               &                    & Zeitpunkt des Startes der Sitzplatzanzahl \\ \hline
		till                & int               &                    & Zeitpunkt des Endes der Sitzplatzanzahl \\ \hline
		seats               & int               &                    & Sitzplatzanzahl \\ \hline
	\end{tabular}
\end{table}
\paragraph{Antwort}Die Antwort ist wie folgt aufgebaut:
\begin{table}[H]
	\begin{tabular}{|c|c|c|p{6.5cm}|}
		\hline
		\textbf{Paramtername} & \textbf{Datentyp} & \textbf{Konstante} & \textbf{Kurzbeschreibung}                                                                                               \\ \hline
		result              & string           &                 & Erfolgreich wenn Wert {\glqq ack\grqq} ist \\ \hline
		Code                & int              &                 & Erfolgreich wenn Wert {\glqq 0\grqq} ist \\ \hline
		type                & string           & asc             & Sitzplatzanzahl hinzufügen \\ \hline
		poi\_id             & int              &                 & Identifikator eines Interessenpunktes \\ \hline
		from                & int              &                 & Zeitpunkt des Startes der Sitzplatzanzahl \\ \hline
		till                & int              &                 & Zeitpunkt des Endes der Sitzplatzanzahl \\ \hline
		seats               & int              &                 & Sitzplatzanzahl \\ \hline
	\end{tabular}
\end{table}
\subsubsection{Validierung einer Sitzplatzanzahl}
\paragraph{Kurzbeschreibung}Dieser API-Request wird dazu genutzt um eine Sitzplatzanzahl zu validieren.
\paragraph{Anfrage}Folgende Daten werden zu Anfrage benötigt:
\begin{table}[H]
	\begin{tabular}{|c|c|c|p{6.5cm}|}
		\hline
		\textbf{Paramtername} & \textbf{Datentyp} & \textbf{Konstante} & \textbf{Kurzbeschreibung}                                                                                               \\ \hline
		type                & string            & vsc                & Sitzplatzanzahl validieren \\ \hline
		SEATID              & int               &                    & Identifikator der Sitzplatzanzahl \\ \hline
	\end{tabular}
\end{table}
\paragraph{Antwort}Die Antwort ist wie folgt aufgebaut:
\begin{table}[H]
	\begin{tabular}{|c|c|c|p{6.5cm}|}
		\hline
		\textbf{Paramtername} & \textbf{Datentyp} & \textbf{Konstante} & \textbf{Kurzbeschreibung}                                                                                               \\ \hline
		result              & string           &                 & Erfolgreich wenn Wert {\glqq ack\grqq} ist \\ \hline
		Code                & int              &                 & Erfolgreich wenn Wert {\glqq 0\grqq} ist \\ \hline
		type                & string           & vsc             & Sitzplatzanzahl validieren \\ \hline
		SEATID              & int              &                 & Identifikator der Sitzplatzanzahl \\ \hline
	\end{tabular}
\end{table}
\subsubsection{Löschung einer Sitzplatzanzahl}
\paragraph{Kurzbeschreibung}Dieser API-Request wird dazu genutzt um eine Sitzplatzanzahl zu löschen.
\paragraph{Anfrage}Folgende Daten werden zu Anfrage benötigt:
\begin{table}[H]
	\begin{tabular}{|c|c|c|p{6.5cm}|}
		\hline
		\textbf{Paramtername} & \textbf{Datentyp} & \textbf{Konstante} & \textbf{Kurzbeschreibung}                                                                                               \\ \hline
		type                & string            & dsc                & Sitzplatzanzahl löschen \\ \hline
		SEATID              & int               &                    & Identifikator der Sitzplatzanzahl \\ \hline
	\end{tabular}
\end{table}
\paragraph{Antwort}Die Antwort ist wie folgt aufgebaut:
\begin{table}[H]
	\begin{tabular}{|c|c|c|p{6.5cm}|}
		\hline
		\textbf{Paramtername} & \textbf{Datentyp} & \textbf{Konstante} & \textbf{Kurzbeschreibung}                                                                                               \\ \hline
		result              & string           &                 & Erfolgreich wenn Wert {\glqq ack\grqq} ist \\ \hline
		Code                & int              &                 & Erfolgreich wenn Wert {\glqq 0\grqq} ist \\ \hline
		type                & string           & dsc             & Sitzplatzanzahl löschen \\ \hline
		SEATID              & int              &                 & Identifikator der Sitzplatzanzahl \\ \hline
	\end{tabular}
\end{table}
\subsubsection{Speichern einer Sitzplatzanzahl}
\paragraph{Kurzbeschreibung}Dieser API-Request wird dazu genutzt um eine bestehende Sitzplatzanzahl zu Speichern.
\paragraph{Anfrage}Folgende Daten werden zu Anfrage benötigt:
\begin{table}[H]
	\begin{tabular}{|c|c|c|p{6.5cm}|}
		\hline
		\textbf{Paramtername} & \textbf{Datentyp} & \textbf{Konstante} & \textbf{Kurzbeschreibung}                                                                                               \\ \hline
		type                & string            & usc                & Sitzplatzanzahl löschen \\ \hline
		id                  & int               &                    & Identifikator der Sitzplatzanzahl \\ \hline
		start               & int               &                    & Zeitpunkt des Startes der Sitzplatzanzahl \\ \hline
		end                 & int               &                    & Zeitpunkt des Endes der Sitzplatzanzahl \\ \hline
		seats               & int               &                    & Sitzplatzanzahl \\ \hline
	\end{tabular}
\end{table}
\paragraph{Antwort}Die Antwort ist wie folgt aufgebaut:
\begin{table}[H]
	\begin{tabular}{|c|c|c|p{6.5cm}|}
		\hline
		\textbf{Paramtername} & \textbf{Datentyp} & \textbf{Konstante} & \textbf{Kurzbeschreibung}                                                                                               \\ \hline
		result              & string           &                 & Erfolgreich wenn Wert {\glqq ack\grqq} ist \\ \hline
		Code                & int              &                 & Erfolgreich wenn Wert {\glqq 0\grqq} ist \\ \hline
		type                & string           & usc             & Sitzplatzanzahl löschen \\ \hline
		SEATID              & int              &                 & Identifikator der Sitzplatzanzahl \\ \hline
	\end{tabular}
\end{table}

\subsubsection{Saalanzahl zu Interessenpunkt hinzufügen}
\paragraph{Kurzbeschreibung}Dieser API-Request wird dazu genutzt um eine neue Saalanzahl zu einem Interessenpunkt hinzuzufügen.
\paragraph{Anfrage}Folgende Daten werden zu Anfrage benötigt:
\begin{table}[H]
	\begin{tabular}{|c|c|c|p{6.5cm}|}
		\hline
		\textbf{Paramtername} & \textbf{Datentyp} & \textbf{Konstante} & \textbf{Kurzbeschreibung}                                                                                               \\ \hline
		type                & string            & acc                & Saalanzahl hinzufügen \\ \hline
		poi\_id             & int               &                    & Identifikator eines Interessenpunktes \\ \hline
		from                & int               &                    & Zeitpunkt des Startes der Saalanzahl \\ \hline
		till                & int               &                    & Zeitpunkt des Endes der Saalanzahl \\ \hline
		cinemas             & int               &                    & Saalanzahl \\ \hline
	\end{tabular}
\end{table}
\paragraph{Antwort}Die Antwort ist wie folgt aufgebaut:
\begin{table}[H]
	\begin{tabular}{|c|c|c|p{6.5cm}|}
		\hline
		\textbf{Paramtername} & \textbf{Datentyp} & \textbf{Konstante} & \textbf{Kurzbeschreibung}                                                                                               \\ \hline
		result              & string           &                 & Erfolgreich wenn Wert {\glqq ack\grqq} ist \\ \hline
		Code                & int              &                 & Erfolgreich wenn Wert {\glqq 0\grqq} ist \\ \hline
		type                & string           & acc             & Saalanzahl hinzufügen \\ \hline
		poi\_id             & int              &                 & Identifikator eines Interessenpunktes \\ \hline
		from                & int              &                 & Zeitpunkt des Startes der Saalanzahl \\ \hline
		till                & int              &                 & Zeitpunkt des Endes der Saalanzahl \\ \hline
		cinemas             & int              &                 & Saalanzahl \\ \hline
	\end{tabular}
\end{table}
\subsubsection{Validierung einer Saalanzahl}
\paragraph{Kurzbeschreibung}Dieser API-Request wird dazu genutzt um eine Saalanzahl zu validieren.
\paragraph{Anfrage}Folgende Daten werden zu Anfrage benötigt:
\begin{table}[H]
	\begin{tabular}{|c|c|c|p{6.5cm}|}
		\hline
		\textbf{Paramtername} & \textbf{Datentyp} & \textbf{Konstante} & \textbf{Kurzbeschreibung}                                                                                               \\ \hline
		type                & string            & vcc                & Saalanzahl validieren \\ \hline
		CINEMAID            & int               &                    & Identifikator der Saalanzahl \\ \hline
	\end{tabular}
\end{table}
\paragraph{Antwort}Die Antwort ist wie folgt aufgebaut:
\begin{table}[H]
	\begin{tabular}{|c|c|c|p{6.5cm}|}
		\hline
		\textbf{Paramtername} & \textbf{Datentyp} & \textbf{Konstante} & \textbf{Kurzbeschreibung}                                                                                               \\ \hline
		result              & string           &                 & Erfolgreich wenn Wert {\glqq ack\grqq} ist \\ \hline
		Code                & int              &                 & Erfolgreich wenn Wert {\glqq 0\grqq} ist \\ \hline
		type                & string           & vcc             & Saalanzahl validieren \\ \hline
		CINEMAID            & int              &                 & Identifikator der Saalanzahl \\ \hline
	\end{tabular}
\end{table}
\subsubsection{Löschung einer Saalanzahl}
\paragraph{Kurzbeschreibung}Dieser API-Request wird dazu genutzt um eine Saalanzahl zu löschen.
\paragraph{Anfrage}Folgende Daten werden zu Anfrage benötigt:
\begin{table}[H]
	\begin{tabular}{|c|c|c|p{6.5cm}|}
		\hline
		\textbf{Paramtername} & \textbf{Datentyp} & \textbf{Konstante} & \textbf{Kurzbeschreibung}                                                                                               \\ \hline
		type                & string            & dcc                & Saalanzahl löschen \\ \hline
		IDent               & int               &                    & Identifikator der Saalanzahl \\ \hline
	\end{tabular}
\end{table}
\paragraph{Antwort}Die Antwort ist wie folgt aufgebaut:
\begin{table}[H]
	\begin{tabular}{|c|c|c|p{6.5cm}|}
		\hline
		\textbf{Paramtername} & \textbf{Datentyp} & \textbf{Konstante} & \textbf{Kurzbeschreibung}                                                                                               \\ \hline
		result              & string           &                 & Erfolgreich wenn Wert {\glqq ack\grqq} ist \\ \hline
		Code                & int              &                 & Erfolgreich wenn Wert {\glqq 0\grqq} ist \\ \hline
		type                & string           & dcc             & Saalanzahl löschen \\ \hline
		IDent               & int              &                 & Identifikator der Saalanzahl \\ \hline
	\end{tabular}
\end{table}
\subsubsection{Speichern einer Saalanzahl}
\paragraph{Kurzbeschreibung}Dieser API-Request wird dazu genutzt um eine bestehende Saalanzahl zu Speichern.
\paragraph{Anfrage}Folgende Daten werden zu Anfrage benötigt:
\begin{table}[H]
	\begin{tabular}{|c|c|c|p{6.5cm}|}
		\hline
		\textbf{Paramtername} & \textbf{Datentyp} & \textbf{Konstante} & \textbf{Kurzbeschreibung}                                                                                               \\ \hline
		type                & string            & ucc                & Saalanzahl löschen \\ \hline
		id                  & int               &                    & Identifikator der Saalanzahl \\ \hline
		start               & int               &                    & Zeitpunkt des Startes der Saalanzahl \\ \hline
		end                 & int               &                    & Zeitpunkt des Endes der Saalanzahl \\ \hline
		cinemas             & int               &                    & Saalanzahl \\ \hline
	\end{tabular}
\end{table}
\paragraph{Antwort}Die Antwort ist wie folgt aufgebaut:
\begin{table}[H]
	\begin{tabular}{|c|c|c|p{6.5cm}|}
		\hline
		\textbf{Paramtername} & \textbf{Datentyp} & \textbf{Konstante} & \textbf{Kurzbeschreibung}                                                                                               \\ \hline
		result              & string           &                 & Erfolgreich wenn Wert {\glqq ack\grqq} ist \\ \hline
		Code                & int              &                 & Erfolgreich wenn Wert {\glqq 0\grqq} ist \\ \hline
		type                & string           & ucc             & Saalanzahl löschen \\ \hline
		cinemas             & int              &                 & Identifikator der Saalanzahl \\ \hline
	\end{tabular}
\end{table}

\subsubsection{Validierung des Spielstättentyp}
\paragraph{Kurzbeschreibung}Dieser API-Request wird dazu genutzt um den Spielstättentyp eines Interessenpunktes zu validieren.
\paragraph{Anfrage}Folgende Daten werden zu Anfrage benötigt:
\begin{table}[H]
	\begin{tabular}{|c|c|c|p{6.5cm}|}
		\hline
		\textbf{Paramtername} & \textbf{Datentyp} & \textbf{Konstante} & \textbf{Kurzbeschreibung}                                                                                               \\ \hline
		type                & string            & vty                & Validierung Spielstättentyp \\ \hline
		POIID               & int               &                    & Identifikator des Interessenpunktes \\ \hline
	\end{tabular}
\end{table}
\paragraph{Antwort}Die Antwort ist wie folgt aufgebaut:
\begin{table}[H]
	\begin{tabular}{|c|c|c|p{6.5cm}|}
		\hline
		\textbf{Paramtername} & \textbf{Datentyp} & \textbf{Konstante} & \textbf{Kurzbeschreibung}                                                                                               \\ \hline
		result              & string           &                 & Erfolgreich wenn Wert {\glqq ack\grqq} ist \\ \hline
		Code                & int              &                 & Erfolgreich wenn Wert {\glqq 0\grqq} ist \\ \hline
		type                & string           & vty             & Validierung Spielstättentyp \\ \hline
		POIID               & int              &                 & Identifikator des Interessenpunktes \\ \hline
	\end{tabular}
\end{table}
\subsubsection{Abfrage Nutzer als Gast}
\paragraph{Kurzbeschreibung}Dieser API-Request wird dazu genutzt um abzufragen, ob aktueller Nutzer Gast ist.
\paragraph{Anfrage}Folgende Daten werden zu Anfrage benötigt:
\begin{table}[H]
	\begin{tabular}{|c|c|c|p{6.5cm}|}
		\hline
		\textbf{Paramtername} & \textbf{Datentyp} & \textbf{Konstante} & \textbf{Kurzbeschreibung}                                                                                               \\ \hline
		type                & string            & asg                & Gastabfrage \\ \hline
		POIID               & int               &                    & Identifikator des Interessenpunktes \\ \hline
	\end{tabular}
\end{table}
\paragraph{Antwort}Die Antwort ist wie folgt aufgebaut:
\begin{table}[H]
	\begin{tabular}{|c|c|c|p{6.5cm}|}
		\hline
		\textbf{Paramtername} & \textbf{Datentyp} & \textbf{Konstante} & \textbf{Kurzbeschreibung}                                                                                               \\ \hline
		result              & string           &                 & Erfolgreich wenn Wert {\glqq ack\grqq} ist \\ \hline
		Code                & int              &                 & Erfolgreich wenn Wert {\glqq 0\grqq} ist \\ \hline
		data                & bool             &                 & Wahr, wenn Nutzer Gast ist \\ \hline
	\end{tabular}
\end{table}
\subsubsection{Abfrage von Statistikdaten}
\paragraph{Kurzbeschreibung}Dieser API-Request wird dazu genutzt um Statistikdaten abzufragen.
\paragraph{Anfrage}Folgende Daten werden zu Anfrage benötigt:
\begin{table}[H]
	\begin{tabular}{|c|c|c|p{6.5cm}|}
		\hline
		\textbf{Paramtername} & \textbf{Datentyp} & \textbf{Konstante} & \textbf{Kurzbeschreibung}                                                                                               \\ \hline
		type                & string            & gsd                & Statistikdaten Abfragen\\ \hline
		data                & array             &                    & Informationen zur Abfrage \\ \hline
	\end{tabular}
\end{table}
\subparagraph{data}Dieses Array enthält Einträge in der nachstehend dargestellten Form haben:
\begin{table}[H]
	\begin{tabular}{|c|c|c|p{6.5cm}|}
		\hline
		\textbf{Paramtername} & \textbf{Datentyp} & \textbf{Konstante} & \textbf{Kurzbeschreibung}    \\ \hline
		data                   & array           &                 & Abfrage von Zeiträumen \\ \hline
		src                    & string          &                 & Quelle der statistischen Informationen \\ \hline
	\end{tabular}
\end{table}
\subparagraph{data}Dieses Array enthält Elemente mit Einträgen in der nachstehend dargestellten Form haben:
\begin{table}[H]
	\begin{tabular}{|c|c|c|p{6.5cm}|}
		\hline
		\textbf{Paramtername} & \textbf{Datentyp} & \textbf{Konstante} & \textbf{Kurzbeschreibung}    \\ \hline
		type                   & char            &                 & Zeiteinheit (D: Tag, W: Week, M: Month, Y:Year) \\ \hline
		Amount                 & int             &                 & Anzahl an Zeiteinheiten \\ \hline
		ID                     & int             &                 & fortlaufender Identifikator \\ \hline
	\end{tabular}
\end{table}
\paragraph{Antwort}Die Antwort ist wie folgt aufgebaut:
\begin{table}[H]
	\begin{tabular}{|c|c|c|p{6.5cm}|}
		\hline
		\textbf{Paramtername} & \textbf{Datentyp} & \textbf{Konstante} & \textbf{Kurzbeschreibung}                                                                                               \\ \hline
		result              & string           &                 & Erfolgreich wenn Wert {\glqq ack\grqq} ist \\ \hline
		Code                & int              &                 & Erfolgreich wenn Wert {\glqq 0\grqq} ist \\ \hline
		data                & array            &                 & Ergebnisse \\ \hline
	\end{tabular}
\end{table}
\subparagraph{data}Dieses Array enthält Elemente mit Einträgen in der nachstehend dargestellten Form haben:
\begin{table}[H]
	\begin{tabular}{|c|c|c|p{6.5cm}|}
		\hline
		\textbf{Paramtername} & \textbf{Datentyp} & \textbf{Konstante} & \textbf{Kurzbeschreibung}    \\ \hline
		type                   & char            &                 & Zeiteinheit (D: Tag, W: Week, M: Month, Y:Year) \\ \hline
		Amount                 & int             &                 & Anzahl an Zeiteinheiten \\ \hline
		ID                     & int             &                 & fortlaufender Identifikator \\ \hline
		data                   & array           &                 & Darstellungsinformationen für Chart.js \\ \hline
	\end{tabular}
\end{table}
\subparagraph{Anmerkungen}Für den Aufbau der Darstellungsinformationen ist die Dokumentation von Chart.js zu konsultieren.

\subsubsection{Geschichte freigeben}
\paragraph{Kurzbeschreibung}Dieser API-Request wird dazu genutzt um eine durch einen Nutzer veröffentlichte Geschichte freizugeben.
\paragraph{Anfrage}Folgende Daten werden zu Anfrage benötigt:
\begin{table}[H]
	\begin{tabular}{|c|c|c|p{6.5cm}|}
		\hline
		\textbf{Paramtername} & \textbf{Datentyp} & \textbf{Konstante} & \textbf{Kurzbeschreibung}                                                                                               \\ \hline
		type                & string            & asa                & Geschichte freigeben \\ \hline
		story\_token        & string            &                    & Identifikator einer Geschichte \\ \hline
	\end{tabular}
\end{table}
\paragraph{Antwort}Die Antwort ist wie folgt aufgebaut:
\begin{table}[H]
	\begin{tabular}{|c|c|c|p{6.5cm}|}
		\hline
		\textbf{Paramtername} & \textbf{Datentyp} & \textbf{Konstante} & \textbf{Kurzbeschreibung}                                                                                               \\ \hline
		result              & string           &                 & Erfolgreich wenn Wert {\glqq ack\grqq} ist \\ \hline
		Code                & int              &                 & Erfolgreich wenn Wert {\glqq 0\grqq} ist \\ \hline
	\end{tabular}
\end{table}
\subsubsection{Geschichte sperren}
\paragraph{Kurzbeschreibung}Dieser API-Request wird dazu genutzt um eine durch einen Nutzer veröffentlichte Geschichte zu sperren.
\paragraph{Anfrage}Folgende Daten werden zu Anfrage benötigt:
\begin{table}[H]
	\begin{tabular}{|c|c|c|p{6.5cm}|}
		\hline
		\textbf{Paramtername} & \textbf{Datentyp} & \textbf{Konstante} & \textbf{Kurzbeschreibung}                                                                                               \\ \hline
		type                & string            & das                & Geschichte sperren \\ \hline
		story\_token        & string            &                    & Identifikator einer Geschichte \\ \hline
	\end{tabular}
\end{table}
\paragraph{Antwort}Die Antwort ist wie folgt aufgebaut:
\begin{table}[H]
	\begin{tabular}{|c|c|c|p{6.5cm}|}
		\hline
		\textbf{Paramtername} & \textbf{Datentyp} & \textbf{Konstante} & \textbf{Kurzbeschreibung}                                                                                               \\ \hline
		result              & string           &                 & Erfolgreich wenn Wert {\glqq ack\grqq} ist \\ \hline
		Code                & int              &                 & Erfolgreich wenn Wert {\glqq 0\grqq} ist \\ \hline
	\end{tabular}
\end{table}

\subsubsection{Namen eines Interessenpunktes abfragen}
\paragraph{Kurzbeschreibung}Dieser API-Request wird dazu genutzt um alle Namen eines Interessenpunktes ab zu fragen.
\paragraph{Anfrage}Folgende Daten werden zu Anfrage benötigt:
\begin{table}[H]
	\begin{tabular}{|c|c|c|p{6.5cm}|}
		\hline
		\textbf{Paramtername} & \textbf{Datentyp} & \textbf{Konstante} & \textbf{Kurzbeschreibung}                                                                                               \\ \hline
		type                & string            & snp                & Namen abfragen \\ \hline
		poi\_id             & int               &                    & Identifikator eines Interessenpunktes \\ \hline
	\end{tabular}
\end{table}
\paragraph{Antwort}Die Antwort ist wie folgt aufgebaut:
\begin{table}[H]
	\begin{tabular}{|c|c|c|p{6.5cm}|}
		\hline
		\textbf{Paramtername} & \textbf{Datentyp} & \textbf{Konstante} & \textbf{Kurzbeschreibung}                                                                                               \\ \hline
		result              & string           &                 & Erfolgreich wenn Wert {\glqq ack\grqq} ist \\ \hline
		Code                & int              &                 & Erfolgreich wenn Wert {\glqq 0\grqq} ist \\ \hline
		data                & array            &                 & Angefragte Informationen \\ \hline
	\end{tabular}
\end{table}
\subparagraph{data}Dieses Array enthält Elemente mit Einträgen in der nachstehend dargestellten Form haben:
\begin{table}[H]
	\begin{tabular}{|c|c|c|p{6.5cm}|}
		\hline
		\textbf{Paramtername} & \textbf{Datentyp} & \textbf{Konstante} & \textbf{Kurzbeschreibung}    \\ \hline
		ID                     & int             &                 & Identifikator des Namens (ab Position >0) \\ \hline
		editable               & bool            &                 & Wahr, wenn Eintrag bearbeitbar (ab Position >0) \\ \hline
		end                    & int             &                 & Ende der Nutzungszeit des Namens \\ \hline
		name                   & string          &                 & Name \\ \hline
		start                  & string          &                 & Anfang der Nutzungszeit des Namens \\ \hline
		validatable            & bool            &                 & Wahr, wenn Eintrag validierbar (ab Position >0) \\ \hline
		deleted                & bool            &                 & Wahr, wenn Eintrag als gelöscht gilt \\ \hline
	\end{tabular}
\end{table}

\subsubsection{Betreiber eines Interessenpunktes abfragen}
\paragraph{Kurzbeschreibung}Dieser API-Request wird dazu genutzt um alle Betreiber eines Interessenpunktes ab zu fragen.
\paragraph{Anfrage}Folgende Daten werden zu Anfrage benötigt:
\begin{table}[H]
	\begin{tabular}{|c|c|c|p{6.5cm}|}
		\hline
		\textbf{Paramtername} & \textbf{Datentyp} & \textbf{Konstante} & \textbf{Kurzbeschreibung}                                                                                               \\ \hline
		type                & string            & sop                & Betreiber abfragen \\ \hline
		poi\_id             & int               &                    & Identifikator eines Interessenpunktes \\ \hline
	\end{tabular}
\end{table}
\paragraph{Antwort}Die Antwort ist wie folgt aufgebaut:
\begin{table}[H]
	\begin{tabular}{|c|c|c|p{6.5cm}|}
		\hline
		\textbf{Paramtername} & \textbf{Datentyp} & \textbf{Konstante} & \textbf{Kurzbeschreibung}                                                                                               \\ \hline
		result              & string           &                 & Erfolgreich wenn Wert {\glqq ack\grqq} ist \\ \hline
		Code                & int              &                 & Erfolgreich wenn Wert {\glqq 0\grqq} ist \\ \hline
		data                & array            &                 & Angefragte Informationen \\ \hline
	\end{tabular}
\end{table}
\subparagraph{data}Dieses Array enthält Elemente mit Einträgen in der nachstehend dargestellten Form haben:
\begin{table}[H]
	\begin{tabular}{|c|c|c|p{6.5cm}|}
		\hline
		\textbf{Paramtername} & \textbf{Datentyp} & \textbf{Konstante} & \textbf{Kurzbeschreibung}    \\ \hline
		ID                     & int             &                 & Identifikator des Betreibers \\ \hline
		editable               & bool            &                 & Wahr, wenn Eintrag bearbeitbar \\ \hline
		end                    & int             &                 & Ende der Nutzungszeit Betreiber \\ \hline
		Operator               & string          &                 & Betreiber \\ \hline
		start                  & string          &                 & Anfang der Nutzungszeit durch Betreiber \\ \hline
		validatable            & bool            &                 & Wahr, wenn Eintrag validierbar \\ \hline
		deleted                & bool            &                 & Wahr, wenn Eintrag als gelöscht gilt \\ \hline
	\end{tabular}
\end{table}

\subsubsection{Historisch Adressen eines Interessenpunktes abfragen}
\paragraph{Kurzbeschreibung}Dieser API-Request wird dazu genutzt um alle historischen Adressen eines Interessenpunktes ab zu fragen.
\paragraph{Anfrage}Folgende Daten werden zu Anfrage benötigt:
\begin{table}[H]
	\begin{tabular}{|c|c|c|p{6.5cm}|}
		\hline
		\textbf{Paramtername} & \textbf{Datentyp} & \textbf{Konstante} & \textbf{Kurzbeschreibung}                                                                                               \\ \hline
		type                & string            & shp                & Historisch Adressen abfragen \\ \hline
		poi\_id             & int               &                    & Identifikator eines Interessenpunktes \\ \hline
	\end{tabular}
\end{table}
\paragraph{Antwort}Die Antwort ist wie folgt aufgebaut:
\begin{table}[H]
	\begin{tabular}{|c|c|c|p{6.5cm}|}
		\hline
		\textbf{Paramtername} & \textbf{Datentyp} & \textbf{Konstante} & \textbf{Kurzbeschreibung}                                                                                               \\ \hline
		result              & string           &                 & Erfolgreich wenn Wert {\glqq ack\grqq} ist \\ \hline
		Code                & int              &                 & Erfolgreich wenn Wert {\glqq 0\grqq} ist \\ \hline
		data                & array            &                 & Angefragte Informationen \\ \hline
	\end{tabular}
\end{table}
\subparagraph{data}Dieses Array enthält Elemente mit Einträgen in der nachstehend dargestellten Form haben:
\begin{table}[H]
	\begin{tabular}{|c|c|c|p{6.5cm}|}
		\hline
		\textbf{Paramtername} & \textbf{Datentyp} & \textbf{Konstante} & \textbf{Kurzbeschreibung}    \\ \hline
		ID                     & int             &                 & Identifikator der historischen Adresse \\ \hline
		editable               & bool            &                 & Wahr, wenn Eintrag bearbeitbar \\ \hline
		end                    & int             &                 & Ende der Nutzungszeit der Adresse \\ \hline
		start                  & string          &                 & Anfang der Nutzungszeit der Adresse \\ \hline
		validatable            & bool            &                 & Wahr, wenn Eintrag validierbar \\ \hline
		City                   & string          &                 & Stadt \\ \hline
		Housenumber            & string          &                 & Hausnummer \\ \hline
		Postalcode             & int             &                 & Postleitzahl \\ \hline
		Streetname             & string          &                 & Straßenname \\ \hline
		deleted                & bool            &                 & Wahr, wenn Eintrag als gelöscht gilt \\ \hline
	\end{tabular}
\end{table}


\subsubsection{Gastabfrage}
\paragraph{Kurzbeschreibung}Dieser API-Request wird dazu genutzt um abzufragen, ob Nutzer Gast ist.
\paragraph{Anfrage}Folgende Daten werden zu Anfrage benötigt:
\begin{table}[H]
	\begin{tabular}{|c|c|c|p{6.5cm}|}
		\hline
		\textbf{Paramtername} & \textbf{Datentyp} & \textbf{Konstante} & \textbf{Kurzbeschreibung}                                                                                               \\ \hline
		type                & string            & gue                & Gastabfrage \\ \hline
	\end{tabular}
\end{table}
\paragraph{Antwort}Die Antwort ist wie folgt aufgebaut:
\begin{table}[H]
	\begin{tabular}{|c|c|c|p{6.5cm}|}
		\hline
		\textbf{Paramtername} & \textbf{Datentyp} & \textbf{Konstante} & \textbf{Kurzbeschreibung}                                                                                               \\ \hline
		result              & string           &                 & Erfolgreich wenn Wert {\glqq ack\grqq} ist \\ \hline
		Code                & int              &                 & Erfolgreich wenn Wert {\glqq 0\grqq} ist \\ \hline
		data                & bool             &                 & Wahr, wenn Nutzer Gast ist \\ \hline
	\end{tabular}
\end{table}

\subsubsection{Saalanzahlen eines Interessenpunktes abfragen}
\paragraph{Kurzbeschreibung}Dieser API-Request wird dazu genutzt um alle Saalanzahlen eines Interessenpunktes ab zu fragen.
\paragraph{Anfrage}Folgende Daten werden zu Anfrage benötigt:
\begin{table}[H]
	\begin{tabular}{|c|c|c|p{6.5cm}|}
		\hline
		\textbf{Paramtername} & \textbf{Datentyp} & \textbf{Konstante} & \textbf{Kurzbeschreibung}                                                                                               \\ \hline
		type                & string            & scp                & Saalanzahlen abfragen \\ \hline
		poi\_id             & int               &                    & Identifikator eines Interessenpunktes \\ \hline
	\end{tabular}
\end{table}
\paragraph{Antwort}Die Antwort ist wie folgt aufgebaut:
\begin{table}[H]
	\begin{tabular}{|c|c|c|p{6.5cm}|}
		\hline
		\textbf{Paramtername} & \textbf{Datentyp} & \textbf{Konstante} & \textbf{Kurzbeschreibung}                                                                                               \\ \hline
		result              & string           &                 & Erfolgreich wenn Wert {\glqq ack\grqq} ist \\ \hline
		Code                & int              &                 & Erfolgreich wenn Wert {\glqq 0\grqq} ist \\ \hline
		data                & array            &                 & Angefragte Informationen \\ \hline
	\end{tabular}
\end{table}
\subparagraph{data}Dieses Array enthält Elemente mit Einträgen in der nachstehend dargestellten Form haben:
\begin{table}[H]
	\begin{tabular}{|c|c|c|p{6.5cm}|}
		\hline
		\textbf{Paramtername} & \textbf{Datentyp} & \textbf{Konstante} & \textbf{Kurzbeschreibung}    \\ \hline
		ID                     & int             &                 & Identifikator der Saalanzahl \\ \hline
		editable               & bool            &                 & Wahr, wenn Eintrag bearbeitbar \\ \hline
		end                    & int             &                 & Ende \\ \hline
		cinemas                & string          &                 & Saalanzahl \\ \hline
		start                  & string          &                 & Anfang \\ \hline
		validatable            & bool            &                 & Wahr, wenn Eintrag validierbar \\ \hline
		deleted                & bool            &                 & Wahr, wenn Eintrag als gelöscht gilt \\ \hline
	\end{tabular}
\end{table}

\subsubsection{Sitzplatzanzahl eines Interessenpunktes abfragen}
\paragraph{Kurzbeschreibung}Dieser API-Request wird dazu genutzt um alle Sitzplatzanzahlen eines Interessenpunktes ab zu fragen.
\paragraph{Anfrage}Folgende Daten werden zu Anfrage benötigt:
\begin{table}[H]
	\begin{tabular}{|c|c|c|p{6.5cm}|}
		\hline
		\textbf{Paramtername} & \textbf{Datentyp} & \textbf{Konstante} & \textbf{Kurzbeschreibung}                                                                                               \\ \hline
		type                & string            & ssp                & Sitzplatzanzahlen abfragen \\ \hline
		poi\_id             & int               &                    & Identifikator eines Interessenpunktes \\ \hline
	\end{tabular}
\end{table}
\paragraph{Antwort}Die Antwort ist wie folgt aufgebaut:
\begin{table}[H]
	\begin{tabular}{|c|c|c|p{6.5cm}|}
		\hline
		\textbf{Paramtername} & \textbf{Datentyp} & \textbf{Konstante} & \textbf{Kurzbeschreibung}                                                                                               \\ \hline
		result              & string           &                 & Erfolgreich wenn Wert {\glqq ack\grqq} ist \\ \hline
		Code                & int              &                 & Erfolgreich wenn Wert {\glqq 0\grqq} ist \\ \hline
		data                & array            &                 & Angefragte Informationen \\ \hline
	\end{tabular}
\end{table}
\subparagraph{data}Dieses Array enthält Elemente mit Einträgen in der nachstehend dargestellten Form haben:
\begin{table}[H]
	\begin{tabular}{|c|c|c|p{6.5cm}|}
		\hline
		\textbf{Paramtername} & \textbf{Datentyp} & \textbf{Konstante} & \textbf{Kurzbeschreibung}    \\ \hline
		ID                     & int             &                 & Identifikator der Sitzplatzanzahl \\ \hline
		editable               & bool            &                 & Wahr, wenn Eintrag bearbeitbar \\ \hline
		end                    & int             &                 & Ende \\ \hline
		seats                  & string          &                 & Sitzplatzanzahl \\ \hline
		start                  & string          &                 & Anfang \\ \hline
		validatable            & bool            &                 & Wahr, wenn Eintrag validierbar \\ \hline
		deleted                & bool            &                 & Wahr, wenn Eintrag als gelöscht gilt \\ \hline
	\end{tabular}
\end{table}

\subsubsection{Geschichtsverknüpfungen eines Interessenpunktes abfragen}
\paragraph{Kurzbeschreibung}Dieser API-Request wird dazu genutzt um alle Geschichtsverknüpfungen eines Interessenpunktes ab zu fragen.
\paragraph{Anfrage}Folgende Daten werden zu Anfrage benötigt:
\begin{table}[H]
	\begin{tabular}{|c|c|c|p{6.5cm}|}
		\hline
		\textbf{Paramtername} & \textbf{Datentyp} & \textbf{Konstante} & \textbf{Kurzbeschreibung}                                                                                               \\ \hline
		type                & string            & slp                & Geschichtsverknüpfungen abfragen \\ \hline
		poi\_id             & int               &                    & Identifikator eines Interessenpunktes \\ \hline
	\end{tabular}
\end{table}
\paragraph{Antwort}Die Antwort ist wie folgt aufgebaut:
\begin{table}[H]
	\begin{tabular}{|c|c|c|p{6.5cm}|}
		\hline
		\textbf{Paramtername} & \textbf{Datentyp} & \textbf{Konstante} & \textbf{Kurzbeschreibung}                                                                                               \\ \hline
		result              & string           &                 & Erfolgreich wenn Wert {\glqq ack\grqq} ist \\ \hline
		Code                & int              &                 & Erfolgreich wenn Wert {\glqq 0\grqq} ist \\ \hline
		data                & array            &                 & Angefragte Informationen \\ \hline
	\end{tabular}
\end{table}
\subparagraph{data}Dieses Array enthält Elemente mit Einträgen in der nachstehend dargestellten Form haben:
\begin{table}[H]
	\begin{tabular}{|c|c|c|p{6.5cm}|}
		\hline
		\textbf{Paramtername} & \textbf{Datentyp} & \textbf{Konstante} & \textbf{Kurzbeschreibung}    \\ \hline
		id                     & int             &                 & Identifikator des Links \\ \hline
		LinkDeletable          & bool            &                 & Wahr, wenn Link löschbar \\ \hline
		LinkValidated          & bool            &                 & Wahr, wenn Link validiert \\ \hline
		name                   & string          &                 & Name des Erstellers \\ \hline
		date                   & timestamp       &                 & Erstellungsdatum \\ \hline
		title                  & string          &                 & Titel der Geschichte \\ \hline
		story                  & string          &                 & Inhalt der Geschichte \\ \hline
		token                  & string          &                 & Identifikator der Geschichte \\ \hline
		validate               & bool            &                 & Validierungsstatus der Geschichte \\ \hline
		deleted                & bool            &                 & Gibt an, ob Link als gelöscht gilt \\ \hline
		restrictions           & bool            &                 & Wahr, wenn Abhängigkeiten als gelöscht gelten \\ \hline
	\end{tabular}
\end{table}

\subsubsection{Liste der unverknüpften Geschichten eines Interessenpunktes }
\paragraph{Kurzbeschreibung}Dieser API-Request wird dazu genutzt um eine Liste der unverknüpften Geschichten eines Interessenpunktes aburagen.
\paragraph{Anfrage}Folgende Daten werden zu Anfrage benötigt:
\begin{table}[H]
	\begin{tabular}{|c|c|c|p{6.5cm}|}
		\hline
		\textbf{Paramtername} & \textbf{Datentyp} & \textbf{Konstante} & \textbf{Kurzbeschreibung}                                                                                               \\ \hline
		type                & string            & gsp                & Liste abfragen \\ \hline
		poi\_id             & int               &                    & Identifikator eines Interessenpunktes \\ \hline
	\end{tabular}
\end{table}
\paragraph{Antwort}Die Antwort ist wie folgt aufgebaut:
\begin{table}[H]
	\begin{tabular}{|c|c|c|p{6.5cm}|}
		\hline
		\textbf{Paramtername} & \textbf{Datentyp} & \textbf{Konstante} & \textbf{Kurzbeschreibung}                                                                                               \\ \hline
		result              & string           &                 & Erfolgreich wenn Wert {\glqq ack\grqq} ist \\ \hline
		Code                & int              &                 & Erfolgreich wenn Wert {\glqq 0\grqq} ist \\ \hline
		data                & array            &                 & Angefragte Informationen \\ \hline
	\end{tabular}
\end{table}
\subparagraph{data}Dieses Array enthält Elemente mit Einträgen in der nachstehend dargestellten Form haben:
\begin{table}[H]
	\begin{tabular}{|c|c|c|p{6.5cm}|}
		\hline
		\textbf{Paramtername} & \textbf{Datentyp} & \textbf{Konstante} & \textbf{Kurzbeschreibung}    \\ \hline
		title                  & string          &                 & Titel der Geschichte \\ \hline
		token                  & string          &                 & Identifikator der Geschichte \\ \hline
	\end{tabular}
\end{table}

\subsubsection{Hauptbild eines Interessenpunktes }
\paragraph{Kurzbeschreibung}Dieser API-Request wird dazu genutzt um das Hauptbild eines Interessenpunktes abzufragen.
\paragraph{Anfrage}Folgende Daten werden zu Anfrage benötigt:
\begin{table}[H]
	\begin{tabular}{|c|c|c|p{6.5cm}|}
		\hline
		\textbf{Paramtername} & \textbf{Datentyp} & \textbf{Konstante} & \textbf{Kurzbeschreibung}                                                                                               \\ \hline
		type                & string            & plp                & Hauptbild abfragen \\ \hline
		poi\_id             & int               &                    & Identifikator eines Interessenpunktes \\ \hline
	\end{tabular}
\end{table}
\paragraph{Antwort}Die Antwort ist wie folgt aufgebaut:
\begin{table}[H]
	\begin{tabular}{|c|c|c|p{6.5cm}|}
		\hline
		\textbf{Paramtername} & \textbf{Datentyp} & \textbf{Konstante} & \textbf{Kurzbeschreibung}                                                                                               \\ \hline
		result              & string           &                 & Erfolgreich wenn Wert {\glqq ack\grqq} ist \\ \hline
		Code                & int              &                 & Erfolgreich wenn Wert {\glqq 0\grqq} ist \\ \hline
		data                & string           &                 & URI des Hauptbildes \\ \hline
		deleted             & bool             &                 & Gibt an, ob das Bild als gelöscht gilt \\ \hline
		source              & string           &                 & Quellenangabe \\ \hline
		sourceType          & string           &                 & Typ der Quelle \\ \hline
	\end{tabular}
\end{table}

\subsubsection{Zusätzliche Bilder eines Interessenpunktes}
\paragraph{Kurzbeschreibung}Dieser API-Request wird dazu genutzt um zusätzliche Bilder eines Interessenpunktes zu laden.
\paragraph{Anfrage}Folgende Daten werden zu Anfrage benötigt:
\begin{table}[H]
	\begin{tabular}{|c|c|c|p{6.5cm}|}
		\hline
		\textbf{Paramtername} & \textbf{Datentyp} & \textbf{Konstante} & \textbf{Kurzbeschreibung}                                                                                               \\ \hline
		type                & string            & apl                & Liste abfragen \\ \hline
		poi\_id             & int               &                    & Identifikator eines Interessenpunktes \\ \hline
	\end{tabular}
\end{table}
\paragraph{Antwort}Die Antwort ist wie folgt aufgebaut:
\begin{table}[H]
	\begin{tabular}{|c|c|c|p{6.5cm}|}
		\hline
		\textbf{Paramtername} & \textbf{Datentyp} & \textbf{Konstante} & \textbf{Kurzbeschreibung}                                                                                               \\ \hline
		result              & string           &                 & Erfolgreich wenn Wert {\glqq ack\grqq} ist \\ \hline
		Code                & int              &                 & Erfolgreich wenn Wert {\glqq 0\grqq} ist \\ \hline
		data                & array            &                 & Angefragte Informationen \\ \hline
	\end{tabular}
\end{table}
\subparagraph{data}Dieses Array enthält Elemente mit Einträgen in der nachstehend dargestellten Form haben:
\begin{table}[H]
	\begin{tabular}{|c|c|c|p{6.5cm}|}
		\hline
		\textbf{Paramtername} & \textbf{Datentyp} & \textbf{Konstante} & \textbf{Kurzbeschreibung}    \\ \hline
		deletable              & bool            &                 & Wahr, wenn Link löschbar \\ \hline
		deleted                & bool            &                 & Wahr, wenn Link als gelöscht gilt \\ \hline
		description            & string          &                 & Bildbeschreibung \\ \hline
		fullsize               & string          &                 & URI des Vollbildes \\ \hline
		id                     & int             &                 & Identifikator des Bildes \\ \hline
		identifier             & string          &                 & zweiter Identifikator des Bildes \\ \hline
		ppid                   & int             &                 & Identifikator des Links \\ \hline
		preview                & string          &                 & URI des Vorschaubildes \\ \hline
		title                  & string          &                 & Titel des Bildes \\ \hline
		token                  & array           &                 & Informationen zum Laden des Bildes \\ \hline
		username               & string          &                 & Nutzername des Erstellers des Bildes \\ \hline
		valUsers               & array           &                 & Array mit Nutzernamen der Validatoren \\ \hline
		validated              & bool            &                 & Validierungsstatus des Links \\ \hline
		validationValue        & bool            &                 & Validierungsstatus des Bildes \\ \hline
		restrictions           & bool            &                 & Wahr, wenn Abhängigkeiten als gelöscht gelten \\ \hline
	\end{tabular}
\end{table}

\subsubsection{Kommentare eines Interessenpunktes}
\paragraph{Kurzbeschreibung}Dieser API-Request wird dazu genutzt um Kommentare eines Interessenpunktes ab zu fragen.
\paragraph{Anfrage}Folgende Daten werden zu Anfrage benötigt:
\begin{table}[H]
	\begin{tabular}{|c|c|c|p{6.5cm}|}
		\hline
		\textbf{Paramtername} & \textbf{Datentyp} & \textbf{Konstante} & \textbf{Kurzbeschreibung}                                                                                               \\ \hline
		type                & string            & lcp                & Kommentare abfragen \\ \hline
		poi\_id             & int               &                    & Identifikator eines Interessenpunktes \\ \hline
	\end{tabular}
\end{table}
\paragraph{Antwort}Die Antwort ist wie folgt aufgebaut:
\begin{table}[H]
	\begin{tabular}{|c|c|c|p{6.5cm}|}
		\hline
		\textbf{Paramtername} & \textbf{Datentyp} & \textbf{Konstante} & \textbf{Kurzbeschreibung}                                                                                               \\ \hline
		result              & string           &                 & Erfolgreich wenn Wert {\glqq ack\grqq} ist \\ \hline
		Code                & int              &                 & Erfolgreich wenn Wert {\glqq 0\grqq} ist \\ \hline
		data                & array            &                 & Angefragte Informationen \\ \hline
	\end{tabular}
\end{table}
\subparagraph{data}Dieses Array Einträge in der nachstehend dargestellten Form haben:
\begin{table}[H]
	\begin{tabular}{|c|c|c|p{6.5cm}|}
		\hline
		\textbf{Paramtername} & \textbf{Datentyp} & \textbf{Konstante} & \textbf{Kurzbeschreibung}    \\ \hline
		deleteComments         & bool            &                 & Wahr, wenn Nutzer Kommentare löschen darf \\ \hline
		poi\_name              & string          &                 & Name des Interessenpunktes \\ \hline
		comments               & array           &                 & Kommentare \\ \hline
	\end{tabular}
\end{table}
\subparagraph{comments}Dieses Array enthält Elemente mit Einträgen in der nachstehend dargestellten Form haben:
\begin{table}[H]
	\begin{tabular}{|c|c|c|p{6.5cm}|}
		\hline
		\textbf{Paramtername} & \textbf{Datentyp} & \textbf{Konstante} & \textbf{Kurzbeschreibung}    \\ \hline
		comment\_id            & int             &                 & Identifikator des Kommentars \\ \hline
		content                & string          &                 & Inhalt des Kommentars \\ \hline
		name                   & string          &                 & Ersteller des Kommentars \\ \hline
		timestamp              & timestamp       &                 & Erstellungszeitpunkt des Kommentars \\ \hline
		deleted                & bool            &                 & Gibt an, ob Kommentar als gelöscht gilt \\ \hline
		deletable              & bool            &                 & Gibt an, ob Kommentar für Nutzer löschbar ist \\ \hline
	\end{tabular}
\end{table}

\subsubsection{Aktivierungsstatus der Biographienfunktion}
\paragraph{Kurzbeschreibung}Dieser API-Request wird dazu genutzt den Status der Biographienfunktion abzufragen.
\paragraph{Anfrage}Folgende Daten werden zu Anfrage benötigt:
\begin{table}[H]
	\begin{tabular}{|c|c|c|p{6.5cm}|}
		\hline
		\textbf{Paramtername} & \textbf{Datentyp} & \textbf{Konstante} & \textbf{Kurzbeschreibung}                                                                                               \\ \hline
		type                & string            & cse                & Kommentare abfragen \\ \hline
	\end{tabular}
\end{table}
\paragraph{Antwort}Die Antwort ist wie folgt aufgebaut:
\begin{table}[H]
	\begin{tabular}{|c|c|c|p{6.5cm}|}
		\hline
		\textbf{Paramtername} & \textbf{Datentyp} & \textbf{Konstante} & \textbf{Kurzbeschreibung}                                                                                               \\ \hline
		result              & string           &                 & Erfolgreich wenn Wert {\glqq ack\grqq} ist \\ \hline
		Code                & int              &                 & Erfolgreich wenn Wert {\glqq 0\grqq} ist \\ \hline
		data                & bool             &                 & Wahr, wenn Biographienfunktion aktiv \\ \hline
	\end{tabular}
\end{table}

\subsubsection{Captcha-Code anfordern}
\paragraph{Kurzbeschreibung}Dieser API-Request wird dazu genutzt einen Captcha-Code anzufordern.
\paragraph{Anfrage}Folgende Daten werden zu Anfrage benötigt:
\begin{table}[H]
	\begin{tabular}{|c|c|c|p{6.5cm}|}
		\hline
		\textbf{Paramtername} & \textbf{Datentyp} & \textbf{Konstante} & \textbf{Kurzbeschreibung}                                                                                               \\ \hline
		type                & string            & cpa                & Captcha-Code anfordern \\ \hline
	\end{tabular}
\end{table}
\paragraph{Antwort}Die Antwort ist wie folgt aufgebaut:
\begin{table}[H]
	\begin{tabular}{|c|c|c|p{6.5cm}|}
		\hline
		\textbf{Paramtername} & \textbf{Datentyp} & \textbf{Konstante} & \textbf{Kurzbeschreibung}                                                                                               \\ \hline
		result              & string           &                 & Erfolgreich wenn Wert {\glqq ack\grqq} ist \\ \hline
		Code                & int              &                 & Erfolgreich wenn Wert {\glqq 0\grqq} ist \\ \hline
		data                & string           &                 & Base64-Codiertes Captcha Bild im JPEG-Format \\ \hline
	\end{tabular}
\end{table}

\subsubsection{Kontaktnachricht senden}
\paragraph{Kurzbeschreibung}Dieser API-Request wird dazu genutzt einen Kontaktnachricht zu versenden.
\paragraph{Anfrage}Folgende Daten werden zu Anfrage benötigt:
\begin{table}[H]
	\begin{tabular}{|c|c|c|p{6.5cm}|}
		\hline
		\textbf{Paramtername} & \textbf{Datentyp} & \textbf{Konstante} & \textbf{Kurzbeschreibung}                                                                                               \\ \hline
		type                & string            & cmg                & Captcha-Code anfordern \\ \hline
		cap                 & string            &                    & Eingabe des gelesenen Captcha-Codes \\ \hline
		email               & string            &                    & Nutzermailadresse (Wahlweise, wenn Nutzer Gast ist) \\ \hline
		msg                 & string            &                    & Mailinhalt \\ \hline
		title               & string            &                    & Betreff der Mail \\ \hline
	\end{tabular}
\end{table}
\paragraph{Antwort}Die Antwort ist wie folgt aufgebaut:
\begin{table}[H]
	\begin{tabular}{|c|c|c|p{6.5cm}|}
		\hline
		\textbf{Paramtername} & \textbf{Datentyp} & \textbf{Konstante} & \textbf{Kurzbeschreibung}                                                                                               \\ \hline
		result              & string           &                 & Erfolgreich wenn Wert {\glqq ack\grqq} ist \\ \hline
		Code                & int              &                 & Erfolgreich wenn Wert {\glqq 0\grqq} ist \\ \hline
	\end{tabular}
\end{table}
\subsubsection{Verknüpfung löschen zwischen Interessenpunkt und Bild}
\paragraph{Kurzbeschreibung}Dieser API-Request wird dazu genutzt um eine Verknüpfung zwischen einem Bild und einem Interessenpunkt zu löschen.
\paragraph{Anfrage}Folgende Daten werden zu Anfrage benötigt:
\begin{table}[H]
	\begin{tabular}{|c|c|c|p{6.5cm}|}
		\hline
		\textbf{Paramtername} & \textbf{Datentyp} & \textbf{Konstante} & \textbf{Kurzbeschreibung}                                                                                               \\ \hline
		type                & string            & fdp                & Link löschen \\ \hline
		IDent               & int               &                    & Identifikator des Links \\ \hline
	\end{tabular}
\end{table}
\paragraph{Antwort}Die Antwort ist wie folgt aufgebaut:
\begin{table}[H]
	\begin{tabular}{|c|c|c|p{6.5cm}|}
		\hline
		\textbf{Paramtername} & \textbf{Datentyp} & \textbf{Konstante} & \textbf{Kurzbeschreibung}                                                                                               \\ \hline
		result              & string           &                 & Erfolgreich wenn Wert {\glqq ack\grqq} ist \\ \hline
		Code                & int              &                 & Erfolgreich wenn Wert {\glqq 0\grqq} ist \\ \hline
	\end{tabular}
\end{table}
\subsubsection{Verknüpfung zwischen Interessenpunkt und Bild wiederherstellen}
\paragraph{Kurzbeschreibung}Dieser API-Request wird dazu genutzt um eine Verknüpfung zwischen einem Bild und einem Interessenpunkt wiederherzustellen.
\paragraph{Anfrage}Folgende Daten werden zu Anfrage benötigt:
\begin{table}[H]
	\begin{tabular}{|c|c|c|p{6.5cm}|}
		\hline
		\textbf{Paramtername} & \textbf{Datentyp} & \textbf{Konstante} & \textbf{Kurzbeschreibung}                                                                                               \\ \hline
		type                & string            & rdp                & Link wiederherstellen \\ \hline
		IDent               & int               &                    & Identifikator des Links \\ \hline
	\end{tabular}
\end{table}
\paragraph{Antwort}Die Antwort ist wie folgt aufgebaut:
\begin{table}[H]
	\begin{tabular}{|c|c|c|p{6.5cm}|}
		\hline
		\textbf{Paramtername} & \textbf{Datentyp} & \textbf{Konstante} & \textbf{Kurzbeschreibung}                                                                                               \\ \hline
		result              & string           &                 & Erfolgreich wenn Wert {\glqq ack\grqq} ist \\ \hline
		Code                & int              &                 & Erfolgreich wenn Wert {\glqq 0\grqq} ist \\ \hline
	\end{tabular}
\end{table}
\subsubsection{Namen eines Interessenpunktes löschen}
\paragraph{Kurzbeschreibung}Dieser API-Request wird dazu genutzt um einen Namen eines Interessenpunktes final zu löschen.
\paragraph{Anfrage}Folgende Daten werden zu Anfrage benötigt:
\begin{table}[H]
	\begin{tabular}{|c|c|c|p{6.5cm}|}
		\hline
		\textbf{Paramtername} & \textbf{Datentyp} & \textbf{Konstante} & \textbf{Kurzbeschreibung}                                                                                               \\ \hline
		type                & string            & fna                & Name löschen \\ \hline
		IDent               & int               &                    & Identifikator des Namen \\ \hline
	\end{tabular}
\end{table}
\paragraph{Antwort}Die Antwort ist wie folgt aufgebaut:
\begin{table}[H]
	\begin{tabular}{|c|c|c|p{6.5cm}|}
		\hline
		\textbf{Paramtername} & \textbf{Datentyp} & \textbf{Konstante} & \textbf{Kurzbeschreibung}                                                                                               \\ \hline
		result              & string           &                 & Erfolgreich wenn Wert {\glqq ack\grqq} ist \\ \hline
		Code                & int              &                 & Erfolgreich wenn Wert {\glqq 0\grqq} ist \\ \hline
	\end{tabular}
\end{table}
\subsubsection{Namen eines Interessenpunktes wiederherstellen}
\paragraph{Kurzbeschreibung}Dieser API-Request wird dazu genutzt um einen Namen eines Interessenpunktes wiederherzustellen.
\paragraph{Anfrage}Folgende Daten werden zu Anfrage benötigt:
\begin{table}[H]
	\begin{tabular}{|c|c|c|p{6.5cm}|}
		\hline
		\textbf{Paramtername} & \textbf{Datentyp} & \textbf{Konstante} & \textbf{Kurzbeschreibung}                                                                                               \\ \hline
		type                & string            & rna                & Name wiederherstellen \\ \hline
		IDent               & int               &                    & Identifikator des Namen \\ \hline
	\end{tabular}
\end{table}
\paragraph{Antwort}Die Antwort ist wie folgt aufgebaut:
\begin{table}[H]
	\begin{tabular}{|c|c|c|p{6.5cm}|}
		\hline
		\textbf{Paramtername} & \textbf{Datentyp} & \textbf{Konstante} & \textbf{Kurzbeschreibung}                                                                                               \\ \hline
		result              & string           &                 & Erfolgreich wenn Wert {\glqq ack\grqq} ist \\ \hline
		Code                & int              &                 & Erfolgreich wenn Wert {\glqq 0\grqq} ist \\ \hline
	\end{tabular}
\end{table}
\subsubsection{Betreiber eines Interessenpunktes löschen}
\paragraph{Kurzbeschreibung}Dieser API-Request wird dazu genutzt um einen Betreiber eines Interessenpunktes final zu löschen.
\paragraph{Anfrage}Folgende Daten werden zu Anfrage benötigt:
\begin{table}[H]
	\begin{tabular}{|c|c|c|p{6.5cm}|}
		\hline
		\textbf{Paramtername} & \textbf{Datentyp} & \textbf{Konstante} & \textbf{Kurzbeschreibung}                                                                                               \\ \hline
		type                & string            & fop                & Betreiber löschen \\ \hline
		IDent               & int               &                    & Identifikator des Betreibers \\ \hline
	\end{tabular}
\end{table}
\paragraph{Antwort}Die Antwort ist wie folgt aufgebaut:
\begin{table}[H]
	\begin{tabular}{|c|c|c|p{6.5cm}|}
		\hline
		\textbf{Paramtername} & \textbf{Datentyp} & \textbf{Konstante} & \textbf{Kurzbeschreibung}                                                                                               \\ \hline
		result              & string           &                 & Erfolgreich wenn Wert {\glqq ack\grqq} ist \\ \hline
		Code                & int              &                 & Erfolgreich wenn Wert {\glqq 0\grqq} ist \\ \hline
	\end{tabular}
\end{table}
\subsubsection{Betreiber eines Interessenpunktes wiederherstellen}
\paragraph{Kurzbeschreibung}Dieser API-Request wird dazu genutzt um einen Betreiber eines Interessenpunktes wiederherzustellen.
\paragraph{Anfrage}Folgende Daten werden zu Anfrage benötigt:
\begin{table}[H]
	\begin{tabular}{|c|c|c|p{6.5cm}|}
		\hline
		\textbf{Paramtername} & \textbf{Datentyp} & \textbf{Konstante} & \textbf{Kurzbeschreibung}                                                                                               \\ \hline
		type                & string            & rop                & Betreiber wiederherstellen \\ \hline
		IDent               & int               &                    & Identifikator des Betreibers \\ \hline
	\end{tabular}
\end{table}
\paragraph{Antwort}Die Antwort ist wie folgt aufgebaut:
\begin{table}[H]
	\begin{tabular}{|c|c|c|p{6.5cm}|}
		\hline
		\textbf{Paramtername} & \textbf{Datentyp} & \textbf{Konstante} & \textbf{Kurzbeschreibung}                                                                                               \\ \hline
		result              & string           &                 & Erfolgreich wenn Wert {\glqq ack\grqq} ist \\ \hline
		Code                & int              &                 & Erfolgreich wenn Wert {\glqq 0\grqq} ist \\ \hline
	\end{tabular}
\end{table}
\subsubsection{Sitzplatzanzahl eines Interessenpunktes löschen}
\paragraph{Kurzbeschreibung}Dieser API-Request wird dazu genutzt um eine Sitzplatzanzahl eines Interessenpunktes final zu löschen.
\paragraph{Anfrage}Folgende Daten werden zu Anfrage benötigt:
\begin{table}[H]
	\begin{tabular}{|c|c|c|p{6.5cm}|}
		\hline
		\textbf{Paramtername} & \textbf{Datentyp} & \textbf{Konstante} & \textbf{Kurzbeschreibung}                                                                                               \\ \hline
		type                & string            & fsc                & Sitzplatzanzahl löschen \\ \hline
		IDent               & int               &                    & Identifikator der Sitzplatzanzahl \\ \hline
	\end{tabular}
\end{table}
\paragraph{Antwort}Die Antwort ist wie folgt aufgebaut:
\begin{table}[H]
	\begin{tabular}{|c|c|c|p{6.5cm}|}
		\hline
		\textbf{Paramtername} & \textbf{Datentyp} & \textbf{Konstante} & \textbf{Kurzbeschreibung}                                                                                               \\ \hline
		result              & string           &                 & Erfolgreich wenn Wert {\glqq ack\grqq} ist \\ \hline
		Code                & int              &                 & Erfolgreich wenn Wert {\glqq 0\grqq} ist \\ \hline
	\end{tabular}
\end{table}
\subsubsection{Sitzplatzanzahl eines Interessenpunktes wiederherstellen}
\paragraph{Kurzbeschreibung}Dieser API-Request wird dazu genutzt um eine Sitzplatzanzahl eines Interessenpunktes wiederherzustellen.
\paragraph{Anfrage}Folgende Daten werden zu Anfrage benötigt:
\begin{table}[H]
	\begin{tabular}{|c|c|c|p{6.5cm}|}
		\hline
		\textbf{Paramtername} & \textbf{Datentyp} & \textbf{Konstante} & \textbf{Kurzbeschreibung}                                                                                               \\ \hline
		type                & string            & rsc                & Sitzplatzanzahl wiederherstellen \\ \hline
		IDent               & int               &                    & Identifikator der Sitzplatzanzahl \\ \hline
	\end{tabular}
\end{table}
\paragraph{Antwort}Die Antwort ist wie folgt aufgebaut:
\begin{table}[H]
	\begin{tabular}{|c|c|c|p{6.5cm}|}
		\hline
		\textbf{Paramtername} & \textbf{Datentyp} & \textbf{Konstante} & \textbf{Kurzbeschreibung}                                                                                               \\ \hline
		result              & string           &                 & Erfolgreich wenn Wert {\glqq ack\grqq} ist \\ \hline
		Code                & int              &                 & Erfolgreich wenn Wert {\glqq 0\grqq} ist \\ \hline
	\end{tabular}
\end{table}
\subsubsection{Saalanzahl eines Interessenpunktes löschen}
\paragraph{Kurzbeschreibung}Dieser API-Request wird dazu genutzt um eine Saalanzahl eines Interessenpunktes final zu löschen.
\paragraph{Anfrage}Folgende Daten werden zu Anfrage benötigt:
\begin{table}[H]
	\begin{tabular}{|c|c|c|p{6.5cm}|}
		\hline
		\textbf{Paramtername} & \textbf{Datentyp} & \textbf{Konstante} & \textbf{Kurzbeschreibung}                                                                                               \\ \hline
		type                & string            & fcc                & Saalanzahl löschen \\ \hline
		IDent               & int               &                    & Identifikator der Saalanzahl \\ \hline
	\end{tabular}
\end{table}
\paragraph{Antwort}Die Antwort ist wie folgt aufgebaut:
\begin{table}[H]
	\begin{tabular}{|c|c|c|p{6.5cm}|}
		\hline
		\textbf{Paramtername} & \textbf{Datentyp} & \textbf{Konstante} & \textbf{Kurzbeschreibung}                                                                                               \\ \hline
		result              & string           &                 & Erfolgreich wenn Wert {\glqq ack\grqq} ist \\ \hline
		Code                & int              &                 & Erfolgreich wenn Wert {\glqq 0\grqq} ist \\ \hline
	\end{tabular}
\end{table}
\subsubsection{Saalanzahl eines Interessenpunktes wiederherstellen}
\paragraph{Kurzbeschreibung}Dieser API-Request wird dazu genutzt um eine Saalanzahl eines Interessenpunktes wiederherzustellen.
\paragraph{Anfrage}Folgende Daten werden zu Anfrage benötigt:
\begin{table}[H]
	\begin{tabular}{|c|c|c|p{6.5cm}|}
		\hline
		\textbf{Paramtername} & \textbf{Datentyp} & \textbf{Konstante} & \textbf{Kurzbeschreibung}                                                                                               \\ \hline
		type                & string            & rcc                & Saalanzahl wiederherstellen \\ \hline
		IDent               & int               &                    & Identifikator der Saalanzahl \\ \hline
	\end{tabular}
\end{table}
\paragraph{Antwort}Die Antwort ist wie folgt aufgebaut:
\begin{table}[H]
	\begin{tabular}{|c|c|c|p{6.5cm}|}
		\hline
		\textbf{Paramtername} & \textbf{Datentyp} & \textbf{Konstante} & \textbf{Kurzbeschreibung}                                                                                               \\ \hline
		result              & string           &                 & Erfolgreich wenn Wert {\glqq ack\grqq} ist \\ \hline
		Code                & int              &                 & Erfolgreich wenn Wert {\glqq 0\grqq} ist \\ \hline
	\end{tabular}
\end{table}
\subsubsection{Historischen Adresse eines Interessenpunktes löschen}
\paragraph{Kurzbeschreibung}Dieser API-Request wird dazu genutzt um eine Historischen Adresse eines Interessenpunktes final zu löschen.
\paragraph{Anfrage}Folgende Daten werden zu Anfrage benötigt:
\begin{table}[H]
	\begin{tabular}{|c|c|c|p{6.5cm}|}
		\hline
		\textbf{Paramtername} & \textbf{Datentyp} & \textbf{Konstante} & \textbf{Kurzbeschreibung}                                                                                               \\ \hline
		type                & string            & fha                & Historischen Adresse löschen \\ \hline
		IDent               & int               &                    & Identifikator der historischen Adresse \\ \hline
	\end{tabular}
\end{table}
\paragraph{Antwort}Die Antwort ist wie folgt aufgebaut:
\begin{table}[H]
	\begin{tabular}{|c|c|c|p{6.5cm}|}
		\hline
		\textbf{Paramtername} & \textbf{Datentyp} & \textbf{Konstante} & \textbf{Kurzbeschreibung}                                                                                               \\ \hline
		result              & string           &                 & Erfolgreich wenn Wert {\glqq ack\grqq} ist \\ \hline
		Code                & int              &                 & Erfolgreich wenn Wert {\glqq 0\grqq} ist \\ \hline
	\end{tabular}
\end{table}
\subsubsection{Historischen Adresse eines Interessenpunktes wiederherstellen}
\paragraph{Kurzbeschreibung}Dieser API-Request wird dazu genutzt um eine Historischen Adresse eines Interessenpunktes wiederherzustellen.
\paragraph{Anfrage}Folgende Daten werden zu Anfrage benötigt:
\begin{table}[H]
	\begin{tabular}{|c|c|c|p{6.5cm}|}
		\hline
		\textbf{Paramtername} & \textbf{Datentyp} & \textbf{Konstante} & \textbf{Kurzbeschreibung}                                                                                               \\ \hline
		type                & string            & rha                & Historischen Adresse wiederherstellen \\ \hline
		IDent               & int               &                    & Identifikator der historischen Adresse \\ \hline
	\end{tabular}
\end{table}
\paragraph{Antwort}Die Antwort ist wie folgt aufgebaut:
\begin{table}[H]
	\begin{tabular}{|c|c|c|p{6.5cm}|}
		\hline
		\textbf{Paramtername} & \textbf{Datentyp} & \textbf{Konstante} & \textbf{Kurzbeschreibung}                                                                                               \\ \hline
		result              & string           &                 & Erfolgreich wenn Wert {\glqq ack\grqq} ist \\ \hline
		Code                & int              &                 & Erfolgreich wenn Wert {\glqq 0\grqq} ist \\ \hline
	\end{tabular}
\end{table}
\subsubsection{Verknüpfung löschen zwischen Interessenpunkt und Geschichte}
\paragraph{Kurzbeschreibung}Dieser API-Request wird dazu genutzt um eine Verknüpfung zwischen einer Geschichte und einem Interessenpunkt zu löschen.
\paragraph{Anfrage}Folgende Daten werden zu Anfrage benötigt:
\begin{table}[H]
	\begin{tabular}{|c|c|c|p{6.5cm}|}
		\hline
		\textbf{Paramtername} & \textbf{Datentyp} & \textbf{Konstante} & \textbf{Kurzbeschreibung}                                                                                               \\ \hline
		type                & string            & fsp                & Link löschen \\ \hline
		IDent               & int               &                    & Identifikator des Links \\ \hline
	\end{tabular}
\end{table}
\paragraph{Antwort}Die Antwort ist wie folgt aufgebaut:
\begin{table}[H]
	\begin{tabular}{|c|c|c|p{6.5cm}|}
		\hline
		\textbf{Paramtername} & \textbf{Datentyp} & \textbf{Konstante} & \textbf{Kurzbeschreibung}                                                                                               \\ \hline
		result              & string           &                 & Erfolgreich wenn Wert {\glqq ack\grqq} ist \\ \hline
		Code                & int              &                 & Erfolgreich wenn Wert {\glqq 0\grqq} ist \\ \hline
	\end{tabular}
\end{table}
\subsubsection{Verknüpfung zwischen Interessenpunkt und Geschichte wiederherstellen}
\paragraph{Kurzbeschreibung}Dieser API-Request wird dazu genutzt um eine Verknüpfung zwischen einer Geschichte und einem Interessenpunkt wiederherzustellen.
\paragraph{Anfrage}Folgende Daten werden zu Anfrage benötigt:
\begin{table}[H]
	\begin{tabular}{|c|c|c|p{6.5cm}|}
		\hline
		\textbf{Paramtername} & \textbf{Datentyp} & \textbf{Konstante} & \textbf{Kurzbeschreibung}                                                                                               \\ \hline
		type                & string            & rsp                & Link wiederherstellen \\ \hline
		IDent               & int               &                    & Identifikator des Links \\ \hline
	\end{tabular}
\end{table}
\paragraph{Antwort}Die Antwort ist wie folgt aufgebaut:
\begin{table}[H]
	\begin{tabular}{|c|c|c|p{6.5cm}|}
		\hline
		\textbf{Paramtername} & \textbf{Datentyp} & \textbf{Konstante} & \textbf{Kurzbeschreibung}                                                                                               \\ \hline
		result              & string           &                 & Erfolgreich wenn Wert {\glqq ack\grqq} ist \\ \hline
		Code                & int              &                 & Erfolgreich wenn Wert {\glqq 0\grqq} ist \\ \hline
	\end{tabular}
\end{table}
\subsubsection{Kommentar eines Interessenpunktes löschen}
\paragraph{Kurzbeschreibung}Dieser API-Request wird dazu genutzt um einen Kommentar eines Interessenpunktes final zu löschen.
\paragraph{Anfrage}Folgende Daten werden zu Anfrage benötigt:
\begin{table}[H]
	\begin{tabular}{|c|c|c|p{6.5cm}|}
		\hline
		\textbf{Paramtername} & \textbf{Datentyp} & \textbf{Konstante} & \textbf{Kurzbeschreibung}                                                                                               \\ \hline
		type                & string            & fcp                & Kommentar löschen \\ \hline
		IDent               & int               &                    & Identifikator des Kommentars \\ \hline
	\end{tabular}
\end{table}
\paragraph{Antwort}Die Antwort ist wie folgt aufgebaut:
\begin{table}[H]
	\begin{tabular}{|c|c|c|p{6.5cm}|}
		\hline
		\textbf{Paramtername} & \textbf{Datentyp} & \textbf{Konstante} & \textbf{Kurzbeschreibung}                                                                                               \\ \hline
		result              & string           &                 & Erfolgreich wenn Wert {\glqq ack\grqq} ist \\ \hline
		Code                & int              &                 & Erfolgreich wenn Wert {\glqq 0\grqq} ist \\ \hline
	\end{tabular}
\end{table}
\subsubsection{Kommentar eines Interessenpunktes wiederherstellen}
\paragraph{Kurzbeschreibung}Dieser API-Request wird dazu genutzt um einen Kommentar eines Interessenpunktes wiederherzustellen.
\paragraph{Anfrage}Folgende Daten werden zu Anfrage benötigt:
\begin{table}[H]
	\begin{tabular}{|c|c|c|p{6.5cm}|}
		\hline
		\textbf{Paramtername} & \textbf{Datentyp} & \textbf{Konstante} & \textbf{Kurzbeschreibung}                                                                                               \\ \hline
		type                & string            & rcp                & Kommentar wiederherstellen \\ \hline
		IDent               & int               &                    & Identifikator des Kommentars \\ \hline
	\end{tabular}
\end{table}
\paragraph{Antwort}Die Antwort ist wie folgt aufgebaut:
\begin{table}[H]
	\begin{tabular}{|c|c|c|p{6.5cm}|}
		\hline
		\textbf{Paramtername} & \textbf{Datentyp} & \textbf{Konstante} & \textbf{Kurzbeschreibung}                                                                                               \\ \hline
		result              & string           &                 & Erfolgreich wenn Wert {\glqq ack\grqq} ist \\ \hline
		Code                & int              &                 & Erfolgreich wenn Wert {\glqq 0\grqq} ist \\ \hline
	\end{tabular}
\end{table}
\subsubsection{Interessenpunkt löschen}
\paragraph{Kurzbeschreibung}Dieser API-Request wird dazu genutzt um einen Interessenpunkt final zu löschen.
\paragraph{Anfrage}Folgende Daten werden zu Anfrage benötigt:
\begin{table}[H]
	\begin{tabular}{|c|c|c|p{6.5cm}|}
		\hline
		\textbf{Paramtername} & \textbf{Datentyp} & \textbf{Konstante} & \textbf{Kurzbeschreibung}                                                                                               \\ \hline
		type                & string            & fpi                & Interessenpunkt löschen \\ \hline
		IDent               & int               &                    & Identifikator des Interessenpunktes \\ \hline
	\end{tabular}
\end{table}
\paragraph{Antwort}Die Antwort ist wie folgt aufgebaut:
\begin{table}[H]
	\begin{tabular}{|c|c|c|p{6.5cm}|}
		\hline
		\textbf{Paramtername} & \textbf{Datentyp} & \textbf{Konstante} & \textbf{Kurzbeschreibung}                                                                                               \\ \hline
		result              & string           &                 & Erfolgreich wenn Wert {\glqq ack\grqq} ist \\ \hline
		Code                & int              &                 & Erfolgreich wenn Wert {\glqq 0\grqq} ist \\ \hline
	\end{tabular}
\end{table}
\subsubsection{Interessenpunkt wiederherstellen}
\paragraph{Kurzbeschreibung}Dieser API-Request wird dazu genutzt um einen Interessenpunkt wiederherzustellen.
\paragraph{Anfrage}Folgende Daten werden zu Anfrage benötigt:
\begin{table}[H]
	\begin{tabular}{|c|c|c|p{6.5cm}|}
		\hline
		\textbf{Paramtername} & \textbf{Datentyp} & \textbf{Konstante} & \textbf{Kurzbeschreibung}                                                                                               \\ \hline
		type                & string            & rpi                & Interessenpunkt wiederherstellen \\ \hline
		IDent               & int               &                    & Identifikator des Interessenpunktes \\ \hline
	\end{tabular}
\end{table}
\paragraph{Antwort}Die Antwort ist wie folgt aufgebaut:
\begin{table}[H]
	\begin{tabular}{|c|c|c|p{6.5cm}|}
		\hline
		\textbf{Paramtername} & \textbf{Datentyp} & \textbf{Konstante} & \textbf{Kurzbeschreibung}                                                                                               \\ \hline
		result              & string           &                 & Erfolgreich wenn Wert {\glqq ack\grqq} ist \\ \hline
		Code                & int              &                 & Erfolgreich wenn Wert {\glqq 0\grqq} ist \\ \hline
	\end{tabular}
\end{table}

\subsubsection{Geschichte löschen}
\paragraph{Kurzbeschreibung}Dieser API-Request wird dazu genutzt um eine Geschichte final zu löschen.
\paragraph{Anfrage}Folgende Daten werden zu Anfrage benötigt:
\begin{table}[H]
	\begin{tabular}{|c|c|c|p{6.5cm}|}
		\hline
		\textbf{Paramtername} & \textbf{Datentyp} & \textbf{Konstante} & \textbf{Kurzbeschreibung}                                                                                               \\ \hline
		type                & string            & fst               & Geschichte löschen \\ \hline
		IDent               & string            &                   & Identifikator der Geschichte \\ \hline
	\end{tabular}
\end{table}
\paragraph{Antwort}Die Antwort ist wie folgt aufgebaut:
\begin{table}[H]
	\begin{tabular}{|c|c|c|p{6.5cm}|}
		\hline
		\textbf{Paramtername} & \textbf{Datentyp} & \textbf{Konstante} & \textbf{Kurzbeschreibung}                                                                                               \\ \hline
		result              & string           &                 & Erfolgreich wenn Wert {\glqq ack\grqq} ist \\ \hline
		Code                & int              &                 & Erfolgreich wenn Wert {\glqq 0\grqq} ist \\ \hline
	\end{tabular}
\end{table}
\subsubsection{Geschichte wiederherstellen}
\paragraph{Kurzbeschreibung}Dieser API-Request wird dazu genutzt um eine Geschichte wiederherzustellen.
\paragraph{Anfrage}Folgende Daten werden zu Anfrage benötigt:
\begin{table}[H]
	\begin{tabular}{|c|c|c|p{6.5cm}|}
		\hline
		\textbf{Paramtername} & \textbf{Datentyp} & \textbf{Konstante} & \textbf{Kurzbeschreibung}                                                                                               \\ \hline
		type                & string            & rst                & Geschichte wiederherstellen \\ \hline
		IDent               & string            &                    & Identifikator der Geschichte \\ \hline
	\end{tabular}
\end{table}
\paragraph{Antwort}Die Antwort ist wie folgt aufgebaut:
\begin{table}[H]
	\begin{tabular}{|c|c|c|p{6.5cm}|}
		\hline
		\textbf{Paramtername} & \textbf{Datentyp} & \textbf{Konstante} & \textbf{Kurzbeschreibung}                                                                                               \\ \hline
		result              & string           &                 & Erfolgreich wenn Wert {\glqq ack\grqq} ist \\ \hline
		Code                & int              &                 & Erfolgreich wenn Wert {\glqq 0\grqq} ist \\ \hline
	\end{tabular}
\end{table}
\subsubsection{Ankündigung Hinzufügen}
\paragraph{Kurzbeschreibung}Dieser API-Request wird dazu genutzt um eine Ankündigung an zu legen.
\paragraph{Anfrage}Folgende Daten werden zu Anfrage benötigt:
\begin{table}[H]
	\begin{tabular}{|c|c|c|p{6.5cm}|}
		\hline
		\textbf{Paramtername} & \textbf{Datentyp} & \textbf{Konstante} & \textbf{Kurzbeschreibung}                                                                                               \\ \hline
		type                & string            & aan                & Ankündigung Hinzufügen \\ \hline
		title               & string            &                    & Titel der Ankündigung \\ \hline
		content             & string            &                    & Inhalt der Ankündigung \\ \hline
		start               & string            &                    & Starttag der Ankündigung \\ \hline
		end                 & string            &                    & Endtag  der Ankündigung \\ \hline
	\end{tabular}
\end{table}
\paragraph{Antwort}Die Antwort ist wie folgt aufgebaut:
\begin{table}[H]
	\begin{tabular}{|c|c|c|p{6.5cm}|}
		\hline
		\textbf{Paramtername} & \textbf{Datentyp} & \textbf{Konstante} & \textbf{Kurzbeschreibung}                                                                                               \\ \hline
		result              & string           &                 & Erfolgreich wenn Wert {\glqq ack\grqq} ist \\ \hline
		Code                & int              &                 & Erfolgreich wenn Wert {\glqq 0\grqq} ist \\ \hline
	\end{tabular}
\end{table}
\subsubsection{Ankündigung Abfragen}
\paragraph{Kurzbeschreibung}Dieser API-Request wird dazu genutzt um eine Ankündigung abzufragen.
\paragraph{Anfrage}Folgende Daten werden zu Anfrage benötigt:
\begin{table}[H]
	\begin{tabular}{|c|c|c|p{6.5cm}|}
		\hline
		\textbf{Paramtername} & \textbf{Datentyp} & \textbf{Konstante} & \textbf{Kurzbeschreibung}                                                                                               \\ \hline
		type                & string            & gan                & Ankündigung Abfragen \\ \hline
		id                  & id                &                    & Identifikator der Ankündigung \\ \hline
	\end{tabular}
\end{table}
\paragraph{Antwort}Die Antwort ist wie folgt aufgebaut:
\begin{table}[H]
	\begin{tabular}{|c|c|c|p{6.5cm}|}
		\hline
		\textbf{Paramtername} & \textbf{Datentyp} & \textbf{Konstante} & \textbf{Kurzbeschreibung}                                                                                               \\ \hline
		result              & string           &                 & Erfolgreich wenn Wert {\glqq ack\grqq} ist \\ \hline
		Code                & int              &                 & Erfolgreich wenn Wert {\glqq 0\grqq} ist \\ \hline
		data                & array            &                 & Strukturierte Daten \\ \hline
	\end{tabular}
\end{table}
\subparagraph{data}Dieses Array enthält Einträge in der nachstehend dargestellten Form haben:
\begin{table}[H]
	\begin{tabular}{|c|c|c|p{6.5cm}|}
		\hline
		\textbf{Paramtername} & \textbf{Datentyp} & \textbf{Konstante} & \textbf{Kurzbeschreibung}    \\ \hline
		id      & int               &                 & Identifikator der Ankündigung\\ \hline
		title   & string            &                 & Titel der Ankündigung \\ \hline
		content & string            &                 & Inhalt der Ankündigung \\ \hline
		start   & string            &                 & Startzeitpunkt der Ankündigung \\ \hline
		end     & string            &                 & Endzeit der Ankündigung \\ \hline
	\end{tabular}
\end{table}
\subsubsection{Ankündigung Ändern}
\paragraph{Kurzbeschreibung}Dieser API-Request wird dazu genutzt um eine Ankündigung zu ändern.
\paragraph{Anfrage}Folgende Daten werden zu Anfrage benötigt:
\begin{table}[H]
	\begin{tabular}{|c|c|c|p{6.5cm}|}
		\hline
		\textbf{Paramtername} & \textbf{Datentyp} & \textbf{Konstante} & \textbf{Kurzbeschreibung}                                                                                               \\ \hline
		type                & string            & uan                & Ankündigung ändern \\ \hline
		title               & string            &                    & Titel der Ankündigung \\ \hline
		content             & string            &                    & Inhalt der Ankündigung \\ \hline
		start               & string            &                    & Starttag der Ankündigung \\ \hline
		end                 & string            &                    & Endtag  der Ankündigung \\ \hline
		id                  & int               &                    & Identifikator der Ankündigung \\ \hline
	\end{tabular}
\end{table}
\paragraph{Antwort}Die Antwort ist wie folgt aufgebaut:
\begin{table}[H]
	\begin{tabular}{|c|c|c|p{6.5cm}|}
		\hline
		\textbf{Paramtername} & \textbf{Datentyp} & \textbf{Konstante} & \textbf{Kurzbeschreibung}                                                                                               \\ \hline
		result              & string           &                 & Erfolgreich wenn Wert {\glqq ack\grqq} ist \\ \hline
		Code                & int              &                 & Erfolgreich wenn Wert {\glqq 0\grqq} ist \\ \hline
	\end{tabular}
\end{table}
\subsubsection{Ankündigung Löschen}
\paragraph{Kurzbeschreibung}Dieser API-Request wird dazu genutzt um eine Ankündigung abzufragen.
\paragraph{Anfrage}Folgende Daten werden zu Anfrage benötigt:
\begin{table}[H]
	\begin{tabular}{|c|c|c|p{6.5cm}|}
		\hline
		\textbf{Paramtername} & \textbf{Datentyp} & \textbf{Konstante} & \textbf{Kurzbeschreibung}                                                                                               \\ \hline
		type                & string            & dan                & Ankündigung Löschen \\ \hline
		id                  & id                &                    & Identifikator der Ankündigung \\ \hline
	\end{tabular}
\end{table}
\paragraph{Antwort}Die Antwort ist wie folgt aufgebaut:
\begin{table}[H]
	\begin{tabular}{|c|c|c|p{6.5cm}|}
		\hline
		\textbf{Paramtername} & \textbf{Datentyp} & \textbf{Konstante} & \textbf{Kurzbeschreibung}                                                                                               \\ \hline
		result              & string           &                 & Erfolgreich wenn Wert {\glqq ack\grqq} ist \\ \hline
		Code                & int              &                 & Erfolgreich wenn Wert {\glqq 0\grqq} ist \\ \hline
	\end{tabular}
\end{table}
\subsubsection{Aktuelle Ankündigungen abfragen}
\paragraph{Kurzbeschreibung}Dieser API-Request wird dazu genutzt um die aktuellen Ankündigungen abzufragen.
\paragraph{Anfrage}Folgende Daten werden zu Anfrage benötigt:
\begin{table}[H]
	\begin{tabular}{|c|c|c|p{6.5cm}|}
		\hline
		\textbf{Paramtername} & \textbf{Datentyp} & \textbf{Konstante} & \textbf{Kurzbeschreibung}                                                                                               \\ \hline
		type                & string            & gca                & Ankündigung Löschen \\ \hline
	\end{tabular}
\end{table}
\paragraph{Antwort}Die Antwort ist wie folgt aufgebaut:
\begin{table}[H]
	\begin{tabular}{|c|c|c|p{6.5cm}|}
		\hline
		\textbf{Paramtername} & \textbf{Datentyp} & \textbf{Konstante} & \textbf{Kurzbeschreibung}                                                                                               \\ \hline
		result              & string           &                 & Erfolgreich wenn Wert {\glqq ack\grqq} ist \\ \hline
		Code                & int              &                 & Erfolgreich wenn Wert {\glqq 0\grqq} ist \\ \hline
		data                & array            &                 & Strukturierte Daten \\ \hline
	\end{tabular}
\end{table}
\subparagraph{data}Dieses Array enthält Einträge in der nachstehend dargestellten Form haben:
\begin{table}[H]
	\begin{tabular}{|c|c|c|p{6.5cm}|}
		\hline
		\textbf{Paramtername} & \textbf{Datentyp} & \textbf{Konstante} & \textbf{Kurzbeschreibung}    \\ \hline
		id      & int               &                 & Identifikator der Ankündigung\\ \hline
		title   & string            &                 & Titel der Ankündigung \\ \hline
		content & string            &                 & Inhalt der Ankündigung \\ \hline
	\end{tabular}
\end{table}
\subsubsection{Aktuelle Ankündigungen aktivieren}
\paragraph{Kurzbeschreibung}Dieser API-Request wird dazu genutzt um die aktuellen Ankündigungen abzufragen.
\paragraph{Anfrage}Folgende Daten werden zu Anfrage benötigt:
\begin{table}[H]
	\begin{tabular}{|c|c|c|p{6.5cm}|}
		\hline
		\textbf{Paramtername} & \textbf{Datentyp} & \textbf{Konstante} & \textbf{Kurzbeschreibung}                                                                                               \\ \hline
		type                & string            & saa                & Ankündigung Löschen \\ \hline
		id                  & int               &                    & Identifikator der Ankündigung \\ \hline
	\end{tabular}
\end{table}
\paragraph{Antwort}Die Antwort ist wie folgt aufgebaut:
\begin{table}[H]
	\begin{tabular}{|c|c|c|p{6.5cm}|}
		\hline
		\textbf{Paramtername} & \textbf{Datentyp} & \textbf{Konstante} & \textbf{Kurzbeschreibung}                                                                                               \\ \hline
		result              & string           &                 & Erfolgreich wenn Wert {\glqq ack\grqq} ist \\ \hline
		Code                & int              &                 & Erfolgreich wenn Wert {\glqq 0\grqq} ist \\ \hline
	\end{tabular}
\end{table}
\subsubsection{Quelle zu Interessenpunkt hinzufügen}
\paragraph{Kurzbeschreibung}Dieser API-Request wird dazu genutzt um einem Interessenpunkt eine neue Quelle hinzuzufügen.
\paragraph{Anfrage}Folgende Daten werden zu Anfrage benötigt:
\begin{table}[H]
	\begin{tabular}{|c|c|c|p{6.5cm}|}
		\hline
		\textbf{Paramtername} & \textbf{Datentyp} & \textbf{Konstante} & \textbf{Kurzbeschreibung}                                                                                               \\ \hline
		type                & string            & asp                & Quelle hinzufügen \\ \hline
		typeSource          & int               &                    & Indentifikator des Typs der Quelle \\ \hline
		source              & string            &                    & Quellenangabe \\ \hline
		relation            & int               &                    & Identifikator des Informationsbezugs der Quelle \\ \hline
		poiid               & int               &                    & Identifikator des Interessenpunktes \\ \hline
	\end{tabular}
\end{table}
\paragraph{Antwort}Die Antwort ist wie folgt aufgebaut:
\begin{table}[H]
	\begin{tabular}{|c|c|c|p{6.5cm}|}
		\hline
		\textbf{Paramtername} & \textbf{Datentyp} & \textbf{Konstante} & \textbf{Kurzbeschreibung}                                                                                               \\ \hline
		result              & string           &                 & Erfolgreich wenn Wert {\glqq ack\grqq} ist \\ \hline
		Code                & int              &                 & Erfolgreich wenn Wert {\glqq 0\grqq} ist \\ \hline
	\end{tabular}
\end{table}
\subsubsection{Quelle zu Interessenpunkt abrufen}
\paragraph{Kurzbeschreibung}Dieser API-Request wird dazu genutzt um einem Interessenpunkt eine neue Quelle hinzuzufügen.
\paragraph{Anfrage}Folgende Daten werden zu Anfrage benötigt:
\begin{table}[H]
	\begin{tabular}{|c|c|c|p{6.5cm}|}
		\hline
		\textbf{Paramtername} & \textbf{Datentyp} & \textbf{Konstante} & \textbf{Kurzbeschreibung}                                                                                               \\ \hline
		type                & string            & grp                & Ankündigung Löschen \\ \hline
		poiid               & int               &                    & Identifikator des Interessenpunktes \\ \hline
	\end{tabular}
\end{table}
\paragraph{Antwort}Die Antwort ist wie folgt aufgebaut:
\begin{table}[H]
	\begin{tabular}{|c|c|c|p{6.5cm}|}
		\hline
		\textbf{Paramtername} & \textbf{Datentyp} & \textbf{Konstante} & \textbf{Kurzbeschreibung}                                                                                               \\ \hline
		result              & string           &                 & Erfolgreich wenn Wert {\glqq ack\grqq} ist \\ \hline
		code                & int              &                 & Erfolgreich wenn Wert {\glqq 0\grqq} ist \\ \hline
		data                & array            &                 & Strukturierte angeforderte Daten \\ \hline
	\end{tabular}
\end{table}
\subparagraph{data}Dieses Array enthält Einträge in der nachstehend dargestellten Form haben:
\begin{table}[H]
	\begin{tabular}{|c|c|c|p{6.5cm}|}
		\hline
		\textbf{Paramtername} & \textbf{Datentyp} & \textbf{Konstante} & \textbf{Kurzbeschreibung}    \\ \hline
		id         & int               &                 & Identifikator der Quelle \\ \hline
		type       & string            &                 & Typ der Quelle \\ \hline
		typeid     & int               &                 & Identifikator des Typs der Quelle \\ \hline
		source     & string            &                 & Inhalt der Ankündigung \\ \hline
		relation   & string            &                 & Bezug der Quelle \\ \hline
		relationid & int               &                 & Identifikator des Bezugs der Quelle \\ \hline
		editable   & boolean           &                 & Nutzer kann Quellenangabe ändern \\ \hline
		deleted    & boolean           &                 & gibt an, ob Quelle als gelöscht zählt \\ \hline
	\end{tabular}
\end{table}
\subsubsection{Alle Quellenbezüge abfragen}
\paragraph{Kurzbeschreibung}Dieser API-Request wird dazu genutzt um alle Bezüge von Quellen abzufragen.
\paragraph{Anfrage}Folgende Daten werden zu Anfrage benötigt:
\begin{table}[H]
	\begin{tabular}{|c|c|c|p{6.5cm}|}
		\hline
		\textbf{Paramtername} & \textbf{Datentyp} & \textbf{Konstante} & \textbf{Kurzbeschreibung}                                                                                               \\ \hline
		type                & string            & grs                & Quellenbeziehungen abfragen \\ \hline
	\end{tabular}
\end{table}
\paragraph{Antwort}Die Antwort ist wie folgt aufgebaut:
\begin{table}[H]
	\begin{tabular}{|c|c|c|p{6.5cm}|}
		\hline
		\textbf{Paramtername} & \textbf{Datentyp} & \textbf{Konstante} & \textbf{Kurzbeschreibung}                                                                                               \\ \hline
		result              & string           &                 & Erfolgreich wenn Wert {\glqq ack\grqq} ist \\ \hline
		code                & int              &                 & Erfolgreich wenn Wert {\glqq 0\grqq} ist \\ \hline
		data                & array            &                 & Strukturierte angeforderte Daten \\ \hline
	\end{tabular}
\end{table}
\subparagraph{data}Dieses Array enthält Einträge in der nachstehend dargestellten Form haben:
\begin{table}[H]
	\begin{tabular}{|c|c|c|p{6.5cm}|}
		\hline
		\textbf{Paramtername} & \textbf{Datentyp} & \textbf{Konstante} & \textbf{Kurzbeschreibung}    \\ \hline
		id         & int               &                 & Identifikator des Bezugs der Quelle \\ \hline
		name       & string            &                 & Name des Bezugs der Quelle \\ \hline
	\end{tabular}
\end{table}
\subsubsection{Alle Quellentypen abfragen}
\paragraph{Kurzbeschreibung}Dieser API-Request wird dazu genutzt um alle Typen von Quellen abzufragen.
\paragraph{Anfrage}Folgende Daten werden zu Anfrage benötigt:
\begin{table}[H]
	\begin{tabular}{|c|c|c|p{6.5cm}|}
		\hline
		\textbf{Paramtername} & \textbf{Datentyp} & \textbf{Konstante} & \textbf{Kurzbeschreibung}                                                                                               \\ \hline
		type                & string            & gts                & Quellenbeziehungen abfragen \\ \hline
	\end{tabular}
\end{table}
\paragraph{Antwort}Die Antwort ist wie folgt aufgebaut:
\begin{table}[H]
	\begin{tabular}{|c|c|c|p{6.5cm}|}
		\hline
		\textbf{Paramtername} & \textbf{Datentyp} & \textbf{Konstante} & \textbf{Kurzbeschreibung}                                                                                               \\ \hline
		result              & string           &                 & Erfolgreich wenn Wert {\glqq ack\grqq} ist \\ \hline
		code                & int              &                 & Erfolgreich wenn Wert {\glqq 0\grqq} ist \\ \hline
		data                & array            &                 & Strukturierte angeforderte Daten \\ \hline
	\end{tabular}
\end{table}
\subparagraph{data}Dieses Array enthält Einträge in der nachstehend dargestellten Form haben:
\begin{table}[H]
	\begin{tabular}{|c|c|c|p{6.5cm}|}
		\hline
		\textbf{Paramtername} & \textbf{Datentyp} & \textbf{Konstante} & \textbf{Kurzbeschreibung}    \\ \hline
		id         & int               &                 & Identifikator des Bezugs der Quelle \\ \hline
		name       & string            &                 & Name des Bezugs der Quelle \\ \hline
	\end{tabular}
\end{table}
\subsubsection{Quelleneintrag ändern}
\paragraph{Kurzbeschreibung}Dieser API-Request wird dazu genutzt um eine Quelle zu ändern.
\paragraph{Anfrage}Folgende Daten werden zu Anfrage benötigt:
\begin{table}[H]
	\begin{tabular}{|c|c|c|p{6.5cm}|}
		\hline
		\textbf{Paramtername} & \textbf{Datentyp} & \textbf{Konstante} & \textbf{Kurzbeschreibung}                                                                                               \\ \hline
		type                & string            & usp                & Quelleneintrag ändern \\ \hline
		id                  & int               &                    & Identifikator der Quelle \\ \hline
		typeSource          & int               &                    & Identifikator des Typs der Quelle \\ \hline
		source              & string            &                    & Quellenangabe \\ \hline
		relation            & int               &                    & Identifikator des Informationsbezugs der Quelle \\ \hline
	\end{tabular}
\end{table}
\paragraph{Antwort}Die Antwort ist wie folgt aufgebaut:
\begin{table}[H]
	\begin{tabular}{|c|c|c|p{6.5cm}|}
		\hline
		\textbf{Paramtername} & \textbf{Datentyp} & \textbf{Konstante} & \textbf{Kurzbeschreibung}                                                                                               \\ \hline
		result              & string           &                 & Erfolgreich wenn Wert {\glqq ack\grqq} ist \\ \hline
		Code                & int              &                 & Erfolgreich wenn Wert {\glqq 0\grqq} ist \\ \hline
	\end{tabular}
\end{table}
\subsubsection{Quelleneintrag löschen}
\paragraph{Kurzbeschreibung}Dieser API-Request wird dazu genutzt um eine Quelle zu löschen oder als gelöscht zu markieren.
\paragraph{Anfrage}Folgende Daten werden zu Anfrage benötigt:
\begin{table}[H]
	\begin{tabular}{|c|c|c|p{6.5cm}|}
		\hline
		\textbf{Paramtername} & \textbf{Datentyp} & \textbf{Konstante} & \textbf{Kurzbeschreibung}                                                                                               \\ \hline
		type                & string            & des                & Quelleneintrag löschen \\ \hline
		id                  & int               &                    & Identifikator der Quelle \\ \hline
	\end{tabular}
\end{table}
\paragraph{Antwort}Die Antwort ist wie folgt aufgebaut:
\begin{table}[H]
	\begin{tabular}{|c|c|c|p{6.5cm}|}
		\hline
		\textbf{Paramtername} & \textbf{Datentyp} & \textbf{Konstante} & \textbf{Kurzbeschreibung}                                                                                               \\ \hline
		result              & string           &                 & Erfolgreich wenn Wert {\glqq ack\grqq} ist \\ \hline
		Code                & int              &                 & Erfolgreich wenn Wert {\glqq 0\grqq} ist \\ \hline
	\end{tabular}
\end{table}
\subsubsection{Quelleneintrag endgültig löschen}
\paragraph{Kurzbeschreibung}Dieser API-Request wird dazu genutzt um eine Quelle zu endgültig zu löschen.
\paragraph{Anfrage}Folgende Daten werden zu Anfrage benötigt:
\begin{table}[H]
	\begin{tabular}{|c|c|c|p{6.5cm}|}
		\hline
		\textbf{Paramtername} & \textbf{Datentyp} & \textbf{Konstante} & \textbf{Kurzbeschreibung}                                                                                               \\ \hline
		type                & string            & fds                & Quelleneintrag endgültig löschen \\ \hline
		id                  & int               &                    & Identifikator der Quelle \\ \hline
	\end{tabular}
\end{table}
\paragraph{Antwort}Die Antwort ist wie folgt aufgebaut:
\begin{table}[H]
	\begin{tabular}{|c|c|c|p{6.5cm}|}
		\hline
		\textbf{Paramtername} & \textbf{Datentyp} & \textbf{Konstante} & \textbf{Kurzbeschreibung}                                                                                               \\ \hline
		result              & string           &                 & Erfolgreich wenn Wert {\glqq ack\grqq} ist \\ \hline
		Code                & int              &                 & Erfolgreich wenn Wert {\glqq 0\grqq} ist \\ \hline
	\end{tabular}
\end{table}
\subsubsection{Quelleneintrag Wiederherstellen}
\paragraph{Kurzbeschreibung}Dieser API-Request wird dazu genutzt um eine Quelle wiederherzustellen.
\paragraph{Anfrage}Folgende Daten werden zu Anfrage benötigt:
\begin{table}[H]
	\begin{tabular}{|c|c|c|p{6.5cm}|}
		\hline
		\textbf{Paramtername} & \textbf{Datentyp} & \textbf{Konstante} & \textbf{Kurzbeschreibung}                                                                                               \\ \hline
		type                & string            & rso                & Quelleneintrag wiederherstellen \\ \hline
		id                  & int               &                    & Identifikator der Quelle \\ \hline
	\end{tabular}
\end{table}
\paragraph{Antwort}Die Antwort ist wie folgt aufgebaut:
\begin{table}[H]
	\begin{tabular}{|c|c|c|p{6.5cm}|}
		\hline
		\textbf{Paramtername} & \textbf{Datentyp} & \textbf{Konstante} & \textbf{Kurzbeschreibung}                                                                                               \\ \hline
		result              & string           &                 & Erfolgreich wenn Wert {\glqq ack\grqq} ist \\ \hline
		Code                & int              &                 & Erfolgreich wenn Wert {\glqq 0\grqq} ist \\ \hline
	\end{tabular}
\end{table}
\subsubsection{Interessenpunkt validieren}
\paragraph{Kurzbeschreibung}Dieser API-Request wird dazu genutzt um einen Interessenpunkt zu validieren.
\paragraph{Anfrage}Folgende Daten werden zu Anfrage benötigt:
\begin{table}[H]
	\begin{tabular}{|c|c|c|p{6.5cm}|}
		\hline
		\textbf{Paramtername} & \textbf{Datentyp} & \textbf{Konstante} & \textbf{Kurzbeschreibung}                                                                                               \\ \hline
		type                & string            & vpi                & Interessenpunkt validieren \\ \hline
		id                  & int               &                    & Identifikator des Interessenpunktes \\ \hline
	\end{tabular}
\end{table}
\paragraph{Antwort}Die Antwort ist wie folgt aufgebaut:
\begin{table}[H]
	\begin{tabular}{|c|c|c|p{6.5cm}|}
		\hline
		\textbf{Paramtername} & \textbf{Datentyp} & \textbf{Konstante} & \textbf{Kurzbeschreibung}                                                                                               \\ \hline
		result              & string           &                 & Erfolgreich wenn Wert {\glqq ack\grqq} ist \\ \hline
		Code                & int              &                 & Erfolgreich wenn Wert {\glqq 0\grqq} ist \\ \hline
	\end{tabular}
\end{table}
\subsubsection{Direkt löschen abfragen}
\paragraph{Kurzbeschreibung}Dieser API-Request wird dazu genutzt um zu prüfen, ob Daten direkt gelöscht werden.
\paragraph{Anfrage}Folgende Daten werden zu Anfrage benötigt:
\begin{table}[H]
	\begin{tabular}{|c|c|c|p{6.5cm}|}
		\hline
		\textbf{Paramtername} & \textbf{Datentyp} & \textbf{Konstante} & \textbf{Kurzbeschreibung}                                                                                               \\ \hline
		type                & string            & vpi                & Interessenpunkt validieren \\ \hline
	\end{tabular}
\end{table}
\paragraph{Antwort}Die Antwort ist wie folgt aufgebaut:
\begin{table}[H]
	\begin{tabular}{|c|c|c|p{6.5cm}|}
		\hline
		\textbf{Paramtername} & \textbf{Datentyp} & \textbf{Konstante} & \textbf{Kurzbeschreibung}                                                                                               \\ \hline
		result              & string           &                 & Erfolgreich wenn Wert {\glqq ack\grqq} ist \\ \hline
		Code                & int              &                 & Erfolgreich wenn Wert {\glqq 0\grqq} ist \\ \hline
		data                & bool             &                 & Wenn Wahr, werden Daten direkt gelöscht \\ \hline
	\end{tabular}
\end{table}
\subsubsection{Hauptbild eines Interessenpunktes ändern}
\paragraph{Kurzbeschreibung}Dieser API-Request wird dazu genutzt um zu prüfen, ob Daten direkt gelöscht werden.
\paragraph{Anfrage}Folgende Daten werden zu Anfrage benötigt:
\begin{table}[H]
	\begin{tabular}{|c|c|c|p{6.5cm}|}
		\hline
		\textbf{Paramtername} & \textbf{Datentyp} & \textbf{Konstante} & \textbf{Kurzbeschreibung}                                                                                               \\ \hline
		type                & string            & emp                & Hauptbild ändern \\ \hline
		poiid				& int				& 					 & Identifikator eines Interessenpunktes \\ \hline
		token				& string			&					 & Identifikator eines Bildes \\ \hline
	\end{tabular}
\end{table}
\paragraph{Antwort}Die Antwort ist wie folgt aufgebaut:
\begin{table}[H]
	\begin{tabular}{|c|c|c|p{6.5cm}|}
		\hline
		\textbf{Paramtername} & \textbf{Datentyp} & \textbf{Konstante} & \textbf{Kurzbeschreibung}                                                                                               \\ \hline
		result              & string           &                 & Erfolgreich wenn Wert {\glqq ack\grqq} ist \\ \hline
		Code                & int              &                 & Erfolgreich wenn Wert {\glqq 0\grqq} ist \\ \hline
		data                & array            &                 & Leeres Array \\ \hline
	\end{tabular}
\end{table}
\subsubsection{Mailadresse bereits existent}
\paragraph{Kurzbeschreibung}Dieser API-Request wird dazu genutzt um zu prüfen, ob eine Mailadresse bereits verwendet wird.
\paragraph{Anfrage}Folgende Daten werden zu Anfrage benötigt:
\begin{table}[H]
	\begin{tabular}{|c|c|c|p{6.5cm}|}
		\hline
		\textbf{Paramtername} & \textbf{Datentyp} & \textbf{Konstante} & \textbf{Kurzbeschreibung}                                                                                               \\ \hline
		type                & string            & cma                & Mailadresse prüfen \\ \hline
		mail				& string			& 					 & Mailadresse\\ \hline
	\end{tabular}
\end{table}
\paragraph{Antwort}Die Antwort ist wie folgt aufgebaut:
\begin{table}[H]
	\begin{tabular}{|c|c|c|p{6.5cm}|}
		\hline
		\textbf{Paramtername} & \textbf{Datentyp} & \textbf{Konstante} & \textbf{Kurzbeschreibung}                                                                                               \\ \hline
		result              & string           &                 & Erfolgreich wenn Wert {\glqq ack\grqq} ist \\ \hline
		Code                & int              &                 & Erfolgreich wenn Wert {\glqq 0\grqq} ist \\ \hline
		data                & bool             &                 & True, wenn Mailadresse verwendet wird \\ \hline
	\end{tabular}
\end{table}
\chapter{Datenbank-Spezifikation}
\section{Tabellen-Übersicht}
\begin{longtable}[H]{|l|p{9cm}|}
	\hline
	\textbf{Api-Befehl} 			& \textbf{Kurzbeschreibung}              \\ \hline
	kino\_\_address\_validate 		& Validierungsdaten für historische Adressen \\ \hline
	kino\_\_announcement            & Tabelle mit Daten für Ankündigungen \\ \hline
	kino\_\_cinema\_type\_validate	& Validierungsdaten für den Typ einer Spielstätte \\ \hline
	kino\_\_cinema\_types 			& Typen einer Spielstätte \\ \hline
	kino\_\_cinemas 				& Saalanzahlen der Interessenpunkte \\ \hline
	kino\_\_cinemas\_validate 		& Validierungsdaten zu Saalanzahlen von Interessenpunkten\\ \hline
	kino\_\_comments 				& Kommentare der Interessenpunkte \\ \hline
	kino\_\_current\_adr\_validate 	& Validierungsdaten zu aktuellen Adressen von Interessenpunkten \\ \hline
	kino\_\_hist\_adr 				& Historische Adressen zu Interessenpunkten \\ \hline
	kino\_\_history\_validate 		& Validierungsdaten zur Historie von Interessenpunkten \\ \hline
	kino\_\_name\_validate 			& Validierungsdaten zu Namen von Interessenpunkten \\ \hline
	kino\_\_names 					& Namen von Interessenpunkten \\ \hline
	kino\_\_operator\_validate 		& Validierungsdaten zu Betreibern von Interessenpunkten \\ \hline
	kino\_\_operators 				& Betreiber von Interessenpunkten \\ \hline
	kino\_\_poi\_pictures 			& Bild-Interessenpunkt-Verknüpfungen\\ \hline
	kino\_\_poi\_pictures\_validate & Validierungsdaten zu Bild-Verknüpfungen von Interessenpunkten \\ \hline
	kino\_\_poi\_sources            & Quelleninformationen zu Interessempunkt \\ \hline
	kino\_\_poi\_story 				& Geschichte--Interessenpunkt-Verknüpfungen \\ \hline
	kino\_\_poi\_story\_validate 	& Validierungsdaten zu Geschichtsverknüpfungen von Interessenpunkten \\ \hline
	kino\_\_pois 					& Grundlegende Daten zu Interessenpunkten \\ \hline
	kino\_\_seats 					& Sitzplatzanzahlen von Interessenpunkten\\ \hline
	kino\_\_seats\_validate 		& Validierungsdaten zu Sitzplatzanzahlen von Interessenpunkten \\ \hline
	kino\_\_source\_relation        & Tabelle mit Namen von Bezugsinformationen zu Quellen \\ \hline
	kino\_\_source\_type            & Typen von Quellen \\ \hline
	kino\_\_timespan\_validate 		& Validierungsdaten zum Betriebszeitraum von Interessenpunkten \\ \hline
	kino\_\_user-login 				& Nutzerdaten \\ \hline
	kino\_\_validate 				& Validierungsdaten von Interessenpunkten \\ \hline
	kino\_\_visitors 				& Statistische Aufruferdaten \\ \hline
\end{longtable}
\newpage
\section{Erläuterung}
\subsection{Abkürzungen}
\begin{table}[H]
	\begin{tabular}{|c|p{12cm}|}
		\hline
		\textbf{Abkürzung} & \textbf{Bedeutung} \\ \hline
		PRI & Primary Key \\ \hline
		FOR & Foreign Key / Fremdschlüssel \\ \hline
	\end{tabular}
\end{table}
\subsection{Aufbau}
In den folgenden Abschnitten wird zuerst etwas über die Verwendung der Tabelle gesagt. Anschließend ist noch der Aufbau detailliert geschildert. Das Feld {\glqq Null\grqq} besagt, ob dieser Wert in der Tabelle den Wert {\glqq null\grqq} annehmen darf. Im Feld {\glqq Key\grqq} ist zu sehen ob dieser Wert als Schlüssel verwendet wird. Sofern dies ein Fremdschlüssel ({\glqq FOR\grqq}) ist, ist in der nächsten Tabelle zu finden auf welches Feld welcher Tabelle dieser sich bezieht.
\section{Tabellen}
\subsection{kino\_\_address\_validate}
\subsubsection{Verwendung} Diese Tabelle wird verwendet um alle Validierungen von historischen Adressen zu speichern. Hierzu wird die ID der historischen Adresse und die Nutzer-ID des validierenden Nutzers benötigt.
\subsubsection{Inhalt}
\begin{table}[H]
	\begin{tabular}{|c|c|c|c|c|p{3.5cm}|}
		\hline
		\textbf{Feldname} & \textbf{Datentyp} & \textbf{Null} & \textbf{Standardwert} & \textbf{Key}   & \textbf{Besonderheiten} \\ \hline
		id & int & NO &  & PRI & auto\_increment  \\ \hline
		address\_id & int & NO &  & FOR &  \\ \hline
		uid & int & NO &  & FOR &  \\ \hline
		value & int & NO &  &  &  \\ \hline
		date & timestamp & NO & current\_timestamp() &  &  \\ \hline
	\end{tabular}
\end{table}
\subsubsection{Beschreibung}
\begin{table}[H]
	\begin{tabular}{|c|p{12cm}|}
		\hline
		\textbf{Feldname} & \textbf{Beschreibung} \\ \hline
		id & Identifikator der Validierung \\ \hline
		address\_id & Identifikator der validierten Adresse \\ \hline
		uid & Identifikator des validierenden Nutzers \\ \hline
		value & Wertung der Validierung \\ \hline
		date & Zeitstempel der Validierung \\ \hline
	\end{tabular}
\end{table}
\subsubsection{Fremdschlüssel}
\begin{table}[H]
	\begin{tabular}{|c|p{12.5cm}|}
		\hline
		\textbf{Feldname} & \textbf{Fremd-Feld} \\ \hline
		address\_id & kino\_\_hist\_adr.ID \\ \hline
		uid & kino\_\_user-login.id \\ \hline
	\end{tabular}
\end{table}
\subsection{kino\_\_announcement}
\subsubsection{Verwendung} Diese Tabelle wird verwendet um alle Daten für Ankündigungen zu speichern.
\subsubsection{Inhalt}
\begin{table}[H]
	\begin{tabular}{|c|c|c|c|c|p{3.5cm}|}
		\hline
		\textbf{Feldname} & \textbf{Datentyp} & \textbf{Null} & \textbf{Standardwert} & \textbf{Key}   & \textbf{Besonderheiten} \\ \hline
		id & int & NO &  & PRI & auto\_increment  \\ \hline
		title & mediumtext & NO &  &  &  \\ \hline
		content & longtext & NO &  &  &  \\ \hline
		start & date & NO &  &  &  \\ \hline
		end & date & NO &  &  &  \\ \hline
		creator & int & NO &  & FOR &  \\ \hline
	\end{tabular}
\end{table}
\subsubsection{Beschreibung}
\begin{table}[H]
	\begin{tabular}{|c|p{12cm}|}
		\hline
		\textbf{Feldname} & \textbf{Beschreibung} \\ \hline
		id & Identifikator der Ankündigung \\ \hline
		title & Titel der Ankündigung \\ \hline
		content & Inhalt der Ankündigung \\ \hline
		start & Startzeitpunkt, ab wann die Ankündigung angezeigt wird \\ \hline
		end & Endzeitpunkt, ab wann die Ankündigung nicht mehr Angezeigt wird \\ \hline
		creator & Ersteller der Ankündigung \\ \hline
	\end{tabular}
\end{table}
\subsubsection{Fremdschlüssel}
\begin{table}[H]
	\begin{tabular}{|c|p{12.5cm}|}
		\hline
		\textbf{Feldname} & \textbf{Fremd-Feld} \\ \hline
		creator & kino\_\_user-login.id \\ \hline
	\end{tabular}
\end{table}
\subsection{kino\_\_cinema\_type\_validate}
\subsubsection{Verwendung} Diese Tabelle wird verwendet um alle Validierungen zum Typ eines Interessenpunktes zu speichern. Hierzu wird die ID des Interessenpunktes und die Nutzer-ID des validierenden Nutzers benötigt.
\subsubsection{Inhalt}
\begin{table}[H]
	\begin{tabular}{|c|c|c|c|c|p{3.5cm}|}
		\hline
		\textbf{Feldname} & \textbf{Datentyp} & \textbf{Null} & \textbf{Standardwert} & \textbf{Key}   & \textbf{Besonderheiten} \\ \hline
		id & int & NO &  & PRI & auto\_increment \\ \hline
		poi\_id & int & NO &  & FOR &  \\ \hline
		uid & int & NO &  & FOR &  \\ \hline
		value & int & NO &  &  &  \\ \hline
		date & timestamp & NO & current\_timestamp() &  &  \\ \hline
	\end{tabular}
\end{table}
\subsubsection{Beschreibung}
\begin{table}[H]
	\begin{tabular}{|c|p{12cm}|}
		\hline
		\textbf{Feldname} & \textbf{Beschreibung} \\ \hline
		id & Identifikator der Validierung \\ \hline
		poi\_id & Identifikator des Interessenpunktes \\ \hline
		uid & Identifikator des validierenden Nutzers \\ \hline
		value & Wertung der Validierung \\ \hline
		date & Zeitstempel der Validierung \\ \hline
	\end{tabular}
\end{table}
\subsubsection{Fremdschlüssel}
\begin{table}[H]
	\begin{tabular}{|c|p{12.5cm}|}
		\hline
		\textbf{Feldname} & \textbf{Fremd-Feld} \\ \hline
		poi\_id & kino\_\_pois.poi\_id \\ \hline
		uid & kino\_\_user-login.id \\ \hline
	\end{tabular}
\end{table}
\subsection{kino\_\_cinema\_types}
\subsubsection{Verwendung} Diese Tabelle wird verwendet um alle verschiedenen Typen eines Interessenpunktes aufzulisten. Diese Tabelle dient als Referenz für die Interessenpunkte.
\subsubsection{Inhalt}
\begin{table}[H]
	\begin{tabular}{|c|c|c|c|c|p{3.5cm}|}
		\hline
		\textbf{Feldname} & \textbf{Datentyp} & \textbf{Null} & \textbf{Standardwert} & \textbf{Key}   & \textbf{Besonderheiten} \\ \hline
		id & int & NO &  & PRI & auto\_increment \\ \hline
		name & mediumtext & NO &  &  &  \\ \hline
	\end{tabular}
\end{table}
\subsubsection{Beschreibung}
\begin{table}[H]
	\begin{tabular}{|c|p{12cm}|}
		\hline
		\textbf{Feldname} & \textbf{Beschreibung} \\ \hline
		id & Identifikator des Typs \\ \hline
		name & Name des Typs \\ \hline
	\end{tabular}
\end{table}
\subsubsection{Fremdschlüssel}
In dieser Tabelle sind keine Fremdschlüssel vorhanden.
\subsection{kino\_\_cinemas}
\subsubsection{Verwendung} Diese Tabelle wird verwendet um die Saalanzahlen von Interessenpunkten zu speichern. Hierzu wird die ID des Interessenpunktes benötigt und die Nutzer-ID des anlegenden Nutzers benötigt.
\subsubsection{Inhalt}
\begin{table}[H]
	\begin{tabular}{|c|c|c|c|c|p{3.5cm}|}
		\hline
		\textbf{Feldname} & \textbf{Datentyp} & \textbf{Null} & \textbf{Standardwert} & \textbf{Key}   & \textbf{Besonderheiten} \\ \hline
		ID & int & NO &  & PRI & auto\_increment \\ \hline
		POI\_ID & int & NO &  & FOR &  \\ \hline
		cinemas & text & NO &  &  &  \\ \hline
		Start & int & YES & NULL &  &  \\ \hline
		End & int & YES & NULL &  &  \\ \hline
		creator & int & NO &  & FOR &  \\ \hline
		creationdate & datetime & NO & current\_timestamp() &  &  \\ \hline
		source & text & YES & NULL &  &  \\ \hline
		points\_received & tinyint & NO & 0 &  &  \\ \hline
		deleted & tinyint & NO & 0 &  &  \\ \hline
	\end{tabular}
\end{table}
\subsubsection{Beschreibung}
\begin{table}[H]
	\begin{tabular}{|c|p{12cm}|}
		\hline
		\textbf{Feldname} & \textbf{Beschreibung} \\ \hline
		ID & Identifikator des Eintrags der Saalanzahl \\ \hline
		POI\_ID & Identifikator des zugehörigen Interessenpunktes \\ \hline
		cinemas & Saalanzahl \\ \hline
		Start & Beginn dieser Saalanzahl \\ \hline
		End & Ende dieser Saalanzahl \\ \hline
		creator & Nutzeridentifikator des Erstellers des Eintrags \\ \hline
		creationdate & Erstellungs- beziehungsweise Änderungsdatum \\ \hline
		source & Möglichkeit der Quellenangabe \\ \hline
		points\_received & Status Erhalt der Punkte (Veraltet) \\ \hline
		deleted & Status des Löschens \\ \hline
	\end{tabular}
\end{table}
\subsubsection{Fremdschlüssel}
\begin{table}[H]
	\begin{tabular}{|c|p{12.5cm}|}
		\hline
		\textbf{Feldname} & \textbf{Fremd-Feld} \\ \hline
		POI\_ID & kino\_\_pois.poi\_id \\ \hline
		creator & kino\_\_user-login.id \\ \hline
	\end{tabular}
\end{table}
\subsection{kino\_\_cinemas\_validate}
\subsubsection{Verwendung} Diese Tabelle wird verwendet um alle Validierungen von Saalanzahlen zu speichern. Hierzu wird die ID der Saalanzahl und die Nutzer-ID des validierenden Nutzers benötigt.
\subsubsection{Inhalt}
\begin{table}[H]
	\begin{tabular}{|c|c|c|c|c|p{3.5cm}|}
		\hline
		\textbf{Feldname} & \textbf{Datentyp} & \textbf{Null} & \textbf{Standardwert} & \textbf{Key}   & \textbf{Besonderheiten} \\ \hline
		id & int & NO &  & PRI & auto\_increment \\ \hline
		cinemas\_id & int & NO &  & FOR & \\ \hline
		uid & int & NO &  & FOR & \\ \hline
		value & int & NO &  &  & \\ \hline
		date & timestamp & NO & current\_timestamp() &  & \\ \hline
	\end{tabular}
\end{table}
\subsubsection{Beschreibung}
\begin{table}[H]
	\begin{tabular}{|c|p{12cm}|}
		\hline
		\textbf{Feldname} & \textbf{Beschreibung} \\ \hline
		id & Identifikator der Validierung \\ \hline
		cinemas\_id & Identifikator der validierten Saalanzahl \\ \hline
		uid & Identifikator des validierenden Nutzers \\ \hline
		value & Wertung der Validierung \\ \hline
		date & Zeitstempel der Validierung \\ \hline
	\end{tabular}
\end{table}
\subsubsection{Fremdschlüssel}
\begin{table}[H]
	\begin{tabular}{|c|p{12.5cm}|}
		\hline
		\textbf{Feldname} & \textbf{Fremd-Feld} \\ \hline
		cinemas\_id & kino\_\_cinemas.ID \\ \hline
		uid & kino\_\_user-login.id \\ \hline
	\end{tabular}
\end{table}
\subsection{kino\_\_comments}
\subsubsection{Verwendung} Diese Tabelle wird verwendet um Kommentare zu Interessenpunkten zu speichern. Hierzu wird die ID des Interessenpunktes und die Nutzer-ID benötigt.
\subsubsection{Inhalt}
\begin{table}[H]
	\begin{tabular}{|c|c|c|c|c|p{3.5cm}|}
		\hline
		\textbf{Feldname} & \textbf{Datentyp} & \textbf{Null} & \textbf{Standardwert} & \textbf{Key}   & \textbf{Besonderheiten} \\ \hline
		comment\_id & int & NO &  & PRI & auto\_increment \\ \hline
		timestamp & datetime & NO & current\_timestamp() &  & \\ \hline
		user\_id & int & NO &  & FOR & \\ \hline
		poi\_id & int & NO &  & FOR & \\ \hline
		content & text & NO &  &  & \\ \hline
		deleted & tinyint & NO & 0 &  & \\ \hline
	\end{tabular}
\end{table}
\subsubsection{Beschreibung}
\begin{table}[H]
	\begin{tabular}{|c|p{12cm}|}
		\hline
		\textbf{Feldname} & \textbf{Beschreibung} \\ \hline
		comment\_id & Identifikator des Kommentars \\ \hline
		timestamp & Zeitstempel des Erstellen oder Änderns \\ \hline
		user\_id & Nutzeridentifikator des Erstellers \\ \hline
		poi\_id & Identifikator des zugehörigen Interessenpunktes \\ \hline
		content & Inhalt des Kommentars \\ \hline
		deleted & Status der Löschung des Kommentars \\ \hline
	\end{tabular}
\end{table}
\subsubsection{Fremdschlüssel}
\begin{table}[H]
	\begin{tabular}{|c|p{12.5cm}|}
		\hline
		\textbf{Feldname} & \textbf{Fremd-Feld} \\ \hline
		poi\_id & kino\_\_pois.poi\_id \\ \hline
		user\_id & kino\_\_user-login.id \\ \hline
	\end{tabular}
\end{table}
\subsection{kino\_\_current\_adr\_validate}
\subsubsection{Verwendung} Diese Tabelle wird verwendet um alle Validierungen zur aktuellen Adresse eines Interessenpunktes zu speichern. Hierzu wird die ID des Interessenpunktes und die Nutzer-ID des validierenden Nutzers benötigt.
\subsubsection{Inhalt}
\begin{table}[H]
	\begin{tabular}{|c|c|c|c|c|p{3.5cm}|}
		\hline
		\textbf{Feldname} & \textbf{Datentyp} & \textbf{Null} & \textbf{Standardwert} & \textbf{Key}   & \textbf{Besonderheiten} \\ \hline
		id & int & NO &  & PRI & auto\_increment \\ \hline
		poi\_id & int & NO &  & FOR & \\ \hline
		uid & int & NO &  & FOR & \\ \hline
		value & int & NO &  &  & \\ \hline
		date & timestamp & NO & current\_timestamp() &  & \\ \hline
	\end{tabular}
\end{table}
\subsubsection{Beschreibung}
\begin{table}[H]
	\begin{tabular}{|c|p{12cm}|}
		\hline
		\textbf{Feldname} & \textbf{Beschreibung} \\ \hline
		id & Identifikator der Validierung \\ \hline
		poi\_id & Identifikator der zugehörigen Interessenpunktes \\ \hline
		uid & Identifikator des validierenden Nutzers \\ \hline
		value & Wertung der Validierung \\ \hline
		date & Zeitstempel der Validierung \\ \hline
	\end{tabular}
\end{table}
\subsubsection{Fremdschlüssel}
\begin{table}[H]
	\begin{tabular}{|c|p{12.5cm}|}
		\hline
		\textbf{Feldname} & \textbf{Fremd-Feld} \\ \hline
		poi\_id & kino\_\_pois.poi\_id \\ \hline
		uid & kino\_\_user-login.id \\ \hline
	\end{tabular}
\end{table}
\subsection{kino\_\_hist\_adr}
\subsubsection{Verwendung} Diese Tabelle wird verwendet um historische Adresseb zu Interessenpunkten zu speichern. Hierzu wird die ID des Interessenpunktes und die Nutzer-ID benötigt.
\subsubsection{Inhalt}
\begin{table}[H]
	\begin{tabular}{|c|c|c|c|c|p{3.5cm}|}
		\hline
		\textbf{Feldname} & \textbf{Datentyp} & \textbf{Null} & \textbf{Standardwert} & \textbf{Key}   & \textbf{Besonderheiten} \\ \hline
		ID & int & NO &  & PRI & auto\_increment \\ \hline
		POI\_ID & int & NO &  & FOR & \\ \hline
		City & text & YES & NULL &  & \\ \hline
		Postalcode & varchar & YES & NULL &  & \\ \hline
		Streetname & text & YES & NULL &  & \\ \hline
		Housenumber & text & YES & NULL &  & \\ \hline
		start & int & YES & NULL &  & \\ \hline
		end & int & YES & NULL &  & \\ \hline
		creator & int & NO &  & FOR & \\ \hline
		creationdate & datetime & NO & current\_timestamp() &  & \\ \hline
		source & text & YES & NULL &  & \\ \hline
		points\_received & tinyint & NO & 0 &  & \\ \hline
		deleted & tinyint & NO & 0 &  & \\ \hline
	\end{tabular}
\end{table}
\subsubsection{Beschreibung}
\begin{table}[H]
	\begin{tabular}{|c|p{12cm}|}
		\hline
		\textbf{Feldname} & \textbf{Beschreibung} \\ \hline
		ID & Identifikator der historischen Adresse \\ \hline
		POI\_ID & Identifikator des zugehörigen Interessenpunktes \\ \hline
		City & Ortsname der Adresse \\ \hline
		Postalcode & Postleitzahl der Adresse \\ \hline
		Streetname & Straßenname der Adresse \\ \hline
		Housenumber & Hausnummer der Adresse \\ \hline
		start & Begin der Nutzung der Adresse \\ \hline
		end & Ende der Nutzung der Adresse \\ \hline
		creator & Nutzeridentifikator des Erstellers \\ \hline
		creationdate & Erstellungsdatum \\ \hline
		source & Quellenangabe \\ \hline
		points\_received & Status des Erhalts von Punkten (veraltet) \\ \hline
		deleted & Status der Löschung \\ \hline
	\end{tabular}
\end{table}
\subsubsection{Fremdschlüssel}
\begin{table}[H]
	\begin{tabular}{|c|p{12.5cm}|}
		\hline
		\textbf{Feldname} & \textbf{Fremd-Feld} \\ \hline
		POI\_ID & kino\_\_pois.poi\_id \\ \hline
		creator & kino\_\_user-login.id \\ \hline
	\end{tabular}
\end{table}
\subsection{kino\_\_history\_validate}
\subsubsection{Verwendung} Diese Tabelle wird verwendet um alle Validierungen der Geschichte eines Interessenpunktes zu speichern. Hierzu wird die ID des Interessenpunktes und die Nutzer-ID des validierenden Nutzers benötigt.
\subsubsection{Inhalt}
\begin{table}[H]
	\begin{tabular}{|c|c|c|c|c|p{3.5cm}|}
		\hline
		\textbf{Feldname} & \textbf{Datentyp} & \textbf{Null} & \textbf{Standardwert} & \textbf{Key}   & \textbf{Besonderheiten} \\ \hline
		id & int & NO &  & PRI & auto\_increment \\ \hline
		poi\_id & int & NO &  & FOR & \\ \hline
		uid & int & NO &  & FOR & \\ \hline
		value & int & NO &  &  & \\ \hline
		date & timestamp & NO & current\_timestamp() &  & \\ \hline
	\end{tabular}
\end{table}
\subsubsection{Beschreibung}
\begin{table}[H]
	\begin{tabular}{|c|p{12cm}|}
		\hline
		\textbf{Feldname} & \textbf{Beschreibung} \\ \hline
		id & Identifikator der Validierung \\ \hline
		poi\_id & Identifikator des zugehörigen Interessenpunktes \\ \hline
		uid & Identifikator des validierenden Nutzers \\ \hline
		value & Wertung der Validierung \\ \hline
		date & Zeitstempel der Validierung \\ \hline
	\end{tabular}
\end{table}
\subsubsection{Fremdschlüssel}
\begin{table}[H]
	\begin{tabular}{|c|p{12.5cm}|}
		\hline
		\textbf{Feldname} & \textbf{Fremd-Feld} \\ \hline
		poi\_id & kino\_\_pois.poi\_id \\ \hline
		uid & kino\_\_user-login.id \\ \hline
	\end{tabular}
\end{table}
\subsection{kino\_\_name\_validate}
\subsubsection{Verwendung} Diese Tabelle wird verwendet um alle Validierungen von Namen der Interessenpunkte zu speichern. Hierzu wird die ID des Namens und die Nutzer-ID des validierenden Nutzers benötigt.
\subsubsection{Inhalt}
\begin{table}[H]
	\begin{tabular}{|c|c|c|c|c|p{3.5cm}|}
		\hline
		\textbf{Feldname} & \textbf{Datentyp} & \textbf{Null} & \textbf{Standardwert} & \textbf{Key}   & \textbf{Besonderheiten} \\ \hline
		id & int & NO &  & PRI & auto\_increment \\ \hline
		name\_id & int & NO &  & FOR & \\ \hline
		uid & int & NO &  & FOR & \\ \hline
		value & int & NO &  &  & \\ \hline
		date & timestamp & NO & current\_timestamp() &  & \\ \hline
	\end{tabular}
\end{table}
\subsubsection{Beschreibung}
\begin{table}[H]
	\begin{tabular}{|c|p{12cm}|}
		\hline
		\textbf{Feldname} & \textbf{Beschreibung} \\ \hline
		id & Identifikator der Validierung \\ \hline
		name\_id & Identifikator des validierten Namens \\ \hline
		uid & Identifikator des validierenden Nutzers \\ \hline
		value & Wertung der Validierung \\ \hline
		date & Zeitstempel der Validierung \\ \hline
	\end{tabular}
\end{table}
\subsubsection{Fremdschlüssel}
\begin{table}[H]
	\begin{tabular}{|c|p{12.5cm}|}
		\hline
		\textbf{Feldname} & \textbf{Fremd-Feld} \\ \hline
		name\_id & kino\_\_names.ID \\ \hline
		uid & kino\_\_user-login.id \\ \hline
	\end{tabular}
\end{table}
\subsection{kino\_\_names}
\subsubsection{Verwendung} Diese Tabelle wird verwendet um Namen von Interessenpunkten zu speichern. Hierzu wird die ID des Interessenpunktes und die Nutzer-ID benötigt.
\subsubsection{Inhalt}
\begin{table}[H]
	\begin{tabular}{|c|c|c|c|c|p{3.5cm}|}
		\hline
		\textbf{Feldname} & \textbf{Datentyp} & \textbf{Null} & \textbf{Standardwert} & \textbf{Key}   & \textbf{Besonderheiten} \\ \hline
		ID & int & NO &  & PRI & auto\_increment \\ \hline
		POI\_ID & int & NO &  & FOR & \\ \hline
		Name & text & NO &  &  & \\ \hline
		Start & int & YES & NULL &  & \\ \hline
		End & int & YES & NULL &  & \\ \hline
		creator & int & NO &  & FOR & \\ \hline
		creationdate & datetime & NO & current\_timestamp() &  & \\ \hline
		source & text & YES & NULL &  & \\ \hline
		points\_received & tinyint & NO & 0 &  & \\ \hline
		deleted & tinyint & NO & 0 &  & \\ \hline
	\end{tabular}
\end{table}
\subsubsection{Beschreibung}
\begin{table}[H]
	\begin{tabular}{|c|p{12cm}|}
		\hline
		\textbf{Feldname} & \textbf{Beschreibung} \\ \hline
		ID & Identifikator des Namens \\ \hline
		POI\_ID & Identifikator des zugehörigen Interessenpunktes \\ \hline
		Name & Name \\ \hline
		Start & Start der Nutzung des Namens \\ \hline
		End & Ende der Nutzung des Namens \\ \hline
		creator & Nutzeridentifikator des Erstellers des Eintrags \\ \hline
		creationdate & Erstellungsdatum \\ \hline
		source & Quelle der Information \\ \hline
		points\_received & Status der Punkte für Namen (veraltet) \\ \hline
		deleted & Status der Löschung \\ \hline
	\end{tabular}
\end{table}
\subsubsection{Fremdschlüssel}
\begin{table}[H]
	\begin{tabular}{|c|p{12.5cm}|}
		\hline
		\textbf{Feldname} & \textbf{Fremd-Feld} \\ \hline
		POI\_ID & kino\_\_pois.poi\_id \\ \hline
		creator & kino\_\_user-login.id \\ \hline
	\end{tabular}
\end{table}
\subsection{kino\_\_operator\_validate}
\subsubsection{Verwendung} Diese Tabelle wird verwendet um alle Validierungen von Betreibern der Interessenpunkte zu speichern. Hierzu wird die ID des Betreibers und die Nutzer-ID des validierenden Nutzers benötigt.
\subsubsection{Inhalt}
\begin{table}[H]
	\begin{tabular}{|c|c|c|c|c|p{3.5cm}|}
		\hline
		\textbf{Feldname} & \textbf{Datentyp} & \textbf{Null} & \textbf{Standardwert} & \textbf{Key}   & \textbf{Besonderheiten} \\ \hline
		id & int & NO &  & PRI & auto\_increment \\ \hline
		operator\_id & int & NO &  & FOR & \\ \hline
		uid & int & NO &  & FOR & \\ \hline
		value & int & NO &  &  & \\ \hline
		date & timestamp & NO & current\_timestamp() &  & \\ \hline
	\end{tabular}
\end{table}
\subsubsection{Beschreibung}
\begin{table}[H]
	\begin{tabular}{|c|p{12cm}|}
		\hline
		\textbf{Feldname} & \textbf{Beschreibung} \\ \hline
		id & Identifikator der Validierung \\ \hline
		operator\_id & Identifikator des validierten Betreibers \\ \hline
		uid & Identifikator des validierenden Nutzers \\ \hline
		value & Wertung der Validierung \\ \hline
		date & Zeitstempel der Validierung \\ \hline
	\end{tabular}
\end{table}
\subsubsection{Fremdschlüssel}
\begin{table}[H]
	\begin{tabular}{|c|p{12.5cm}|}
		\hline
		\textbf{Feldname} & \textbf{Fremd-Feld} \\ \hline
		operator\_id & kino\_\_operators.ID \\ \hline
		uid & kino\_\_user-login.id \\ \hline
	\end{tabular}
\end{table}
\subsection{kino\_\_operators}
\subsubsection{Verwendung} Diese Tabelle wird verwendet um Betreiber von Interessenpunkten zu speichern. Hierzu wird die ID des Interessenpunktes und die Nutzer-ID benötigt.
\subsubsection{Inhalt}
\begin{table}[H]
	\begin{tabular}{|c|c|c|c|c|p{3.5cm}|}
		\hline
		\textbf{Feldname} & \textbf{Datentyp} & \textbf{Null} & \textbf{Standardwert} & \textbf{Key}   & \textbf{Besonderheiten} \\ \hline
		ID & int & NO &  & PRI & auto\_increment \\ \hline
		POI\_ID & int & NO &  & FOR & \\ \hline
		Operator & text & NO &  &  & \\ \hline
		start & int & YES & NULL &  & \\ \hline
		end & int & YES & NULL &  & \\ \hline
		creator & int & NO &  & FOR & \\ \hline
		creationdate & datetime & NO & current\_timestamp() &  & \\ \hline
		source & text & YES & NULL &  & \\ \hline
		points\_received & tinyint & NO & 0 &  & \\ \hline
		deleted & tinyint & NO & 0 &  & \\ \hline
	\end{tabular}
\end{table}
\subsubsection{Beschreibung}
\begin{table}[H]
	\begin{tabular}{|c|p{12cm}|}
		\hline
		\textbf{Feldname} & \textbf{Beschreibung} \\ \hline
		ID & Identifikator des Betreibers \\ \hline
		POI\_ID & Identifikator des zugehörigen Interessenpunktes \\ \hline
		Operator & Name des Betreibers \\ \hline
		start & Start der Nutzung durch Betreiber \\ \hline
		end & Ende der Nutzung durch Betreiber \\ \hline
		creator & Nutzeridentifikator des Erstellers des Eintrags \\ \hline
		creationdate & Erstellungsdatum \\ \hline
		source & Quelle der Information \\ \hline
		points\_received & Status der Punktevergabe (veraltet) \\ \hline
		deleted & Status der Löschung \\ \hline
	\end{tabular}
\end{table}
\subsubsection{Fremdschlüssel}
\begin{table}[H]
	\begin{tabular}{|c|p{12.5cm}|}
		\hline
		\textbf{Feldname} & \textbf{Fremd-Feld} \\ \hline
		POI\_ID & kino\_\_pois.poi\_id \\ \hline
		creator & kino\_\_user-login.id \\ \hline
	\end{tabular}
\end{table}
\subsection{kino\_\_poi\_pictures}
\subsubsection{Verwendung} Diese Tabelle wird verwendet um Bilder Interessenpunkten zu zuordnen. Hierzu wird die ID des Interessenpunktes, der Identifikator des Bildes und die Nutzer-ID benötigt.
\subsubsection{Inhalt}
\begin{table}[H]
	\begin{tabular}{|c|c|c|c|c|p{3.5cm}|}
		\hline
		\textbf{Feldname} & \textbf{Datentyp} & \textbf{Null} & \textbf{Standardwert} & \textbf{Key}   & \textbf{Besonderheiten} \\ \hline
		id & int & NO &  & PRI & auto\_increment \\ \hline
		picture\_id & varchar & NO &  & FOR & \\ \hline
		poi\_id & int & NO &  & FOR & \\ \hline
		creator & int & NO &  & FOR & \\ \hline
		creationdate & datetime & NO & current\_timestamp() &  & \\ \hline
		deleted & tinyint & NO & 0 &  & \\ \hline
		poiDel & tinyint & NO & 0 &  & \\ \hline
		picDel & tinyint & NO & 0 &  & \\ \hline
	\end{tabular}
\end{table}
\subsubsection{Beschreibung}
\begin{table}[H]
	\begin{tabular}{|c|p{12cm}|}
		\hline
		\textbf{Feldname} & \textbf{Beschreibung} \\ \hline
		id & Identifikator des Links zwischen Interessenpunkt und Bild \\ \hline
		picture\_id & alphanumerischer Identifikator des Bildes \\ \hline
		poi\_id & Identifikator des Interessenpunktes \\ \hline
		creator & Nutzeridentifikator des Erstellers des Links \\ \hline
		creationdate & Erstellungs- beziehungsweise Änderungsdatum \\ \hline
		deleted & Status der Löschung des Links \\ \hline
		poiDel & Status der Restriktion durch Status der Löschung des Interessenpunkt \\ \hline
		picDel & Status der Restriktion durch Status der Löschung des Bildes \\ \hline
	\end{tabular}
\end{table}
\subsubsection{Fremdschlüssel}
\begin{table}[H]
	\begin{tabular}{|c|p{12.5cm}|}
		\hline
		\textbf{Feldname} & \textbf{Fremd-Feld} \\ \hline
		poi\_id & kino\_\_pois.poi\_id \\ \hline
		creator & kino\_\_user-login.id \\ \hline
		picture\_id & Ist ein Fremdschlüssel in der {\glqq COSP\grqq}-Datenbank \\ \hline
	\end{tabular}
\end{table}
\subsection{kino\_\_poi\_pictures\_validate}
\subsubsection{Verwendung} Diese Tabelle wird verwendet um alle Validierungen für einen Link zwischen einem Interessenpunkt und einem Bild zu speichern. Hierzu wird die ID des Links und die Nutzer-ID des validierenden Nutzers benötigt.
\subsubsection{Inhalt}
\begin{table}[H]
	\begin{tabular}{|c|c|c|c|c|p{3.5cm}|}
		\hline
		\textbf{Feldname} & \textbf{Datentyp} & \textbf{Null} & \textbf{Standardwert} & \textbf{Key}   & \textbf{Besonderheiten} \\ \hline
		id & int & NO &  & PRI & auto\_increment \\ \hline
		link-id-poi-pic & int & NO &  & FOR & \\ \hline
		value & int & NO &  &  & \\ \hline
		creator & int & NO &  & FOR & \\ \hline
		creationdate & datetime & NO & current\_timestamp() &  & \\ \hline
	\end{tabular}
\end{table}
\subsubsection{Beschreibung}
\begin{table}[H]
	\begin{tabular}{|c|p{12cm}|}
		\hline
		\textbf{Feldname} & \textbf{Beschreibung} \\ \hline
		id & Identifikator der Validierung \\ \hline
		link-id-poi-pic & Identifikator der validierten Links zwischen einem Interessenpunkt und einem Bild \\ \hline
		uid & Identifikator des validierenden Nutzers \\ \hline
		value & Wertung der Validierung \\ \hline
		date & Zeitstempel der Validierung \\ \hline
	\end{tabular}
\end{table}
\subsubsection{Fremdschlüssel}
\begin{table}[H]
	\begin{tabular}{|c|p{12.5cm}|}
		\hline
		\textbf{Feldname} & \textbf{Fremd-Feld} \\ \hline
		link-id-poi-pic & kino\_\_poi\_pictures.id \\ \hline
		creator & kino\_\_user-login.id \\ \hline
	\end{tabular}
\end{table}
\subsection{kino\_\_poi\_sources}
\subsubsection{Verwendung} Diese Tabelle wird verwendet um Geschichten Interessenpunkten zu zuordnen. Hierzu wird die ID des Interessenpunktes, der Identifikator der Geschichte und die Nutzer-ID benötigt.
\subsubsection{Inhalt}
\begin{table}[H]
	\begin{tabular}{|c|c|c|c|c|p{3.5cm}|}
		\hline
		\textbf{Feldname} & \textbf{Datentyp} & \textbf{Null} & \textbf{Standardwert} & \textbf{Key}   & \textbf{Besonderheiten} \\ \hline
		id & int & NO &  & PRI & auto\_increment \\ \hline
		poiid & int & NO &  & FOR & \\ \hline
		source & mediumtext & NO &  &  & \\ \hline
		typeid & int & NO &  & FOR & \\ \hline
		relationid & int & NO &  & FOR & \\ \hline
		creator & int & NO &  & FOR & \\ \hline
		creationdate & timestamp & NO & current\_timestamp() &  & \\ \hline
	\end{tabular}
\end{table}
\subsubsection{Beschreibung}
\begin{table}[H]
	\begin{tabular}{|c|p{12cm}|}
		\hline
		\textbf{Feldname} & \textbf{Beschreibung} \\ \hline
		id & Identifikator der Quelle \\ \hline
		poiid & Identifikator des zugehörigen Interessenpunktes\\ \hline
		source & Textuelle Beschreibung der Quelle \\ \hline
		typeid & Identifikator des Typs der Quelle \\ \hline
		relationid & Identifikator der Tabelle mit Namen des Bezugs der Quelle \\ \hline
		creator & Nutzeridentifikator des erstellenden Nutzers \\ \hline
		creationdate & Erstellungs- beziehungsweise Änderungsdatum \\ \hline
	\end{tabular}
\end{table}
\subsubsection{Fremdschlüssel}
\begin{table}[H]
	\begin{tabular}{|c|p{12.5cm}|}
		\hline
		\textbf{Feldname} & \textbf{Fremd-Feld} \\ \hline
		poiid & kino\_\_pois.poi\_id \\ \hline
		creator & kino\_\_user-login.id \\ \hline
		typeid & kino\_\_source\_type.id \\ \hline
		relationid & kino\_\_source\_relation.id \\ \hline
	\end{tabular}
\end{table}
\subsection{kino\_\_poi\_story}
\subsubsection{Verwendung} Diese Tabelle wird verwendet um Geschichten Interessenpunkten zu zuordnen. Hierzu wird die ID des Interessenpunktes, der Identifikator der Geschichte und die Nutzer-ID benötigt.
\subsubsection{Inhalt}
\begin{table}[H]
	\begin{tabular}{|c|c|c|c|c|p{3.5cm}|}
		\hline
		\textbf{Feldname} & \textbf{Datentyp} & \textbf{Null} & \textbf{Standardwert} & \textbf{Key}   & \textbf{Besonderheiten} \\ \hline
		id & int & NO &  & PRI & auto\_increment \\ \hline
		poi\_id & int & NO &  & FOR & \\ \hline
		story\_token & varchar & NO &  &  & \\ \hline
		creator & int & NO &  & FOR & \\ \hline
		creationdate & timestamp & NO & current\_timestamp() &  & \\ \hline
		deleted & tinyint & NO & 0 &  & \\ \hline
		poiDel & tinyint & NO & 0 &  & \\ \hline
		storyDel & tinyint & NO & 0 &  & \\ \hline
	\end{tabular}
\end{table}
\subsubsection{Beschreibung}
\begin{table}[H]
	\begin{tabular}{|c|p{12cm}|}
		\hline
		\textbf{Feldname} & \textbf{Beschreibung} \\ \hline
		id & Identifikator des Links zwischen Interessenpunktes und Geschichte \\ \hline
		poi\_id & Identifikator des zugehörigen Interessenpunktes\\ \hline
		story\_token & alphanumerischer Identifikator der zugehörigen Geschichte \\ \hline
		creator & Nutzeridentifikator des erstellenden Nutzers \\ \hline
		creationdate & Erstellungs- beziehungsweise Änderungsdatum \\ \hline
		deleted & Status der Löschung des Links zwischen einem Interessenpunkt und einer Geschichte \\ \hline
		poiDel & Status der Restriktion durch Status der Löschung des Interessenpunkt \\ \hline
		storyDel & Status der Restriktion durch Status der Löschung der Geschichte \\ \hline
	\end{tabular}
\end{table}
\subsubsection{Fremdschlüssel}
\begin{table}[H]
	\begin{tabular}{|c|p{12.5cm}|}
		\hline
		\textbf{Feldname} & \textbf{Fremd-Feld} \\ \hline
		poi\_id & kino\_\_pois.poi\_id \\ \hline
		creator & kino\_\_user-login.id \\ \hline
		story\_token & Ist ein Fremdschlüssel in der {\glqq COSP\grqq}-Datenbank \\ \hline
	\end{tabular}
\end{table}
\subsection{kino\_\_poi\_story\_validate}
\subsubsection{Verwendung} Diese Tabelle wird verwendet um alle Validierungen für einen Link zwischen einem Interessenpunkt und einer Geschichte zu speichern. Hierzu wird die ID des Links und die Nutzer-ID des validierenden
\subsubsection{Inhalt}
\begin{table}[H]
	\begin{tabular}{|c|c|c|c|c|p{3.5cm}|}
		\hline
		\textbf{Feldname} & \textbf{Datentyp} & \textbf{Null} & \textbf{Standardwert} & \textbf{Key}   & \textbf{Besonderheiten} \\ \hline
		id & int & NO &  & PRI & auto\_increment \\ \hline
		story\_poi\_link\_id & int & NO &  & FOR & \\ \hline
		uid & int & NO &  & FOR & \\ \hline
		value & int & NO &  &  & \\ \hline
		date & timestamp & NO & current\_timestamp() &  & \\ \hline
	\end{tabular}
\end{table}
\subsubsection{Beschreibung}
\begin{table}[H]
	\begin{tabular}{|c|p{12cm}|}
		\hline
		\textbf{Feldname} & \textbf{Beschreibung} \\ \hline
		id & Identifikator der Validierung \\ \hline
		story\_poi\_link\_id & Identifikator des validierten Links zwischen einer Geschichte und einem Interessenpunkt \\ \hline
		uid & Identifikator des validierenden Nutzers \\ \hline
		value & Wertung der Validierung \\ \hline
		date & Zeitstempel der Validierung \\ \hline
	\end{tabular}
\end{table}
\subsubsection{Fremdschlüssel}
\begin{table}[H]
	\begin{tabular}{|c|p{12.5cm}|}
		\hline
		\textbf{Feldname} & \textbf{Fremd-Feld} \\ \hline
		story\_poi\_link\_id & kino\_\_poi\_story.id \\ \hline
		uid & kino\_\_user-login.id \\ \hline
	\end{tabular}
\end{table}
\subsection{kino\_\_pois}
\subsubsection{Verwendung} Diese Tabelle wird verwendet um alle grundlegenden Daten eines Interessenpunktes zu speichern. Hierzu wird die Nutzer-ID des anlegenden benötigt.
\subsubsection{Inhalt}
\begin{longtable}[H]{|c|c|c|c|c|p{2.9cm}|}
		\hline
		\textbf{Feldname} & \textbf{Datentyp} & \textbf{Null} & \textbf{Standardwert} & \textbf{Key}   & \textbf{Besonderheiten} \\ \hline
		poi\_id & int & NO &  & PRI & auto\_increment \\ \hline
		name & varchar & NO &  &  & \\ \hline
		lng & double & NO &  &  & \\ \hline
		lat & double & NO &  &  & \\ \hline
		City & text & YES & NULL &  & \\ \hline
		Postalcode & varchar & YES & NULL &  & \\ \hline
		Streetname & text & YES & NULL &  & \\ \hline
		Housenumber & text & YES & NULL &  & \\ \hline
		picture & varchar & YES & NULL &  & \\ \hline
		start & int & YES & NULL &  & \\ \hline
		end & int & YES & NULL &  & \\ \hline
		category & int & NO & 0 &  & \\ \hline
		history & text & YES & NULL &  & \\ \hline
		type & int & YES & NULL & FOR & \\ \hline
		user\_id & int & NO &  & FOR & \\ \hline
		creationDate & datetime & NO & current\_timestamp() &  & \\ \hline
		creator\_timespan & int & YES & NULL & FOR & \\ \hline
		creationdate\_timespan & datetime & NO & current\_timestamp() &  & \\ \hline
		creator\_currentAddress & int & YES & NULL & FOR & \\ \hline
		creationdate\_currentAddress & datetime & NO & current\_timestamp() &  & \\ \hline
		creator\_history & int & YES & NULL & FOR & \\ \hline
		creatoiondate\_history & datetime & NO & current\_timestamp() &  & \\ \hline
		creator\_type & int & YES & NULL & FOR & \\ \hline
		creationdate\_type & datetime & NO & current\_timestamp() &  & \\ \hline
		deleted & tinyint & NO & 0 &  & \\ \hline
		deletedPic & tinyint & NO & 0 &  & \\ \hline
\end{longtable}
\subsubsection{Beschreibung}
\begin{longtable}[H]{|c|p{11cm}|}
	\hline
	\textbf{Feldname} & \textbf{Beschreibung} \\ \hline
	poi\_id & Identifikator des Interessenpunktes \\ \hline
	name & primärer Name des Interessenpunktes \\ \hline
	lng & Längengrad des Interessenpunktes \\ \hline
	lat & Breitengrad des Interessenpunktes \\ \hline
	City & Ortsname der aktuellen Adresse \\ \hline
	Postalcode & Postleitzahl der aktuellen Adresse \\ \hline
	Streetname & Straßenname der aktuellen Adresse \\ \hline
	Housenumber & Hausnummer der aktuellen Adresse \\ \hline
	picture & alphanumerischer Identifikator des zugewiesenen Hauptbildes \\ \hline
	start & Start der Nutzungsdauer des Interessenpunktes \\ \hline
	end & Ende der Nutzungsdauer des Interessenpunktes \\ \hline
	category & numerische Kategorie des Interessenpunktes \\ \hline
	history & Geschichte des Interessenpunktes \\ \hline
	type & Identifikator des Typ des Interessenpunktes \\ \hline
	user\_id & Nutzeridentifikator des erstellenden Nutzers \\ \hline
	creationDate & Datum der Erstellung der Stammdaten des Interessenpunktes \\ \hline
	creator\_timespan & Nutzeridentifikator des Erstellers oder Änderers der Nutzungszeitspanne des Interessenpunktes \\ \hline
	creationdate\_timespan & Zeitstempel des Erstellens oder des Änderns der Nutzungszeitspanne des Interessenpunktes \\ \hline
	creator\_currentAddress & Nutzeridentifikator des Erstellers oder Änderers der aktuellen Adresse des Interessenpunktes \\ \hline
	creationdate\_currentAddress & Zeitstempel des Erstellens oder des Änderns der aktuellen Adresse des Interessenpunktes \\ \hline
	creator\_history & Nutzeridentifikator des Erstellers oder Änderers der Geschichte des Interessenpunktes \\ \hline
	creatoiondate\_history & Zeitstempel des Erstellens oder des Änderns der Geschichte des Interessenpunktes \\ \hline
	creator\_type & Nutzeridentifikator des Erstellers oder Änderers des Typs des Interessenpunktes \\ \hline
	creationdate\_type & Zeitstempel des Erstellens oder des Änderns des Typs des Interessenpunktes \\ \hline
	deleted & Status der Löschung \\ \hline
	deletedPic & Status der Löschung des Hauptbildes \\ \hline
\end{longtable}
\subsubsection{Fremdschlüssel}
\begin{table}[H]
	\begin{tabular}{|c|p{12.5cm}|}
		\hline
		\textbf{Feldname} & \textbf{Fremd-Feld} \\ \hline
		user\_id & kino\_\_user-login.id \\ \hline
		creator\_timespan & kino\_\_user-login.id \\ \hline
		creator\_currentAddress & kino\_\_user-login.id \\ \hline
		creator\_history & kino\_\_user-login.id \\ \hline
		creator\_type & kino\_\_user-login.id \\ \hline
		picture & Ist ein Fremdschlüssel in der {\glqq COSP\grqq}-Datenbank \\ \hline
	\end{tabular}
\end{table}
\subsection{kino\_\_seats}
\subsubsection{Verwendung} Diese Tabelle wird verwendet um die Sitzplatzanzahl von Interessenpunkten zu speichern. Hierzu wird die ID des Interessenpunktes benötigt.
\subsubsection{Inhalt}
\begin{table}[H]
	\begin{tabular}{|c|c|c|c|c|p{3.5cm}|}
		\hline
		\textbf{Feldname} & \textbf{Datentyp} & \textbf{Null} & \textbf{Standardwert} & \textbf{Key}   & \textbf{Besonderheiten} \\ \hline
		ID & int & NO &  & PRI & auto\_increment \\ \hline
 		POI\_ID & int & NO &  & FOR & \\ \hline
		seats & text & NO &  &  & \\ \hline
		Start & int & YES & NULL &  & \\ \hline
		End & int & YES & NULL &  & \\ \hline
		creator & int & NO &  & FOR & \\ \hline
		creationdate & datetime & NO & current\_timestamp() &  & \\ \hline
		source & text & YES & NULL &  & \\ \hline
		points\_received & tinyint & NO & 0 &  & \\ \hline
		deleted & tinyint & NO & 0 &  & \\ \hline
	\end{tabular}
\end{table}
\subsubsection{Beschreibung}
\begin{table}[H]
	\begin{tabular}{|c|p{12cm}|}
		\hline
		\textbf{Feldname} & \textbf{Beschreibung} \\ \hline
		ID & Identifikator des Sitzanzahl \\ \hline
		POI\_ID & Identifikator des zugehörigen Interessenpunktes \\ \hline
		seats & Sitzplatzanzahl \\ \hline
		Start & Start der Nutzung mit Sitzplatzanzahl \\ \hline
		End & Ende der Nutzung mit Sitzplatzanzahl \\ \hline
		creator & Nutzeridentifikator des erstellenden Nutzers \\ \hline
		creationdate & Zeitstempel der Erstellung beziehungsweise Änderung\\ \hline
		source & Quelle der Information \\ \hline
		points\_received & Status der Punktevergabe (veraltet) \\ \hline
		deleted & Status der Löschung \\ \hline
	\end{tabular}
\end{table}
\subsubsection{Fremdschlüssel}
\begin{table}[H]
	\begin{tabular}{|c|p{12.5cm}|}
		\hline
		\textbf{Feldname} & \textbf{Fremd-Feld} \\ \hline
		POI\_ID & kino\_\_poi\_story.id \\ \hline
		creator & kino\_\_user-login.id \\ \hline
	\end{tabular}
\end{table}
\subsection{kino\_\_seats\_validate}
\subsubsection{Verwendung} Diese Tabelle wird verwendet um alle Validierungen von Sitzplatzanzahlen zu speichern. Hierzu wird die ID der Sitzplatzanzahl und die Nutzer-ID des validierenden Nutzers benötigt.
\subsubsection{Inhalt}
\begin{table}[H]
	\begin{tabular}{|c|c|c|c|c|p{3.5cm}|}
		\hline
		\textbf{Feldname} & \textbf{Datentyp} & \textbf{Null} & \textbf{Standardwert} & \textbf{Key}   & \textbf{Besonderheiten} \\ \hline
		id & int & NO &  & PRI & auto\_increment \\ \hline
		seats\_id & int & NO &  & FOR & \\ \hline
		uid & int & NO &  & FOR & \\ \hline
		value & int & NO &  &  & \\ \hline
		date & timestamp & NO & current\_timestamp() &  & \\ \hline
	\end{tabular}
\end{table}
\subsubsection{Beschreibung}
\begin{table}[H]
	\begin{tabular}{|c|p{12cm}|}
		\hline
		\textbf{Feldname} & \textbf{Beschreibung} \\ \hline
		id & Identifikator der Validierung \\ \hline
		seats\_id & Identifikator der validierten Sitzplatzanzahl \\ \hline
		uid & Identifikator des validierenden Nutzers \\ \hline
		value & Wertung der Validierung \\ \hline
		date & Zeitstempel der Validierung \\ \hline
	\end{tabular}
\end{table}
\subsubsection{Fremdschlüssel}
\begin{table}[H]
	\begin{tabular}{|c|p{12.5cm}|}
		\hline
		\textbf{Feldname} & \textbf{Fremd-Feld} \\ \hline
		uid & kino\_\_user-login.id \\ \hline
		seats\_id & kino\_\_seats.ID \\ \hline
	\end{tabular}
\end{table}
\subsection{kino\_\_source\_relation}
\subsubsection{Verwendung} Diese Tabelle wird verwendet um den Bezug von Quellen zu speichern.
\subsubsection{Inhalt}
\begin{table}[H]
	\begin{tabular}{|c|c|c|c|c|p{3.5cm}|}
		\hline
		\textbf{Feldname} & \textbf{Datentyp} & \textbf{Null} & \textbf{Standardwert} & \textbf{Key}   & \textbf{Besonderheiten} \\ \hline
		id & int & NO &  & PRI & auto\_increment \\ \hline
		name & mediumtext & NO &  &  & \\ \hline
	\end{tabular}
\end{table}
\subsubsection{Beschreibung}
\begin{table}[H]
	\begin{tabular}{|c|p{12cm}|}
		\hline
		\textbf{Feldname} & \textbf{Beschreibung} \\ \hline
		id & Identifikator des Typs einer Quelle \\ \hline
		name & Name des Bezugs einer Quelle \\ \hline
	\end{tabular}
\end{table}
\subsection{kino\_\_source\_type}
\subsubsection{Verwendung} Diese Tabelle wird verwendet um Typen von Quellen zu speichern.
\subsubsection{Inhalt}
\begin{table}[H]
	\begin{tabular}{|c|c|c|c|c|p{3.5cm}|}
		\hline
		\textbf{Feldname} & \textbf{Datentyp} & \textbf{Null} & \textbf{Standardwert} & \textbf{Key}   & \textbf{Besonderheiten} \\ \hline
		id & int & NO &  & PRI & auto\_increment \\ \hline
		name & mediumtext & NO &  &  & \\ \hline
	\end{tabular}
\end{table}
\subsubsection{Beschreibung}
\begin{table}[H]
	\begin{tabular}{|c|p{12cm}|}
		\hline
		\textbf{Feldname} & \textbf{Beschreibung} \\ \hline
		id & Identifikator des Typs einer Quelle \\ \hline
		name & Name des Typs einer Quelle \\ \hline
	\end{tabular}
\end{table}
\subsection{kino\_\_user-login}
\subsubsection{Verwendung} Diese Tabelle wird verwendet um alle Daten zu Nutzern zu speichern.
\subsubsection{Inhalt}
\begin{table}[H]
	\begin{tabular}{|c|c|c|c|c|p{3.5cm}|}
		\hline
		\textbf{Feldname} & \textbf{Datentyp} & \textbf{Null} & \textbf{Standardwert} & \textbf{Key}   & \textbf{Besonderheiten} \\ \hline
		id & int & NO &  & PRI & auto\_increment \\ \hline
		name & varchar & NO &  & UNI & \\ \hline
		password & longtext & NO &  &  & \\ \hline
		firstname & mediumtext & YES & NULL &  & \\ \hline
		lastname & mediumtext & YES & NULL &  & \\ \hline
		email & mediumtext & YES & NULL &  & \\ \hline
		deaktivate & tinyint & NO & 0 &  & \\ \hline
	\end{tabular}
\end{table}
\subsubsection{Beschreibung}
\begin{table}[H]
	\begin{tabular}{|c|p{12cm}|}
		\hline
		\textbf{Feldname} & \textbf{Beschreibung} \\ \hline
		id & Identifikator des Nutzers \\ \hline
		name & Nutzername beziehungsweise Login-Name \\ \hline
		password & Hash des Passwortes des Nutzers \\ \hline
		firstname & Vorname des Nutzers (optional) \\ \hline
		lastname & Nachname des Nutzers (optional) \\ \hline
		email & E-Mailadresse des Nutzers \\ \hline
		deaktivate & Status der Deaktivierung des Nutzers \\ \hline
	\end{tabular}
\end{table}
\subsubsection{Fremdschlüssel}
In dieser Tabelle sind keine Fremdschlüssel vorhanden.
\subsection{kino\_\_validate}
\subsubsection{Verwendung}  Diese Tabelle wird verwendet um alle Validierungen von Stammdaten eines Interessenpunktes zu speichern. Hierzu wird die ID des Interessenpunktes und die Nutzer-ID des validierenden Nutzers benötigt.
\subsubsection{Inhalt}
\begin{table}[H]
	\begin{tabular}{|c|c|c|c|c|p{3.5cm}|}
		\hline
		\textbf{Feldname} & \textbf{Datentyp} & \textbf{Null} & \textbf{Standardwert} & \textbf{Key}   & \textbf{Besonderheiten} \\ \hline
		id & int & NO &  & PRI & auto\_increment \\ \hline
		poi\_id & int & NO &  & FOR & \\ \hline
		uid & int & NO &  & FOR & \\ \hline
		value & int & NO &  &  & \\ \hline
		date & timestamp & NO & current\_timestamp() &  & \\ \hline
	\end{tabular}
\end{table}
\subsubsection{Beschreibung}
\begin{table}[H]
	\begin{tabular}{|c|p{12cm}|}
		\hline
		\textbf{Feldname} & \textbf{Beschreibung} \\ \hline
		id & Identifikator der Validierung \\ \hline
		poi\_id & Identifikator des validierten Interessenpunktes \\ \hline
		uid & Identifikator des validierenden Nutzers \\ \hline
		value & Wertung der Validierung \\ \hline
		date & Zeitstempel der Validierung \\ \hline
	\end{tabular}
\end{table}
\subsubsection{Fremdschlüssel}
\begin{table}[H]
	\begin{tabular}{|c|p{12.5cm}|}
		\hline
		\textbf{Feldname} & \textbf{Fremd-Feld} \\ \hline
		poi\_id & kino\_\_poi\_story.id \\ \hline
		uid & kino\_\_user-login.id \\ \hline
	\end{tabular}
\end{table}
\subsection{kino\_\_visitors}
\subsubsection{Verwendung}  Diese Tabelle wird verwendet um alle Zugriffe auf die Website für statistische Zwecke zu speichern.
\subsubsection{Inhalt}
\begin{table}[H]
	\begin{tabular}{|c|c|c|c|c|p{3.5cm}|}
		\hline
		\textbf{Feldname} & \textbf{Datentyp} & \textbf{Null} & \textbf{Standardwert} & \textbf{Key}   & \textbf{Besonderheiten} \\ \hline
		id & int & NO &  & PRI & auto\_increment \\ \hline
		ip & varchar & NO &  &  & \\ \hline
		date & timestamp & NO & current\_timestamp() &  & \\ \hline
		type & varchar & NO &  &  & \\ \hline
	\end{tabular}
\end{table}
\subsubsection{Beschreibung}
\begin{table}[H]
	\begin{tabular}{|c|p{12cm}|}
		\hline
		\textbf{Feldname} & \textbf{Beschreibung} \\ \hline
		id & Identifikator des Eintrags \\ \hline
		ip & IP-Adresse des Aufrufers \\ \hline
		date & Zeitstempel des Aufrufens \\ \hline
		type & Nutzertyp des Aufrufers (Nutzer oder Gast) \\ \hline
	\end{tabular}
\end{table}
\subsubsection{Fremdschlüssel}
In dieser Tabelle sind keine Fremdschlüssel vorhanden.
\chapter{Datei-Übersicht}
\section{Auflistung}
\begin{longtable}[H]{|c|p{10cm}|}
	\hline
	\textbf{Dateiname} & \textbf{Beschreibung} \\ \hline
	api-functions.php & Enthält alle API-Funktionen der Frontend-API \\ \hline
	authSystem.php & Enthält alle Funktionen zur Authentifikation eines Nutzers \\ \hline
	config-sample.php & Beispielkonfigurationsdatei \\ \hline
	config.php & Konfigurationsdatei (gleicher Aufbau wie config-sample.php) \\ \hline
	basic-db.php & Enthält grundlegende Funktionen für Datenbankzugriff \\ \hline
	cinema-type-db.php & Enthält Funktionen für Zugriff auf Typdaten \\ \hline
	cinemas-db.php & Enthält Funktionen für Zugriff auf Tabelle mit Kinosaalanzahlen \\ \hline
	hist-adr-db.php & Enthält Funktionen für Zugriff auf Tabelle mit historischen Adressen \\ \hline
	inc-db-sub.php & Bindet Dateien für Datenbankzugriffe in der korrekten Reihenfolge ein für nicht Hauptverzeichnisdateien \\ \hline
	inc-db.php & Bindet Dateien für Datenbankzugriffe in der korrekten Reihenfolge ein \\ \hline
	logging.php & Enthält Funktionen für Zugriff auf Tabelle mit Nutzungsstatistiken \\ \hline
	names-db.php & Enthält Funktionen für Zugriff auf Tabelle mit Namen \\ \hline
	operators-db.php & Enthält Funktionen für Zugriff auf Tabelle mit Betreibern \\ \hline
	poi-comment.php & Enthält Funktionen für Zugriff auf Tabelle mit Kommentaren \\ \hline
	poi-db.php & Enthält Funktionen für Zugriff auf Tabelle mit Interessenpunkten \\ \hline
	poi-picture-db.php & Enthält Funktionen für Zugriff auf Tabelle mit Verlinkungen zwischen Interessenpunkten und Bildern \\ \hline
	poi-source-db.php & Enthält Funktionen für Zugriff auf Tabelle mit Quellen zu Interessenpunkten \\ \hline
	poi-story-db.php & Enthält Funktionen für Zugriff auf Tabelle mit Verlinkungen zwischen Geschichten und Interessenpunkten \\ \hline
	poi-val.php & Enthält Funktionen für Zugriff auf Tabelle mit Validierungen von Interessenpunkten \\ \hline
	source-relation-db.php & Enthält Funktionen für Zugriff auf Tabelle mit Beziehungen von Quellen \\ \hline
	source-type-db.php & Enthält Funktionen für Zugriff auf Tabelle mit Typen von Quellen \\ \hline
	seats-db.php & Enthält Funktionen für Zugriff auf Tabelle mit Sitzplatzanzahlen \\ \hline
	statistics-basic-dbfunctions.php & Enthält grundlegende Funktionen für statistischen Datenbankzugriff \\ \hline
	statistics-comments-db.php & Enthält Funktionen für statistischen Zugriff auf Tabelle mit Kommentaren \\ \hline
	statistics-poi-db.php & Enthält Funktionen für statistischen Zugriff auf Tabelle mit Interessenpunkten \\ \hline
	user.php & Enthält Funktionen für Zugriff auf Tabelle mit Nutzerinformationen \\ \hline
	validate-cinemas.php & Enthält Funktionen für Zugriff auf Tabelle mit Validierungsdaten für Saalanzahlen \\ \hline
	validate-curr-adr.php & Enthält Funktionen für Zugriff auf Tabelle mit Validierungsdaten für aktuelle Adressen \\ \hline
	validate-hist-adr.php & Enthält Funktionen für Zugriff auf Tabelle mit Validierungsdaten für historische Adressen\\ \hline
	validate-hist.php & Enthält Funktionen für Zugriff auf Tabelle mit Validierungsdaten für Geschichte eines Interessenpunktes \\ \hline
	validate-name.php & Enthält Funktionen für Zugriff auf Tabelle mit Validierungsdaten für Namen \\ \hline
	validate-operator.php & Enthält Funktionen für Zugriff auf Tabelle mit Validierungsdaten für Betreiber \\ \hline
	validate-poi-picture.php & Enthält Funktionen für Zugriff auf Tabelle mit Validierungsdaten für Links zwischen Interessenpunkten und Bildern \\ \hline
	validate-poi-story.php & Enthält Funktionen für Zugriff auf Tabelle mit Validierungsdaten für Links zwischen Interessenpunkten und Geschichten \\ \hline
	validate-seats.php & Enthält Funktionen für Zugriff auf Tabelle mit Validierungsdaten für Sitzplatzanzahlen \\ \hline
	validate-timespan.php & Enthält Funktionen für Zugriff auf Tabelle mit Validierungsdaten für Nutzungszeitspanne eines Interessenpunktes \\ \hline
	validate-type.php & Enthält Funktionen für Zugriff auf Tabelle mit Validierungsdaten für den Typ eines Interessenpunktes \\ \hline
	de.php & Enthält Sprachbezogene Bezeichnungen für Deutsch \\ \hline
	deletions.php & Enthält Funktionen zum Löschen von Daten \\ \hline
	en.php & Enthält Sprachbezogene Bezeichnungen für Englisch (nicht genutzt) \\ \hline
	functionLib.php & Enthält grundlegende Funktionen die an diversen Stellen genutzt werden \\ \hline
	inc-sub.php & Bindet alle benötigten Dateien in korrekter Reihenfolge ein für nicht Hauptverzeichnisdateien \\ \hline
	inc.php & Bindet alle benötigten Dateien in korrekter Reihenfolge ein \\ \hline
	rapi-functions.php & Enthält Funktionen für Reverse-API \\ \hline
	session.php & Enthält Funktionen und ini-Settings für Sessions \\ \hline
	settings.php & Enthält ini-Settings \\ \hline
	statistic-calc.php & Enthält Funktionen zum berechnen und vorbereiten statistischer Grafiken \\ \hline
	contact.php & Seite mit Mailformular \\ \hline
	editPoi.php & Seite zum Editieren eines Interessenpunktes \\ \hline
	Formular/api.php & Funktionsendpunkt der Frontend-API \\ \hline
	Formular/rapi.php & Funktionsendpunkt der Reverse-API \\ \hline
	hub.php & Dynamisch generierte Übersichtsseite \\ \hline
	impressum.php & Seite mit Impressum \\ \hline
	index.php & Startseite \\ \hline
	ListMaterial.php & Seite zum Auflisten von Bildern \\ \hline
	logoutpage.php & Seite zum Ausloggen des Nutzers \\ \hline
	map.php & Seite mit Kartendarstellung \\ \hline
	MaterialUpload.php & Seite zum Hochladen von Bildern \\ \hline
	poimgmt.php & Seite zum Verwalten von Interessenpunkten \\ \hline
	privacy-policy.php & Seite mit Datenschutzerklärung \\ \hline
	ranklist.php & Seite mit sortierter Auflistung der Top-Nutzer \\ \hline
	registration.php & Seite zur Selbstregistration \\ \hline
	statistics.php & Seite mit Statistiken \\ \hline
	StoryUpload.php & Seite zum Hochladen und Anzeigen einer Geschichte \\ \hline
	Control.Geocoder.js & Funktionen zum Suchen einer Adresse (teilweise geändert zum Original) \\ \hline
	editmap.js & Enthält Funktionen für Änderungsseite eines Interessenpunktes \\ \hline
	kinoMainLib.js & Funktionsbibliothek für diverse JavaScripts \\ \hline
	loadCaptcha.js & Enthält Funktionen zum laden eines Captchas \\ \hline
	loadCookie.js & Enthält Funktionen zum Akzeptieren von Cookies \\ \hline
	mapfunctions.js & Enthält Funktionen, welche nur in der Kartenansicht benötigt werden \\ \hline
	MarkerFunctions.js & Enthält Funktionen, welche bei Markern auf der Karte verwendet werden \\ \hline
	pictureUploadNew.js & Enthält Funktionen welche nur beim Bildupload benötigt werden \\ \hline
	search.js & Enthält Funktionen die an diversen Stellen verwendet werden \\ \hline
	slider.js & Enthält Funktionen um Slider auf Karte zu ermöglichen \\ \hline
	statistics.js & Enthält Funktionen, welche nur auf der Statistik-Subpage verwendet werden \\ \hline
	StoryUpload.js & Enthält Funktionen, welche nur auf der Seite mit Geschichten verwendet werden \\ \hline
	personalArea.js & Enthält  Funktionen, welche für den persönlichen Bereich genutzt werden\\ \hline
	archive.js & Enthält  Funktionen, welche nur auf der Archivseite genutzt werden\\ \hline
	mapFnc.ts & Enthält Funktionen, welche nur auf der Kartenseite verwendet werden \\ \hline
	materialList.ts & Enthält Funktionen, welche nur auf der Archivseite verwendet werden \\ \hline
	poiEdit.ts & Enthält Funktionen, welche nur auf der Änderungsseite von Interessenpunkten verwendet werden \\ \hline
	registration.ts & Enthält Funktionen, welche nur auf der Registrierung-Seite von Interessenpunkten verwendet werden \\ \hline
\end{longtable}
\section{Speicherort}
\subsection{Eigene oder Geänderte Dateien}
Die oben genannten Dateien sind unter folgenden Pfaden zu finden:
\begin{itemize}
	\item {\glqq ./\grqq}
	\item {\glqq ./bin\grqq}
	\item {\glqq ./bin/database\grqq}
	\item {\glqq ./js\grqq}
	\item {\glqq ./tjs\grqq}
\end{itemize}
Die dem Projekt zugehörigen Cascading-Stylesheets, welche bearbeitet oder geschrieben wurden, befinden sich im Ordner {\glqq ./css\grqq}. Alle verwendeten Icons befinden sich im Ordner {\glqq ./images\grqq}.
\subsection{Weitere Dateien}
Alle Dateien unter {\glqq ./jse\grqq} sind aus Bibliotheken übernommen und nicht verändert worden. Sie werden für Bootstrap, Fontawesome, Leaflet, Lightbox oder als Abhängigkeit dieser gebraucht. Für alle Dateien unter {\glqq ./csse\grqq} gilt dies ebenso. Die Dokumentation entsprechender Funktionen ist nicht teil dieser Dokumentation. Für diese Dateien ist die jeweilige Dokumentation zu verwenden. 

\setcounter{secnumdepth}{3}
\chapter{Funktionsübersicht}
Die Funktionen sind nach den Dateien sortiert in denen die Funktionen enthalten sind. Des weiteren ist auch die Reihenfolge der Funktionen mit denen in der Datei übereinstimmend. \\
Hier finden Sie insbesondere eine Kurzbeschreibung aller verfügbaren Funktionen innerhalb des Projektes sowie eine Beschreibung des Aufbaus der meisten Seiten. Die Funktionen sind alle durch einen Dokumentationsblog mit folgenden Bestandteilen Beschrieben:
\begin{itemize}
	\item Parameter
	\item Beschreibung
	\item Vorgehensweise
\end{itemize}
Diese Bestandteile enthalten die Nachfolgend beschriebenen Informationen:
\paragraph{Parameter} Hier finden Sie eine Kurzbeschreibung aller verfügbaren Parameter der Funktion, inklusive aller optionalen. Diese Beschreibung ist meist tabellarisch gehalten und hat folgendes Schema:
\begin{table}[H]
	\begin{tabular}{|c|p{11cm}|}
		\hline
		\textbf{Parametername} & \textbf{Parameterbeschreibung} \\ \hline
		\$Parameter1 & Beschreibung 1 \\ \hline
		\$Parameter2 & Beschreibung 2 \\ \hline
	\end{tabular}
\end{table}
Sollte eine Funktionen einen Parameter des Typs Array haben, so wird dieser im Anschluss bei Bedarf näher aufgeschlüsselt. Dies geschieht in einer tabellarisch ähnlichen Form wie die Beschreibung und Nennung der Parameter.
\paragraph{Beschreibung} Hier finden Sie eine Beschreibung der Funktion, sowie die Quellen für Informationen, welche die Funktion verarbeitet. Dabei können die Quellen direkt oder Indirekt abgefragt werden. Die Quellen sind stets als Auflistung dargestellt. Es wird auch benannt, ob die Funktion Daten aus {\glqq COSP\grqq} abfragt oder an {\glqq COSP\grqq} sendet.
\paragraph{Vorgehensweise} Bei einigen größeren Funktionen ist unter diesem Punkt eine Beschreibung der Vorgehensweise der Funktion zu finden. Diese ist meist Abstrakt gehalten und dient dem groben Verständnis der Funktion.
\newpage
\section{api-functions}
\subsection{Allgemeines} Diese Datei enthält alle durch die API des Frontends aufgerufene Funktionen sowie zusätzliche Funktionen, um die Datenstrukturierung der Antworten zu vereinheitlichen.
\begin{table}[H]
	\begin{tabular}{|c|p{11cm}|}
		\hline
		\textbf{Einbindungspunkt} & inc.php \\ \hline
		\textbf{Einbindungspunkt} & inc-sub.php \\ \hline
	\end{tabular}
\end{table}
Die Datei ist nicht direkt durch den Nutzer aufrufbar, dies wird durch folgenden Code-Ausschnitt sichergestellt:
\begin{lstlisting}[language=php]
if (!defined('NICE_PROJECT')) {
	die('Permission denied.');
}
\end{lstlisting}
Der Globale Wert {\glqq NICE\_PROJECT\grqq} wird durch für den Nutzer valide Aufrufpunkte festgelegt, z.B. {\glqq api.php\grqq}.
\newpage
\subsection{Funktionen}
\subsubsection{PersonalAreaCollection}
\paragraph{Parameter} Die Funktion besitzt folgende Parameter:
\begin{table}[H]
	\begin{tabular}{|c|p{11cm}|}
		\hline
		\textbf{Parametername} & \textbf{Parameterbeschreibung} \\ \hline
		\$username & Nutzername des Nutzers für den Funktion ausgeführt werden soll. \\ \hline
	\end{tabular}
\end{table}
\paragraph{Beschreibung} Die Funktion dient dem ermitteln aller für die Anzeige des Persönlichen Bereiches benötigten Daten aus folgenden Quellen:
\begin{itemize}
	\item Interessenpunkt-Tabelle
	\item Nutzerdaten-Tabelle
	\item Kommentar-Tabelle
\end{itemize}
Es findet bei dieser Funktion kein Abruf von Daten aus {\glqq COSP\grqq} statt. Den so gefundenen Informationen werden zusätzliche Informationen hinzugefügt beziehungsweise Informationen umcodiert. Die Antwort wird als strukturiertes Array an den Aufrufer zurückgegeben.
\subsubsection{deleteUserComment}
\paragraph{Parameter} Die Funktion besitzt folgende Parameter:
\begin{table}[H]
	\begin{tabular}{|c|p{11cm}|}
		\hline
		\textbf{Parametername} & \textbf{Parameterbeschreibung} \\ \hline
		\$cid & Numerischer Identifikator des Kommentars, welcher gelöscht werden soll. \\ \hline
	\end{tabular}
\end{table}
\paragraph{Beschreibung} Die Funktion dient als Trampolin-Funktion um Kommentare zu löschen. Sie hat Auswirkungen auf folgende Tabellen:
\begin{itemize}
	\item Kommentar-Tabelle
\end{itemize}
Es findet bei dieser Funktion kein Abruf von Daten aus {\glqq COSP\grqq} statt. Der Kommentar wird je nach Konfiguration direkt gelöscht oder nur als gelöscht markiert. Die Funktion liefert stets ein positives Ergebnis zurück.
\subsubsection{AddUserComment}
\paragraph{Parameter} Die Funktion besitzt folgende Parameter:
\begin{table}[H]
	\begin{tabular}{|c|p{11cm}|}
		\hline
		\textbf{Parametername} & \textbf{Parameterbeschreibung} \\ \hline
		\$json & Strukturiertes Array  \\ \hline
	\end{tabular}
\end{table}
\subparagraph{\$json}Das Array enthält folgende Elemente:
\begin{table}[H]
	\begin{tabular}{|c|p{11cm}|}
		\hline
		\textbf{Parametername} & \textbf{Parameterbeschreibung} \\ \hline
		comment & Inhalt des Kommentars  \\ \hline
		poi\_id & Identifikator des Interessenpunktes zu welchem Kommentar hinzugefügt werden soll\\ \hline
	\end{tabular}
\end{table}
\paragraph{Beschreibung} Die Funktion fügt einen Kommentar zu einem bestehendem Interessenpunkt hinzu. Die Funktion hat Auswirkungen auf folgende Tabellen:
\begin{itemize}
	\item Kommentar-Tabelle
\end{itemize}
Es findet bei dieser Funktion kein Abruf von Daten aus {\glqq COSP\grqq} statt.
\subsubsection{generateJson}
\paragraph{Parameter} Die Funktion besitzt folgende Parameter:
\begin{table}[H]
	\begin{tabular}{|c|p{11cm}|}
		\hline
		\textbf{Parametername} & \textbf{Parameterbeschreibung} \\ \hline
		\$array & Array, welches zu einem JSON umgewandelt werden soll \\ \hline
	\end{tabular}
\end{table}
\paragraph{Beschreibung} Die Funktion dient dem Umwandeln von PHP-Arrays in JSON-Daten. Dieses wird folgendem Code ausgegeben:
\begin{lstlisting}[language=php]
echo json_encode($array);
\end{lstlisting}
Es findet bei dieser Funktion kein Abruf von Daten aus {\glqq COSP\grqq} statt.
\subsubsection{generateError}
\paragraph{Parameter} Die Funktion besitzt folgende Parameter:
\begin{table}[H]
	\begin{tabular}{|c|p{11cm}|}
		\hline
		\textbf{Parametername} & \textbf{Parameterbeschreibung} \\ \hline
		\$msg & Optionale Fehlermeldung beziehungsweise zusätzliche Informationen \\ \hline
	\end{tabular}
\end{table}
\paragraph{Beschreibung} Die Funktion generiert eine Fehlermeldung der Frontend-API mit Code 1. Es findet bei dieser Funktion kein Abruf von Daten aus {\glqq COSP\grqq} statt.
\subsubsection{generateError2}
\paragraph{Parameter} Die Funktion besitzt folgende Parameter:
\begin{table}[H]
	\begin{tabular}{|c|p{11cm}|}
		\hline
		\textbf{Parametername} & \textbf{Parameterbeschreibung} \\ \hline
		\$msg & Optionale Fehlermeldung beziehungsweise zusätzliche Informationen \\ \hline
	\end{tabular}
\end{table}
\paragraph{Beschreibung} Die Funktion generiert eine Fehlermeldung der Frontend-API mit Code 2. Es findet bei dieser Funktion kein Abruf von Daten aus {\glqq COSP\grqq} statt.
\subsubsection{generateSuccess}
\paragraph{Parameter} Die Funktion besitzt keine Parameter.
\paragraph{Beschreibung} Die Funktion generiert eine Erfolgsmeldung der Frontend-API. Es findet bei dieser Funktion kein Abruf von Daten aus {\glqq COSP\grqq} statt.
\subsubsection{getMinimalMaximalYear}
\paragraph{Parameter} Die Funktion besitzt keine Parameter.
\paragraph{Beschreibung} Die Funktion dient dem ermitteln des Start- und Endjahres des Schiebers zur Zeitintervall-Auswahl auf der Kartenansicht. Es ermittelt die benötigten Daten aus folgender Tabelle:
\begin{itemize}
	\item Interessenpunkt-Tabelle
\end{itemize}
Es findet bei dieser Funktion kein Abruf von Daten aus {\glqq COSP\grqq} statt. Es wird stets ein Erfolg zurück gegeben.
\subsubsection{selectMoreApi}
\paragraph{Parameter} Die Funktion besitzt folgende Parameter:
\begin{table}[H]
	\begin{tabular}{|c|p{11cm}|}
		\hline
		\textbf{Parametername} & \textbf{Parameterbeschreibung} \\ \hline
		\$json & Strukturiertes Array \\ \hline
	\end{tabular}
\end{table}
\subparagraph{\$json}Das Array enthält folgende Elemente:
\begin{table}[H]
	\begin{tabular}{|c|p{11cm}|}
		\hline
		\textbf{Parametername} & \textbf{Parameterbeschreibung} \\ \hline
		poi\_id & Identifikator des Interessenpunktes zu welchem Daten abgerufen werden sollen \\ \hline
	\end{tabular}
\end{table}
\paragraph{Beschreibung} Die Funktion dient dem ermitteln aller für die Anzeige von Daten, um das {\glqq Mehr Anzeigen\grqq}-Modal zu ermöglichen. Die Daten für die Antwort werden zu großen Teilen aus folgende Tabellen bezogen:
\begin{itemize}
	\item Interessenpunkt-Tabelle
	\item Validierungstabelle für Interessenpunkte
	\item Validierungstabelle für aktuelle Adresse
	\item Validierungstabelle für Geschichte des Interessenpunktes
	\item Validierungstabelle für Nutzungszeitspanne
	\item Validierungstabelle für Typ des Interessenpunktes
\end{itemize}
Es findet bei dieser Funktion kein Abruf von Daten aus {\glqq COSP\grqq} statt. Es wird stets eine erfolgreiche Antwort gegeben.
\subsubsection{ShowMoreComments}
\paragraph{Parameter} Die Funktion besitzt folgende Parameter:
\begin{table}[H]
	\begin{tabular}{|c|p{11cm}|}
		\hline
		\textbf{Parametername} & \textbf{Parameterbeschreibung} \\ \hline
		\$json & Strukturiertes Array \\ \hline
	\end{tabular}
\end{table}
\subparagraph{\$json}Das Array enthält folgende Elemente:
\begin{table}[H]
	\begin{tabular}{|c|p{11cm}|}
		\hline
		\textbf{Parametername} & \textbf{Parameterbeschreibung} \\ \hline
		poi\_id & Identifikator des Interessenpunktes zu welchem Kommentare abgerufen werden sollen \\ \hline
	\end{tabular}
\end{table}
\paragraph{Beschreibung} Die Funktion dient dem ermitteln aller Kommentare zu einem bestimmten Interessenpunkt. Hierzu werden Daten aus folgenden Tabellen benötigt:
\begin{itemize}
	\item Nutzerdaten-Tabelle
	\item Kommentar-Tabelle
\end{itemize}
Es findet bei dieser Funktion kein Abruf von Daten aus {\glqq COSP\grqq} statt. es wird stets eine erfolgreiche Antwort gegeben.
\subsubsection{ShowMoreLoadAdditionalPictures}
\paragraph{Parameter} Die Funktion besitzt folgende Parameter:
\begin{table}[H]
	\begin{tabular}{|c|p{11cm}|}
		\hline
		\textbf{Parametername} & \textbf{Parameterbeschreibung} \\ \hline
		\$json & Strukturiertes Array \\ \hline
	\end{tabular}
\end{table}
\subparagraph{\$json}Das Array enthält folgende Elemente:
\begin{table}[H]
	\begin{tabular}{|c|p{11cm}|}
		\hline
		\textbf{Parametername} & \textbf{Parameterbeschreibung} \\ \hline
		poi\_id & Identifikator des Interessenpunktes zu welchem zusätzliche Bilder abgerufen werden sollen \\ \hline
	\end{tabular}
\end{table}
\paragraph{Beschreibung} Die Funktion dient dem ermitteln aller Daten, welche für die Anzeige von zusätzlichen Bildern eines Interessenpunktes notwendig sind. Die Daten Stammen aus folgenden Quellen:
\begin{itemize}
	\item Tabelle mit Links zwischen Interessenpunkten und Bildern
	\item Validierungstabelle für Links zwischen Interessenpunkten und Bildern
	\item COSP
\end{itemize}
Es findet bei dieser Funktion ein Abruf von Daten aus {\glqq COSP\grqq} statt. Das Ergebnis ist stets erfolgreich.
\subsubsection{ShowMoreLoadPicture}
\paragraph{Parameter} Die Funktion besitzt folgende Parameter:
\begin{table}[H]
	\begin{tabular}{|c|p{11cm}|}
		\hline
		\textbf{Parametername} & \textbf{Parameterbeschreibung} \\ \hline
		\$json & Strukturiertes Array \\ \hline
	\end{tabular}
\end{table}
\subparagraph{\$json}Das Array enthält folgende Elemente:
\begin{table}[H]
	\begin{tabular}{|c|p{11cm}|}
		\hline
		\textbf{Parametername} & \textbf{Parameterbeschreibung} \\ \hline
		poi\_id & Identifikator des Interessenpunktes zu welchem das Hauptbild abgerufen werden sollen \\ \hline
	\end{tabular}
\end{table}
\paragraph{Beschreibung} Die Funktion dient dem ermitteln aller für die Anzeige des Hauptbildes eines Interessenpunktes benötigten Daten aus folgenden Quellen:
\begin{itemize}
	\item COSP
\end{itemize}
Es findet bei dieser Funktion ein Abruf von Daten aus {\glqq COSP\grqq} statt. Das Ergebnis ist stets erfolgreich.
\subsubsection{getStoriesForOptionDropDownShowMoreApi}
\paragraph{Parameter} Die Funktion besitzt folgende Parameter:
\begin{table}[H]
	\begin{tabular}{|c|p{11cm}|}
		\hline
		\textbf{Parametername} & \textbf{Parameterbeschreibung} \\ \hline
		\$json & Strukturiertes Array \\ \hline
	\end{tabular}
\end{table}
\subparagraph{\$json}Das Array enthält folgende Elemente:
\begin{table}[H]
	\begin{tabular}{|c|p{11cm}|}
		\hline
		\textbf{Parametername} & \textbf{Parameterbeschreibung} \\ \hline
		poi\_id & Identifikator des Interessenpunktes zu welchem verknüpfbare Geschichten abgerufen werden sollen \\ \hline
	\end{tabular}
\end{table}
\paragraph{Beschreibung} Die Funktion dient dem ermitteln aller nicht verknüpften Geschichten für einen gegebenen Interessenpunkt aus folgenden Quellen:
\begin{itemize}
	\item Tabelle mit Links zwischen Geschichten und Interessenpunkten
	\item COSP
\end{itemize}
Es findet bei dieser Funktion ein Abruf von Daten aus {\glqq COSP\grqq} statt. Das Ergebnis ist stets erfolgreich.
\subsubsection{selectStoriesPoiAPI}
\paragraph{Parameter} Die Funktion besitzt folgende Parameter:
\begin{table}[H]
	\begin{tabular}{|c|p{11cm}|}
		\hline
		\textbf{Parametername} & \textbf{Parameterbeschreibung} \\ \hline
		\$json & Strukturiertes Array \\ \hline
	\end{tabular}
\end{table}
\subparagraph{\$json}Das Array enthält folgende Elemente:
\begin{table}[H]
	\begin{tabular}{|c|p{11cm}|}
		\hline
		\textbf{Parametername} & \textbf{Parameterbeschreibung} \\ \hline
		poi\_id & Identifikator des Interessenpunktes zu welchem verknüpfte Geschichten abgerufen werden sollen \\ \hline
	\end{tabular}
\end{table}
\paragraph{Beschreibung} Die Funktion dient dem ermitteln alle benötigten Daten für das Laden aller mit einem bestimmten Interessenpunkt verknüpften Geschichten aus folgenden Quellen:
\begin{itemize}
	\item Tabelle mit Links zwischen Geschichten und Interessenpunkten
	\item Tabelle mit Validierungsinformationen für Links zwischen Geschichten und Interessenpunkten
	\item COSP
\end{itemize}
Es findet bei dieser Funktion ein Abruf von Daten aus {\glqq COSP\grqq} statt. Das Ergebnis ist stets erfolgreich.
\subsubsection{selectSeatsPoiAPI}
\paragraph{Parameter} Die Funktion besitzt folgende Parameter:
\begin{table}[H]
	\begin{tabular}{|c|p{11cm}|}
		\hline
		\textbf{Parametername} & \textbf{Parameterbeschreibung} \\ \hline
		\$json & Strukturiertes Array \\ \hline
	\end{tabular}
\end{table}
\subparagraph{\$json}Das Array enthält folgende Elemente:
\begin{table}[H]
	\begin{tabular}{|c|p{11cm}|}
		\hline
		\textbf{Parametername} & \textbf{Parameterbeschreibung} \\ \hline
		poi\_id & Identifikator des Interessenpunktes zu welchem alle Sitzplatzanzahlen abgerufen werden sollen \\ \hline
	\end{tabular}
\end{table}
\paragraph{Beschreibung} Die Funktion dient dem ermitteln aller Sitzplatzanzahlen für einen gegebenen Interessenpunkt aus folgenden Quellen:
\begin{itemize}
	\item Sitzplatzanzahl-Tabelle
	\item Tabelle mit Validierungsinformationen zu Sitzplatzanzahlen
	\item Nutzertabelle
\end{itemize}
Es findet bei dieser Funktion kein Abruf von Daten aus {\glqq COSP\grqq} statt. Das Ergebnis ist stets erfolgreich.
\subsubsection{selectCinemasPoiAPI}
\paragraph{Parameter} Die Funktion besitzt folgende Parameter:
\begin{table}[H]
	\begin{tabular}{|c|p{11cm}|}
		\hline
		\textbf{Parametername} & \textbf{Parameterbeschreibung} \\ \hline
		\$json & Strukturiertes Array \\ \hline
	\end{tabular}
\end{table}
\subparagraph{\$json}Das Array enthält folgende Elemente:
\begin{table}[H]
	\begin{tabular}{|c|p{11cm}|}
		\hline
		\textbf{Parametername} & \textbf{Parameterbeschreibung} \\ \hline
		poi\_id & Identifikator des Interessenpunktes zu welchem alle Saalanzahlen abgerufen werden sollen \\ \hline
	\end{tabular}
\end{table}
\paragraph{Beschreibung} Die Funktion dient dem ermitteln aller Saalanzahlen für einen gegebenen Interessenpunkt aus folgenden Quellen:
\begin{itemize}
	\item Saalanzahl-Tabelle
	\item Tabelle mit Validierungsinformationen zu Saalanzahlen
	\item Nutzertabelle
\end{itemize}
Es findet bei dieser Funktion kein Abruf von Daten aus {\glqq COSP\grqq} statt. Das Ergebnis ist stets erfolgreich.
\subsubsection{isGuestAPI}
\paragraph{Parameter} Die Funktion besitzt keine Parameter.
\paragraph{Beschreibung} Die Funktion dient dem ermitteln ob der aktuelle Nutzer ein Gast ist. Es findet bei dieser Funktion kein Abruf von Daten aus {\glqq COSP\grqq} statt. Das Ergebnis ist stets erfolgreich.
\subsubsection{selectHistAddrPoiAPI}
\paragraph{Parameter} Die Funktion besitzt folgende Parameter:
\begin{table}[H]
	\begin{tabular}{|c|p{11cm}|}
		\hline
		\textbf{Parametername} & \textbf{Parameterbeschreibung} \\ \hline
		\$json & Strukturiertes Array \\ \hline
	\end{tabular}
\end{table}
\subparagraph{\$json}Das Array enthält folgende Elemente:
\begin{table}[H]
	\begin{tabular}{|c|p{11cm}|}
		\hline
		\textbf{Parametername} & \textbf{Parameterbeschreibung} \\ \hline
		poi\_id & Identifikator des Interessenpunktes zu welchem alle historischen Adressen abgerufen werden sollen \\ \hline
	\end{tabular}
\end{table}
\paragraph{Beschreibung} Die Funktion dient dem ermitteln aller historischen Adressen für einen gegebenen Interessenpunkt aus folgenden Quellen:
\begin{itemize}
	\item Tabelle mit historischen Adressen
	\item Tabelle mit Validierungsinformationen zu historischen Adressen
	\item Nutzertabelle
\end{itemize}
Es findet bei dieser Funktion kein Abruf von Daten aus {\glqq COSP\grqq} statt. Das Ergebnis ist stets erfolgreich.
\subsubsection{selectNamesPoiAPI}
\paragraph{Parameter} Die Funktion besitzt folgende Parameter:
\begin{table}[H]
	\begin{tabular}{|c|p{11cm}|}
		\hline
		\textbf{Parametername} & \textbf{Parameterbeschreibung} \\ \hline
		\$json & Strukturiertes Array \\ \hline
	\end{tabular}
\end{table}
\subparagraph{\$json}Das Array enthält folgende Elemente:
\begin{table}[H]
	\begin{tabular}{|c|p{11cm}|}
		\hline
		\textbf{Parametername} & \textbf{Parameterbeschreibung} \\ \hline
		poi\_id & Identifikator des Interessenpunktes zu welchem alle Namen abgerufen werden sollen \\ \hline
	\end{tabular}
\end{table}
\paragraph{Beschreibung} Die Funktion dient dem ermitteln aller Namen für einen gegebenen Interessenpunkt aus folgenden Quellen:
\begin{itemize}
	\item Tabelle mit Namen von Interessenpunkten
	\item Tabelle mit Validierungsinformationen zu Namen
	\item Nutzertabelle
\end{itemize}
Es findet bei dieser Funktion kein Abruf von Daten aus {\glqq COSP\grqq} statt. Das Ergebnis ist stets erfolgreich.
\subsubsection{selectOperatorsPoiAPI}
\paragraph{Parameter} Die Funktion besitzt folgende Parameter:
\begin{table}[H]
	\begin{tabular}{|c|p{11cm}|}
		\hline
		\textbf{Parametername} & \textbf{Parameterbeschreibung} \\ \hline
		\$json & Strukturiertes Array \\ \hline
	\end{tabular}
\end{table}
\subparagraph{\$json}Das Array enthält folgende Elemente:
\begin{table}[H]
	\begin{tabular}{|c|p{11cm}|}
		\hline
		\textbf{Parametername} & \textbf{Parameterbeschreibung} \\ \hline
		poi\_id & Identifikator des Interessenpunktes zu welchem alle Betreiber abgerufen werden sollen \\ \hline
	\end{tabular}
\end{table}
\paragraph{Beschreibung} Die Funktion dient dem ermitteln aller Betreiber für einen gegebenen Interessenpunkt aus folgenden Quellen:
\begin{itemize}
	\item Tabelle mit Betreibern von Interessenpunkten
	\item Tabelle mit Validierungsinformationen zu Betreibern
	\item Nutzertabelle
\end{itemize}
Es findet bei dieser Funktion kein Abruf von Daten aus {\glqq COSP\grqq} statt. Das Ergebnis ist stets erfolgreich.
\subsubsection{getPoisForUserApi}
\paragraph{Parameter} Die Funktion besitzt keine Parameter.
\paragraph{Beschreibung} Die Funktion dient dem ermitteln aller für einen Nutzer anzeigbaren Interessenpunkte. Die Daten hierfür Stammen aus folgenden Quellen:
\begin{itemize}
	\item Tabelle mit Interessenpunkten
	\item Tabelle mit Validierungsinformationen zu Interessenpunkten
	\item Nutzertabelle
\end{itemize}
Es findet bei dieser Funktion kein Abruf von Daten aus {\glqq COSP\grqq} statt. Das Ergebnis ist stets erfolgreich.
\subsubsection{getAllStoriesDataApi}
\paragraph{Parameter} Die Funktion besitzt keine Parameter.
\paragraph{Beschreibung} Die Funktion dient dem ermitteln aller Daten zum Abrufen aller für den Nutzer freigegebenen Geschichten. Die Daten hierfür Stammen aus folgenden Quellen:
\begin{itemize}
	\item COSP
\end{itemize}
Es findet bei dieser Funktion ein Abruf von Daten aus {\glqq COSP\grqq} statt. Das Ergebnis ist stets erfolgreich.
\subsubsection{addUserStory}\label{api-functions:addUserStory}
\paragraph{Parameter} Die Funktion besitzt folgende Parameter:
\begin{table}[H]
	\begin{tabular}{|c|p{11cm}|}
		\hline
		\textbf{Parametername} & \textbf{Parameterbeschreibung} \\ \hline
		\$json & Strukturiertes Array \\ \hline
	\end{tabular}
\end{table}
\subparagraph{\$json}Das Array enthält folgende Elemente:
\begin{table}[H]
	\begin{tabular}{|c|p{11cm}|}
		\hline
		\textbf{Parametername} & \textbf{Parameterbeschreibung} \\ \hline
		story & Inhalt der Geschichte \\ \hline
		title & Titel der Geschichte \\ \hline
		rights & Status der Abgabe der Rechte \\ \hline
	\end{tabular}
\end{table}
\paragraph{Beschreibung} Die Funktion dient dem hinzufügen einer neuen Geschichte. Sie hat Auswirkungen auf folgende Quellen:
\begin{itemize}
	\item COSP
\end{itemize}
Es findet bei dieser Funktion kein Abruf von Daten aus {\glqq COSP\grqq} statt. Es werden jedoch Daten an {\glqq COSP\grqq} gesendet. Das Ergebnis ist stets erfolgreich.
\subsubsection{deletePointOfInterestViaAPI}
\paragraph{Parameter} Die Funktion besitzt folgende Parameter:
\begin{table}[H]
	\begin{tabular}{|c|p{11cm}|}
		\hline
		\textbf{Parametername} & \textbf{Parameterbeschreibung} \\ \hline
		\$json & Strukturiertes Array \\ \hline
	\end{tabular}
\end{table}
\subparagraph{\$json}Das Array enthält folgende Elemente:
\begin{table}[H]
	\begin{tabular}{|c|p{11cm}|}
		\hline
		\textbf{Parametername} & \textbf{Parameterbeschreibung} \\ \hline
		poiid & Identifikator des Interessenpunktes welcher gelöscht werden sollen \\ \hline
	\end{tabular}
\end{table}
\paragraph{Beschreibung} Die Funktion dient dem Löschen eines Interessenpunktes inklusive aller Daten und Verknüpfungen. Die Abfrage hat auf verschiedene Quellen Auswirkungen. Es findet bei dieser Funktion kein Abruf von Daten aus {\glqq COSP\grqq} statt.
\subsubsection{getCommentByCommentID}
\paragraph{Parameter} Die Funktion besitzt folgende Parameter:
\begin{table}[H]
	\begin{tabular}{|c|p{11cm}|}
		\hline
		\textbf{Parametername} & \textbf{Parameterbeschreibung} \\ \hline
		\$json & Strukturiertes Array \\ \hline
	\end{tabular}
\end{table}
\subparagraph{\$json}Das Array enthält folgende Elemente:
\begin{table}[H]
	\begin{tabular}{|c|p{11cm}|}
		\hline
		\textbf{Parametername} & \textbf{Parameterbeschreibung} \\ \hline
		commentid & Identifikator des abgefragten Kommentars \\ \hline
	\end{tabular}
\end{table}
\paragraph{Beschreibung} Die Funktion dient der Abfrage eines bestimmten Kommentars. Es werden Daten aus folgenden Quellen benötigt:
\begin{itemize}
	\item Nutzerdaten-Tabelle
	\item Kommentar-Tabelle
\end{itemize}
Es findet bei dieser Funktion kein Abruf von Daten aus {\glqq COSP\grqq} statt. Das Ergebnis ist stets erfolgreich.
\subsubsection{DataSingleMaterial}
\paragraph{Parameter} Die Funktion besitzt folgende Parameter:
\begin{table}[H]
	\begin{tabular}{|c|p{11cm}|}
		\hline
		\textbf{Parametername} & \textbf{Parameterbeschreibung} \\ \hline
		\$json & Strukturiertes Array \\ \hline
	\end{tabular}
\end{table}
\subparagraph{\$json}Das Array enthält folgende Elemente:
\begin{table}[H]
	\begin{tabular}{|c|p{11cm}|}
		\hline
		\textbf{Parametername} & \textbf{Parameterbeschreibung} \\ \hline
		token & Identifikator eines Bildes \\ \hline
	\end{tabular}
\end{table}
\paragraph{Beschreibung} Die Funktion dient dem umwandeln von Abrufdaten für eine Abfrage eines Bildes aus {\glqq COSP\grqq} in die Abruf-URL. Es findet bei dieser Funktion kein Abruf von Daten aus {\glqq COSP\grqq} statt. Das Ergebnis ist stets erfolgreich.
\subsubsection{saveSingleMaterialViaAPI}
\paragraph{Parameter} Die Funktion besitzt folgende Parameter:
\begin{table}[H]
	\begin{tabular}{|c|p{11cm}|}
		\hline
		\textbf{Parametername} & \textbf{Parameterbeschreibung} \\ \hline
		\$json & Strukturiertes Array \\ \hline
	\end{tabular}
\end{table}
\subparagraph{\$json}Das Array enthält folgende Elemente:
\begin{table}[H]
	\begin{tabular}{|c|p{11cm}|}
		\hline
		\textbf{Parametername} & \textbf{Parameterbeschreibung} \\ \hline
		token & Identifikator des Bildes \\ \hline
		title & Titel des Bildes \\ \hline
		description & Beschreibung des Bildes \\ \hline
	\end{tabular}
\end{table}
\paragraph{Beschreibung} Die Funktion dient dem Speichern geänderter Daten eines Bildes. Es hat Auswirkungen auf folgende Quellen:
\begin{itemize}
	\item COSP
\end{itemize}
Es findet bei dieser Funktion kein Abruf von Daten aus {\glqq COSP\grqq} statt. Es werden jedoch Daten an {\glqq COSP\grqq} übertragen. Das Ergebnis ist stets erfolgreich.
\subsubsection{SaveDataForEditedStoryAPI}
\paragraph{Parameter} Die Funktion besitzt folgende Parameter:
\begin{table}[H]
	\begin{tabular}{|c|p{11cm}|}
		\hline
		\textbf{Parametername} & \textbf{Parameterbeschreibung} \\ \hline
		\$json & Strukturiertes Array \\ \hline
	\end{tabular}
\end{table}
\subparagraph{\$json}Das Array enthält folgende Elemente:
\begin{table}[H]
	\begin{tabular}{|c|p{11cm}|}
		\hline
		\textbf{Parametername} & \textbf{Parameterbeschreibung} \\ \hline
		token & Identifikator der Geschichte \\ \hline
		title & Titel der Geschichte \\ \hline
		story & Inhalt der Geschichte \\ \hline
	\end{tabular}
\end{table}
\paragraph{Beschreibung} Die Funktion dient dem Speichern geänderter Daten einer Geschichte. Es hat Auswirkungen auf folgende Quellen:
\begin{itemize}
	\item COSP
\end{itemize}
Es findet bei dieser Funktion kein Abruf von Daten aus {\glqq COSP\grqq} statt. Es werden jedoch Daten an {\glqq COSP\grqq} übertragen. Das Ergebnis ist stets erfolgreich.
\subsubsection{SaveOperatorNewAPI}
\paragraph{Parameter} Die Funktion besitzt folgende Parameter:
\begin{table}[H]
	\begin{tabular}{|c|p{11cm}|}
		\hline
		\textbf{Parametername} & \textbf{Parameterbeschreibung} \\ \hline
		\$json & Strukturiertes Array \\ \hline
	\end{tabular}
\end{table}
\subparagraph{\$json}Das Array enthält folgende Elemente:
\begin{table}[H]
	\begin{tabular}{|c|p{11cm}|}
		\hline
		\textbf{Parametername} & \textbf{Parameterbeschreibung} \\ \hline
		poi\_id & Identifikator des Interessenpunktes zu welchem Kommentare abgerufen werden sollen \\ \hline
		from & Startjahr der Nutzung durch Betreiber \\ \hline
		till & Endjahr der Nutzung durch Betreiber \\ \hline
		operator & Name des Betreibers \\ \hline
	\end{tabular}
\end{table}
\paragraph{Beschreibung} Die Funktion dient dem Hinzufügen eines neuen Betreibers. Die Funktion hat Auswirkung auf folgenden Quellen:
\begin{itemize}
	\item Betreiber-Tabelle
\end{itemize}
Es findet bei dieser Funktion kein Abruf von Daten aus {\glqq COSP\grqq} statt.
\subsubsection{SaveNameNewAPI}
\paragraph{Parameter} Die Funktion besitzt folgende Parameter:
\begin{table}[H]
	\begin{tabular}{|c|p{11cm}|}
		\hline
		\textbf{Parametername} & \textbf{Parameterbeschreibung} \\ \hline
		\$json & Strukturiertes Array \\ \hline
	\end{tabular}
\end{table}
\subparagraph{\$json}Das Array enthält folgende Elemente:
\begin{table}[H]
	\begin{tabular}{|c|p{11cm}|}
		\hline
		\textbf{Parametername} & \textbf{Parameterbeschreibung} \\ \hline
		poi\_id & Identifikator des Interessenpunktes zu welchem Kommentare abgerufen werden sollen \\ \hline
		from & Startjahr der Nutzung des Namens \\ \hline
		till & Endjahr der Nutzung des Namens \\ \hline
		name & Name \\ \hline
	\end{tabular}
\end{table}
\paragraph{Beschreibung} Die Funktion dient dem Hinzufügen eines neuen Namens. Die Funktion hat Auswirkung auf folgenden Quellen:
\begin{itemize}
	\item Namen-Tabelle
\end{itemize}
Es findet bei dieser Funktion kein Abruf von Daten aus {\glqq COSP\grqq} statt.
\subsubsection{SaveHistoricalAddressNewAPI}
\paragraph{Parameter} Die Funktion besitzt folgende Parameter:
\begin{table}[H]
	\begin{tabular}{|c|p{11cm}|}
		\hline
		\textbf{Parametername} & \textbf{Parameterbeschreibung} \\ \hline
		\$json & Strukturiertes Array \\ \hline
	\end{tabular}
\end{table}
\subparagraph{\$json}Das Array enthält folgende Elemente:
\begin{table}[H]
	\begin{tabular}{|c|p{11cm}|}
		\hline
		\textbf{Parametername} & \textbf{Parameterbeschreibung} \\ \hline
		poi\_id & Identifikator des Interessenpunktes zu welchem Kommentare abgerufen werden sollen \\ \hline
		from & Startjahr der Nutzung der Adresse \\ \hline
		till & Endjahr der Nutzung der Adresse \\ \hline
		streetname & Straßenname der Adresse \\ \hline
		housenumber & Hausnummer der Adresse \\ \hline
		city & Ortsname der Adresse \\ \hline
		postalcode & Postleitzahl der Adresse \\ \hline
	\end{tabular}
\end{table}
\paragraph{Beschreibung} Die Funktion dient dem Hinzufügen einer neuen historischen Adresse. Die Funktion hat Auswirkung auf folgenden Quellen:
\begin{itemize}
	\item Tabelle mit historischen Adressen
\end{itemize}
Es findet bei dieser Funktion kein Abruf von Daten aus {\glqq COSP\grqq} statt. Das Ergebnis ist stets erfolgreich.
\subsubsection{validateTimeSpanApi}
\paragraph{Parameter} Die Funktion besitzt folgende Parameter:
\begin{table}[H]
	\begin{tabular}{|c|p{11cm}|}
		\hline
		\textbf{Parametername} & \textbf{Parameterbeschreibung} \\ \hline
		\$json & Strukturiertes Array \\ \hline
	\end{tabular}
\end{table}
\subparagraph{\$json}Das Array enthält folgende Elemente:
\begin{table}[H]
	\begin{tabular}{|c|p{11cm}|}
		\hline
		\textbf{Parametername} & \textbf{Parameterbeschreibung} \\ \hline
		POIID & Identifikator des zugehörigen Interessenpunktes bei welchem Zeitspanne validiert werden sollen \\ \hline
	\end{tabular}
\end{table}
\paragraph{Beschreibung} Die Funktion dient der Validierung einer Zeitspanne eines bestimmten Interessenpunktes und hat Auswirkungen auf folgenden Quellen:
\begin{itemize}
	\item Tabelle mit Validierungsinformationen zu Zeitspannen
	\item COSP
\end{itemize}
Es findet bei dieser Funktion kein Abruf von Daten aus {\glqq COSP\grqq} statt. Es werden jedoch Daten an {\glqq COSP\grqq} übermittelt. Das Ergebnis ist stets erfolgreich.
\subsubsection{validateCurrentAddressApi}
\paragraph{Parameter} Die Funktion besitzt folgende Parameter:
\begin{table}[H]
	\begin{tabular}{|c|p{11cm}|}
		\hline
		\textbf{Parametername} & \textbf{Parameterbeschreibung} \\ \hline
		\$json & Strukturiertes Array \\ \hline
	\end{tabular}
\end{table}
\subparagraph{\$json}Das Array enthält folgende Elemente:
\begin{table}[H]
	\begin{tabular}{|c|p{11cm}|}
		\hline
		\textbf{Parametername} & \textbf{Parameterbeschreibung} \\ \hline
		POIID & Identifikator des zugehörigen Interessenpunktes bei welchem die aktuelle Adresse validiert werden sollen \\ \hline
	\end{tabular}
\end{table}
\paragraph{Beschreibung} Die Funktion dient der Validierung einer aktuellen Adresse eines bestimmten Interessenpunktes und hat Auswirkungen auf folgenden Quellen:
\begin{itemize}
	\item Tabelle mit Validierungsinformationen zur aktuellen Adresse
	\item COSP
\end{itemize}
Es findet bei dieser Funktion kein Abruf von Daten aus {\glqq COSP\grqq} statt. Es werden jedoch Daten an {\glqq COSP\grqq} übermittelt. Das Ergebnis ist stets erfolgreich.
\subsubsection{validateHistoryApi}
\paragraph{Parameter} Die Funktion besitzt folgende Parameter:
\begin{table}[H]
	\begin{tabular}{|c|p{11cm}|}
		\hline
		\textbf{Parametername} & \textbf{Parameterbeschreibung} \\ \hline
		\$json & Strukturiertes Array \\ \hline
	\end{tabular}
\end{table}
\subparagraph{\$json}Das Array enthält folgende Elemente:
\begin{table}[H]
	\begin{tabular}{|c|p{11cm}|}
		\hline
		\textbf{Parametername} & \textbf{Parameterbeschreibung} \\ \hline
		POIID & Identifikator des zugehörigen Interessenpunktes bei welchem Geschichte validiert werden sollen \\ \hline
	\end{tabular}
\end{table}
\paragraph{Beschreibung} Die Funktion dient der Validierung einer Geschichte eines bestimmten Interessenpunktes und hat Auswirkungen auf folgenden Quellen:
\begin{itemize}
	\item Tabelle mit Validierungsinformationen zu Geschichten von Interessenpunkten
	\item COSP
\end{itemize}
Es findet bei dieser Funktion kein Abruf von Daten aus {\glqq COSP\grqq} statt. Es werden jedoch Daten an {\glqq COSP\grqq} übermittelt. Das Ergebnis ist stets erfolgreich.
\subsubsection{validatePoiNamesApi}
\paragraph{Parameter} Die Funktion besitzt folgende Parameter:
\begin{table}[H]
	\begin{tabular}{|c|p{11cm}|}
		\hline
		\textbf{Parametername} & \textbf{Parameterbeschreibung} \\ \hline
		\$json & Strukturiertes Array \\ \hline
	\end{tabular}
\end{table}
\subparagraph{\$json}Das Array enthält folgende Elemente:
\begin{table}[H]
	\begin{tabular}{|c|p{11cm}|}
		\hline
		\textbf{Parametername} & \textbf{Parameterbeschreibung} \\ \hline
		NAMEID & Identifikator des zu validierenden Namens \\ \hline
	\end{tabular}
\end{table}
\paragraph{Beschreibung} Die Funktion dient der Validierung eines bestimmten Namens und hat Auswirkungen auf folgenden Quellen:
\begin{itemize}
	\item Tabelle mit Validierungsinformationen zu Namen
	\item COSP
\end{itemize}
Es findet bei dieser Funktion kein Abruf von Daten aus {\glqq COSP\grqq} statt. Es werden jedoch Daten an {\glqq COSP\grqq} übermittelt. Das Ergebnis ist stets erfolgreich.
\subsubsection{validatePoiOperatorsApi}
\paragraph{Parameter} Die Funktion besitzt folgende Parameter:
\begin{table}[H]
	\begin{tabular}{|c|p{11cm}|}
		\hline
		\textbf{Parametername} & \textbf{Parameterbeschreibung} \\ \hline
		\$json & Strukturiertes Array \\ \hline
	\end{tabular}
\end{table}
\subparagraph{\$json}Das Array enthält folgende Elemente:
\begin{table}[H]
	\begin{tabular}{|c|p{11cm}|}
		\hline
		\textbf{Parametername} & \textbf{Parameterbeschreibung} \\ \hline
		OPERATORID & Identifikator des zu validierenden Betreibers \\ \hline
	\end{tabular}
\end{table}
\paragraph{Beschreibung} Die Funktion dient der Validierung eines bestimmten Betreibers und hat Auswirkungen auf folgenden Quellen:
\begin{itemize}
	\item Tabelle mit Validierungsinformationen zu Betreibern
	\item COSP
\end{itemize}
Es findet bei dieser Funktion kein Abruf von Daten aus {\glqq COSP\grqq} statt. Es werden jedoch Daten an {\glqq COSP\grqq} übermittelt. Das Ergebnis ist stets erfolgreich.
\subsubsection{validatePoiHistAddressApi}
\paragraph{Parameter} Die Funktion besitzt folgende Parameter:
\begin{table}[H]
	\begin{tabular}{|c|p{11cm}|}
		\hline
		\textbf{Parametername} & \textbf{Parameterbeschreibung} \\ \hline
		\$json & Strukturiertes Array \\ \hline
	\end{tabular}
\end{table}
\subparagraph{\$json}Das Array enthält folgende Elemente:
\begin{table}[H]
	\begin{tabular}{|c|p{11cm}|}
		\hline
		\textbf{Parametername} & \textbf{Parameterbeschreibung} \\ \hline
		ADDRESSID & Identifikator der zu validierenden historischen Adresse \\ \hline
	\end{tabular}
\end{table}
\paragraph{Beschreibung} Die Funktion dient der Validierung einer bestimmten historischen Adresse und hat Auswirkungen auf folgenden Quellen:
\begin{itemize}
	\item Tabelle mit Validierungsinformationen zu historischen Adressen
	\item COSP
\end{itemize}
Es findet bei dieser Funktion kein Abruf von Daten aus {\glqq COSP\grqq} statt. Es werden jedoch Daten an {\glqq COSP\grqq} übermittelt. Das Ergebnis ist stets erfolgreich.
\subsubsection{deleteNameApi}
\paragraph{Parameter} Die Funktion besitzt folgende Parameter:
\begin{table}[H]
	\begin{tabular}{|c|p{11cm}|}
		\hline
		\textbf{Parametername} & \textbf{Parameterbeschreibung} \\ \hline
		\$json & Strukturiertes Array \\ \hline
	\end{tabular}
\end{table}
\subparagraph{\$json}Das Array enthält folgende Elemente:
\begin{table}[H]
	\begin{tabular}{|c|p{11cm}|}
		\hline
		\textbf{Parametername} & \textbf{Parameterbeschreibung} \\ \hline
		IDent & Identifikator des zu löschenden Namens \\ \hline
	\end{tabular}
\end{table}
\paragraph{Beschreibung} Die Funktion dient der Löschung eines bestimmten Namens und hat Auswirkungen auf folgenden Quellen:
\begin{itemize}
	\item Tabelle mit Validierungsinformationen zu Namen
	\item Tabelle mit Namen
\end{itemize}
Es findet bei dieser Funktion kein Abruf von Daten aus {\glqq COSP\grqq} statt. Das Ergebnis ist stets erfolgreich.
\subsubsection{deleteOperatorApi}
\paragraph{Parameter} Die Funktion besitzt folgende Parameter:
\begin{table}[H]
	\begin{tabular}{|c|p{11cm}|}
		\hline
		\textbf{Parametername} & \textbf{Parameterbeschreibung} \\ \hline
		\$json & Strukturiertes Array \\ \hline
	\end{tabular}
\end{table}
\subparagraph{\$json}Das Array enthält folgende Elemente:
\begin{table}[H]
	\begin{tabular}{|c|p{11cm}|}
		\hline
		\textbf{Parametername} & \textbf{Parameterbeschreibung} \\ \hline
		IDent & Identifikator des zu löschenden Betreibers \\ \hline
	\end{tabular}
\end{table}
\paragraph{Beschreibung} Die Funktion dient der Löschung eines bestimmten Betreibers und hat Auswirkungen auf folgenden Quellen:
\begin{itemize}
	\item Tabelle mit Validierungsinformationen zu Betreibern
	\item Tabelle mit Betreibern
\end{itemize}
Es findet bei dieser Funktion kein Abruf von Daten aus {\glqq COSP\grqq} statt. Das Ergebnis ist stets erfolgreich.
\subsubsection{deleteHistAddressApi}
\paragraph{Parameter} Die Funktion besitzt folgende Parameter:
\begin{table}[H]
	\begin{tabular}{|c|p{11cm}|}
		\hline
		\textbf{Parametername} & \textbf{Parameterbeschreibung} \\ \hline
		\$json & Strukturiertes Array \\ \hline
	\end{tabular}
\end{table}
\subparagraph{\$json}Das Array enthält folgende Elemente:
\begin{table}[H]
	\begin{tabular}{|c|p{11cm}|}
		\hline
		\textbf{Parametername} & \textbf{Parameterbeschreibung} \\ \hline
		IDent & Identifikator der zu löschenden historischen Adresse \\ \hline
	\end{tabular}
\end{table}
\paragraph{Beschreibung} Die Funktion dient der Löschung einer bestimmten historischen Adresse und hat Auswirkungen auf folgenden Quellen:
\begin{itemize}
	\item Tabelle mit Validierungsinformationen zu historischen Adressen
	\item Tabelle mit historischen Adressen
\end{itemize}
Es findet bei dieser Funktion kein Abruf von Daten aus {\glqq COSP\grqq} statt. Das Ergebnis ist stets erfolgreich.
\subsubsection{UpdateOperatorApi}
\paragraph{Parameter} Die Funktion besitzt folgende Parameter:
\begin{table}[H]
	\begin{tabular}{|c|p{11cm}|}
		\hline
		\textbf{Parametername} & \textbf{Parameterbeschreibung} \\ \hline
		\$json & Strukturiertes Array \\ \hline
	\end{tabular}
\end{table}
\subparagraph{\$json}Das Array enthält folgende Elemente:
\begin{table}[H]
	\begin{tabular}{|c|p{11cm}|}
		\hline
		\textbf{Parametername} & \textbf{Parameterbeschreibung} \\ \hline
		id & Identifikator des Betreibers \\ \hline
		from & Startjahr der Nutzung durch Betreiber \\ \hline
		till & Endjahr der Nutzung durch Betreiber \\ \hline
		operator & Name des Betreibers \\ \hline
	\end{tabular}
\end{table}
\paragraph{Beschreibung} Die Funktion dient dem Ändern eines Betreibers. Die Funktion hat Auswirkung auf folgenden Quellen:
\begin{itemize}
	\item Betreiber-Tabelle
	\item Tabelle mit Validierungsinformationen zu Betreibern
\end{itemize}
Es findet bei dieser Funktion kein Abruf von Daten aus {\glqq COSP\grqq} statt. Das Ergebnis ist stets erfolgreich.
\subsubsection{SaveNameNewAPI}
\paragraph{Parameter} Die Funktion besitzt folgende Parameter:
\begin{table}[H]
	\begin{tabular}{|c|p{11cm}|}
		\hline
		\textbf{Parametername} & \textbf{Parameterbeschreibung} \\ \hline
		\$json & Strukturiertes Array \\ \hline
	\end{tabular}
\end{table}
\subparagraph{\$json}Das Array enthält folgende Elemente:
\begin{table}[H]
	\begin{tabular}{|c|p{11cm}|}
		\hline
		\textbf{Parametername} & \textbf{Parameterbeschreibung} \\ \hline
		id & Identifikator des Namens \\ \hline
		from & Startjahr der Nutzung des Namens \\ \hline
		till & Endjahr der Nutzung des Namens \\ \hline
		name & Name \\ \hline
	\end{tabular}
\end{table}
\paragraph{Beschreibung} Die Funktion dient dem Ändern eines neuen Namens. Die Funktion hat Auswirkung auf folgenden Quellen:
\begin{itemize}
	\item Namen-Tabelle
	\item Tabelle mit Validierungsinformationen zu Namen
\end{itemize}
Es findet bei dieser Funktion kein Abruf von Daten aus {\glqq COSP\grqq} statt. Das Ergebnis ist stets erfolgreich.
\subsubsection{SaveHistoricalAddressNewAPI}
\paragraph{Parameter} Die Funktion besitzt folgende Parameter:
\begin{table}[H]
	\begin{tabular}{|c|p{11cm}|}
		\hline
		\textbf{Parametername} & \textbf{Parameterbeschreibung} \\ \hline
		\$json & Strukturiertes Array \\ \hline
	\end{tabular}
\end{table}
\subparagraph{\$json}Das Array enthält folgende Elemente:
\begin{table}[H]
	\begin{tabular}{|c|p{11cm}|}
		\hline
		\textbf{Parametername} & \textbf{Parameterbeschreibung} \\ \hline
		id & Identifikator der Adresse \\ \hline
		from & Startjahr der Nutzung der Adresse \\ \hline
		till & Endjahr der Nutzung der Adresse \\ \hline
		streetname & Straßenname der Adresse \\ \hline
		housenumber & Hausnummer der Adresse \\ \hline
		city & Ortsname der Adresse \\ \hline
		postalcode & Postleitzahl der Adresse \\ \hline
	\end{tabular}
\end{table}
\paragraph{Beschreibung} Die Funktion dient dem Ändern einer neuen historischen Adresse. Die Funktion hat Auswirkung auf folgenden Quellen:
\begin{itemize}
	\item Tabelle mit historischen Adressen
	\item Tabelle mit Validierungsinformationen zu historischen Adressen
\end{itemize}
Es findet bei dieser Funktion kein Abruf von Daten aus {\glqq COSP\grqq} statt. Das Ergebnis ist stets erfolgreich.
\subsubsection{deleteMaterialApi}
\paragraph{Parameter} Die Funktion besitzt folgende Parameter:
\begin{table}[H]
	\begin{tabular}{|c|p{11cm}|}
		\hline
		\textbf{Parametername} & \textbf{Parameterbeschreibung} \\ \hline
		\$json & Strukturiertes Array \\ \hline
	\end{tabular}
\end{table}
\subparagraph{\$json}Das Array enthält folgende Elemente:
\begin{table}[H]
	\begin{tabular}{|c|p{11cm}|}
		\hline
		\textbf{Parametername} & \textbf{Parameterbeschreibung} \\ \hline
		token & Identifikator des zu löschenden Bildes \\ \hline
	\end{tabular}
\end{table}
\paragraph{Beschreibung} Die Funktion dient dem Löschen eines Bildes und hat auf folgenden Quellen Auswirkungen:
\begin{itemize}
	\item COSP
\end{itemize}
Es findet bei dieser Funktion kein Abruf von Daten aus {\glqq COSP\grqq} statt. Es werden jedoch Daten an {\glqq COSP\grqq} gesendet. Das Ergebnis ist stets erfolgreich.
\subsubsection{getPoiTitleAPI}
\paragraph{Parameter} Die Funktion besitzt folgende Parameter:
\begin{table}[H]
	\begin{tabular}{|c|p{11cm}|}
		\hline
		\textbf{Parametername} & \textbf{Parameterbeschreibung} \\ \hline
		\$json & Strukturiertes Array \\ \hline
	\end{tabular}
\end{table}
\subparagraph{\$json}Das Array enthält folgende Elemente:
\begin{table}[H]
	\begin{tabular}{|c|p{11cm}|}
		\hline
		\textbf{Parametername} & \textbf{Parameterbeschreibung} \\ \hline
		storytoken & Identifikator einer bestimmten Geschichte \\ \hline
	\end{tabular}
\end{table}
\paragraph{Beschreibung} Die Funktion dient dem ermitteln aller Titel von nicht mit der Geschichte verknüpften Interessenpunkten und nutzt folgenden Quellen:
\begin{itemize}
	\item Tabelle mit Links zwischen Interessenpunkten und Geschichten
\end{itemize}
Es findet bei dieser Funktion kein Abruf von Daten aus {\glqq COSP\grqq} statt. Das Ergebnis ist stets erfolgreich.
\subsubsection{addStoryPoiLinkApi}
\paragraph{Parameter} Die Funktion besitzt folgende Parameter:
\begin{table}[H]
	\begin{tabular}{|c|p{11cm}|}
		\hline
		\textbf{Parametername} & \textbf{Parameterbeschreibung} \\ \hline
		\$json & Strukturiertes Array \\ \hline
	\end{tabular}
\end{table}
\subparagraph{\$json}Das Array enthält folgende Elemente:
\begin{table}[H]
	\begin{tabular}{|c|p{11cm}|}
		\hline
		\textbf{Parametername} & \textbf{Parameterbeschreibung} \\ \hline
		poiid & Identifikator des Interessenpunktes zu welchem Link mit Geschichte hinzugefügt werden sollen \\ \hline
		storytoken & Identifikator der Geschichte \\ \hline
	\end{tabular}
\end{table}
\paragraph{Beschreibung} Die Funktion dient dem Hinzufügen eines Links zwischen einer Geschichte und einem Interessenpunkt. Die Funktion hat Auswirkungen auf folgenden Quellen:
\begin{itemize}
	\item Tabelle mit Links zwischen Interessenpunkten und Geschichten
\end{itemize}
Es findet bei dieser Funktion kein Abruf von Daten aus {\glqq COSP\grqq} statt. Das Ergebnis ist stets erfolgreich.
\subsubsection{addStoryPoiLinkApi}
\paragraph{Parameter} Die Funktion besitzt folgende Parameter:
\begin{table}[H]
	\begin{tabular}{|c|p{11cm}|}
		\hline
		\textbf{Parametername} & \textbf{Parameterbeschreibung} \\ \hline
		\$json & Strukturiertes Array \\ \hline
	\end{tabular}
\end{table}
\subparagraph{\$json}Das Array enthält folgende Elemente:
\begin{table}[H]
	\begin{tabular}{|c|p{11cm}|}
		\hline
		\textbf{Parametername} & \textbf{Parameterbeschreibung} \\ \hline
		storytoken & Identifikator der Geschichte \\ \hline
	\end{tabular}
\end{table}
\paragraph{Beschreibung} Die Funktion dient der Abfrage eines Links zwischen einer Geschichte und einem Interessenpunkt. Die Funktion nutzt auf folgenden Quellen:
\begin{itemize}
	\item Tabelle mit Links zwischen Interessenpunkten und Geschichten
	\item Tabelle mit Validierungsinformationen zu Links zwischen Interessenpunkten und Geschichten
\end{itemize}
Es findet bei dieser Funktion kein Abruf von Daten aus {\glqq COSP\grqq} statt. Das Ergebnis ist stets erfolgreich.
\subsubsection{validatePoiStoryLinkDataApi}
\paragraph{Parameter} Die Funktion besitzt folgende Parameter:
\begin{table}[H]
	\begin{tabular}{|c|p{11cm}|}
		\hline
		\textbf{Parametername} & \textbf{Parameterbeschreibung} \\ \hline
		\$json & Strukturiertes Array \\ \hline
	\end{tabular}
\end{table}
\subparagraph{\$json}Das Array enthält folgende Elemente:
\begin{table}[H]
	\begin{tabular}{|c|p{11cm}|}
		\hline
		\textbf{Parametername} & \textbf{Parameterbeschreibung} \\ \hline
		poiStoryId & Identifikator des Links zwischen einem Interessenpunkte und einer Geschichte,welcher validiert werden sollen \\ \hline
	\end{tabular}
\end{table}
\paragraph{Beschreibung} Die Funktion dient dem Validieren eines Links zwischen einer Geschichte und einem Interessenpunkt. Die Funktion hat Auswirkungen auf folgende Quellen:
\begin{itemize}
	\item Tabelle mit Validierungsinformationen zu Links zwischen Interessenpunkten und Geschichten
\end{itemize}
Es findet bei dieser Funktion kein Abruf von Daten aus {\glqq COSP\grqq} statt. Es findet jedoch eine Übertragung von Daten an {\glqq COSP\grqq} statt. Das Ergebnis ist stets erfolgreich.
\subsubsection{deleteUserStoryApi}
\paragraph{Parameter} Die Funktion besitzt folgende Parameter:
\begin{table}[H]
	\begin{tabular}{|c|p{11cm}|}
		\hline
		\textbf{Parametername} & \textbf{Parameterbeschreibung} \\ \hline
		\$json & Strukturiertes Array \\ \hline
	\end{tabular}
\end{table}
\subparagraph{\$json}Das Array enthält folgende Elemente:
\begin{table}[H]
	\begin{tabular}{|c|p{11cm}|}
		\hline
		\textbf{Parametername} & \textbf{Parameterbeschreibung} \\ \hline
		story\_token & Identifikator der zu löschenden Geschichte \\ \hline
	\end{tabular}
\end{table}
\paragraph{Beschreibung} Die Funktion dient dem Löschen einer Geschichte und hat Auswirkungen auf folgende Quellen:
\begin{itemize}
	\item COSP
\end{itemize}
Es findet bei dieser Funktion kein Abruf von Daten aus {\glqq COSP\grqq} statt. Es findet jedoch eine Übertragung von Daten an {\glqq COSP\grqq} statt. Das Ergebnis ist stets erfolgreich.
\subsubsection{CheckAddressApi}
\paragraph{Parameter} Die Funktion besitzt folgende Parameter:
\begin{table}[H]
	\begin{tabular}{|c|p{11cm}|}
		\hline
		\textbf{Parametername} & \textbf{Parameterbeschreibung} \\ \hline
		\$json & Strukturiertes Array \\ \hline
	\end{tabular}
\end{table}
\subparagraph{\$json}Das Array enthält folgende Elemente:
\begin{table}[H]
	\begin{tabular}{|c|p{11cm}|}
		\hline
		\textbf{Parametername} & \textbf{Parameterbeschreibung} \\ \hline
		st & Straßenname einer Adresse \\ \hline
		hn & Hausnummer einer Adresse \\ \hline
		ct & Ortsname einer Adresse \\ \hline
		pc & Postleitzahl einer Adresse \\ \hline
	\end{tabular}
\end{table}
\paragraph{Beschreibung} Die Funktion dient der Prüfung, ob eine Adresse bereits dem System bekannt ist und nutzt hierfür folgende Quellen:
\begin{itemize}
	\item Interessenpunkt-Tabelle
	\item Tabelle mit historischen Adressen
\end{itemize}
Es findet bei dieser Funktion kein Abruf von Daten aus {\glqq COSP\grqq} statt. Das Ergebnis ist stets erfolgreich.
\subsubsection{getAllPicturesListAPI}
\paragraph{Parameter} Die Funktion besitzt folgende Parameter:
\begin{table}[H]
	\begin{tabular}{|c|p{11cm}|}
		\hline
		\textbf{Parametername} & \textbf{Parameterbeschreibung} \\ \hline
		\$incomplete & Legt fest ob die Rückgabe nicht an die Frontend-API erfolgt \\ \hline
	\end{tabular}
\end{table}
\paragraph{Beschreibung} Die Funktion dient dem ermitteln aller Daten für die Anzeige aller Bilder als Vorschau oder Vollbild. Hierfür werden folgende Quellen benutzt:
\begin{itemize}
	\item COSP
\end{itemize}
Es findet bei dieser Funktion ein Abruf von Daten aus {\glqq COSP\grqq} statt. Die Antwort wird als strukturiertes Array an den Aufrufer zurückgegeben.
\subsubsection{addPicturetoPoi}
\paragraph{Parameter} Die Funktion besitzt folgende Parameter:
\begin{table}[H]
	\begin{tabular}{|c|p{11cm}|}
		\hline
		\textbf{Parametername} & \textbf{Parameterbeschreibung} \\ \hline
		\$json & Strukturiertes Array \\ \hline
	\end{tabular}
\end{table}
\subparagraph{\$json}Das Array enthält folgende Elemente:
\begin{table}[H]
	\begin{tabular}{|c|p{11cm}|}
		\hline
		\textbf{Parametername} & \textbf{Parameterbeschreibung} \\ \hline
		data & alphanumerischer Bildidentifikator \\ \hline
		poi  & numerischer Identifikator des Interessenpunktes \\ \hline
	\end{tabular}
\end{table}
\paragraph{Beschreibung} Die Funktion fügt einem Interessenpunkt zusätzliche Bilder hinzu beziehungsweise fügt sie einem Bild verknüpfte Interessenpunkte hinzu. Die Funktion hat dabei Auswirkungen auf folgende Quellen:
\begin{itemize}
	\item Tabelle mit Links zwischen Bildern und Interessenpunkten
\end{itemize}
Es findet bei dieser Funktion kein Abruf von Daten aus {\glqq COSP\grqq} statt. Die Antwort wird als strukturiertes Array an den Aufrufer zurückgegeben.
\subsubsection{insertValidatePicturePoiApi}
\paragraph{Parameter} Die Funktion besitzt folgende Parameter:
\begin{table}[H]
	\begin{tabular}{|c|p{11cm}|}
		\hline
		\textbf{Parametername} & \textbf{Parameterbeschreibung} \\ \hline
		\$json & Strukturiertes Array \\ \hline
	\end{tabular}
\end{table}
\subparagraph{\$json}Das Array enthält folgende Elemente:
\begin{table}[H]
	\begin{tabular}{|c|p{11cm}|}
		\hline
		\textbf{Parametername} & \textbf{Parameterbeschreibung} \\ \hline
		id   & Identifikator eines Links zwischen einem Bild und einem Interessenpunkt \\ \hline
	\end{tabular}
\end{table}
\paragraph{Beschreibung} Die Funktion fügt eine Validierung einem Link zwischen einem Interessenpunkt und einer Geschichte hinzu. Die Funktion hat dabei Auswirkungen auf folgende Quellen:
\begin{itemize}
	\item Tabelle mit Validierungsinformationen zu Links zwischen Bildern und Interessenpunkten
\end{itemize}
Es findet bei dieser Funktion kein Abruf von Daten aus {\glqq COSP\grqq} statt. Es werden Daten an {\glqq COSP\grqq} übermittelt. Die Antwort wird als strukturiertes Array an den Aufrufer zurückgegeben und ist stetes erfolgreich.
\subsubsection{deletePoiPicLinkApi}
\paragraph{Parameter} Die Funktion besitzt folgende Parameter:
\begin{table}[H]
	\begin{tabular}{|c|p{11cm}|}
		\hline
		\textbf{Parametername} & \textbf{Parameterbeschreibung} \\ \hline
		\$json & Strukturiertes Array \\ \hline
	\end{tabular}
\end{table}
\subparagraph{\$json}Das Array enthält folgende Elemente:
\begin{table}[H]
	\begin{tabular}{|c|p{11cm}|}
		\hline
		\textbf{Parametername} & \textbf{Parameterbeschreibung} \\ \hline
		id   & Identifikator eines Links zwischen einem Bild und einem Interessenpunkt \\ \hline
	\end{tabular}
\end{table}
\paragraph{Beschreibung} Die Funktion entfernt einen Link zwischen einem Interessenpunkt und einem Bild. Die Funktion hat dabei Auswirkungen auf folgende Quellen:
\begin{itemize}
	\item Tabelle mit Links zwischen Bildern und Interessenpunkten
	\item Tabelle mit Validierungsinformationen zu Links zwischen Bildern und Interessenpunkten
\end{itemize}
Es findet bei dieser Funktion kein Abruf von Daten aus {\glqq COSP\grqq} statt. Die Antwort wird als strukturiertes Array an den Aufrufer zurückgegeben.
\subsubsection{loadPoiPicLinker}
\paragraph{Parameter} Die Funktion besitzt folgende Parameter:
\begin{table}[H]
	\begin{tabular}{|c|p{11cm}|}
		\hline
		\textbf{Parametername} & \textbf{Parameterbeschreibung} \\ \hline
		\$json & Strukturiertes Array \\ \hline
	\end{tabular}
\end{table}
\subparagraph{\$json}Das Array enthält folgende Elemente:
\begin{table}[H]
	\begin{tabular}{|c|p{11cm}|}
		\hline
		\textbf{Parametername} & \textbf{Parameterbeschreibung} \\ \hline
		pictoken  & alphanumerischer Identifikator eines Bildes \\ \hline
	\end{tabular}
\end{table}
\paragraph{Beschreibung} Die Funktion lädt Verknüpfungsdaten von einem Bild mit Interessenpunkten und liefert sowohl eine Liste der Verknüpften als auch der unverknüpften Interessenpunkte zurück. Die Funktion nutzt folgende Quellen:
\begin{itemize}
	\item Tabelle mit Links zwischen Bildern und Interessenpunkten
	\item Tabelle mit Validierungsinformationen zu Links zwischen Bildern und Interessenpunkten
\end{itemize}
Es findet bei dieser Funktion kein Abruf von Daten aus {\glqq COSP\grqq} statt. Die Antwort wird als strukturiertes Array an den Aufrufer zurückgegeben und ist stets erfolgreich.
\subsubsection{getUapiUrl}
\paragraph{Parameter} Die Funktion besitzt keine Parameter.
\paragraph{Beschreibung} Die Funktion sendet die Adresse der COSP User-API an das Frontend. Die Funktion nutzt folgende Quellen:
\begin{itemize}
	\item Konfigurationsdatei
\end{itemize}
Es findet bei dieser Funktion kein Abruf von Daten aus {\glqq COSP\grqq} statt. Die Antwort wird als strukturiertes Array an den Aufrufer zurückgegeben und ist stets erfolgreich.
\subsubsection{SaveSeatsNewAPI}
\paragraph{Parameter} Die Funktion besitzt folgende Parameter:
\begin{table}[H]
	\begin{tabular}{|c|p{11cm}|}
		\hline
		\textbf{Parametername} & \textbf{Parameterbeschreibung} \\ \hline
		\$json & Strukturiertes Array \\ \hline
	\end{tabular}
\end{table}
\subparagraph{\$json}Das Array enthält folgende Elemente:
\begin{table}[H]
	\begin{tabular}{|c|p{11cm}|}
		\hline
		\textbf{Parametername} & \textbf{Parameterbeschreibung} \\ \hline
		poi\_id  & Identifikator des Interessenpunktes \\ \hline
		from     & Startjahr \\ \hline
		till     & Endjahr \\ \hline
		seats    & Sitzplatzanzahl \\ \hline
	\end{tabular}
\end{table}
\paragraph{Beschreibung} Die Funktion speichert eine neue Sitzplatzanzahl eines Interessenpunktes. Die Funktion hat Auswirkung auf folgende Quellen:
\begin{itemize}
	\item Tabelle mit Sitzplatzanzahlen
\end{itemize}
Es findet bei dieser Funktion kein Abruf von Daten aus {\glqq COSP\grqq} statt. Die Antwort wird als strukturiertes Array an den Aufrufer zurückgegeben und ist stets erfolgreich.
\subsubsection{validatePoiSeatsApi}
\paragraph{Parameter} Die Funktion besitzt folgende Parameter:
\begin{table}[H]
	\begin{tabular}{|c|p{11cm}|}
		\hline
		\textbf{Parametername} & \textbf{Parameterbeschreibung} \\ \hline
		\$json & Strukturiertes Array \\ \hline
	\end{tabular}
\end{table}
\subparagraph{\$json}Das Array enthält folgende Elemente:
\begin{table}[H]
	\begin{tabular}{|c|p{11cm}|}
		\hline
		\textbf{Parametername} & \textbf{Parameterbeschreibung} \\ \hline
		SEATID   & Identifikator der Sitzplatzanzahl \\ \hline
	\end{tabular}
\end{table}
\paragraph{Beschreibung} Die Funktion validiert eine Sitzplatzanzahl eines Interessenpunktes. Die Funktion hat Auswirkung auf folgende Quellen:
\begin{itemize}
	\item Tabelle mit Validierungsinformationen zu Sitzplatzanzahlen
\end{itemize}
Es findet bei dieser Funktion kein Abruf von Daten aus {\glqq COSP\grqq} statt. Es werden jedoch Daten an {\glqq COSP\grqq} gesendet. Die Antwort wird als strukturiertes Array an den Aufrufer zurückgegeben und ist stets erfolgreich.
\subsubsection{deleteSeatsApi}
\paragraph{Parameter} Die Funktion besitzt folgende Parameter:
\begin{table}[H]
	\begin{tabular}{|c|p{11cm}|}
		\hline
		\textbf{Parametername} & \textbf{Parameterbeschreibung} \\ \hline
		\$json & Strukturiertes Array \\ \hline
	\end{tabular}
\end{table}
\subparagraph{\$json}Das Array enthält folgende Elemente:
\begin{table}[H]
	\begin{tabular}{|c|p{11cm}|}
		\hline
		\textbf{Parametername} & \textbf{Parameterbeschreibung} \\ \hline
		IDent    & Identifikator der Sitzplatzanzahl \\ \hline
	\end{tabular}
\end{table}
\paragraph{Beschreibung} Die Funktion löscht eine Sitzplatzanzahl eines Interessenpunktes. Die Funktion hat Auswirkung auf folgende Quellen:
\begin{itemize}
	\item Tabelle mit Sitzplatzanzahlen
	\item Tabelle mit Validierungsinformationen zu Sitzplatzanzahlen
\end{itemize}
Es findet bei dieser Funktion kein Abruf von Daten aus {\glqq COSP\grqq} statt. Die Antwort wird als strukturiertes Array an den Aufrufer zurückgegeben und ist stets erfolgreich.
\subsubsection{UpdateSeatsApi}
\paragraph{Parameter} Die Funktion besitzt folgende Parameter:
\begin{table}[H]
	\begin{tabular}{|c|p{11cm}|}
		\hline
		\textbf{Parametername} & \textbf{Parameterbeschreibung} \\ \hline
		\$json & Strukturiertes Array \\ \hline
	\end{tabular}
\end{table}
\subparagraph{\$json}Das Array enthält folgende Elemente:
\begin{table}[H]
	\begin{tabular}{|c|p{11cm}|}
		\hline
		\textbf{Parametername} & \textbf{Parameterbeschreibung} \\ \hline
		id       & Identifikator der Sitzplatzanzahl \\ \hline
		from     & Startjahr \\ \hline
		till     & Endjahr \\ \hline
		seats    & Sitzplatzanzahl \\ \hline
	\end{tabular}
\end{table}
\paragraph{Beschreibung} Die Funktion aktualisiert eine Sitzplatzanzahl. Die Funktion hat Auswirkung auf folgende Quellen:
\begin{itemize}
	\item Tabelle mit Sitzplatzanzahlen
	\item Tabelle mit Validierungsinformationen zu Sitzplatzanzahlen
\end{itemize}
Es findet bei dieser Funktion kein Abruf von Daten aus {\glqq COSP\grqq} statt. Die Antwort wird als strukturiertes Array an den Aufrufer zurückgegeben und ist stets erfolgreich.
\subsubsection{SaveCinemasNewAPI}
\paragraph{Parameter} Die Funktion besitzt folgende Parameter:
\begin{table}[H]
	\begin{tabular}{|c|p{11cm}|}
		\hline
		\textbf{Parametername} & \textbf{Parameterbeschreibung} \\ \hline
		\$json & Strukturiertes Array \\ \hline
	\end{tabular}
\end{table}
\subparagraph{\$json}Das Array enthält folgende Elemente:
\begin{table}[H]
	\begin{tabular}{|c|p{11cm}|}
		\hline
		\textbf{Parametername} & \textbf{Parameterbeschreibung} \\ \hline
		poi\_id  & Identifikator des Interessenpunktes \\ \hline
		from     & Startjahr \\ \hline
		till     & Endjahr \\ \hline
		cinemas  & Saalanzahl \\ \hline
	\end{tabular}
\end{table}
\paragraph{Beschreibung} Die Funktion speichert eine neue Saalanzahl eines Interessenpunktes. Die Funktion hat Auswirkung auf folgende Quellen:
\begin{itemize}
	\item Tabelle mit Saalanzahlen
\end{itemize}
Es findet bei dieser Funktion kein Abruf von Daten aus {\glqq COSP\grqq} statt. Die Antwort wird als strukturiertes Array an den Aufrufer zurückgegeben und ist stets erfolgreich.
\subsubsection{validatePoiCinemasApi}
\paragraph{Parameter} Die Funktion besitzt folgende Parameter:
\begin{table}[H]
	\begin{tabular}{|c|p{11cm}|}
		\hline
		\textbf{Parametername} & \textbf{Parameterbeschreibung} \\ \hline
		\$json & Strukturiertes Array \\ \hline
	\end{tabular}
\end{table}
\subparagraph{\$json}Das Array enthält folgende Elemente:
\begin{table}[H]
	\begin{tabular}{|c|p{11cm}|}
		\hline
		\textbf{Parametername} & \textbf{Parameterbeschreibung} \\ \hline
		CINEMAID   & Identifikator der Saalanzahl \\ \hline
	\end{tabular}
\end{table}
\paragraph{Beschreibung} Die Funktion validiert eine Saalanzahl eines Interessenpunktes. Die Funktion hat Auswirkung auf folgende Quellen:
\begin{itemize}
	\item Tabelle mit Validierungsinformationen zu Saalanzahlen
\end{itemize}
Es findet bei dieser Funktion kein Abruf von Daten aus {\glqq COSP\grqq} statt. Die Antwort wird als strukturiertes Array an den Aufrufer zurückgegeben und ist stets erfolgreich.
\subsubsection{deleteCinemasApi}
\paragraph{Parameter} Die Funktion besitzt folgende Parameter:
\begin{table}[H]
	\begin{tabular}{|c|p{11cm}|}
		\hline
		\textbf{Parametername} & \textbf{Parameterbeschreibung} \\ \hline
		\$json & Strukturiertes Array \\ \hline
	\end{tabular}
\end{table}
\subparagraph{\$json}Das Array enthält folgende Elemente:
\begin{table}[H]
	\begin{tabular}{|c|p{11cm}|}
		\hline
		\textbf{Parametername} & \textbf{Parameterbeschreibung} \\ \hline
		IDent    & Identifikator der Saalanzahl \\ \hline
	\end{tabular}
\end{table}
\paragraph{Beschreibung} Die Funktion löscht eine Saalanzahl eines Interessenpunktes. Die Funktion hat Auswirkung auf folgende Quellen:
\begin{itemize}
	\item Tabelle mit Saalanzahlen
	\item Tabelle mit Validierungsinformationen zu Saalanzahlen
\end{itemize}
Es findet bei dieser Funktion kein Abruf von Daten aus {\glqq COSP\grqq} statt. Die Antwort wird als strukturiertes Array an den Aufrufer zurückgegeben und ist stets erfolgreich.
\subsubsection{UpdateCinemasApi}
\paragraph{Parameter} Die Funktion besitzt folgende Parameter:
\begin{table}[H]
	\begin{tabular}{|c|p{11cm}|}
		\hline
		\textbf{Parametername} & \textbf{Parameterbeschreibung} \\ \hline
		\$json & Strukturiertes Array \\ \hline
	\end{tabular}
\end{table}
\subparagraph{\$json}Das Array enthält folgende Elemente:
\begin{table}[H]
	\begin{tabular}{|c|p{11cm}|}
		\hline
		\textbf{Parametername} & \textbf{Parameterbeschreibung} \\ \hline
		id       & Identifikator der Saalanzahl \\ \hline
		from     & Startjahr \\ \hline
		till     & Endjahr \\ \hline
		cinemas  & Saalanzahl \\ \hline
	\end{tabular}
\end{table}
\paragraph{Beschreibung} Die Funktion aktualisiert eine Saalanzahl. Die Funktion hat Auswirkung auf folgende Quellen:
\begin{itemize}
	\item Tabelle mit Saalanzahlen
	\item Tabelle mit Validierungsinformationen zu Saalanzahlen
\end{itemize}
Es findet bei dieser Funktion kein Abruf von Daten aus {\glqq COSP\grqq} statt. Die Antwort wird als strukturiertes Array an den Aufrufer zurückgegeben und ist stets erfolgreich.
\subsubsection{validateTypeApi}
\paragraph{Parameter} Die Funktion besitzt folgende Parameter:
\begin{table}[H]
	\begin{tabular}{|c|p{11cm}|}
		\hline
		\textbf{Parametername} & \textbf{Parameterbeschreibung} \\ \hline
		\$json & Strukturiertes Array \\ \hline
	\end{tabular}
\end{table}
\subparagraph{\$json}Das Array enthält folgende Elemente:
\begin{table}[H]
	\begin{tabular}{|c|p{11cm}|}
		\hline
		\textbf{Parametername} & \textbf{Parameterbeschreibung} \\ \hline
		POIID & Identifikator eines Interessenpunktes \\ \hline
	\end{tabular}
\end{table}
\paragraph{Beschreibung} Die Funktion validiert den Typ eines Interessenpunktes. Die Funktion hat Auswirkungen auf folgende Quellen:
\begin{itemize}
	\item Tabelle mit Validierungsdaten zum Typ eines Interessenpunktes
\end{itemize}
Es findet bei dieser Funktion kein Abruf von Daten aus {\glqq COSP\grqq} statt. Die Antwort wird als strukturiertes Array an den Aufrufer zurückgegeben.
\subsubsection{isUserGuest}
\paragraph{Parameter} Die Funktion besitzt keine Parameter.
\paragraph{Beschreibung} Die Funktion bestimmt, ob ein Nutzer die Rolle {\glqq Gast\grqq} hat. Es findet bei dieser Funktion kein Abruf von Daten aus {\glqq COSP\grqq} statt. Die Antwort wird als strukturiertes Array an den Aufrufer zurückgegeben.
\subsubsection{getStatisticalDataAPI}
\paragraph{Parameter} Die Funktion besitzt folgende Parameter:
\begin{table}[H]
	\begin{tabular}{|c|p{11cm}|}
		\hline
		\textbf{Parametername} & \textbf{Parameterbeschreibung} \\ \hline
		\$json & Strukturiertes Array \\ \hline
	\end{tabular}
\end{table}
\subparagraph{\$json}Das Array enthält folgende Elemente:
\begin{table}[H]
	\begin{tabular}{|c|p{11cm}|}
		\hline
		\textbf{Parametername} & \textbf{Parameterbeschreibung} \\ \hline
		data & Strukturiertes Array \\ \hline
	\end{tabular}
\end{table}
\subparagraph{data}Das Array enthält unter anderem folgende Elemente:
\begin{table}[H]
	\begin{tabular}{|c|p{11cm}|}
		\hline
		\textbf{Parametername} & \textbf{Parameterbeschreibung} \\ \hline
		src & Quelle der statistischen Daten \\ \hline
	\end{tabular}
\end{table}
\paragraph{Beschreibung} Die Funktion dient dem ermitteln statistischer Daten für Statistik-Seite. Die Funktion nutzt folgende Quellen:
\begin{itemize}
	\item Tabelle mit statistischen Daten
	\item Tabelle mit Kommentaren
	\item Tabelle mit Interessenpunkten
\end{itemize}
Es findet bei dieser Funktion kein Abruf von Daten aus {\glqq COSP\grqq} statt. Die Antwort wird als strukturiertes Array an den Aufrufer zurückgegeben.
\subsubsection{approveUserStoryAPI}
\paragraph{Parameter} Die Funktion besitzt folgende Parameter:
\begin{table}[H]
	\begin{tabular}{|c|p{11cm}|}
		\hline
		\textbf{Parametername} & \textbf{Parameterbeschreibung} \\ \hline
		\$json & Strukturiertes Array \\ \hline
	\end{tabular}
\end{table}
\subparagraph{\$json}Das Array enthält folgende Elemente:
\begin{table}[H]
	\begin{tabular}{|c|p{11cm}|}
		\hline
		\textbf{Parametername} & \textbf{Parameterbeschreibung} \\ \hline
		story\_token & alphanumerischer Identifikator einer Geschichte \\ \hline
	\end{tabular}
\end{table}
\paragraph{Beschreibung} Die Funktion die Funktion schaltet eine Nutzergeschichte frei. Die Funktion hat Auswirkungen auf folgende Quellen:
\begin{itemize}
	\item COSP
\end{itemize}
Es findet bei dieser Funktion kein Abruf von Daten aus {\glqq COSP\grqq} statt. Es werden jedoch Daten an {\glqq COSP\grqq} übermittelt. Die Antwort wird als strukturiertes Array an den Aufrufer zurückgegeben.
\subsubsection{disapproveUserStoryAPI}
\paragraph{Parameter} Die Funktion besitzt folgende Parameter:
\begin{table}[H]
	\begin{tabular}{|c|p{11cm}|}
		\hline
		\textbf{Parametername} & \textbf{Parameterbeschreibung} \\ \hline
		\$json & Strukturiertes Array \\ \hline
	\end{tabular}
\end{table}
\subparagraph{\$json}Das Array enthält folgende Elemente:
\begin{table}[H]
	\begin{tabular}{|c|p{11cm}|}
		\hline
		\textbf{Parametername} & \textbf{Parameterbeschreibung} \\ \hline
		story\_token & alphanumerischer Identifikator einer Geschichte \\ \hline
	\end{tabular}
\end{table}
\paragraph{Beschreibung} Die Funktion die Funktion sperrt eine Nutzergeschichte. Die Funktion hat Auswirkungen auf folgende Quellen:
\begin{itemize}
	\item COSP
\end{itemize}
Es findet bei dieser Funktion kein Abruf von Daten aus {\glqq COSP\grqq} statt. Es werden jedoch Daten an {\glqq COSP\grqq} übermittelt. Die Antwort wird als strukturiertes Array an den Aufrufer zurückgegeben.
\subsubsection{GetStateOfStories}
\paragraph{Parameter} Die Funktion besitzt keine Parameter.
\paragraph{Beschreibung} Die Funktion fragt die Freischaltung der Funktion {\glqq Nutzergeschichten\grqq} ab. Die Funktion nutzt folgende Quellen:
\begin{itemize}
	\item Konfigurationsdatei
\end{itemize}
Es findet bei dieser Funktion kein Abruf von Daten aus {\glqq COSP\grqq} statt. Die Antwort wird als strukturiertes Array an den Aufrufer zurückgegeben.
\subsubsection{GetCaptchaAPI}
\paragraph{Parameter} Die Funktion besitzt keine Parameter.
\paragraph{Beschreibung} Die Funktion stellt einen Captcha-Code zur Verfügung. Die Funktion nutzt folgende Quellen:
\begin{itemize}
	\item COSP
\end{itemize}
Es findet bei dieser Funktion ein Abruf von Daten aus {\glqq COSP\grqq} statt. Die Antwort wird als strukturiertes Array an den Aufrufer zurückgegeben.
\subsubsection{sendContactMessageAPI}
\paragraph{Parameter} Die Funktion besitzt folgende Parameter:
\begin{table}[H]
	\begin{tabular}{|c|p{11cm}|}
		\hline
		\textbf{Parametername} & \textbf{Parameterbeschreibung} \\ \hline
		\$json & Strukturiertes Array \\ \hline
	\end{tabular}
\end{table}
\subparagraph{\$json}Das Array enthält folgende Elemente:
\begin{table}[H]
	\begin{tabular}{|c|p{11cm}|}
		\hline
		\textbf{Parametername} & \textbf{Parameterbeschreibung} \\ \hline
		cap   & Inhalt des Captchas \\ \hline
		email & E-Mailadresse des Senders \\ \hline
		msg   & Inhalt der Nachricht \\ \hline
		title & Titel der Nachricht \\ \hline
	\end{tabular}
\end{table}
\paragraph{Beschreibung} Die Funktion sendet eine Kontaktnachricht an die Administratoren der Website. Es findet bei dieser Funktion kein Abruf von Daten aus {\glqq COSP\grqq} statt. Es werden jedoch Daten an {\glqq COSP\grqq} übermittelt. Die Antwort wird als strukturiertes Array an den Aufrufer zurückgegeben.
\subsubsection{finalDeletePoiPic}
\paragraph{Parameter} Die Funktion besitzt folgende Parameter:
\begin{table}[H]
	\begin{tabular}{|c|p{11cm}|}
		\hline
		\textbf{Parametername} & \textbf{Parameterbeschreibung} \\ \hline
		\$json & Strukturiertes Array \\ \hline
	\end{tabular}
\end{table}
\subparagraph{\$json}Das Array enthält folgende Elemente:
\begin{table}[H]
	\begin{tabular}{|c|p{11cm}|}
		\hline
		\textbf{Parametername} & \textbf{Parameterbeschreibung} \\ \hline
		IDent & Identifikator des Links zwischen Bild und Interessenpunkt \\ \hline
	\end{tabular}
\end{table}
\paragraph{Beschreibung} Die Funktion dient dem endgültigen löschen eines Links zwischen einem Interessenpunkt und einem Bild. Die Funktion hat Auswirkungen auf folgende Quellen:
\begin{itemize}
	\item Tabelle mit Links zwischen Interessenpunkten und Bildern
	\item Tabelle mit Validierungsinformationen zu Links zwischen Interessenpunkten und Bildern
\end{itemize}
Es findet bei dieser Funktion kein Abruf von Daten aus {\glqq COSP\grqq} statt. Die Antwort wird als strukturiertes Array an den Aufrufer zurückgegeben.
\subsubsection{RestorePoiPicLink}
\paragraph{Parameter} Die Funktion besitzt folgende Parameter:
\begin{table}[H]
	\begin{tabular}{|c|p{11cm}|}
		\hline
		\textbf{Parametername} & \textbf{Parameterbeschreibung} \\ \hline
		\$json & Strukturiertes Array \\ \hline
	\end{tabular}
\end{table}
\subparagraph{\$json}Das Array enthält folgende Elemente:
\begin{table}[H]
	\begin{tabular}{|c|p{11cm}|}
		\hline
		\textbf{Parametername} & \textbf{Parameterbeschreibung} \\ \hline
		IDent & Identifikator des Links zwischen Bild und Interessenpunkt \\ \hline
	\end{tabular}
\end{table}
\paragraph{Beschreibung} Die Funktion dient dem wiederherstellen gelöschter eines Links zwischen einem Interessenpunkt und einem Bild. Die Funktion hat Auswirkungen auf folgende Quellen:
\begin{itemize}
	\item Tabelle mit Links zwischen Interessenpunkten und Bildern
\end{itemize}
Es findet bei dieser Funktion kein Abruf von Daten aus {\glqq COSP\grqq} statt. Die Antwort wird als strukturiertes Array an den Aufrufer zurückgegeben.
\subsubsection{RestorePoiName}
\paragraph{Parameter} Die Funktion besitzt folgende Parameter:
\begin{table}[H]
	\begin{tabular}{|c|p{11cm}|}
		\hline
		\textbf{Parametername} & \textbf{Parameterbeschreibung} \\ \hline
		\$json & Strukturiertes Array \\ \hline
	\end{tabular}
\end{table}
\subparagraph{\$json}Das Array enthält folgende Elemente:
\begin{table}[H]
	\begin{tabular}{|c|p{11cm}|}
		\hline
		\textbf{Parametername} & \textbf{Parameterbeschreibung} \\ \hline
		IDent & Identifikator eines Namen \\ \hline
	\end{tabular}
\end{table}
\paragraph{Beschreibung} Die Funktion dient dem wiederherstellen eines gelöschten Namens. Die Funktion hat Auswirkungen auf folgende Quellen:
\begin{itemize}
	\item Tabelle mit Namen
\end{itemize}
Es findet bei dieser Funktion kein Abruf von Daten aus {\glqq COSP\grqq} statt. Die Antwort wird als strukturiertes Array an den Aufrufer zurückgegeben.
\subsubsection{FinalDeletePoiName}
\paragraph{Parameter} Die Funktion besitzt folgende Parameter:
\begin{table}[H]
	\begin{tabular}{|c|p{11cm}|}
		\hline
		\textbf{Parametername} & \textbf{Parameterbeschreibung} \\ \hline
		\$json & Strukturiertes Array \\ \hline
	\end{tabular}
\end{table}
\subparagraph{\$json}Das Array enthält folgende Elemente:
\begin{table}[H]
	\begin{tabular}{|c|p{11cm}|}
		\hline
		\textbf{Parametername} & \textbf{Parameterbeschreibung} \\ \hline
		IDent & Identifikator eines Namen \\ \hline
	\end{tabular}
\end{table}
\paragraph{Beschreibung} Die Funktion dient dem endgültigen löschen eines Namens. Die Funktion hat Auswirkungen auf folgende Quellen:
\begin{itemize}
	\item Tabelle mit Namen
	\item Tabelle mit Validierungsinformationen zu Namen
\end{itemize}
Es findet bei dieser Funktion kein Abruf von Daten aus {\glqq COSP\grqq} statt. Die Antwort wird als strukturiertes Array an den Aufrufer zurückgegeben.
\subsubsection{RestorePoiOperator}
\paragraph{Parameter} Die Funktion besitzt folgende Parameter:
\begin{table}[H]
	\begin{tabular}{|c|p{11cm}|}
		\hline
		\textbf{Parametername} & \textbf{Parameterbeschreibung} \\ \hline
		\$json & Strukturiertes Array \\ \hline
	\end{tabular}
\end{table}
\subparagraph{\$json}Das Array enthält folgende Elemente:
\begin{table}[H]
	\begin{tabular}{|c|p{11cm}|}
		\hline
		\textbf{Parametername} & \textbf{Parameterbeschreibung} \\ \hline
		IDent & Identifikator eines Betreibers \\ \hline
	\end{tabular}
\end{table}
\paragraph{Beschreibung} Die Funktion dient dem wiederherstellen eines gelöschten Betreibers. Die Funktion hat Auswirkungen auf folgende Quellen:
\begin{itemize}
	\item Tabelle mit Betreibern
\end{itemize}
Es findet bei dieser Funktion kein Abruf von Daten aus {\glqq COSP\grqq} statt. Die Antwort wird als strukturiertes Array an den Aufrufer zurückgegeben.
\subsubsection{FinalDeletePoiOperator}
\paragraph{Parameter} Die Funktion besitzt folgende Parameter:
\begin{table}[H]
	\begin{tabular}{|c|p{11cm}|}
		\hline
		\textbf{Parametername} & \textbf{Parameterbeschreibung} \\ \hline
		\$json & Strukturiertes Array \\ \hline
	\end{tabular}
\end{table}
\subparagraph{\$json}Das Array enthält folgende Elemente:
\begin{table}[H]
	\begin{tabular}{|c|p{11cm}|}
		\hline
		\textbf{Parametername} & \textbf{Parameterbeschreibung} \\ \hline
		IDent & Identifikator eines Betreibers \\ \hline
	\end{tabular}
\end{table}
\paragraph{Beschreibung} Die Funktion dient dem endgültigen löschen eines Betreibers. Die Funktion hat Auswirkungen auf folgende Quellen:
\begin{itemize}
	\item Tabelle mit Betreibern
	\item Tabelle mit Validierungsinformationen zu Betreibern
\end{itemize}
Es findet bei dieser Funktion kein Abruf von Daten aus {\glqq COSP\grqq} statt. Die Antwort wird als strukturiertes Array an den Aufrufer zurückgegeben.
\subsubsection{RestorePoiSeats}
\paragraph{Parameter} Die Funktion besitzt folgende Parameter:
\begin{table}[H]
	\begin{tabular}{|c|p{11cm}|}
		\hline
		\textbf{Parametername} & \textbf{Parameterbeschreibung} \\ \hline
		\$json & Strukturiertes Array \\ \hline
	\end{tabular}
\end{table}
\subparagraph{\$json}Das Array enthält folgende Elemente:
\begin{table}[H]
	\begin{tabular}{|c|p{11cm}|}
		\hline
		\textbf{Parametername} & \textbf{Parameterbeschreibung} \\ \hline
		IDent & Identifikator einer Sitzplatzanzahl \\ \hline
	\end{tabular}
\end{table}
\paragraph{Beschreibung} Die Funktion dient dem wiederherstellen einer gelöschten Sitzplatzanzahl. Die Funktion hat Auswirkungen auf folgende Quellen:
\begin{itemize}
	\item Tabelle mit Sitzplatzanzahlen
\end{itemize}
Es findet bei dieser Funktion kein Abruf von Daten aus {\glqq COSP\grqq} statt. Die Antwort wird als strukturiertes Array an den Aufrufer zurückgegeben.
\subsubsection{FinalDeletePoiSeats}
\paragraph{Parameter} Die Funktion besitzt folgende Parameter:
\begin{table}[H]
	\begin{tabular}{|c|p{11cm}|}
		\hline
		\textbf{Parametername} & \textbf{Parameterbeschreibung} \\ \hline
		\$json & Strukturiertes Array \\ \hline
	\end{tabular}
\end{table}
\subparagraph{\$json}Das Array enthält folgende Elemente:
\begin{table}[H]
	\begin{tabular}{|c|p{11cm}|}
		\hline
		\textbf{Parametername} & \textbf{Parameterbeschreibung} \\ \hline
		IDent & Identifikator einer Sitzplatzanzahl \\ \hline
	\end{tabular}
\end{table}
\paragraph{Beschreibung} Die Funktion dient dem endgültigen löschen einer Sitzplatzanzahl. Die Funktion hat Auswirkungen auf folgende Quellen:
\begin{itemize}
	\item Tabelle mit Sitzplatzanzahlen
	\item Tabelle mit Validierungsinformationen zu Sitzplatzanzahlen
\end{itemize}
Es findet bei dieser Funktion kein Abruf von Daten aus {\glqq COSP\grqq} statt. Die Antwort wird als strukturiertes Array an den Aufrufer zurückgegeben.
\subsubsection{RestorePoiCinemas}
\paragraph{Parameter} Die Funktion besitzt folgende Parameter:
\begin{table}[H]
	\begin{tabular}{|c|p{11cm}|}
		\hline
		\textbf{Parametername} & \textbf{Parameterbeschreibung} \\ \hline
		\$json & Strukturiertes Array \\ \hline
	\end{tabular}
\end{table}
\subparagraph{\$json}Das Array enthält folgende Elemente:
\begin{table}[H]
	\begin{tabular}{|c|p{11cm}|}
		\hline
		\textbf{Parametername} & \textbf{Parameterbeschreibung} \\ \hline
		IDent & Identifikator einer Saalanzahl \\ \hline
	\end{tabular}
\end{table}
\paragraph{Beschreibung} Die Funktion dient dem wiederherstellen einer gelöschten Saalanzahl. Die Funktion hat Auswirkungen auf folgende Quellen:
\begin{itemize}
	\item Tabelle mit Saalanzahlen
\end{itemize}
Es findet bei dieser Funktion kein Abruf von Daten aus {\glqq COSP\grqq} statt. Die Antwort wird als strukturiertes Array an den Aufrufer zurückgegeben.
\subsubsection{FinalDeletePoiCinemas}
\paragraph{Parameter} Die Funktion besitzt folgende Parameter:
\begin{table}[H]
	\begin{tabular}{|c|p{11cm}|}
		\hline
		\textbf{Parametername} & \textbf{Parameterbeschreibung} \\ \hline
		\$json & Strukturiertes Array \\ \hline
	\end{tabular}
\end{table}
\subparagraph{\$json}Das Array enthält folgende Elemente:
\begin{table}[H]
	\begin{tabular}{|c|p{11cm}|}
		\hline
		\textbf{Parametername} & \textbf{Parameterbeschreibung} \\ \hline
		IDent & Identifikator einer Saalanzahl \\ \hline
	\end{tabular}
\end{table}
\paragraph{Beschreibung} Die Funktion dient dem endgültigen löschen einer Saalanzahl. Die Funktion hat Auswirkungen auf folgende Quellen:
\begin{itemize}
	\item Tabelle mit Saalanzahlen
	\item Tabelle mit Validierungsinformationen zu Saalanzahlen
\end{itemize}
Es findet bei dieser Funktion kein Abruf von Daten aus {\glqq COSP\grqq} statt. Die Antwort wird als strukturiertes Array an den Aufrufer zurückgegeben.
\subsubsection{RestorePoiHistAddr}
\paragraph{Parameter} Die Funktion besitzt folgende Parameter:
\begin{table}[H]
	\begin{tabular}{|c|p{11cm}|}
		\hline
		\textbf{Parametername} & \textbf{Parameterbeschreibung} \\ \hline
		\$json & Strukturiertes Array \\ \hline
	\end{tabular}
\end{table}
\subparagraph{\$json}Das Array enthält folgende Elemente:
\begin{table}[H]
	\begin{tabular}{|c|p{11cm}|}
		\hline
		\textbf{Parametername} & \textbf{Parameterbeschreibung} \\ \hline
		IDent & Identifikator einer historischen Adresse \\ \hline
	\end{tabular}
\end{table}
\paragraph{Beschreibung} Die Funktion dient dem wiederherstellen einer gelöschten historischen Adresse. Die Funktion hat Auswirkungen auf folgende Quellen:
\begin{itemize}
	\item Tabelle mit historischen Adressen
\end{itemize}
Es findet bei dieser Funktion kein Abruf von Daten aus {\glqq COSP\grqq} statt. Die Antwort wird als strukturiertes Array an den Aufrufer zurückgegeben.
\subsubsection{FinalDeletePoiHistAddr}
\paragraph{Parameter} Die Funktion besitzt folgende Parameter:
\begin{table}[H]
	\begin{tabular}{|c|p{11cm}|}
		\hline
		\textbf{Parametername} & \textbf{Parameterbeschreibung} \\ \hline
		\$json & Strukturiertes Array \\ \hline
	\end{tabular}
\end{table}
\subparagraph{\$json}Das Array enthält folgende Elemente:
\begin{table}[H]
	\begin{tabular}{|c|p{11cm}|}
		\hline
		\textbf{Parametername} & \textbf{Parameterbeschreibung} \\ \hline
		IDent & Identifikator einer historischen Adresse \\ \hline
	\end{tabular}
\end{table}
\paragraph{Beschreibung} Die Funktion dient dem endgültigen löschen einer historischen Adresse. Die Funktion hat Auswirkungen auf folgende Quellen:
\begin{itemize}
	\item Tabelle mit historischen Adressen
	\item Tabelle mit Validierungsinformationen zu historischen Adressen
\end{itemize}
Es findet bei dieser Funktion kein Abruf von Daten aus {\glqq COSP\grqq} statt. Die Antwort wird als strukturiertes Array an den Aufrufer zurückgegeben.
\subsubsection{RestorePoiStoryLink}
\paragraph{Parameter} Die Funktion besitzt folgende Parameter:
\begin{table}[H]
	\begin{tabular}{|c|p{11cm}|}
		\hline
		\textbf{Parametername} & \textbf{Parameterbeschreibung} \\ \hline
		\$json & Strukturiertes Array \\ \hline
	\end{tabular}
\end{table}
\subparagraph{\$json}Das Array enthält folgende Elemente:
\begin{table}[H]
	\begin{tabular}{|c|p{11cm}|}
		\hline
		\textbf{Parametername} & \textbf{Parameterbeschreibung} \\ \hline
		IDent & Identifikator des Links zwischen Geschichte und Interessenpunkt \\ \hline
	\end{tabular}
\end{table}
\paragraph{Beschreibung} Die Funktion dient dem wiederherstellen gelöschter eines Links zwischen einem Interessenpunkt und einer Geschichte. Die Funktion hat Auswirkungen auf folgende Quellen:
\begin{itemize}
	\item Tabelle mit Links zwischen Interessenpunkten und Geschichten
\end{itemize}
Es findet bei dieser Funktion kein Abruf von Daten aus {\glqq COSP\grqq} statt. Die Antwort wird als strukturiertes Array an den Aufrufer zurückgegeben.
\subsubsection{FinalDeletePoiStoryLink}
\paragraph{Parameter} Die Funktion besitzt folgende Parameter:
\begin{table}[H]
	\begin{tabular}{|c|p{11cm}|}
		\hline
		\textbf{Parametername} & \textbf{Parameterbeschreibung} \\ \hline
		\$json & Strukturiertes Array \\ \hline
	\end{tabular}
\end{table}
\subparagraph{\$json}Das Array enthält folgende Elemente:
\begin{table}[H]
	\begin{tabular}{|c|p{11cm}|}
		\hline
		\textbf{Parametername} & \textbf{Parameterbeschreibung} \\ \hline
		IDent & Identifikator des Links zwischen Geschichte und Interessenpunkt \\ \hline
	\end{tabular}
\end{table}
\paragraph{Beschreibung} Die Funktion dient dem endgültigen löschen eines Links zwischen einem Interessenpunkt und einer Geschichte. Die Funktion hat Auswirkungen auf folgende Quellen:
\begin{itemize}
	\item Tabelle mit Links zwischen Interessenpunkten und Geschichten
	\item Tabelle mit Validierungsinformationen zu Links zwischen Interessenpunkten und Geschichten
\end{itemize}
Es findet bei dieser Funktion kein Abruf von Daten aus {\glqq COSP\grqq} statt. Die Antwort wird als strukturiertes Array an den Aufrufer zurückgegeben.
\subsubsection{RestorePoiComment}
\paragraph{Parameter} Die Funktion besitzt folgende Parameter:
\begin{table}[H]
	\begin{tabular}{|c|p{11cm}|}
		\hline
		\textbf{Parametername} & \textbf{Parameterbeschreibung} \\ \hline
		\$json & Strukturiertes Array \\ \hline
	\end{tabular}
\end{table}
\subparagraph{\$json}Das Array enthält folgende Elemente:
\begin{table}[H]
	\begin{tabular}{|c|p{11cm}|}
		\hline
		\textbf{Parametername} & \textbf{Parameterbeschreibung} \\ \hline
		IDent & Identifikator eines Kommentars \\ \hline
	\end{tabular}
\end{table}
\paragraph{Beschreibung} Die Funktion dient dem wiederherstellen eines gelöschten Kommentars. Die Funktion hat Auswirkungen auf folgende Quellen:
\begin{itemize}
	\item Tabelle mit Kommentaren
\end{itemize}
Es findet bei dieser Funktion kein Abruf von Daten aus {\glqq COSP\grqq} statt. Die Antwort wird als strukturiertes Array an den Aufrufer zurückgegeben.
\subsubsection{FinalDeletePoiComment}
\paragraph{Parameter} Die Funktion besitzt folgende Parameter:
\begin{table}[H]
	\begin{tabular}{|c|p{11cm}|}
		\hline
		\textbf{Parametername} & \textbf{Parameterbeschreibung} \\ \hline
		\$json & Strukturiertes Array \\ \hline
	\end{tabular}
\end{table}
\subparagraph{\$json}Das Array enthält folgende Elemente:
\begin{table}[H]
	\begin{tabular}{|c|p{11cm}|}
		\hline
		\textbf{Parametername} & \textbf{Parameterbeschreibung} \\ \hline
		IDent & Identifikator eines Kommentars \\ \hline
	\end{tabular}
\end{table}
\paragraph{Beschreibung} Die Funktion dient dem endgültigen löschen eines Kommentars. Die Funktion hat Auswirkungen auf folgende Quellen:
\begin{itemize}
	\item Tabelle mit Kommentaren
\end{itemize}
Es findet bei dieser Funktion kein Abruf von Daten aus {\glqq COSP\grqq} statt. Die Antwort wird als strukturiertes Array an den Aufrufer zurückgegeben.
\subsubsection{RestorePoiAPI}
\paragraph{Parameter} Die Funktion besitzt folgende Parameter:
\begin{table}[H]
	\begin{tabular}{|c|p{11cm}|}
		\hline
		\textbf{Parametername} & \textbf{Parameterbeschreibung} \\ \hline
		\$json & Strukturiertes Array \\ \hline
	\end{tabular}
\end{table}
\subparagraph{\$json}Das Array enthält folgende Elemente:
\begin{table}[H]
	\begin{tabular}{|c|p{11cm}|}
		\hline
		\textbf{Parametername} & \textbf{Parameterbeschreibung} \\ \hline
		IDent & Identifikator eines Interessenpunktes \\ \hline
	\end{tabular}
\end{table}
\paragraph{Beschreibung} Die Funktion dient dem wiederherstellen eines gelöschten Interessenpunktes. Die Funktion hat Auswirkungen auf folgende Quellen:
\begin{itemize}
	\item Tabelle mit Interessenpunkten
\end{itemize}
Es findet bei dieser Funktion kein Abruf von Daten aus {\glqq COSP\grqq} statt. Die Antwort wird als strukturiertes Array an den Aufrufer zurückgegeben.
\subsubsection{FinalDeletePoi}
\paragraph{Parameter} Die Funktion besitzt folgende Parameter:
\begin{table}[H]
	\begin{tabular}{|c|p{11cm}|}
		\hline
		\textbf{Parametername} & \textbf{Parameterbeschreibung} \\ \hline
		\$json & Strukturiertes Array \\ \hline
	\end{tabular}
\end{table}
\subparagraph{\$json}Das Array enthält folgende Elemente:
\begin{table}[H]
	\begin{tabular}{|c|p{11cm}|}
		\hline
		\textbf{Parametername} & \textbf{Parameterbeschreibung} \\ \hline
		IDent & Identifikator eines Interessenpunktes \\ \hline
	\end{tabular}
\end{table}
\paragraph{Beschreibung} Die Funktion dient dem endgültigen löschen eines Interessenpunktes. Die Funktion hat Auswirkungen auf folgende Quellen:
\begin{itemize}
	\item Tabelle mit Interessenpunkten
	\item Tabelle mit Validierungsinforationen zu Interessenpunkten
\end{itemize}
Es findet bei dieser Funktion kein Abruf von Daten aus {\glqq COSP\grqq} statt. Die Antwort wird als strukturiertes Array an den Aufrufer zurückgegeben.
\subsubsection{RestoreStoryAPI}
\paragraph{Parameter} Die Funktion besitzt folgende Parameter:
\begin{table}[H]
	\begin{tabular}{|c|p{11cm}|}
		\hline
		\textbf{Parametername} & \textbf{Parameterbeschreibung} \\ \hline
		\$json & Strukturiertes Array \\ \hline
	\end{tabular}
\end{table}
\subparagraph{\$json}Das Array enthält folgende Elemente:
\begin{table}[H]
	\begin{tabular}{|c|p{11cm}|}
		\hline
		\textbf{Parametername} & \textbf{Parameterbeschreibung} \\ \hline
		IDent & Identifikator einer Geschichte \\ \hline
	\end{tabular}
\end{table}
\paragraph{Beschreibung} Die Funktion dient dem wiederherstellen einer gelöschten Geschichte. Die Funktion hat Auswirkungen auf folgende Quellen:
\begin{itemize}
	\item COSP
\end{itemize}
Es findet bei dieser Funktion kein Abruf von Daten aus {\glqq COSP\grqq} statt. Es werden jedoch Daten an {\glqq COSP\grqq} übermittelt. Die Antwort wird als strukturiertes Array an den Aufrufer zurückgegeben.
\subsubsection{FinalDeleteStory}
\paragraph{Parameter} Die Funktion besitzt folgende Parameter:
\begin{table}[H]
	\begin{tabular}{|c|p{11cm}|}
		\hline
		\textbf{Parametername} & \textbf{Parameterbeschreibung} \\ \hline
		\$json & Strukturiertes Array \\ \hline
	\end{tabular}
\end{table}
\subparagraph{\$json}Das Array enthält folgende Elemente:
\begin{table}[H]
	\begin{tabular}{|c|p{11cm}|}
		\hline
		\textbf{Parametername} & \textbf{Parameterbeschreibung} \\ \hline
		IDent & Identifikator einer Geschichte \\ \hline
	\end{tabular}
\end{table}
\paragraph{Beschreibung} Die Funktion dient dem endgültigen löschen einer Geschichte. Die Funktion hat Auswirkungen auf folgende Quellen:
\begin{itemize}
	\item COSP
\end{itemize}
Es findet bei dieser Funktion kein Abruf von Daten aus {\glqq COSP\grqq} statt. Es werden jedoch Daten an {\glqq COSP\grqq} übermittelt. Die Antwort wird als strukturiertes Array an den Aufrufer zurückgegeben.
\subsubsection{RestorePictureAPI}
\paragraph{Parameter} Die Funktion besitzt folgende Parameter:
\begin{table}[H]
	\begin{tabular}{|c|p{11cm}|}
		\hline
		\textbf{Parametername} & \textbf{Parameterbeschreibung} \\ \hline
		\$json & Strukturiertes Array \\ \hline
	\end{tabular}
\end{table}
\subparagraph{\$json}Das Array enthält folgende Elemente:
\begin{table}[H]
	\begin{tabular}{|c|p{11cm}|}
		\hline
		\textbf{Parametername} & \textbf{Parameterbeschreibung} \\ \hline
		IDent & Identifikator eines Bildes \\ \hline
	\end{tabular}
\end{table}
\paragraph{Beschreibung} Die Funktion dient dem wiederherstellen eines gelöschten Bildes. Die Funktion hat Auswirkungen auf folgende Quellen:
\begin{itemize}
	\item COSP
\end{itemize}
Es findet bei dieser Funktion kein Abruf von Daten aus {\glqq COSP\grqq} statt. Es werden jedoch Daten an {\glqq COSP\grqq} übermittelt. Die Antwort wird als strukturiertes Array an den Aufrufer zurückgegeben.
\subsubsection{FinalPictureStory}
\paragraph{Parameter} Die Funktion besitzt folgende Parameter:
\begin{table}[H]
	\begin{tabular}{|c|p{11cm}|}
		\hline
		\textbf{Parametername} & \textbf{Parameterbeschreibung} \\ \hline
		\$json & Strukturiertes Array \\ \hline
	\end{tabular}
\end{table}
\subparagraph{\$json}Das Array enthält folgende Elemente:
\begin{table}[H]
	\begin{tabular}{|c|p{11cm}|}
		\hline
		\textbf{Parametername} & \textbf{Parameterbeschreibung} \\ \hline
		IDent & Identifikator eines Bildes \\ \hline
	\end{tabular}
\end{table}
\paragraph{Beschreibung} Die Funktion dient dem endgültigen löschen eines Bildes. Die Funktion hat Auswirkungen auf folgende Quellen:
\begin{itemize}
	\item COSP
\end{itemize}
Es findet bei dieser Funktion kein Abruf von Daten aus {\glqq COSP\grqq} statt. Es werden jedoch Daten an {\glqq COSP\grqq} übermittelt. Die Antwort wird als strukturiertes Array an den Aufrufer zurückgegeben.
\subsubsection{addAnnouncementAPI}
\paragraph{Parameter} Die Funktion besitzt folgende Parameter:
\begin{table}[H]
	\begin{tabular}{|c|p{11cm}|}
		\hline
		\textbf{Parametername} & \textbf{Parameterbeschreibung} \\ \hline
		\$json & Strukturiertes Array \\ \hline
	\end{tabular}
\end{table}
\subparagraph{\$json}Das Array enthält folgende Elemente:
\begin{table}[H]
	\begin{tabular}{|c|p{11cm}|}
		\hline
		\textbf{Parametername} & \textbf{Parameterbeschreibung} \\ \hline
		title   & Titel der Ankündigung \\ \hline
		content & Inhalt den Ankündigung \\ \hline
		start   & Starttage der Ankündigung \\ \hline
		end     & Endtage der Ankündigung \\ \hline
	\end{tabular}
\end{table}
\paragraph{Beschreibung} Die Funktion fügt eine neue Ankündigung hinzu. Die Funktion hat Auswirkungen auf folgende Quellen:
\begin{itemize}
	\item Tabelle mit Ankündigungen
\end{itemize}
Es findet bei dieser Funktion kein Abruf von Daten aus {\glqq COSP\grqq} statt. Es werden jedoch Daten an {\glqq COSP\grqq} übermittelt. Die Antwort wird als strukturiertes Array an den Aufrufer zurückgegeben.
\subsubsection{getAnnouncementAPI}
\paragraph{Parameter} Die Funktion besitzt folgende Parameter:
\begin{table}[H]
	\begin{tabular}{|c|p{11cm}|}
		\hline
		\textbf{Parametername} & \textbf{Parameterbeschreibung} \\ \hline
		\$json & Strukturiertes Array \\ \hline
	\end{tabular}
\end{table}
\subparagraph{\$json}Das Array enthält folgende Elemente:
\begin{table}[H]
	\begin{tabular}{|c|p{11cm}|}
		\hline
		\textbf{Parametername} & \textbf{Parameterbeschreibung} \\ \hline
		id   & Identifikator der Ankündigung \\ \hline
	\end{tabular}
\end{table}
\paragraph{Beschreibung} Die Funktion ruft Daten einer Ankündigung ab. Die Funktion nutzt folgende Quellen:
\begin{itemize}
	\item Tabelle mit Ankündigungen
\end{itemize}
Es findet bei dieser Funktion kein Abruf von Daten aus {\glqq COSP\grqq} statt. Es werden jedoch Daten an {\glqq COSP\grqq} übermittelt. Die Antwort wird als strukturiertes Array an den Aufrufer zurückgegeben.
\subsubsection{addAnnouncementAPI}
\paragraph{Parameter} Die Funktion besitzt folgende Parameter:
\begin{table}[H]
	\begin{tabular}{|c|p{11cm}|}
		\hline
		\textbf{Parametername} & \textbf{Parameterbeschreibung} \\ \hline
		\$json & Strukturiertes Array \\ \hline
	\end{tabular}
\end{table}
\subparagraph{\$json}Das Array enthält folgende Elemente:
\begin{table}[H]
	\begin{tabular}{|c|p{11cm}|}
		\hline
		\textbf{Parametername} & \textbf{Parameterbeschreibung} \\ \hline
		id      & Identifikator der Ankündigung \\ \hline
		title   & Titel der Ankündigung \\ \hline
		content & Inhalt den Ankündigung \\ \hline
		start   & Starttage der Ankündigung \\ \hline
		end     & Endtage der Ankündigung \\ \hline
	\end{tabular}
\end{table}
\paragraph{Beschreibung} Die Funktion fügt eine neue Ankündigung hinzu. Die Funktion hat Auswirkungen auf folgende Quellen:
\begin{itemize}
	\item Tabelle mit Ankündigungen
\end{itemize}
Es findet bei dieser Funktion kein Abruf von Daten aus {\glqq COSP\grqq} statt. Es werden jedoch Daten an {\glqq COSP\grqq} übermittelt. Die Antwort wird als strukturiertes Array an den Aufrufer zurückgegeben.
\subsubsection{deleteAnnouncementAPI}
\paragraph{Parameter} Die Funktion besitzt folgende Parameter:
\begin{table}[H]
	\begin{tabular}{|c|p{11cm}|}
		\hline
		\textbf{Parametername} & \textbf{Parameterbeschreibung} \\ \hline
		\$json & Strukturiertes Array \\ \hline
	\end{tabular}
\end{table}
\subparagraph{\$json}Das Array enthält folgende Elemente:
\begin{table}[H]
	\begin{tabular}{|c|p{11cm}|}
		\hline
		\textbf{Parametername} & \textbf{Parameterbeschreibung} \\ \hline
		id   & Identifikator der Ankündigung \\ \hline
	\end{tabular}
\end{table}
\paragraph{Beschreibung} Die Funktion löscht Daten einer Ankündigung. Die Funktion nutzt folgende Quellen:
\begin{itemize}
	\item Tabelle mit Ankündigungen
\end{itemize}
Es findet bei dieser Funktion kein Abruf von Daten aus {\glqq COSP\grqq} statt. Es werden jedoch Daten an {\glqq COSP\grqq} übermittelt. Die Antwort wird als strukturiertes Array an den Aufrufer zurückgegeben.
\subsubsection{setAktivationAnnouncement}
\paragraph{Parameter} Die Funktion besitzt folgende Parameter:
\begin{table}[H]
	\begin{tabular}{|c|p{11cm}|}
		\hline
		\textbf{Parametername} & \textbf{Parameterbeschreibung} \\ \hline
		\$json & Strukturiertes Array \\ \hline
	\end{tabular}
\end{table}
\subparagraph{\$json}Das Array enthält folgende Elemente:
\begin{table}[H]
	\begin{tabular}{|c|p{11cm}|}
		\hline
		\textbf{Parametername} & \textbf{Parameterbeschreibung} \\ \hline
		id      & Identifikator der Ankündigung \\ \hline
		state   & Aktivierungsstatus der Ankündigung \\ \hline
	\end{tabular}
\end{table}
\paragraph{Beschreibung} Die Funktion setzt den Aktivierungsstatus einer bestimmten Ankündigung. Die Funktion hat Auswirkungen auf folgende Quellen:
\begin{itemize}
	\item Tabelle mit Ankündigungen
\end{itemize}
Es findet bei dieser Funktion kein Abruf von Daten aus {\glqq COSP\grqq} statt. Es werden jedoch Daten an {\glqq COSP\grqq} übermittelt. Die Antwort wird als strukturiertes Array an den Aufrufer zurückgegeben.
\subsubsection{addSourcePoiAPI}
\paragraph{Parameter} Die Funktion besitzt folgende Parameter:
\begin{table}[H]
	\begin{tabular}{|c|p{11cm}|}
		\hline
		\textbf{Parametername} & \textbf{Parameterbeschreibung} \\ \hline
		\$json & Strukturiertes Array \\ \hline
	\end{tabular}
\end{table}
\subparagraph{\$json}Das Array enthält folgende Elemente:
\begin{table}[H]
	\begin{tabular}{|c|p{11cm}|}
		\hline
		\textbf{Parametername} & \textbf{Parameterbeschreibung} \\ \hline
		typeSource & Indentifikator des Typs der Quelle \\ \hline
		source     & Quellenangabe \\ \hline
		relation   & Identifikator des Informationsbezugs der Quelle \\ \hline
		poiid      & Identifikator des Interessenpunktes \\ \hline
	\end{tabular}
\end{table}
\paragraph{Beschreibung} Die Funktion fügt einem Interessenpunkt eine neue Quelle hinzu. Die Funktion hat Auswirkungen auf folgende Quellen:
\begin{itemize}
	\item Tabelle mit Quellenangaben
\end{itemize}
Es findet bei dieser Funktion kein Abruf von Daten aus {\glqq COSP\grqq} statt. Die Antwort wird als strukturiertes Array an den Aufrufer zurückgegeben.
\subsubsection{getSourcePoiAPI}
\paragraph{Parameter} Die Funktion besitzt folgende Parameter:
\begin{table}[H]
	\begin{tabular}{|c|p{11cm}|}
		\hline
		\textbf{Parametername} & \textbf{Parameterbeschreibung} \\ \hline
		\$json & Strukturiertes Array \\ \hline
	\end{tabular}
\end{table}
\subparagraph{\$json}Das Array enthält folgende Elemente:
\begin{table}[H]
	\begin{tabular}{|c|p{11cm}|}
		\hline
		\textbf{Parametername} & \textbf{Parameterbeschreibung} \\ \hline
		poiid      & Identifikator des Interessenpunktes \\ \hline
	\end{tabular}
\end{table}
\paragraph{Beschreibung} Die Funktion fügt einem Interessenpunkt eine neue Quelle hinzu. Die Funktion hat Auswirkungen auf folgende Quellen:
\begin{itemize}
	\item Tabelle mit Quellenangaben
\end{itemize}
Es findet bei dieser Funktion kein Abruf von Daten aus {\glqq COSP\grqq} statt. Die Antwort wird als strukturiertes Array an den Aufrufer zurückgegeben.
\subsubsection{getSourceRelationsAPI}
\paragraph{Parameter} Die Funktion besitzt keine Parameter.
\paragraph{Beschreibung} Die Funktion ruft alle Bezüge von Quellen ab. Die Funktion nutzt folgende Quellen:
\begin{itemize}
	\item Tabelle mit Bezugsangaben von Quellen
\end{itemize}
Es findet bei dieser Funktion kein Abruf von Daten aus {\glqq COSP\grqq} statt. Die Antwort wird als strukturiertes Array an den Aufrufer zurückgegeben.
\subsubsection{getSourceTypeAPI}
\paragraph{Parameter} Die Funktion besitzt keine Parameter.
\paragraph{Beschreibung} Die Funktion ruft alle Bezüge von Quellen ab. Die Funktion nutzt folgende Quellen:
\begin{itemize}
	\item Tabelle mit Bezugsangaben von Quellen
\end{itemize}
Es findet bei dieser Funktion kein Abruf von Daten aus {\glqq COSP\grqq} statt. Die Antwort wird als strukturiertes Array an den Aufrufer zurückgegeben.
\subsubsection{updateSourcePoiAPI}
\paragraph{Parameter} Die Funktion besitzt folgende Parameter:
\begin{table}[H]
	\begin{tabular}{|c|p{11cm}|}
		\hline
		\textbf{Parametername} & \textbf{Parameterbeschreibung} \\ \hline
		\$json & Strukturiertes Array \\ \hline
	\end{tabular}
\end{table}
\subparagraph{\$json}Das Array enthält folgende Elemente:
\begin{table}[H]
	\begin{tabular}{|c|p{11cm}|}
		\hline
		\textbf{Parametername} & \textbf{Parameterbeschreibung} \\ \hline
		typeSource & Indentifikator des Typs der Quelle \\ \hline
		source     & Quellenangabe \\ \hline
		relation   & Identifikator des Informationsbezugs der Quelle \\ \hline
		id         & Identifikator der Quelle \\ \hline
	\end{tabular}
\end{table}
\paragraph{Beschreibung} Die Funktion ändert eine Quelle. Die Funktion hat Auswirkungen auf folgende Quellen:
\begin{itemize}
	\item Tabelle mit Quellenangaben
\end{itemize}
Es findet bei dieser Funktion kein Abruf von Daten aus {\glqq COSP\grqq} statt. Die Antwort wird als strukturiertes Array an den Aufrufer zurückgegeben.
\subsubsection{deleteSourceAPI}
\paragraph{Parameter} Die Funktion besitzt folgende Parameter:
\begin{table}[H]
	\begin{tabular}{|c|p{11cm}|}
		\hline
		\textbf{Parametername} & \textbf{Parameterbeschreibung} \\ \hline
		\$json & Strukturiertes Array \\ \hline
	\end{tabular}
\end{table}
\subparagraph{\$json}Das Array enthält folgende Elemente:
\begin{table}[H]
	\begin{tabular}{|c|p{11cm}|}
		\hline
		\textbf{Parametername} & \textbf{Parameterbeschreibung} \\ \hline
		id & Identifikator einer Quelle \\ \hline
	\end{tabular}
\end{table}
\paragraph{Beschreibung} Die Funktion löscht eine Quelle oder markiert diese als gelöscht. Die Funktion hat Auswirkungen auf folgende Quellen:
\begin{itemize}
	\item Tabelle mit Quellenangaben
\end{itemize}
Es findet bei dieser Funktion kein Abruf von Daten aus {\glqq COSP\grqq} statt. Die Antwort wird als strukturiertes Array an den Aufrufer zurückgegeben.
\subsubsection{finalDeleteSourceAPI}
\paragraph{Parameter} Die Funktion besitzt folgende Parameter:
\begin{table}[H]
	\begin{tabular}{|c|p{11cm}|}
		\hline
		\textbf{Parametername} & \textbf{Parameterbeschreibung} \\ \hline
		\$json & Strukturiertes Array \\ \hline
	\end{tabular}
\end{table}
\subparagraph{\$json}Das Array enthält folgende Elemente:
\begin{table}[H]
	\begin{tabular}{|c|p{11cm}|}
		\hline
		\textbf{Parametername} & \textbf{Parameterbeschreibung} \\ \hline
		id & Identifikator einer Quelle \\ \hline
	\end{tabular}
\end{table}
\paragraph{Beschreibung} Die Funktion löscht eine Quelle endgültig. Die Funktion hat Auswirkungen auf folgende Quellen:
\begin{itemize}
	\item Tabelle mit Quellenangaben
\end{itemize}
Es findet bei dieser Funktion kein Abruf von Daten aus {\glqq COSP\grqq} statt. Die Antwort wird als strukturiertes Array an den Aufrufer zurückgegeben.
\subsubsection{restoreSourceApi}
\paragraph{Parameter} Die Funktion besitzt folgende Parameter:
\begin{table}[H]
	\begin{tabular}{|c|p{11cm}|}
		\hline
		\textbf{Parametername} & \textbf{Parameterbeschreibung} \\ \hline
		\$json & Strukturiertes Array \\ \hline
	\end{tabular}
\end{table}
\subparagraph{\$json}Das Array enthält folgende Elemente:
\begin{table}[H]
	\begin{tabular}{|c|p{11cm}|}
		\hline
		\textbf{Parametername} & \textbf{Parameterbeschreibung} \\ \hline
		id & Identifikator einer Quelle \\ \hline
	\end{tabular}
\end{table}
\paragraph{Beschreibung} Die Funktion stellt eine Quelle wieder her. Die Funktion hat Auswirkungen auf folgende Quellen:
\begin{itemize}
	\item Tabelle mit Quellenangaben
\end{itemize}
Es findet bei dieser Funktion kein Abruf von Daten aus {\glqq COSP\grqq} statt. Die Antwort wird als strukturiertes Array an den Aufrufer zurückgegeben.
\subsubsection{validatePoiAPI}
\paragraph{Parameter} Die Funktion besitzt folgende Parameter:
\begin{table}[H]
	\begin{tabular}{|c|p{11cm}|}
		\hline
		\textbf{Parametername} & \textbf{Parameterbeschreibung} \\ \hline
		\$json & Strukturiertes Array \\ \hline
	\end{tabular}
\end{table}
\subparagraph{\$json}Das Array enthält folgende Elemente:
\begin{table}[H]
	\begin{tabular}{|c|p{11cm}|}
		\hline
		\textbf{Parametername} & \textbf{Parameterbeschreibung} \\ \hline
		id & Identifikator eines Interessenpunktes \\ \hline
	\end{tabular}
\end{table}
\paragraph{Beschreibung} Die Funktion stellt eine Quelle wieder her. Die Funktion hat Auswirkungen auf folgende Quellen:
\begin{itemize}
	\item Tabelle mit Validierungen von Interessenpunkten
\end{itemize}
Es findet bei dieser Funktion kein Abruf von Daten aus {\glqq COSP\grqq} statt. Die Antwort wird als strukturiertes Array an den Aufrufer zurückgegeben.
\subsubsection{getDirectDelete}
\paragraph{Parameter} Die Funktion besitzt keine Parameter.
\paragraph{Beschreibung} Die Funktion gibt wieder, ob direktes Löschen aktiviert ist. Es findet bei dieser Funktion kein Abruf von Daten aus {\glqq COSP\grqq} statt. Die Antwort wird als strukturiertes Array an den Aufrufer zurückgegeben.
\subsubsection{changeMainPicturePoi}
\paragraph{Parameter} Die Funktion besitzt folgende Parameter:
\begin{table}[H]
	\begin{tabular}{|c|p{11cm}|}
		\hline
		\textbf{Parametername} & \textbf{Parameterbeschreibung} \\ \hline
		\$json & Strukturiertes Array \\ \hline
	\end{tabular}
\end{table}
\subparagraph{\$json}Das Array enthält folgende Elemente:
\begin{table}[H]
	\begin{tabular}{|c|p{11cm}|}
		\hline
		\textbf{Parametername} & \textbf{Parameterbeschreibung} \\ \hline
		poiid & Identifikator eines Interessenpunktes \\ \hline
		token & Identifikator eines Bildes \\ \hline
	\end{tabular}
\end{table}
\paragraph{Beschreibung} Die Funktion ändert das Hauptbild eines Interessenpunktes. Die Funktion hat Auswirkungen auf folgende Quellen:
\begin{itemize}
	\item Tabelle mit Validierungen von Interessenpunkten
	\item Tabelle mit Interessenpunkten
\end{itemize}
Es findet bei dieser Funktion kein Abruf von Daten aus {\glqq COSP\grqq} statt. Die Antwort wird als strukturiertes Array an den Aufrufer zurückgegeben.
\subsubsection{checkMailAddressExistent}
\paragraph{Parameter} Die Funktion besitzt folgende Parameter:
\begin{table}[H]
	\begin{tabular}{|c|p{11cm}|}
		\hline
		\textbf{Parametername} & \textbf{Parameterbeschreibung} \\ \hline
		\$json & Strukturiertes Array \\ \hline
	\end{tabular}
\end{table}
\subparagraph{\$json}Das Array enthält folgende Elemente:
\begin{table}[H]
	\begin{tabular}{|c|p{11cm}|}
		\hline
		\textbf{Parametername} & \textbf{Parameterbeschreibung} \\ \hline
		mail & Mailadresse \\ \hline
	\end{tabular}
\end{table}
\paragraph{Beschreibung} Die Funktion prüft ob eine Mailadresse bereits verwendet wird.
Es findet bei dieser Funktion ein Abruf von Daten aus {\glqq COSP\grqq} statt. Die Antwort wird als strukturiertes Array an den Aufrufer zurückgegeben.
\newpage
\section{authSystem}
\subsection{Allgemeines} Diese Datei enthält alle für eine Authentifizierung notwendigen Funktionen.
\begin{table}[H]
	\begin{tabular}{|c|p{11cm}|}
		\hline
		\textbf{Einbindungspunkt} & inc.php \\ \hline
		\textbf{Einbindungspunkt} & inc-sub.php \\ \hline
	\end{tabular}
\end{table}
Die Datei ist nicht direkt durch den Nutzer aufrufbar, dies wird durch folgenden Code-Ausschnitt sichergestellt:
\begin{lstlisting}[language=php]
if (!defined('NICE_PROJECT')) {
	die('Permission denied.');
}
\end{lstlisting}
Der Globale Wert {\glqq NICE\_PROJECT\grqq} wird durch für den Nutzer valide Aufrufpunkte festgelegt, z.B. {\glqq api.php\grqq}.
\newpage
\subsection{Funktionen}
\subsubsection{getAuth}
\paragraph{Parameter} Die Funktion besitzt folgende Parameter:
\begin{table}[H]
	\begin{tabular}{|c|p{11cm}|}
		\hline
		\textbf{Parametername} & \textbf{Parameterbeschreibung} \\ \hline
		\$name        & Name des zu authentifizierenden Nutzers \\ \hline
		\$password    & durch Nutzer eingegebenes Passwort \\ \hline
		\$forceRemote & Fragt Nutzerdaten ausschlielich aus {\glqq COSP\grqq} ab \\ \hline
	\end{tabular}
\end{table}
\paragraph{Beschreibung} Die Funktion prüft die Authentizität eines Nutzers anhand seines eindeutigen Nutzernamens und des eingebenen Passwortes. Die Funktion nutzt folgende Quellen:
\begin{itemize}
	\item Nutzerdaten-Tabelle
	\item COSP
\end{itemize}
Es findet bei dieser Funktion ein Abruf von Daten aus {\glqq COSP\grqq} statt.
\paragraph{Vorgehensweise} Zuerst prüft die Funktion, ob sich der gegebene Nutzer bereits in diesem Modul bekannt ist. Sollte dies nicht der Fall sein, so wird nach dem Nutzer in {\glqq COSP\grqq} gesucht, wird er dort gefunden und er das korrekte Passwort eingegeben hat, so wird automatisiert ein Modulbenutzer erstellt und der Nutzer angemeldet. Sollte er dort nicht gefunden werden oder das Passwort falsch sein, so wird die Authentisierung abgelehnt. Sollte der Nutzer im eigenen System vorhanden sein, so wird zunächst das Passwort mit dem Hash in der Datenbank geprüft, sollte dies Fehlschlagen, so ruft sich die Funktion rekursiv selbst auf und Prüft mit den in {\glqq COSP\grqq} enthaltenen Daten.
\subsubsection{logLogin}
\paragraph{Parameter} Die Funktion besitzt folgende Parameter:
\begin{table}[H]
	\begin{tabular}{|c|p{11cm}|}
		\hline
		\textbf{Parametername} & \textbf{Parameterbeschreibung} \\ \hline
		\$type        & Gibt an ob Nutzer {\glqq guest\grqq} oder {\glqq user\grqq} ist \\ \hline
	\end{tabular}
\end{table}
\paragraph{Beschreibung} Die Funktion dient der Erfassung statistischer Nutzerdaten. Die Funktion hat Auswirkungen auf folgende Quellen:
\begin{itemize}
	\item Tabelle mit statistischen Nutzungsdaten
\end{itemize}
Es findet bei dieser Funktion kein Abruf von Daten aus {\glqq COSP\grqq} statt.
\subsubsection{getGuestAuth}
\paragraph{Parameter} Die Funktion besitzt keine Parameter.
\paragraph{Beschreibung} Die Funktion führt einen Login als Gast durch. 
Es findet bei dieser Funktion kein Abruf von Daten aus {\glqq COSP\grqq} statt.
\subsubsection{GuestAuthData}
\paragraph{Parameter} Die Funktion besitzt keine Parameter.
\paragraph{Beschreibung} Die Funktion liefert alle notwendigen Daten für einen Login als Gast. 
Es findet bei dieser Funktion kein Abruf von Daten aus {\glqq COSP\grqq} statt. Die Funktion liefert ein Strukturiertes Array zurück.
\subsubsection{setSessionData}
\paragraph{Parameter} Die Funktion besitzt folgende Parameter:
\begin{table}[H]
	\begin{tabular}{|c|p{11cm}|}
		\hline
		\textbf{Parametername} & \textbf{Parameterbeschreibung} \\ \hline
		\$userData    & Array mit Daten des Nutzers \\ \hline
		\$external    & gibt an, ob Daten extern angefragt werden sollen aus {\glqq COSP\grqq} \\ \hline
	\end{tabular}
\end{table}
\subparagraph{userData} Das Array besitzt folgende Parameter:
\begin{table}[H]
	\begin{tabular}{|c|p{11cm}|}
		\hline
		\textbf{Parametername} & \textbf{Parameterbeschreibung} \\ \hline
		name      & Nutzername/Nickname \\ \hline
		firstname & Vorname des Nutzers \\ \hline
		lastname  & Nachname des Nutzers \\ \hline
		email     & E-Mailadresse des Nutzers \\ \hline
		role      & Array mit Rollendaten des Nutzers \\ \hline
	\end{tabular}
\end{table}
\subparagraph{role} Das Array besitzt folgende Parameter:
\begin{table}[H]
	\begin{tabular}{|c|p{11cm}|}
		\hline
		\textbf{Parametername} & \textbf{Parameterbeschreibung} \\ \hline
		rolevalue & Wert der Rolle \\ \hline
		rolename  & Name der Rolle \\ \hline
	\end{tabular}
\end{table}
\paragraph{Beschreibung} Die Funktion setzt die Serverseitigen Session-Daten für die Anmeldung des Nutzers.
Es findet bei dieser Funktion ein Abruf von Daten aus {\glqq COSP\grqq} statt.
\subsubsection{getRemoteUserData}
\paragraph{Parameter} Die Funktion besitzt folgende Parameter:
\begin{table}[H]
	\begin{tabular}{|c|p{11cm}|}
		\hline
		\textbf{Parametername} & \textbf{Parameterbeschreibung} \\ \hline
		\$name & Nickname oder Nutzername eines Nutzers \\ \hline
	\end{tabular}
\end{table}
\paragraph{Beschreibung} Die Funktion fragt Benutzerdaten aus {\glqq COSP\grqq} ab. Die Funktion nutzt folgenden Quellen:
\begin{itemize}
	\item COSP
\end{itemize}
Es findet bei dieser Funktion ein Abruf von Daten aus {\glqq COSP\grqq} statt. Die Antwort wird als strukturiertes Array an den Aufrufer zurückgegeben.

\newpage
\section{config-sample / config}
\subsection{Allgemeines} Diese Datei enthält die Beispielkonfiguration beziehungsweise die Konfiguration.
\begin{table}[H]
	\begin{tabular}{|c|p{11cm}|}
		\hline
		\textbf{Einbindungspunkt} & keiner \\ \hline
	\end{tabular}
\end{table}
Die Datei ist nicht direkt durch den Nutzer aufrufbar, dies wird durch folgenden Code-Ausschnitt sichergestellt:
\begin{lstlisting}[language=php]
if (!defined('NICE_PROJECT')) {
	die('Permission denied.');
}
\end{lstlisting}
Der Globale Wert {\glqq NICE\_PROJECT\grqq} wird durch für den Nutzer valide Aufrufpunkte festgelegt, z.B. {\glqq api.php\grqq}. Alle Nachfolgenden {\glqq Funktionen\grqq} sind statische Werte der Klasse {\glqq configsample\grqq}. Für Detailbeschreibungen und Standardwerte siehe \autoref{chapter:config}.
\subsection{Installationsanweisungen} Für einen produktiven Einsatz der Konfigurationsdatei muss sie von {\glqq config-sample.php\grqq} in {\glqq config.php\grqq} kopiert werden. Anschließend muss die Klasse {\glqq configsample\grqq} in {\glqq config\grqq} umbenannt werden.
\newpage
\subsection{Funktionen}
\subsubsection{\$SQL\_SERVER} Setzt den zu verwendenden SQL-Server. Dieser sollte im Optimalfall ein MariaDB-Server sein.
\subsubsection{\$SQL\_USER} Setzt den Nutzernamen am SQL-Server.
\subsubsection{\$SQL\_PASSWORD} Setzt das Passwort des Nutzers am SQL-Server.
\subsubsection{\$SQL\_SCHEMA} Setzt das Schema am SQL-Server.
\subsubsection{\$SQL\_PREFIX} Setzt das zu nutzende Präfix für die SQL-Tabellen.
\subsubsection{\$SQL\_Connector} Bestimmt den SQL-Connector. Momentan ist nur der PDO-Connector implementiert.
\subsubsection{\$PICTURE\_PATH} Setzt den Temporären Pfad zur Zwischenspeicherung von Bildern.
\subsubsection{\$DEBUG} Schaltet Debug-Funktionen frei.
\subsubsection{\$DEBUG\_LEVEL} Setzt das Debug-Level.
\subsubsection{\$PWD\_LENGTH} Setzt die mindestens benötigte Passwortlänge.
\subsubsection{\$PWD\_ALGORITHM} Setzt den Hash-Algorithmus zur Passwortspeicherung.
\subsubsection{\$RANDOM\_STRING\_LENGTH} Setzt die Länge von zufällig generierten Zeichenketten.
\subsubsection{\$CSAPI} Setzt die URI der {\glqq COSP\grqq}-API.
\subsubsection{\$CSTOKEN} Setzt den Authentifizierungstoken der {\glqq COSP\grqq}-API.
\subsubsection{\$USAPI} Setzt die URI der {\glqq COSP\grqq}-Nutzer-API.
\subsubsection{\$ENABLE\_STORIES} Legt die Verfügbarkeit der Geschichten-Funktion fest.
\subsubsection{\$BETA} Schaltet den Beta-Modus an.
\subsubsection{\$MAINTENANCE} Schaltet den Wartungs-Modus an.
\subsubsection{\$ROLE\_GUEST} Setzt den Mindestwert der Rolle {\glqq Gast\grqq}.
\subsubsection{\$ROLE\_UNAUTH\_USER} Setzt den Mindestwert der Rolle {\glqq nicht authentifizierter Nutzer\grqq}.
\subsubsection{\$ROLE\_AUTH\_USER} Setzt den Mindestwert der Rolle {\glqq Nutzer\grqq}.
\subsubsection{\$ROLE\_EMPLOYEE} Setzt den Mindestwert der Rolle {\glqq Mitarbeiter\grqq}.
\subsubsection{\$ROLE\_ADMIN} Setzt den Mindestwert der Rolle {\glqq Administrator\grqq}.
\subsubsection{\$SPECIAL\_CHARS\_CAPTCHA} Schaltet Sonderzeichen in Captchas ein.
\subsubsection{\$PUBLIC\_CONTACT} Schaltet das Kontaktformular für alle frei.
\subsubsection{\$ZENTRAL\_MAIL} Legt die Administratormailadresse des Moduls fest.
\subsubsection{\$IMPRESSUM\_NAME} Setzt den Namen des Verantwortlichen im Impressum.
\subsubsection{\$IMPRESSUM\_STREET} Setzt den Straßennamen und die Hausnummer des Verantwortlichen im Impressum.
\subsubsection{\$IMPRESSUM\_CITY} Setzt den Ortsnamen und die Postleitzahl des Verantwortlichen im Impressum.
\subsubsection{\$PRIVACY\_COMPANY\_NAME} Setzt den Namen der Firma in der Datenschutzerklärung.
\subsubsection{\$PRIVACY\_COMPANY\_STREET} Setzt den Straßennamen und die Hausnummer der Firma in der Datenschutzerklärung.
\subsubsection{\$PRIVACY\_COMPANY\_CITY} Setzt den Ortsnamen und die Postleitzahl der Firma in der Datenschutzerklärung.
\subsubsection{\$PRIVACY\_COMPANY\_FON} Setzt die Telefonnummer der Firma in der Datenschutzerklärung.
\subsubsection{\$PRIVACY\_COMPANY\_FAX} Setzt die Faxnummer der Firma in der Datenschutzerklärung.
\subsubsection{\$PRIVACY\_COMPANY\_MAIL} Setzt die E-Mailadresse der Firma in der Datenschutzerklärung.
\subsubsection{\$PRIVACY\_REP\_NAME} Setzt den Namen des Datenschutzbeauftragten.
\subsubsection{\$PRIVACY\_REP\_POS} Setzt die Positionsbezeichnung des Datenschutzbeauftragten.
\subsubsection{\$PRIVACY\_REP\_STREET} Setzt den Straßennamen und die Hausnummer des Datenschutzbeauftragten.
\subsubsection{\$PRIVACY\_REP\_CITY} Setzt den Ortsnamen und die Postleitzahl des Datenschutzbeauftragten.
\subsubsection{\$PRIVACY\_REP\_FON} Setzt die Telefonnummer des Datenschutzbeauftragten.
\subsubsection{\$PRIVACY\_REP\_FAX} Setzt die Faxnummer des Datenschutzbeauftragten.
\subsubsection{\$PRIVACY\_REP\_MAIL} Setzt die E-Mailadresse des Datenschutzbeauftragten.
\subsubsection{\$DIRECT\_DELETE} Schaltet direktes Löschen frei.

\newpage
\section{announcement-db}
\subsection{Allgemeines} Diese Datei enthält alle Funktionen für die Ankündigungstabelle-Tabelle.
\begin{table}[H]
	\begin{tabular}{|c|p{11cm}|}
		\hline
		\textbf{Einbindungspunkt} & inc-db.php \\ \hline
		\textbf{Einbindungspunkt} & inc-db-sub.php \\ \hline
	\end{tabular}
\end{table}
Die Datei ist nicht direkt durch den Nutzer aufrufbar, dies wird durch folgenden Code-Ausschnitt sichergestellt:
\begin{lstlisting}[language=php]
if (!defined('NICE_PROJECT')) {
	die('Permission denied.');
}
\end{lstlisting}
Der Globale Wert {\glqq NICE\_PROJECT\grqq} wird durch für den Nutzer valide Aufrufpunkte festgelegt, z.B. {\glqq api.php\grqq}.
\newpage
\subsection{Funktionen}
\subsubsection{addAnnouncement}
\paragraph{Parameter} Die Funktion besitzt folgende Parameter:
\begin{table}[H]
	\begin{tabular}{|c|p{11cm}|}
		\hline
		\textbf{Parametername} & \textbf{Parameterbeschreibung} \\ \hline
		\$title & Titel der Ankündigung \\ \hline
		\$content & Inhalt der Ankündigung \\ \hline
		\$start & Starttag der Ankündigung \\ \hline
		\$end & Endtag der Ankündigung \\ \hline
	\end{tabular}
\end{table}
\paragraph{Beschreibung} Die Funktion fügt eine neue Ankündigung hinzu. Die Funktion hat Auswirkungen auf folgende Quellen:
\begin{itemize}
	\item Ankündigungstabelle
\end{itemize}
Es findet bei dieser Funktion kein Abruf von Daten aus {\glqq COSP\grqq} statt. Die Antwort wird als strukturiertes Array an den Aufrufer zurückgegeben.
\subsubsection{getAllAnnouncements}
\paragraph{Parameter} Die Funktion besitzt keine Parameter.
\paragraph{Beschreibung} Die Funktion ruft alle bestehenden Ankündigungen ab. Die Funktion nutzt folgende Quellen:
\begin{itemize}
	\item Ankündigungstabelle
\end{itemize}
Es findet bei dieser Funktion kein Abruf von Daten aus {\glqq COSP\grqq} statt. Die Antwort wird als strukturiertes Array an den Aufrufer zurückgegeben.
\subsubsection{getAnnouncement}
\paragraph{Parameter} Die Funktion besitzt folgende Parameter:
\begin{table}[H]
	\begin{tabular}{|c|p{11cm}|}
		\hline
		\textbf{Parametername} & \textbf{Parameterbeschreibung} \\ \hline
		\$id & Identifikator einer Ankündigung\\ \hline
	\end{tabular}
\end{table}
\paragraph{Beschreibung} Die Funktion ruft die Daten einer bestimmten Ankündigung ab. Die Funktion nutzt folgende Quellen:
\begin{itemize}
	\item Ankündigungstabelle
\end{itemize}
Es findet bei dieser Funktion kein Abruf von Daten aus {\glqq COSP\grqq} statt. Die Antwort wird als strukturiertes Array an den Aufrufer zurückgegeben.
\subsubsection{updateAnnouncement}
\paragraph{Parameter} Die Funktion besitzt folgende Parameter:
\begin{table}[H]
	\begin{tabular}{|c|p{11cm}|}
		\hline
		\textbf{Parametername} & \textbf{Parameterbeschreibung} \\ \hline
		\$id      & Identifikator der Ankündigung \\ \hline
		\$title   & Titel der Ankündigung \\ \hline
		\$content & Inhalt der Ankündigung \\ \hline
		\$start   & Starttag der Ankündigung \\ \hline
		\$end     & Endtag der Ankündigung \\ \hline
	\end{tabular}
\end{table}
\paragraph{Beschreibung} Die Funktion ändert eine Ankündigung. Die Funktion hat Auswirkungen auf folgende Quellen:
\begin{itemize}
	\item Ankündigungstabelle
\end{itemize}
Es findet bei dieser Funktion kein Abruf von Daten aus {\glqq COSP\grqq} statt. Die Antwort wird als strukturiertes Array an den Aufrufer zurückgegeben.
\subsubsection{deleteAnnouncement}
\paragraph{Parameter} Die Funktion besitzt folgende Parameter:
\begin{table}[H]
	\begin{tabular}{|c|p{11cm}|}
		\hline
		\textbf{Parametername} & \textbf{Parameterbeschreibung} \\ \hline
		\$id & Identifikator einer Ankündigung\\ \hline
	\end{tabular}
\end{table}
\paragraph{Beschreibung} Die Funktion löscht die Daten einer bestimmten Ankündigung. Die Funktion nutzt folgende Quellen:
\begin{itemize}
	\item Ankündigungstabelle
\end{itemize}
Es findet bei dieser Funktion kein Abruf von Daten aus {\glqq COSP\grqq} statt. Die Antwort wird als strukturiertes Array an den Aufrufer zurückgegeben.
\subsubsection{getCurrentAnnouncement}
\paragraph{Parameter} Die Funktion besitzt keine Parameter.
\paragraph{Beschreibung} Die Funktion ruft alle aktuellen Ankündigungen ab. Die Funktion nutzt folgende Quellen:
\begin{itemize}
	\item Ankündigungstabelle
\end{itemize}
Es findet bei dieser Funktion kein Abruf von Daten aus {\glqq COSP\grqq} statt. Die Antwort wird als strukturiertes Array an den Aufrufer zurückgegeben.
\subsubsection{updateAktivationStateAnnouncement}
\paragraph{Parameter} Die Funktion besitzt folgende Parameter:
\begin{table}[H]
	\begin{tabular}{|c|p{11cm}|}
		\hline
		\textbf{Parametername} & \textbf{Parameterbeschreibung} \\ \hline
		\$id    & Identifikator einer Ankündigung \\ \hline
		\$state & Aktivierungsstatus einer Ankündigung \\ \hline
	\end{tabular}
\end{table}
\paragraph{Beschreibung} Die Funktion setzt den Aktivierungsstatus einer bestimmten Ankündigung. Die Funktion hat Auswirkungen auf folgende Quellen:
\begin{itemize}
	\item Ankündigungstabelle
\end{itemize}
Es findet bei dieser Funktion kein Abruf von Daten aus {\glqq COSP\grqq} statt. Die Antwort wird als strukturiertes Array an den Aufrufer zurückgegeben.
\newpage
\section{basic-db}
\subsection{Allgemeines} Diese Datei enthält alle grundlegenden Funktionen für den Datenbankzugriff.
\begin{table}[H]
	\begin{tabular}{|c|p{11cm}|}
		\hline
		\textbf{Einbindungspunkt} & inc.php \\ \hline
		\textbf{Einbindungspunkt} & inc-sub.php \\ \hline
	\end{tabular}
\end{table}
Die Datei ist nicht direkt durch den Nutzer aufrufbar, dies wird durch folgenden Code-Ausschnitt sichergestellt:
\begin{lstlisting}[language=php]
if (!defined('NICE_PROJECT')) {
	die('Permission denied.');
}
\end{lstlisting}
Der Globale Wert {\glqq NICE\_PROJECT\grqq} wird durch für den Nutzer valide Aufrufpunkte festgelegt, z.B. {\glqq api.php\grqq}.
\newpage
\subsection{Funktionen}
\subsubsection{getPdo}
\paragraph{Parameter} Die Funktion besitzt keine Parameter.
\paragraph{Beschreibung} Die Funktion instanziiert die PDO-Klasse mit allen notwendigen Informationen. Die Funktion nutzt folgende Quellen:
\begin{itemize}
	\item Konfiguration
\end{itemize}
Es findet bei dieser Funktion kein Abruf von Daten aus {\glqq COSP\grqq} statt. Es wird eine PDO-Instanz zurück gegeben.
\subsubsection{ExecuteStatementWOR}
\paragraph{Parameter} Die Funktion besitzt folgende Parameter:
\begin{table}[H]
	\begin{tabular}{|c|p{11cm}|}
		\hline
		\textbf{Parametername} & \textbf{Parameterbeschreibung} \\ \hline
		\$prep\_stmt & Vorbereitete SQL-Abfrage \\ \hline
		\$params     & Array mit Parametern der Abfrage \\ \hline
	\end{tabular}
\end{table}
\subparagraph{\$params}Das Array enthält Elemente mit folgenden Elementen:
\begin{table}[H]
	\begin{tabular}{|c|p{11cm}|}
		\hline
		\textbf{Parametername} & \textbf{Parameterbeschreibung} \\ \hline
		val & Wert des Parameters \\ \hline
		typ & Typ des Parameters \\ \hline
	\end{tabular}
\end{table}
\paragraph{Beschreibung} Die Funktion dient dem ermitteln aller für die Anzeige des Persönlichen Bereiches benötigten Daten aus folgenden Quellen:
Es findet bei dieser Funktion kein Abruf von Daten aus {\glqq COSP\grqq} statt. Es wird eine Antwort zurück gegeben.
\paragraph{Vorgehensweise} Es werden die Parameter in die vorbereitete Abfrage in der Reihenfolge des Arrays eingebunden. Anschließend wird die Abfrage auf der Datenbank ausgeführt.
\subsubsection{ExecuteStatementWOR}
\paragraph{Parameter} Die Funktion besitzt folgende Parameter:
\begin{table}[H]
	\begin{tabular}{|c|p{11cm}|}
		\hline
		\textbf{Parametername} & \textbf{Parameterbeschreibung} \\ \hline
		\$prep\_stmt  & Vorbereitete SQL-Abfrage \\ \hline
		\$params      & Array mit Parametern der Abfrage \\ \hline
		\$read        & Schaltet das Lesen von Daten ab \\ \hline
		\$disableNull & Schaltet das setzten des Wertes {\glqq null\grqq} ab \\ \hline
	\end{tabular}
\end{table}
\subparagraph{\$params}Das Array enthält Elemente mit folgenden Elementen:
\begin{table}[H]
	\begin{tabular}{|c|p{11cm}|}
		\hline
		\textbf{Parametername} & \textbf{Parameterbeschreibung} \\ \hline
		val & Wert des Parameters \\ \hline
		typ & Typ des Parameters \\ \hline
		nam & Name des Parameters in der vorbereiteten Abfrage \\ \hline
	\end{tabular}
\end{table}
\paragraph{Beschreibung} Die Funktion dient dem ermitteln aller für die Anzeige des Persönlichen Bereiches benötigten Daten aus folgenden Quellen:
Es findet bei dieser Funktion kein Abruf von Daten aus {\glqq COSP\grqq} statt. Es wird eine Antwort zurück gegeben.
\paragraph{Vorgehensweise} Es werden die Parameter in die vorbereitete Abfrage in der Anhand der vergebenen Platzhalter eingebunden. Die Einbindung geschieht in der Reihenfolge der im Array enthaltenen Elemente. Anschließend wird die Abfrage auf der Datenbank ausgeführt. Sofern das Lesen aktiviert ist, werden die Ergebnisse als strukturiertes Array zurück gegeben. Bei Auftreten eines Fehlers ist dieser Rückgabewert.
\newpage
\section{cinema-type-db}
\subsection{Allgemeines} Diese Datei enthält alle Funktionen für die Typen-Tabelle.
\begin{table}[H]
	\begin{tabular}{|c|p{11cm}|}
		\hline
		\textbf{Einbindungspunkt} & inc-db.php \\ \hline
		\textbf{Einbindungspunkt} & inc-db-sub.php \\ \hline
	\end{tabular}
\end{table}
Die Datei ist nicht direkt durch den Nutzer aufrufbar, dies wird durch folgenden Code-Ausschnitt sichergestellt:
\begin{lstlisting}[language=php]
if (!defined('NICE_PROJECT')) {
	die('Permission denied.');
}
\end{lstlisting}
Der Globale Wert {\glqq NICE\_PROJECT\grqq} wird durch für den Nutzer valide Aufrufpunkte festgelegt, z.B. {\glqq api.php\grqq}.
\newpage
\subsection{Funktionen}
\subsubsection{getAllCinemaTypes}
\paragraph{Parameter} Die Funktion besitzt keine Parameter.
\paragraph{Beschreibung} Die Funktion fragt alle verfügbaren Kinotypen ab. Die Funktion nutzt folgenden Quellen:
\begin{itemize}
	\item Typen-Tabelle
\end{itemize}
Es findet bei dieser Funktion kein Abruf von Daten aus {\glqq COSP\grqq} statt. Die Antwort wird als strukturiertes Array an den Aufrufer zurückgegeben.
\subsubsection{getCinemaTypeNameByTypeId}
\paragraph{Parameter} Die Funktion besitzt folgende Parameter:
\begin{table}[H]
	\begin{tabular}{|c|p{11cm}|}
		\hline
		\textbf{Parametername} & \textbf{Parameterbeschreibung} \\ \hline
		\$id & numerischer Identifikator des Types \\ \hline
	\end{tabular}
\end{table}
\paragraph{Beschreibung} Die Funktion fragt den Namen eines Typs ab. Die Funktion nutzt folgenden Quellen:
\begin{itemize}
	\item Typen-Tabelle
\end{itemize}
Es findet bei dieser Funktion kein Abruf von Daten aus {\glqq COSP\grqq} statt. Die Antwort wird als Zeichenkette an den Aufrufer zurückgegeben.


\newpage
\section{cinemas-db}
\subsection{Allgemeines} Diese Datei enthält alle Funktionen, welche die Saalanzahl-Tabelle nutzen.
\begin{table}[H]
	\begin{tabular}{|c|p{11cm}|}
		\hline
		\textbf{Einbindungspunkt} & inc-db.php \\ \hline
		\textbf{Einbindungspunkt} & inc-db-sub.php \\ \hline
	\end{tabular}
\end{table}
Die Datei ist nicht direkt durch den Nutzer aufrufbar, dies wird durch folgenden Code-Ausschnitt sichergestellt:
\begin{lstlisting}[language=php]
if (!defined('NICE_PROJECT')) {
	die('Permission denied.');
}
\end{lstlisting}
Der Globale Wert {\glqq NICE\_PROJECT\grqq} wird durch für den Nutzer valide Aufrufpunkte festgelegt, z.B. {\glqq api.php\grqq}.
\newpage
\subsection{Funktionen}
\subsubsection{insertCinemasOfPOI}
\paragraph{Parameter} Die Funktion besitzt folgende Parameter:
\begin{table}[H]
	\begin{tabular}{|c|p{11cm}|}
		\hline
		\textbf{Parametername} & \textbf{Parameterbeschreibung} \\ \hline
		\$poi\_id       & Identifikator eines Interessenpunktes \\ \hline
		\$start         & Startjahr \\ \hline
		\$end           & Endjahr \\ \hline
		\$cinema\_count & Saalanzahl \\ \hline
	\end{tabular}
\end{table}
\paragraph{Beschreibung} Die Funktion fügt eine neue Sallanzahl hinzu. Die Funktion hat Auswirkungen auf folgende Quellen:
\begin{itemize}
	\item Saalanzahl-Tabelle
\end{itemize}
Es findet bei dieser Funktion kein Abruf von Daten aus {\glqq COSP\grqq} statt. Es wird eine Antwort zurück gegeben.
\subsubsection{getCinemasByPoiId}
\paragraph{Parameter} Die Funktion besitzt folgende Parameter:
\begin{table}[H]
	\begin{tabular}{|c|p{11cm}|}
		\hline
		\textbf{Parametername} & \textbf{Parameterbeschreibung} \\ \hline
		\$poiid       & Identifikator eines Interessenpunktes \\ \hline
	\end{tabular}
\end{table}
\paragraph{Beschreibung} Die Funktion fragt alle zu einem Interessenpunkt zugehörigen Saalanzahlen ab. Die Funktion nutzt folgende Quellen:
\begin{itemize}
	\item Saalanzahl-Tabelle
\end{itemize}
Es findet bei dieser Funktion kein Abruf von Daten aus {\glqq COSP\grqq} statt. Die Antwort wird als strukturiertes Array an den Aufrufer zurückgegeben.
\subsubsection{getCreatorByCinemasID}
\paragraph{Parameter} Die Funktion besitzt folgende Parameter:
\begin{table}[H]
	\begin{tabular}{|c|p{11cm}|}
		\hline
		\textbf{Parametername} & \textbf{Parameterbeschreibung} \\ \hline
		\$cinemas\_id       & Identifikator einer Saalanzahl \\ \hline
	\end{tabular}
\end{table}
\paragraph{Beschreibung} Die Funktion fragt den Ersteller einer Saalanzahl ab. Die Funktion nutzt folgende Quellen:
\begin{itemize}
	\item Saalanzahl-Tabelle
\end{itemize}
Es findet bei dieser Funktion kein Abruf von Daten aus {\glqq COSP\grqq} statt. Die Antwort wird als strukturiertes Array an den Aufrufer zurückgegeben.
\subsubsection{deleteCinemas}
\paragraph{Parameter} Die Funktion besitzt folgende Parameter:
\begin{table}[H]
	\begin{tabular}{|c|p{11cm}|}
		\hline
		\textbf{Parametername} & \textbf{Parameterbeschreibung} \\ \hline
		\$cinemas\_id       & Identifikator einer Saalanzahl \\ \hline
	\end{tabular}
\end{table}
\paragraph{Beschreibung} Die Funktion löscht eine Saalanzahl. Die Funktion hat Auswirkung auf folgende Quellen:
\begin{itemize}
	\item Saalanzahl-Tabelle
\end{itemize}
Es findet bei dieser Funktion kein Abruf von Daten aus {\glqq COSP\grqq} statt. Die Antwort wird als strukturiertes Array an den Aufrufer zurückgegeben.
\subsubsection{updateCinemas}
\paragraph{Parameter} Die Funktion besitzt folgende Parameter:
\begin{table}[H]
	\begin{tabular}{|c|p{11cm}|}
		\hline
		\textbf{Parametername} & \textbf{Parameterbeschreibung} \\ \hline
		\$id       & Identifikator einer Saalanzahl \\ \hline
		\$cinemas  & Saalanzahl \\ \hline
		\$start    & Startjahr \\ \hline
		\$end      & Endjahr \\ \hline
	\end{tabular}
\end{table}
\paragraph{Beschreibung} Die Funktion aktualisiert eine Saalanzahl. Die Funktion hat Auswirkung auf folgende Quellen:
\begin{itemize}
	\item Saalanzahl-Tabelle
\end{itemize}
Es findet bei dieser Funktion kein Abruf von Daten aus {\glqq COSP\grqq} statt. Es wird eine Antwort zurück gegeben.
\subsubsection{updateDeletionStateCinemasById}
\paragraph{Parameter} Die Funktion besitzt folgende Parameter:
\begin{table}[H]
	\begin{tabular}{|c|p{11cm}|}
		\hline
		\textbf{Parametername} & \textbf{Parameterbeschreibung} \\ \hline
		\$id       & Identifikator einer Saalanzahl \\ \hline
		\$state    & Status der Löschung \\ \hline
	\end{tabular}
\end{table}
\paragraph{Beschreibung} Die Funktion setzt den Löschungsstatus einer Saalanzahl. Die Funktion hat Auswirkung auf folgende Quellen:
\begin{itemize}
	\item Saalanzahl-Tabelle
\end{itemize}
Es findet bei dieser Funktion kein Abruf von Daten aus {\glqq COSP\grqq} statt. Es wird eine Antwort zurück gegeben.
\newpage
\section{hist-adr-db}
\subsection{Allgemeines} Diese Datei enthält Funktionen, welche auf die Datenbanktabelle mit historischen Adressen zugreifen.
\begin{table}[H]
	\begin{tabular}{|c|p{11cm}|}
		\hline
		\textbf{Einbindungspunkt} & inc-db.php \\ \hline
		\textbf{Einbindungspunkt} & inc-db-sub.php \\ \hline
	\end{tabular}
\end{table}
Die Datei ist nicht direkt durch den Nutzer aufrufbar, dies wird durch folgenden Code-Ausschnitt sichergestellt:
\begin{lstlisting}[language=php]
if (!defined('NICE_PROJECT')) {
	die('Permission denied.');
}
\end{lstlisting}
Der Globale Wert {\glqq NICE\_PROJECT\grqq} wird durch für den Nutzer valide Aufrufpunkte festgelegt, z.B. {\glqq api.php\grqq}.
\newpage
\subsection{Funktionen}
\subsubsection{insertCinemasOfPOI}
\paragraph{Parameter} Die Funktion besitzt folgende Parameter:
\begin{table}[H]
	\begin{tabular}{|c|p{11cm}|}
		\hline
		\textbf{Parametername} & \textbf{Parameterbeschreibung} \\ \hline
		\$poi\_id       & Identifikator des zugehörigen Interessenpunktes \\ \hline
		\$start         & Startjahr \\ \hline
		\$end           & Endjahr \\ \hline
		\$cinema\_count & Anzahl der Kinosäle \\ \hline
	\end{tabular}
\end{table}
\paragraph{Beschreibung} Die Funktion fügt einem Interessenpunkt eine neue Saalanzahl hinzu. Die Funktion hat Auswirkungen auf folgende Quellen:
\begin{itemize}
	\item Tabelle mit Kinosaalanzahlen
\end{itemize}
Es findet bei dieser Funktion kein Abruf von Daten aus {\glqq COSP\grqq} statt. Es wird eine Antwort zurück gegeben.
\subsubsection{getCinemasByPoiId}
\paragraph{Parameter} Die Funktion besitzt folgende Parameter:
\begin{table}[H]
	\begin{tabular}{|c|p{11cm}|}
		\hline
		\textbf{Parametername} & \textbf{Parameterbeschreibung} \\ \hline
		\$poiid & Identifikator des Interessenpunktes \\ \hline
	\end{tabular}
\end{table}
\paragraph{Beschreibung} Die Funktion ermittelt alle Kinosaalanzahlen eines Interessenpunktes. Die Funktion nutzt folgende Quellen:
\begin{itemize}
	\item Tabelle mit Kinosaalanzahlen
\end{itemize}
Es findet bei dieser Funktion kein Abruf von Daten aus {\glqq COSP\grqq} statt. Die Antwort wird als strukturiertes Array an den Aufrufer zurückgegeben.
\subsubsection{getCreatorByCinemasID}
\paragraph{Parameter} Die Funktion besitzt folgende Parameter:
\begin{table}[H]
	\begin{tabular}{|c|p{11cm}|}
		\hline
		\textbf{Parametername} & \textbf{Parameterbeschreibung} \\ \hline
		\$cinemas\_id & Identifikator der Saalanzahl \\ \hline
	\end{tabular}
\end{table}
\paragraph{Beschreibung} Die Funktion ermittelt den Ersteller einer Kinosaalanzahl. Die Funktion nutzt folgende Quellen:
\begin{itemize}
	\item Tabelle mit Kinosaalanzahlen
	\item Nutzer-Tabelle
\end{itemize}
Es findet bei dieser Funktion kein Abruf von Daten aus {\glqq COSP\grqq} statt. Die Antwort ist eine Zeichenkette.
\subsubsection{deleteCinemas}
\paragraph{Parameter} Die Funktion besitzt folgende Parameter:
\begin{table}[H]
	\begin{tabular}{|c|p{11cm}|}
		\hline
		\textbf{Parametername} & \textbf{Parameterbeschreibung} \\ \hline
		\$cinemas\_id & Identifikator der Saalanzahl \\ \hline
	\end{tabular}
\end{table}
\paragraph{Beschreibung} Die Funktion löscht eine Kinosaalanzahl. Die Funktion hat Auswirkungen auf folgende Quellen:
\begin{itemize}
	\item Tabelle mit Kinosaalanzahlen
\end{itemize}
Es findet bei dieser Funktion kein Abruf von Daten aus {\glqq COSP\grqq} statt. Es wird eine Antwort zurück gegeben.
\subsubsection{updateCinemas}
\paragraph{Parameter} Die Funktion besitzt folgende Parameter:
\begin{table}[H]
	\begin{tabular}{|c|p{11cm}|}
		\hline
		\textbf{Parametername} & \textbf{Parameterbeschreibung} \\ \hline
		\$id      & Identifikator der Saalanzahl \\ \hline
		\$cinemas & Anzahl der Kinosäle \\ \hline
		\$start   & Startjahr \\ \hline
		\$end     & Endjahr \\ \hline
	\end{tabular}
\end{table}
\paragraph{Beschreibung} Die Funktion aktualisiert eine Saalanzahl. Die Funktion hat Auswirkungen auf folgende Quellen:
\begin{itemize}
	\item Tabelle mit Kinosaalanzahlen
\end{itemize}
Es findet bei dieser Funktion kein Abruf von Daten aus {\glqq COSP\grqq} statt. Es wird eine Antwort zurück gegeben.
\subsubsection{updateDeletionStateCinemasById}
\paragraph{Parameter} Die Funktion besitzt folgende Parameter:
\begin{table}[H]
	\begin{tabular}{|c|p{11cm}|}
		\hline
		\textbf{Parametername} & \textbf{Parameterbeschreibung} \\ \hline
		\$id    & Identifikator der Saalanzahl \\ \hline
		\$state & Status der Löschung \\ \hline
	\end{tabular}
\end{table}
\paragraph{Beschreibung} Die Funktion markiert eine Kinosaalanzahl als gelöscht. Die Funktion hat Auswirkungen auf folgende Quellen:
\begin{itemize}
	\item Tabelle mit Kinosaalanzahlen
\end{itemize}
Es findet bei dieser Funktion kein Abruf von Daten aus {\glqq COSP\grqq} statt. Es wird eine Antwort zurück gegeben.
\newpage
\section{inc-db-sub}
\subsection{Allgemeines} Diese Datei Einbindungen aller benötigten Dateien für den Datenbankzugriff.
\begin{table}[H]
	\begin{tabular}{|c|p{11cm}|}
		\hline
		\textbf{Einbindungspunkt} & inc-sub.php \\ \hline
	\end{tabular}
\end{table}
Die Datei ist nicht direkt durch den Nutzer aufrufbar, dies wird durch folgenden Code-Ausschnitt sichergestellt:
\begin{lstlisting}[language=php]
if (!defined('NICE_PROJECT')) {
	die('Permission denied.');
}
\end{lstlisting}
Der Globale Wert {\glqq NICE\_PROJECT\grqq} wird durch für den Nutzer valide Aufrufpunkte festgelegt, z.B. {\glqq api.php\grqq}.
\newpage
\subsection{Einbindungen}
\subsubsection{Grundlegendes}
Nachfolgend zu sehender Code-Block bindet alle benötigten Dateien in korrekter Reihenfolge ein. Beim Einbinden neuer Dateien, sind diese stets an das Ende zu schreiben, außer die Dateien sind Umstrukturierungen bereits existenten Dateien.
\begin{lstlisting}[language=php]
require_once '../bin/database/basic-db.php';
require_once '../bin/database/statistics-basic-dbfunctions.php';
require_once '../bin/database/user.php';
require_once '../bin/database/poi-db.php';
require_once '../bin/database/poi-val.php';
require_once '../bin/database/poi-comment.php';
require_once '../bin/database/validate-hist-adr.php';
require_once '../bin/database/validate-curr-adr.php';
require_once '../bin/database/validate-hist.php';
require_once '../bin/database/validate-type.php';
require_once '../bin/database/validate-name.php';
require_once '../bin/database/validate-operator.php';
require_once '../bin/database/validate-timespan.php';
require_once '../bin/database/validate-poi-story.php';
require_once '../bin/database/validate-seats.php';
require_once '../bin/database/validate-cinemas.php';
require_once '../bin/database/operators-db.php';
require_once '../bin/database/hist-adr-db.php';
require_once '../bin/database/names-db.php';
require_once '../bin/database/seats-db.php';
require_once '../bin/database/cinemas-db.php';
require_once '../bin/database/poi-story-db.php';
require_once '../bin/database/logging.php';
require_once '../bin/database/poi-picture-db.php';
require_once '../bin/database/validate-poi-picture.php';
require_once '../bin/database/cinema-type-db.php';
require_once '../bin/database/statistics-poi-db.php';
require_once '../bin/database/statistics-comments-db.php';
require_once '../bin/database/announcement-db.php';
require_once '../bin/database/source-type-db.php';
require_once '../bin/database/source-relation-db.php';
require_once '../bin/database/poi-source-db.php';
\end{lstlisting}
\subsubsection{Besonderheit}
Die Einbindungen sind immer mit {\glqq ../\grqq} anzufangen, da Sie für Subordner des Hauptordners gedacht sind.
\newpage
\section{inc-db}
\subsection{Allgemeines} Diese Datei Einbindungen aller benötigten Dateien für den Datenbankzugriff.
Die Datei ist nicht direkt durch den Nutzer aufrufbar, dies wird durch folgenden Code-Ausschnitt sichergestellt:
\begin{lstlisting}[language=php]
if (!defined('NICE_PROJECT')) {
	die('Permission denied.');
}
\end{lstlisting}
Der Globale Wert {\glqq NICE\_PROJECT\grqq} wird durch für den Nutzer valide Aufrufpunkte festgelegt, z.B. {\glqq api.php\grqq}.
\newpage
\subsection{Einbindungen}
\subsubsection{Grundlegendes}
Nachfolgend zu sehender Code-Block bindet alle benötigten Dateien in korrekter Reihenfolge ein. Beim Einbinden neuer Dateien, sind diese stets an das Ende zu schreiben, außer die Dateien sind Umstrukturierungen bereits existenten Dateien.
\begin{lstlisting}[language=php]
require_once 'bin/database/basic-db.php';
require_once 'bin/database/statistics-basic-dbfunctions.php';
require_once 'bin/database/user.php';
require_once 'bin/database/poi-db.php';
require_once 'bin/database/poi-val.php';
require_once 'bin/database/poi-comment.php';
require_once 'bin/database/validate-hist-adr.php';
require_once 'bin/database/validate-curr-adr.php';
require_once 'bin/database/validate-hist.php';
require_once 'bin/database/validate-type.php';
require_once 'bin/database/validate-name.php';
require_once 'bin/database/validate-operator.php';
require_once 'bin/database/validate-timespan.php';
require_once 'bin/database/validate-poi-story.php';
require_once 'bin/database/validate-seats.php';
require_once 'bin/database/validate-cinemas.php';
require_once 'bin/database/operators-db.php';
require_once 'bin/database/hist-adr-db.php';
require_once 'bin/database/names-db.php';
require_once 'bin/database/seats-db.php';
require_once 'bin/database/cinemas-db.php';
require_once 'bin/database/poi-story-db.php';
require_once 'bin/database/logging.php';
require_once 'bin/database/poi-picture-db.php';
require_once 'bin/database/validate-poi-picture.php';
require_once 'bin/database/cinema-type-db.php';
require_once 'bin/database/statistics-poi-db.php';
require_once 'bin/database/statistics-comments-db.php';
require_once 'bin/database/announcement-db.php';
require_once 'bin/database/source-type-db.php';
require_once 'bin/database/source-relation-db.php';
require_once 'bin/database/poi-source-db.php';
\end{lstlisting}
\subsubsection{Besonderheit}
Die Einbindungen sind immer vom Hauptordner aus zu erreichen und auch relativ zu diesem anzugeben.
\newpage
\section{logging}
\subsection{Allgemeines} Diese Datei enthält Funktionen zum Abfragen der Tabelle zum erfassen statistischer Nutzer.
\begin{table}[H]
	\begin{tabular}{|c|p{11cm}|}
		\hline
		\textbf{Einbindungspunkt} & inc-db.php \\ \hline
		\textbf{Einbindungspunkt} & inc-db-sub.php \\ \hline
	\end{tabular}
\end{table}
Die Datei ist nicht direkt durch den Nutzer aufrufbar, dies wird durch folgenden Code-Ausschnitt sichergestellt:
\begin{lstlisting}[language=php]
if (!defined('NICE_PROJECT')) {
	die('Permission denied.');
}
\end{lstlisting}
Der Globale Wert {\glqq NICE\_PROJECT\grqq} wird durch für den Nutzer valide Aufrufpunkte festgelegt, z.B. {\glqq api.php\grqq}.
\newpage
\subsection{Funktionen}
\subsubsection{insertLogUniqueVisitors}
\paragraph{Parameter} Die Funktion besitzt folgende Parameter:
\begin{table}[H]
	\begin{tabular}{|c|p{11cm}|}
		\hline
		\textbf{Parametername} & \textbf{Parameterbeschreibung} \\ \hline
		\$ip   & Ip-Adresse des Aufrufers \\ \hline
		\$type & Typ der Nutzung (Gast oder Nutzer) \\ \hline
	\end{tabular}
\end{table}
\paragraph{Beschreibung} Die Funktion trägt eine neue Nutzung in die Tabelle ein. Die Funktion hat Auswirkung auf folgende Quellen:
\begin{itemize}
	\item Tabelle mit statistischen Nutzerdaten
\end{itemize}
Es findet bei dieser Funktion kein Abruf von Daten aus {\glqq COSP\grqq} statt. Die Funktion gibt eine Antwort zurück.
\subsubsection{getStatisticalDataLastWeeks}
\paragraph{Parameter} Die Funktion besitzt folgende Parameter:
\begin{table}[H]
	\begin{tabular}{|c|p{11cm}|}
		\hline
		\textbf{Parametername} & \textbf{Parameterbeschreibung} \\ \hline
		\$number & Anzahl der Wochen \\ \hline
	\end{tabular}
\end{table}
\paragraph{Beschreibung} Die Funktion fragt statistische Daten zu Nutzern für den in Wochen angegeben Zeitraum ab. Die Funktion nutzt folgende Quellen:
\begin{itemize}
	\item Tabelle mit statistischen Nutzerdaten
\end{itemize}
Es findet bei dieser Funktion kein Abruf von Daten aus {\glqq COSP\grqq} statt. Die Antwort wird als strukturiertes Array an den Aufrufer zurückgegeben.
\subsubsection{getStatisticalDataLastMonth}
\paragraph{Parameter} Die Funktion besitzt folgende Parameter:
\begin{table}[H]
	\begin{tabular}{|c|p{11cm}|}
		\hline
		\textbf{Parametername} & \textbf{Parameterbeschreibung} \\ \hline
		\$number & Anzahl der Monate \\ \hline
	\end{tabular}
\end{table}
\paragraph{Beschreibung} Die Funktion fragt statistische Daten zu Nutzern für den in Monaten angegeben Zeitraum ab. Die Funktion nutzt folgende Quellen:
\begin{itemize}
	\item Tabelle mit statistischen Nutzerdaten
\end{itemize}
Es findet bei dieser Funktion kein Abruf von Daten aus {\glqq COSP\grqq} statt. Die Antwort wird als strukturiertes Array an den Aufrufer zurückgegeben.
\subsubsection{getStatisticalDataLastYear}
\paragraph{Parameter} Die Funktion besitzt folgende Parameter:
\begin{table}[H]
	\begin{tabular}{|c|p{11cm}|}
		\hline
		\textbf{Parametername} & \textbf{Parameterbeschreibung} \\ \hline
		\$number & Anzahl der Jahre \\ \hline
	\end{tabular}
\end{table}
\paragraph{Beschreibung} Die Funktion fragt statistische Daten zu Nutzern für den in Jahren angegeben Zeitraum ab. Die Funktion nutzt folgende Quellen:
\begin{itemize}
	\item Tabelle mit statistischen Nutzerdaten
\end{itemize}
Es findet bei dieser Funktion kein Abruf von Daten aus {\glqq COSP\grqq} statt. Die Antwort wird als strukturiertes Array an den Aufrufer zurückgegeben.
\subsubsection{getStatisticalDataLastDays}
\paragraph{Parameter} Die Funktion besitzt folgende Parameter:
\begin{table}[H]
	\begin{tabular}{|c|p{11cm}|}
		\hline
		\textbf{Parametername} & \textbf{Parameterbeschreibung} \\ \hline
		\$number & Anzahl der Tage \\ \hline
	\end{tabular}
\end{table}
\paragraph{Beschreibung} Die Funktion fragt statistische Daten zu Nutzern für den in Tagen angegeben Zeitraum ab. Die Funktion nutzt folgende Quellen:
\begin{itemize}
	\item Tabelle mit statistischen Nutzerdaten
\end{itemize}
Es findet bei dieser Funktion kein Abruf von Daten aus {\glqq COSP\grqq} statt. Die Antwort wird als strukturiertes Array an den Aufrufer zurückgegeben.

\newpage
\section{names-db}
\subsection{Allgemeines} Diese Datei enthält alle Funktionen, welche die Namen-Tabelle benutzen.
\begin{table}[H]
	\begin{tabular}{|c|p{11cm}|}
		\hline
		\textbf{Einbindungspunkt} & inc-db.php \\ \hline
		\textbf{Einbindungspunkt} & inc-db-sub.php \\ \hline
	\end{tabular}
\end{table}
Die Datei ist nicht direkt durch den Nutzer aufrufbar, dies wird durch folgenden Code-Ausschnitt sichergestellt:
\begin{lstlisting}[language=php]
if (!defined('NICE_PROJECT')) {
	die('Permission denied.');
}
\end{lstlisting}
Der Globale Wert {\glqq NICE\_PROJECT\grqq} wird durch für den Nutzer valide Aufrufpunkte festgelegt, z.B. {\glqq api.php\grqq}.
\newpage
\subsection{Funktionen}
\subsubsection{getNamesByPoiId}
\paragraph{Parameter} Die Funktion besitzt folgende Parameter:
\begin{table}[H]
	\begin{tabular}{|c|p{11cm}|}
		\hline
		\textbf{Parametername} & \textbf{Parameterbeschreibung} \\ \hline
		\$poiid & Identifikator eines Interessenpunktes \\ \hline
	\end{tabular}
\end{table}
\paragraph{Beschreibung} Die Funktion fragt alle Namen eines Interessenpunktes ab. Die Funktion nutzt folgende Quellen:
\begin{itemize}
	\item Namen-Tabelle
\end{itemize}
Es findet bei dieser Funktion kein Abruf von Daten aus {\glqq COSP\grqq} statt. Die Antwort wird als strukturiertes Array an den Aufrufer zurückgegeben.
\subsubsection{insertNameOfPOI}
\paragraph{Parameter} Die Funktion besitzt folgende Parameter:
\begin{table}[H]
	\begin{tabular}{|c|p{11cm}|}
		\hline
		\textbf{Parametername} & \textbf{Parameterbeschreibung} \\ \hline
		\$poi\_id & Identifikator des zugehörigen Interessenpunktes \\ \hline
		\$start   & Startjahr \\ \hline
		\$end     & Endjahr \\ \hline
		\$name    & Name des Kinos \\ \hline
	\end{tabular}
\end{table}
\paragraph{Beschreibung} Die Funktion fügt einem Interessenpunkt einen Namen hinzu. Die Funktion hat Auswirkungen auf folgende Quellen:
\begin{itemize}
	\item Namen-Tabelle
\end{itemize}
Es findet bei dieser Funktion kein Abruf von Daten aus {\glqq COSP\grqq} statt. Es gibt einen Rückgabewert.
\subsubsection{getUsernameByNameId}
\paragraph{Parameter} Die Funktion besitzt folgende Parameter:
\begin{table}[H]
	\begin{tabular}{|c|p{11cm}|}
		\hline
		\textbf{Parametername} & \textbf{Parameterbeschreibung} \\ \hline
		\$Nameid & Identifikator des Namens \\ \hline
	\end{tabular}
\end{table}
\paragraph{Beschreibung} Die Funktion fragt den Ersteller eines Namen ab. Die Funktion nutzt folgende Quellen:
\begin{itemize}
	\item Namen-Tabelle
\end{itemize}
Es findet bei dieser Funktion kein Abruf von Daten aus {\glqq COSP\grqq} statt. Die Antwort ist eine Zeichenkette.
\subsubsection{updateName}
\paragraph{Parameter} Die Funktion besitzt folgende Parameter:
\begin{table}[H]
	\begin{tabular}{|c|p{11cm}|}
		\hline
		\textbf{Parametername} & \textbf{Parameterbeschreibung} \\ \hline
		\$id    & Identifikator des Namens \\ \hline
		\$name  & Name des Kinos \\ \hline
		\$start & Startjahr \\ \hline
		\$end   & Endjahr \\ \hline
	\end{tabular}
\end{table}
\paragraph{Beschreibung} Die Funktion aktualisiert einen Namen. Die Funktion hat Auswirkungen auf folgende Quellen:
\begin{itemize}
	\item Namen-Tabelle
\end{itemize}
Es findet bei dieser Funktion kein Abruf von Daten aus {\glqq COSP\grqq} statt. Es gibt einen Rückgabewert.
\subsubsection{deleteName}
\paragraph{Parameter} Die Funktion besitzt folgende Parameter:
\begin{table}[H]
	\begin{tabular}{|c|p{11cm}|}
		\hline
		\textbf{Parametername} & \textbf{Parameterbeschreibung} \\ \hline
		\$name\_id & Identifikator des Namens \\ \hline
	\end{tabular}
\end{table}
\paragraph{Beschreibung} Die Funktion löscht einen Namen. Die Funktion hat Auswirkungen auf folgende Quellen:
\begin{itemize}
	\item Namen-Tabelle
\end{itemize}
Es findet bei dieser Funktion kein Abruf von Daten aus {\glqq COSP\grqq} statt. Es gibt einen Rückgabewert.
\subsubsection{updateDeletionStateNamesByID}
\paragraph{Parameter} Die Funktion besitzt folgende Parameter:
\begin{table}[H]
	\begin{tabular}{|c|p{11cm}|}
		\hline
		\textbf{Parametername} & \textbf{Parameterbeschreibung} \\ \hline
		\$id    & Identifikator des Namens \\ \hline
		\$state & Status der Löschung \\ \hline
	\end{tabular}
\end{table}
\paragraph{Beschreibung} Die Funktion markiert einen Namen als gelöscht. Die Funktion hat Auswirkungen auf folgende Quellen:
\begin{itemize}
	\item Namen-Tabelle
\end{itemize}
Es findet bei dieser Funktion kein Abruf von Daten aus {\glqq COSP\grqq} statt. Es gibt einen Rückgabewert.
\newpage
\section{operators-db}
\subsection{Allgemeines} Diese Datei enthält alle Funktionen, welche die Betreiber-Tabelle benutzen.
\begin{table}[H]
	\begin{tabular}{|c|p{11cm}|}
		\hline
		\textbf{Einbindungspunkt} & inc-db.php \\ \hline
		\textbf{Einbindungspunkt} & inc-db-sub.php \\ \hline
	\end{tabular}
\end{table}
Die Datei ist nicht direkt durch den Nutzer aufrufbar, dies wird durch folgenden Code-Ausschnitt sichergestellt:
\begin{lstlisting}[language=php]
	if (!defined('NICE_PROJECT')) {
		die('Permission denied.');
	}
\end{lstlisting}
Der Globale Wert {\glqq NICE\_PROJECT\grqq} wird durch für den Nutzer valide Aufrufpunkte festgelegt, z.B. {\glqq api.php\grqq}.
\newpage
\subsection{Funktionen}
\subsubsection{getOpertorsByPoiId}
\paragraph{Parameter} Die Funktion besitzt folgende Parameter:
\begin{table}[H]
	\begin{tabular}{|c|p{11cm}|}
		\hline
		\textbf{Parametername} & \textbf{Parameterbeschreibung} \\ \hline
		\$poiid & Identifikator eines Interessenpunktes \\ \hline
	\end{tabular}
\end{table}
\paragraph{Beschreibung} Die Funktion sucht alle Betreiber eines Interessenpunktes. Die Funktion nutzt hierzu folgende Quellen:
\begin{itemize}
	\item Betreiber-Tabelle
\end{itemize}
Es findet bei dieser Funktion kein Abruf von Daten aus {\glqq COSP\grqq} statt. Die Antwort wird als strukturiertes Array an den Aufrufer zurückgegeben.
\subsubsection{insertOperator}
\paragraph{Parameter} Die Funktion besitzt folgende Parameter:
\begin{table}[H]
	\begin{tabular}{|c|p{11cm}|}
		\hline
		\textbf{Parametername} & \textbf{Parameterbeschreibung} \\ \hline
		\$poi\_id  & Identifikator des zugehörigen Interessenpunktes \\ \hline
		\$start    & Startjahr \\ \hline
		\$end      & Endjahr \\ \hline
		\$operator & Name des Betreibers \\ \hline
	\end{tabular}
\end{table}
\paragraph{Beschreibung} Die Funktion fügt einem Interessenpunkt einen neuen Betreiber hinzu. Die Funktion hat Auswirkungen auf folgende Quellen:
\begin{itemize}
	\item Betreiber-Tabelle
\end{itemize}
Es findet bei dieser Funktion kein Abruf von Daten aus {\glqq COSP\grqq} statt. Es gibt eine Antwort.
\subsubsection{getCreatorByOperatorID}
\paragraph{Parameter} Die Funktion besitzt folgende Parameter:
\begin{table}[H]
	\begin{tabular}{|c|p{11cm}|}
		\hline
		\textbf{Parametername} & \textbf{Parameterbeschreibung} \\ \hline
		\$operator\_id & Identifikator des Betreibers \\ \hline
	\end{tabular}
\end{table}
\paragraph{Beschreibung} Die Funktion sucht der Ersteller des gegebenen Betreibers. Die Funktion nutzt hierzu folgende Quellen:
\begin{itemize}
	\item Betreiber-Tabelle
\end{itemize}
Es findet bei dieser Funktion kein Abruf von Daten aus {\glqq COSP\grqq} statt. Der Rückgabewert ist eine Zeichenkette.
\subsubsection{deleteOperator}
\paragraph{Parameter} Die Funktion besitzt folgende Parameter:
\begin{table}[H]
	\begin{tabular}{|c|p{11cm}|}
		\hline
		\textbf{Parametername} & \textbf{Parameterbeschreibung} \\ \hline
		\$operator\_id  & Identifikator des Betreibers \\ \hline
	\end{tabular}
\end{table}
\paragraph{Beschreibung} Die Funktion löscht einen Betreiber. Die Funktion hat Auswirkungen auf folgende Quellen:
\begin{itemize}
	\item Betreiber-Tabelle
\end{itemize}
Es findet bei dieser Funktion kein Abruf von Daten aus {\glqq COSP\grqq} statt. Es gibt eine Antwort.
\subsubsection{updateOperator}
\paragraph{Parameter} Die Funktion besitzt folgende Parameter:
\begin{table}[H]
	\begin{tabular}{|c|p{11cm}|}
		\hline
		\textbf{Parametername} & \textbf{Parameterbeschreibung} \\ \hline
		\$id       & Identifikator des Betreibers \\ \hline
		\$operator & Name des Betreibers \\ \hline
		\$start    & Startjahr \\ \hline
		\$end      & Endjahr \\ \hline
	\end{tabular}
\end{table}
\paragraph{Beschreibung} Die Funktion aktualisiert einen Betreiber. Die Funktion hat Auswirkungen auf folgende Quellen:
\begin{itemize}
	\item Betreiber-Tabelle
\end{itemize}
Es findet bei dieser Funktion kein Abruf von Daten aus {\glqq COSP\grqq} statt. Es gibt eine Antwort.
\subsubsection{updateDeletionStateOperatorsById}
\paragraph{Parameter} Die Funktion besitzt folgende Parameter:
\begin{table}[H]
	\begin{tabular}{|c|p{11cm}|}
		\hline
		\textbf{Parametername} & \textbf{Parameterbeschreibung} \\ \hline
		\$id    & Identifikator des Betreibers \\ \hline
		\$state & Status der Löschung \\ \hline
	\end{tabular}
\end{table}
\paragraph{Beschreibung} Die Funktion markiert einen Betreiber als gelöscht. Die Funktion hat Auswirkungen auf folgende Quellen:
\begin{itemize}
	\item Betreiber-Tabelle
\end{itemize}
Es findet bei dieser Funktion kein Abruf von Daten aus {\glqq COSP\grqq} statt. Es gibt eine Antwort.
\newpage
\section{poi-comment}
\subsection{Allgemeines} Diese Datei enthält alle Funktionen, welche die Kommentar-Tabelle benutzen.
\begin{table}[H]
	\begin{tabular}{|c|p{11cm}|}
		\hline
		\textbf{Einbindungspunkt} & inc-db.php \\ \hline
		\textbf{Einbindungspunkt} & inc-db-sub.php \\ \hline
	\end{tabular}
\end{table}
Die Datei ist nicht direkt durch den Nutzer aufrufbar, dies wird durch folgenden Code-Ausschnitt sichergestellt:
\begin{lstlisting}[language=php]
	if (!defined('NICE_PROJECT')) {
		die('Permission denied.');
	}
\end{lstlisting}
Der Globale Wert {\glqq NICE\_PROJECT\grqq} wird durch für den Nutzer valide Aufrufpunkte festgelegt, z.B. {\glqq api.php\grqq}.
\newpage
\subsection{Funktionen}
\subsubsection{insertComment}
\paragraph{Parameter} Die Funktion besitzt folgende Parameter:
\begin{table}[H]
	\begin{tabular}{|c|p{11cm}|}
		\hline
		\textbf{Parametername} & \textbf{Parameterbeschreibung} \\ \hline
		\$data & Array mit Eingabewerten \\ \hline
	\end{tabular}
\end{table}
\subparagraph{\$data}Das Array enthält folgende Elemente:
\begin{table}[H]
	\begin{tabular}{|c|p{11cm}|}
		\hline
		\textbf{Parametername} & \textbf{Parameterbeschreibung} \\ \hline
		poi\_id & Identifikator eines Interessenpunktes  \\ \hline
		comment & Inhalt des Kommentars \\ \hline
	\end{tabular}
\end{table}
\paragraph{Beschreibung} Die Funktion fügt einem Interessenpunkt einen Kommentar hinzu. Die Funktion hat Auswirkungen auf folgende Quellen:
\begin{itemize}
	\item Kommentar-Tabelle
\end{itemize}
Es findet bei dieser Funktion kein Abruf von Daten aus {\glqq COSP\grqq} statt. Es gibt einen Rückgabewert.
\subsubsection{selectComments}
\paragraph{Parameter} Die Funktion besitzt folgende Parameter:
\begin{table}[H]
	\begin{tabular}{|c|p{11cm}|}
		\hline
		\textbf{Parametername} & \textbf{Parameterbeschreibung} \\ \hline
		\$poi\_id & Identifikator eines Interessenpunktes \\ \hline
	\end{tabular}
\end{table}
\paragraph{Beschreibung} Die Funktion fragt alle Kommentare zu einem gegebenen Interessenpunkt ab. Die Funktion nutzt folgende Quellen:
\begin{itemize}
	\item Kommentar-Tabelle
\end{itemize}
Es findet bei dieser Funktion kein Abruf von Daten aus {\glqq COSP\grqq} statt. Die Antwort wird als strukturiertes Array an den Aufrufer zurückgegeben.
\subsubsection{deleteComment}
\paragraph{Parameter} Die Funktion besitzt folgende Parameter:
\begin{table}[H]
	\begin{tabular}{|c|p{11cm}|}
		\hline
		\textbf{Parametername} & \textbf{Parameterbeschreibung} \\ \hline
		\$id & Identifikator des Kommentars \\ \hline
	\end{tabular}
\end{table}
\paragraph{Beschreibung} Die Funktion löscht einen Kommentar. Die Funktion hat Auswirkungen auf folgende Quellen:
\begin{itemize}
	\item Kommentar-Tabelle
\end{itemize}
Es findet bei dieser Funktion kein Abruf von Daten aus {\glqq COSP\grqq} statt. Es gibt einen Rückgabewert.
\subsubsection{deleteCommentByPOI}
\paragraph{Parameter} Die Funktion besitzt folgende Parameter:
\begin{table}[H]
	\begin{tabular}{|c|p{11cm}|}
		\hline
		\textbf{Parametername} & \textbf{Parameterbeschreibung} \\ \hline
		\$poiid & Identifikator des Interessenpunktes \\ \hline
	\end{tabular}
\end{table}
\paragraph{Beschreibung} Die Funktion löscht alle Kommentare eines Interessenpunktes. Die Funktion hat Auswirkungen auf folgende Quellen:
\begin{itemize}
	\item Kommentar-Tabelle
\end{itemize}
Es findet bei dieser Funktion kein Abruf von Daten aus {\glqq COSP\grqq} statt. Es gibt einen Rückgabewert.
\subsubsection{selectCommentsByCommentID}
\paragraph{Parameter} Die Funktion besitzt folgende Parameter:
\begin{table}[H]
	\begin{tabular}{|c|p{11cm}|}
		\hline
		\textbf{Parametername} & \textbf{Parameterbeschreibung} \\ \hline
		\$commentID & Identifikator eines Kommentars \\ \hline
	\end{tabular}
\end{table}
\paragraph{Beschreibung} Die Funktion fragt einen bestimmten Kommentar ab. Die Funktion nutzt folgende Quellen:
\begin{itemize}
	\item Kommentar-Tabelle
\end{itemize}
Es findet bei dieser Funktion kein Abruf von Daten aus {\glqq COSP\grqq} statt. Die Antwort wird als strukturiertes Array an den Aufrufer zurückgegeben.
\subsubsection{selectCommentsByCommentID}
\paragraph{Parameter} Die Funktion besitzt folgende Parameter:
\begin{table}[H]
	\begin{tabular}{|c|p{11cm}|}
		\hline
		\textbf{Parametername} & \textbf{Parameterbeschreibung} \\ \hline
		\$commentID      & Identifikator eines Kommentars \\ \hline
		\$commentContent & Inhalt des Kommentars \\ \hline
	\end{tabular}
\end{table}
\paragraph{Beschreibung} Die Funktion aktualisiert einen bestimmten Kommentar. Die Funktion hat Auswirkungen folgende Quellen:
\begin{itemize}
	\item Kommentar-Tabelle
\end{itemize}
Es findet bei dieser Funktion kein Abruf von Daten aus {\glqq COSP\grqq} statt. Es gibt einen Rückgabewert.
\subsubsection{getAllCommentsOfUser}
\paragraph{Parameter} Die Funktion besitzt folgende Parameter:
\begin{table}[H]
	\begin{tabular}{|c|p{11cm}|}
		\hline
		\textbf{Parametername} & \textbf{Parameterbeschreibung} \\ \hline
		\$username & Nutzername \\ \hline
	\end{tabular}
\end{table}
\paragraph{Beschreibung} Die Funktion fragt alle Kommentare eines Nutzers ab. Die Funktion nutzt folgende Quellen:
\begin{itemize}
	\item Kommentar-Tabelle
\end{itemize}
Es findet bei dieser Funktion kein Abruf von Daten aus {\glqq COSP\grqq} statt. Die Antwort wird als strukturiertes Array an den Aufrufer zurückgegeben.
\subsubsection{updateDeletionStateCommentByPoiid}
\paragraph{Parameter} Die Funktion besitzt folgende Parameter:
\begin{table}[H]
	\begin{tabular}{|c|p{11cm}|}
		\hline
		\textbf{Parametername} & \textbf{Parameterbeschreibung} \\ \hline
		\$poiid & Identifikator eines Interessenpunktes \\ \hline
		\$state & Status der Löschung \\ \hline
	\end{tabular}
\end{table}
\paragraph{Beschreibung} Die Funktion markiert alle Kommentare eines Interessenpunktes als gelöscht. Die Funktion hat Auswirkungen auf folgende Quellen:
\begin{itemize}
	\item Kommentar-Tabelle
\end{itemize}
Es findet bei dieser Funktion kein Abruf von Daten aus {\glqq COSP\grqq} statt. Es gibt einen Rückgabewert.
\subsubsection{updateDeletionStateCommentById}
\paragraph{Parameter} Die Funktion besitzt folgende Parameter:
\begin{table}[H]
	\begin{tabular}{|c|p{11cm}|}
		\hline
		\textbf{Parametername} & \textbf{Parameterbeschreibung} \\ \hline
		\$id    & Identifikator eines Kommentars \\ \hline
		\$state & Status der Löschung \\ \hline
	\end{tabular}
\end{table}
\paragraph{Beschreibung} Die Funktion markiert einen Kommentare als gelöscht. Die Funktion hat Auswirkungen auf folgende Quellen:
\begin{itemize}
	\item Kommentar-Tabelle
\end{itemize}
Es findet bei dieser Funktion kein Abruf von Daten aus {\glqq COSP\grqq} statt. Es gibt einen Rückgabewert.
\newpage
\section{poi-db}
\subsection{Allgemeines} Diese Datei enthält alle Funktionen, welche die Interessenpunkt-Tabelle benutzen.
\begin{table}[H]
	\begin{tabular}{|c|p{11cm}|}
		\hline
		\textbf{Einbindungspunkt} & inc-db.php \\ \hline
		\textbf{Einbindungspunkt} & inc-db-sub.php \\ \hline
	\end{tabular}
\end{table}
Die Datei ist nicht direkt durch den Nutzer aufrufbar, dies wird durch folgenden Code-Ausschnitt sichergestellt:
\begin{lstlisting}[language=php]
	if (!defined('NICE_PROJECT')) {
		die('Permission denied.');
	}
\end{lstlisting}
Der Globale Wert {\glqq NICE\_PROJECT\grqq} wird durch für den Nutzer valide Aufrufpunkte festgelegt, z.B. {\glqq api.php\grqq}.
\newpage
\subsection{Funktionen}
\subsubsection{insertPoi}
\paragraph{Parameter} Die Funktion besitzt folgende Parameter:
\begin{table}[H]
	\begin{tabular}{|c|p{11cm}|}
		\hline
		\textbf{Parametername} & \textbf{Parameterbeschreibung} \\ \hline
		\$data & Array mit Eingabewerten \\ \hline
	\end{tabular}
\end{table}
\subparagraph{\$data}Das Array enthält folgende Elemente:
\begin{table}[H]
	\begin{tabular}{|c|p{11cm}|}
		\hline
		\textbf{Parametername} & \textbf{Parameterbeschreibung} \\ \hline
		name        & Name des Interessenpunktes \\ \hline
		lng         & Längengrad \\ \hline
		lat         & Breitengrad \\ \hline
		city        & Ortsname der aktuellen Adresse \\ \hline
		postalcode  & Postleitzahl der aktuellen Adresse \\ \hline
		streetname  & Straßenname der aktuellen Adresse \\ \hline
		housenumber & Hausnummer der aktuellen Adresse \\ \hline
		picture     & alphanumerischer Identifikator des Hauptbildes \\ \hline
		start       & Startjahr der Nutzung \\ \hline
		end         & Endjahr der Nutzung \\ \hline
		category    & Kategorie des Interessenpunktes \\ \hline
		history     & Geschichte des Interessenpunktes \\ \hline
		ctype       & Typ des Interessenpunktes \\ \hline
		duty        & Legt fest, ob das Kino aktuell noch in Betrieb ist \\ \hline
	\end{tabular}
\end{table}
\paragraph{Beschreibung} Die Funktion fügt einen neuen Interessenpunkt hinzu. Die Funktion hat Auswirkungen auf folgende Quellen:
\begin{itemize}
	\item Interessenpunkt-Tabelle
\end{itemize}
Es findet bei dieser Funktion kein Abruf von Daten aus {\glqq COSP\grqq} statt. Es gibt einen Rückgabewert.
\subsubsection{updatePoi}
\paragraph{Parameter} Die Funktion besitzt folgende Parameter:
\begin{table}[H]
	\begin{tabular}{|c|p{11cm}|}
		\hline
		\textbf{Parametername} & \textbf{Parameterbeschreibung} \\ \hline
		\$data & Array mit Eingabewerten \\ \hline
	\end{tabular}
\end{table}
\subparagraph{\$data}Das Array enthält folgende Elemente:
\begin{table}[H]
	\begin{tabular}{|c|p{11cm}|}
		\hline
		\textbf{Parametername} & \textbf{Parameterbeschreibung} \\ \hline
		name        & Name des Interessenpunktes \\ \hline
		lng         & Längengrad \\ \hline
		lat         & Breitengrad \\ \hline
		city        & Ortsname der aktuellen Adresse \\ \hline
		postalcode  & Postleitzahl der aktuellen Adresse \\ \hline
		streetname  & Straßenname der aktuellen Adresse \\ \hline
		housenumber & Hausnummer der aktuellen Adresse \\ \hline
		start       & Startjahr der Nutzung \\ \hline
		end         & Endjahr der Nutzung \\ \hline
		history     & Geschichte des Interessenpunktes \\ \hline
		id          & Identifikator des Interessenpunktes \\ \hline
		ctype       & Typ des Interessenpunktes \\ \hline
	\end{tabular}
\end{table}
\paragraph{Beschreibung} Die Funktion aktualisiert einen Interessenpunkt. Die Funktion hat Auswirkungen auf folgende Quellen:
\begin{itemize}
	\item Interessenpunkt-Tabelle
\end{itemize}
Es findet bei dieser Funktion kein Abruf von Daten aus {\glqq COSP\grqq} statt. Es gibt einen Rückgabewert.
\subsubsection{updatePoiCreatorTimespan}
\paragraph{Parameter} Die Funktion besitzt folgende Parameter:
\begin{table}[H]
	\begin{tabular}{|c|p{11cm}|}
		\hline
		\textbf{Parametername} & \textbf{Parameterbeschreibung} \\ \hline
		\$id & Identifikator des Interessenpunktes \\ \hline
	\end{tabular}
\end{table}
\paragraph{Beschreibung} Die Funktion aktualisiert den Erstellungszeitpunkt und den Ersteller der Zeitspanne des Interessenpunktes. Die Funktion hat Auswirkungen auf folgende Quellen:
\begin{itemize}
	\item Interessenpunkt-Tabelle
\end{itemize}
Es findet bei dieser Funktion kein Abruf von Daten aus {\glqq COSP\grqq} statt. Es gibt keinen Rückgabewert.
\subsubsection{updatePoiCreatorCurrentAddress}
\paragraph{Parameter} Die Funktion besitzt folgende Parameter:
\begin{table}[H]
	\begin{tabular}{|c|p{11cm}|}
		\hline
		\textbf{Parametername} & \textbf{Parameterbeschreibung} \\ \hline
		\$id & Identifikator des Interessenpunktes \\ \hline
	\end{tabular}
\end{table}
\paragraph{Beschreibung} Die Funktion aktualisiert den Erstellungszeitpunkt und den Ersteller der aktuellen Adresse des Interessenpunktes. Die Funktion hat Auswirkungen auf folgende Quellen:
\begin{itemize}
	\item Interessenpunkt-Tabelle
\end{itemize}
Es findet bei dieser Funktion kein Abruf von Daten aus {\glqq COSP\grqq} statt. Es gibt keinen Rückgabewert.
\subsubsection{updatePoiCreatorHistory}
\paragraph{Parameter} Die Funktion besitzt folgende Parameter:
\begin{table}[H]
	\begin{tabular}{|c|p{11cm}|}
		\hline
		\textbf{Parametername} & \textbf{Parameterbeschreibung} \\ \hline
		\$id & Identifikator des Interessenpunktes \\ \hline
	\end{tabular}
\end{table}
\paragraph{Beschreibung} Die Funktion aktualisiert den Erstellungszeitpunkt und den Ersteller der Geschichte des Interessenpunktes. Die Funktion hat Auswirkungen auf folgende Quellen:
\begin{itemize}
	\item Interessenpunkt-Tabelle
\end{itemize}
Es findet bei dieser Funktion kein Abruf von Daten aus {\glqq COSP\grqq} statt. Es gibt keinen Rückgabewert.
\subsubsection{updatePoiCreatorType}
\paragraph{Parameter} Die Funktion besitzt folgende Parameter:
\begin{table}[H]
	\begin{tabular}{|c|p{11cm}|}
		\hline
		\textbf{Parametername} & \textbf{Parameterbeschreibung} \\ \hline
		\$id & Identifikator des Interessenpunktes \\ \hline
	\end{tabular}
\end{table}
\paragraph{Beschreibung} Die Funktion aktualisiert den Erstellungszeitpunkt und den Ersteller der Typs des Interessenpunktes. Die Funktion hat Auswirkungen auf folgende Quellen:
\begin{itemize}
	\item Interessenpunkt-Tabelle
\end{itemize}
Es findet bei dieser Funktion kein Abruf von Daten aus {\glqq COSP\grqq} statt. Es gibt keinen Rückgabewert.
\subsubsection{updatePoiCreator}
\paragraph{Parameter} Die Funktion besitzt folgende Parameter:
\begin{table}[H]
	\begin{tabular}{|c|p{11cm}|}
		\hline
		\textbf{Parametername} & \textbf{Parameterbeschreibung} \\ \hline
		\$id & Identifikator des Interessenpunktes \\ \hline
	\end{tabular}
\end{table}
\paragraph{Beschreibung} Die Funktion aktualisiert den Erstellungszeitpunkt und den Ersteller des Interessenpunktes. Die Funktion hat Auswirkungen auf folgende Quellen:
\begin{itemize}
	\item Interessenpunkt-Tabelle
\end{itemize}
Es findet bei dieser Funktion kein Abruf von Daten aus {\glqq COSP\grqq} statt. Es gibt keinen Rückgabewert.
\subsubsection{getAllPois}
\paragraph{Parameter} Die Funktion besitzt folgende Parameter:
\begin{table}[H]
	\begin{tabular}{|c|p{11cm}|}
		\hline
		\textbf{Parametername} & \textbf{Parameterbeschreibung} \\ \hline
		\$api & Gibt an, ob Aufrufer eine API ist \\ \hline
	\end{tabular}
\end{table}
\paragraph{Beschreibung} Die Funktion fragt alle Interessenpunkt Daten ab, welche für den Nutzer sichtbar sind. Die Funktion nutzt folgende Quellen:
\begin{itemize}
	\item Interessenpunkt-Tabelle
\end{itemize}
Es findet bei dieser Funktion kein Abruf von Daten aus {\glqq COSP\grqq} statt. Die Antwort wird als strukturiertes Array an den Aufrufer zurückgegeben.
\subsubsection{getAllPoisTitle}
\paragraph{Parameter} Die Funktion besitzt keine Parameter,
\paragraph{Beschreibung} Die Funktion fragt die Titel aller Interessenpunkte ab, welche für den Nutzer sichtbar sind. Die Funktion nutzt folgende Quellen:
\begin{itemize}
	\item Interessenpunkt-Tabelle
\end{itemize}
Es findet bei dieser Funktion kein Abruf von Daten aus {\glqq COSP\grqq} statt. Die Antwort wird als strukturiertes Array an den Aufrufer zurückgegeben.
\subsubsection{getPoi}
\paragraph{Parameter} Die Funktion besitzt folgende Parameter:
\begin{table}[H]
	\begin{tabular}{|c|p{11cm}|}
		\hline
		\textbf{Parametername} & \textbf{Parameterbeschreibung} \\ \hline
		\$poiid & Identifikator eines Interessenpunktes \\ \hline
	\end{tabular}
\end{table}
\paragraph{Beschreibung} Die Funktion fragt alle Informationen zu einem bestimmten Interessenpunkt ab. Die Funktion nutzt folgende Quellen:
\begin{itemize}
	\item Interessenpunkt-Tabelle
\end{itemize}
Es findet bei dieser Funktion kein Abruf von Daten aus {\glqq COSP\grqq} statt. Die Antwort wird als strukturiertes Array an den Aufrufer zurückgegeben.
\subsubsection{deletePoi}
\paragraph{Parameter} Die Funktion besitzt folgende Parameter:
\begin{table}[H]
	\begin{tabular}{|c|p{11cm}|}
		\hline
		\textbf{Parametername} & \textbf{Parameterbeschreibung} \\ \hline
		\$poiid & Identifikator eines Interessenpunktes \\ \hline
	\end{tabular}
\end{table}
\paragraph{Beschreibung} Die Funktion löscht einen Interessenpunkt. Die Funktion hat Auswirkungen auf folgende Quellen:
\begin{itemize}
	\item Interessenpunkt-Tabelle
\end{itemize}
Es findet bei dieser Funktion kein Abruf von Daten aus {\glqq COSP\grqq} statt. Die Antwort wird als strukturiertes Array an den Aufrufer zurückgegeben.
\subsubsection{selectMore}
\paragraph{Parameter} Die Funktion besitzt folgende Parameter:
\begin{table}[H]
	\begin{tabular}{|c|p{11cm}|}
		\hline
		\textbf{Parametername} & \textbf{Parameterbeschreibung} \\ \hline
		\$poi\_id & Identifikator eines Interessenpunktes \\ \hline
	\end{tabular}
\end{table}
\paragraph{Beschreibung} Die Funktion fragt alle Informationen zu einem bestimmten Interessenpunkt für das {\glqq Mehr Anzeigen\grqq}-Modal ab. Die Funktion nutzt folgende Quellen:
\begin{itemize}
	\item Interessenpunkt-Tabelle
\end{itemize}
Es findet bei dieser Funktion kein Abruf von Daten aus {\glqq COSP\grqq} statt. Die Antwort wird als strukturiertes Array an den Aufrufer zurückgegeben.
\subsubsection{getCurrentAddresses}
\paragraph{Parameter} Die Funktion besitzt keine Parameter.
\paragraph{Beschreibung} Die Funktion fragt alle aktuellen Adressen ab. Die Funktion nutzt folgende Quellen:
\begin{itemize}
	\item Interessenpunkt-Tabelle
\end{itemize}
Es findet bei dieser Funktion kein Abruf von Daten aus {\glqq COSP\grqq} statt. Die Antwort wird als strukturiertes Array an den Aufrufer zurückgegeben.
\subsubsection{PersonalAreaCollection}
\paragraph{Parameter} Die Funktion besitzt keine Parameter.
\paragraph{Beschreibung} Die Funktion fragt das Minimale und das Maximale Jahr für die Anzeige des Sliders ab. Die Funktion nutzt folgende Quellen:
\begin{itemize}
	\item Interessenpunkt-Tabelle
\end{itemize}
Es findet bei dieser Funktion kein Abruf von Daten aus {\glqq COSP\grqq} statt. Die Antwort wird als strukturiertes Array an den Aufrufer zurückgegeben.
\subsubsection{getAllPoisOfUser}
\paragraph{Parameter} Die Funktion besitzt folgende Parameter:
\begin{table}[H]
	\begin{tabular}{|c|p{11cm}|}
		\hline
		\textbf{Parametername} & \textbf{Parameterbeschreibung} \\ \hline
		\$username & Nutzername des Nutzers für den Funktion ausgeführt werden soll. \\ \hline
	\end{tabular}
\end{table}
\paragraph{Beschreibung} Die Funktion fragt alle Interessenpunkte ab, welche einem Nutzer zugeordnet werden können. Die Funktion nutzt folgende Quellen:
\begin{itemize}
	\item Interessenpunkt-Tabelle
\end{itemize}
Es findet bei dieser Funktion kein Abruf von Daten aus {\glqq COSP\grqq} statt. Die Antwort wird als strukturiertes Array an den Aufrufer zurückgegeben.
\subsubsection{updatePicForPoi}
\paragraph{Parameter} Die Funktion besitzt folgende Parameter:
\begin{table}[H]
	\begin{tabular}{|c|p{11cm}|}
		\hline
		\textbf{Parametername} & \textbf{Parameterbeschreibung} \\ \hline
		\$newtoken & neuer alphanumerischer Identifikator eines Bildes \\ \hline
		\$poiid    & Identifikator eines Interessenpunktes \\ \hline
	\end{tabular}
\end{table}
\paragraph{Beschreibung} Die Funktion aktualisiert das Hauptbild eines Interessenpunktes. Die Funktion hat Auswirkungen auf folgende Quellen:
\begin{itemize}
	\item Interessenpunkt-Tabelle
\end{itemize}
Es findet bei dieser Funktion kein Abruf von Daten aus {\glqq COSP\grqq} statt. Es gibt keinen Rückgabewert.
\subsubsection{getAllPoisWithCertainPicture}
\paragraph{Parameter} Die Funktion besitzt folgende Parameter:
\begin{table}[H]
	\begin{tabular}{|c|p{11cm}|}
		\hline
		\textbf{Parametername} & \textbf{Parameterbeschreibung} \\ \hline
		\$PicToken & alphanumerischer Identifikator eines Bildes \\ \hline
	\end{tabular}
\end{table}
\paragraph{Beschreibung} Die Funktion fragt alle Interessenpunkte mit einem bestimmten Hauptbild ab. Die Funktion nutzt folgende Quellen:
\begin{itemize}
	\item Interessenpunkt-Tabelle
\end{itemize}
Es findet bei dieser Funktion kein Abruf von Daten aus {\glqq COSP\grqq} statt. Die Antwort wird als strukturiertes Array an den Aufrufer zurückgegeben.
\subsubsection{updateDeletionStatePoiByPoiid}
\paragraph{Parameter} Die Funktion besitzt folgende Parameter:
\begin{table}[H]
	\begin{tabular}{|c|p{11cm}|}
		\hline
		\textbf{Parametername} & \textbf{Parameterbeschreibung} \\ \hline
		\$poiid & Identifikator des Interessenpunktes \\ \hline
		\$state & Status der Löschung \\ \hline
	\end{tabular}
\end{table}
\paragraph{Beschreibung} Die Funktion markiert einen Interessenpunkt als gelöscht. Die Funktion nutzt folgende Quellen:
\begin{itemize}
	\item Interessenpunkt-Tabelle
\end{itemize}
Es findet bei dieser Funktion kein Abruf von Daten aus {\glqq COSP\grqq} statt. Es gibt einen Rückgabewert.
\subsubsection{updateDeletionPicStatePoiByPoiid}
\paragraph{Parameter} Die Funktion besitzt folgende Parameter:
\begin{table}[H]
	\begin{tabular}{|c|p{11cm}|}
		\hline
		\textbf{Parametername} & \textbf{Parameterbeschreibung} \\ \hline
		\$poiid & Identifikator des Interessenpunktes \\ \hline
		\$state & Status der Löschung des Hauptbildes \\ \hline
	\end{tabular}
\end{table}
\paragraph{Beschreibung} Die Funktion markiert das Hauptbild eines Interessenpunktes als gelöscht. Die Funktion nutzt folgende Quellen:
\begin{itemize}
	\item Interessenpunkt-Tabelle
\end{itemize}
Es findet bei dieser Funktion kein Abruf von Daten aus {\glqq COSP\grqq} statt. Es gibt einen Rückgabewert.
\newpage
\section{poi-picture-db}
\subsection{Allgemeines} Diese Datei enthält alle Funktionen, welche die Tabelle mit Links zwischen Bildern und Interessenpunkten benutzen.
\begin{table}[H]
	\begin{tabular}{|c|p{11cm}|}
		\hline
		\textbf{Einbindungspunkt} & inc-db.php \\ \hline
		\textbf{Einbindungspunkt} & inc-db-sub.php \\ \hline
	\end{tabular}
\end{table}
Die Datei ist nicht direkt durch den Nutzer aufrufbar, dies wird durch folgenden Code-Ausschnitt sichergestellt:
\begin{lstlisting}[language=php]
	if (!defined('NICE_PROJECT')) {
		die('Permission denied.');
	}
\end{lstlisting}
Der Globale Wert {\glqq NICE\_PROJECT\grqq} wird durch für den Nutzer valide Aufrufpunkte festgelegt, z.B. {\glqq api.php\grqq}.
\newpage
\subsection{Funktionen}
\subsubsection{insertPoiPicture}
\paragraph{Parameter} Die Funktion besitzt folgende Parameter:
\begin{table}[H]
	\begin{tabular}{|c|p{11cm}|}
		\hline
		\textbf{Parametername} & \textbf{Parameterbeschreibung} \\ \hline
		\$pictureId & alphanumerischer Identifikator eines Bildes \\ \hline
		\$PoiId     & Identifikator eines Interessenpunktes \\ \hline
	\end{tabular}
\end{table}
\paragraph{Beschreibung} Die Funktion fügt einen neuen Link zwischen einem Bild und einem Interessenpunkt hinzu. Die Funktion hat Auswirkungen auf folgende Quellen:
\begin{itemize}
	\item Tabelle mit Links zwischen Bildern und Interessenpunkten
\end{itemize}
Es findet bei dieser Funktion kein Abruf von Daten aus {\glqq COSP\grqq} statt. Es gibt einen Rückgabewert.
\subsubsection{getPicturesForPoi}
\paragraph{Parameter} Die Funktion besitzt folgende Parameter:
\begin{table}[H]
	\begin{tabular}{|c|p{11cm}|}
		\hline
		\textbf{Parametername} & \textbf{Parameterbeschreibung} \\ \hline
		\$poiid & Identifikator eines Interessenpunktes \\ \hline
	\end{tabular}
\end{table}
\paragraph{Beschreibung} Die Funktion fragt alle Links zwischen einem bestimmten Interessenpunkt und Bildern ab. Die Funktion nutzt folgende Quellen:
\begin{itemize}
	\item Tabelle mit Links zwischen Bildern und Interessenpunkten
\end{itemize}
Es findet bei dieser Funktion kein Abruf von Daten aus {\glqq COSP\grqq} statt. Die Antwort wird als strukturiertes Array an den Aufrufer zurückgegeben.
\subsubsection{getLinkIdPoiPic}
\paragraph{Parameter} Die Funktion besitzt folgende Parameter:
\begin{table}[H]
	\begin{tabular}{|c|p{11cm}|}
		\hline
		\textbf{Parametername} & \textbf{Parameterbeschreibung} \\ \hline
		\$poiid & Identifikator eines Interessenpunktes \\ \hline
		\$picid & alphanumerischer Identifikator eines Bildes \\ \hline
	\end{tabular}
\end{table}
\paragraph{Beschreibung} Die Funktion fragt den Identifikator eines Links zwischen einem bestimmten Interessenpunkt und einem bestimmten Bilder ab. Die Funktion nutzt folgende Quellen:
\begin{itemize}
	\item Tabelle mit Links zwischen Bildern und Interessenpunkten
\end{itemize}
Es findet bei dieser Funktion kein Abruf von Daten aus {\glqq COSP\grqq} statt. Die Antwort wird als strukturiertes Array an den Aufrufer zurückgegeben.
\subsubsection{getLinkIdsForPoi}
\paragraph{Parameter} Die Funktion besitzt folgende Parameter:
\begin{table}[H]
	\begin{tabular}{|c|p{11cm}|}
		\hline
		\textbf{Parametername} & \textbf{Parameterbeschreibung} \\ \hline
		\$poiid & Identifikator eines Interessenpunktes \\ \hline
	\end{tabular}
\end{table}
\paragraph{Beschreibung} Die Funktion fragt alle IDs von Links zwischen einem bestimmten Interessenpunkt und Bildern ab. Die Funktion nutzt folgende Quellen:
\begin{itemize}
	\item Tabelle mit Links zwischen Bildern und Interessenpunkten
\end{itemize}
Es findet bei dieser Funktion kein Abruf von Daten aus {\glqq COSP\grqq} statt. Die Antwort wird als strukturiertes Array an den Aufrufer zurückgegeben.
\subsubsection{getCreatorByPoiPicId}
\paragraph{Parameter} Die Funktion besitzt folgende Parameter:
\begin{table}[H]
	\begin{tabular}{|c|p{11cm}|}
		\hline
		\textbf{Parametername} & \textbf{Parameterbeschreibung} \\ \hline
		\$pic\_poi\_id & Identifikator eines Links zwischen einem Bild und einem Interessenpunkt \\ \hline
	\end{tabular}
\end{table}
\paragraph{Beschreibung} Die Funktion fragt den Ersteller eines Links zwischen einem Interessenpunkt und einem Bild ab. Die Funktion nutzt folgende Quellen:
\begin{itemize}
	\item Tabelle mit Links zwischen Bildern und Interessenpunkten
\end{itemize}
Es findet bei dieser Funktion kein Abruf von Daten aus {\glqq COSP\grqq} statt. Der Rückgabewert ist eine Zeichenkette.
\subsubsection{deleteCertainPoiPicLink}
\paragraph{Parameter} Die Funktion besitzt folgende Parameter:
\begin{table}[H]
	\begin{tabular}{|c|p{11cm}|}
		\hline
		\textbf{Parametername} & \textbf{Parameterbeschreibung} \\ \hline
		\$lid & Identifikator eines Links zwischen einem Bild und einem Interessenpunkt \\ \hline
	\end{tabular}
\end{table}
\paragraph{Beschreibung} Die Funktion löscht einen Link zwischen einem Interessenpunkt und einem Bild. Die Funktion hat Auswirkungen auf folgende Quellen:
\begin{itemize}
	\item Tabelle mit Links zwischen Bildern und Interessenpunkten
\end{itemize}
Es findet bei dieser Funktion kein Abruf von Daten aus {\glqq COSP\grqq} statt. Es gibt einen Rückgabewert.
\subsubsection{getPoisForPic}
\paragraph{Parameter} Die Funktion besitzt folgende Parameter:
\begin{table}[H]
	\begin{tabular}{|c|p{11cm}|}
		\hline
		\textbf{Parametername} & \textbf{Parameterbeschreibung} \\ \hline
		\$token & alphanumerischer Identifikator eines Bildes \\ \hline
	\end{tabular}
\end{table}
\paragraph{Beschreibung} Die Funktion fragt alle Links zwischen Interessenpunkten und einem bestimmten Bild ab. Die Funktion nutzt folgende Quellen:
\begin{itemize}
	\item Tabelle mit Links zwischen Bildern und Interessenpunkten
\end{itemize}
Es findet bei dieser Funktion kein Abruf von Daten aus {\glqq COSP\grqq} statt. Die Antwort wird als strukturiertes Array an den Aufrufer zurückgegeben.
\subsubsection{getLinkIdsPoiPic}
\paragraph{Parameter} Die Funktion besitzt folgende Parameter:
\begin{table}[H]
	\begin{tabular}{|c|p{11cm}|}
		\hline
		\textbf{Parametername} & \textbf{Parameterbeschreibung} \\ \hline
		\$picid & alphanumerischer Identifikator eines Bildes \\ \hline
	\end{tabular}
\end{table}
\paragraph{Beschreibung} Die Funktion fragt alle IDs von Links zwischen Interessenpunkten und einem bestimmten Bild ab. Die Funktion nutzt folgende Quellen:
\begin{itemize}
	\item Tabelle mit Links zwischen Bildern und Interessenpunkten
\end{itemize}
Es findet bei dieser Funktion kein Abruf von Daten aus {\glqq COSP\grqq} statt. Die Antwort wird als strukturiertes Array an den Aufrufer zurückgegeben.
\subsubsection{updateDeletionStateLinkPoiPicByID}
\paragraph{Parameter} Die Funktion besitzt folgende Parameter:
\begin{table}[H]
	\begin{tabular}{|c|p{11cm}|}
		\hline
		\textbf{Parametername} & \textbf{Parameterbeschreibung} \\ \hline
		\$id    & Identifikator eines Links zwischen einem Bild und einem Interessenpunkt \\ \hline
		\$state & Status der Löschung des Links \\ \hline
	\end{tabular}
\end{table}
\paragraph{Beschreibung} Die Funktion markiert einen Link zwischen einem Interessenpunkt und einem Bild als gelöscht. Die Funktion hat Auswirkungen auf folgende Quellen:
\begin{itemize}
	\item Tabelle mit Links zwischen Bildern und Interessenpunkten
\end{itemize}
Es findet bei dieser Funktion kein Abruf von Daten aus {\glqq COSP\grqq} statt. Es gibt einen Rückgabewert.
\subsubsection{updateDeletionPoiStateLinkPoiPicByID}
\paragraph{Parameter} Die Funktion besitzt folgende Parameter:
\begin{table}[H]
	\begin{tabular}{|c|p{11cm}|}
		\hline
		\textbf{Parametername} & \textbf{Parameterbeschreibung} \\ \hline
		\$id    & Identifikator eines Links zwischen einem Bild und einem Interessenpunkt \\ \hline
		\$state & Status der Löschung des Links \\ \hline
	\end{tabular}
\end{table}
\paragraph{Beschreibung} Die Funktion markiert einen Link zwischen einem Interessenpunkt und einem Bild als gelöscht, aufgrund eines gelöschten Interessenpunktes. Die Funktion hat Auswirkungen auf folgende Quellen:
\begin{itemize}
	\item Tabelle mit Links zwischen Bildern und Interessenpunkten
\end{itemize}
Es findet bei dieser Funktion kein Abruf von Daten aus {\glqq COSP\grqq} statt. Es gibt einen Rückgabewert.
\subsubsection{updateDeletionPicStateLinkPoiPicByID}
\paragraph{Parameter} Die Funktion besitzt folgende Parameter:
\begin{table}[H]
	\begin{tabular}{|c|p{11cm}|}
		\hline
		\textbf{Parametername} & \textbf{Parameterbeschreibung} \\ \hline
		\$id    & Identifikator eines Links zwischen einem Bild und einem Interessenpunkt \\ \hline
		\$state & Status der Löschung des Links \\ \hline
	\end{tabular}
\end{table}
\paragraph{Beschreibung} Die Funktion markiert einen Link zwischen einem Interessenpunkt und einem Bild als gelöscht, aufgrund eines gelöschten Bildes. Die Funktion hat Auswirkungen auf folgende Quellen:
\begin{itemize}
	\item Tabelle mit Links zwischen Bildern und Interessenpunkten
\end{itemize}
Es findet bei dieser Funktion kein Abruf von Daten aus {\glqq COSP\grqq} statt. Es gibt einen Rückgabewert.
\subsubsection{LinkPoiPicRestictedPOI}
\paragraph{Parameter} Die Funktion besitzt folgende Parameter:
\begin{table}[H]
	\begin{tabular}{|c|p{11cm}|}
		\hline
		\textbf{Parametername} & \textbf{Parameterbeschreibung} \\ \hline
		\$id    & Identifikator eines Links zwischen einem Bild und einem Interessenpunkt \\ \hline
	\end{tabular}
\end{table}
\paragraph{Beschreibung} Die Funktion fragt ab, ob eine Restriktion der Wiederherstellung aufgrund eines gelöschten Interessenpunktes besteht. Die Funktion nutzt folgende Quellen:
\begin{itemize}
	\item Tabelle mit Links zwischen Bildern und Interessenpunkten
\end{itemize}
Es findet bei dieser Funktion kein Abruf von Daten aus {\glqq COSP\grqq} statt. Der Rückgabewert ist ein Boolean.
\subsubsection{LinkPoiPicRestictedPic}
\paragraph{Parameter} Die Funktion besitzt folgende Parameter:
\begin{table}[H]
	\begin{tabular}{|c|p{11cm}|}
		\hline
		\textbf{Parametername} & \textbf{Parameterbeschreibung} \\ \hline
		\$id    & Identifikator eines Links zwischen einem Bild und einem Interessenpunkt \\ \hline
	\end{tabular}
\end{table}
\paragraph{Beschreibung} Die Funktion fragt ab, ob eine Restriktion der Wiederherstellung aufgrund eines gelöschten Bildes besteht. Die Funktion nutzt folgende Quellen:
\begin{itemize}
	\item Tabelle mit Links zwischen Bildern und Interessenpunkten
\end{itemize}
Es findet bei dieser Funktion kein Abruf von Daten aus {\glqq COSP\grqq} statt. Der Rückgabewert ist ein Boolean.
\newpage
\section{poi-source-db}
\subsection{Allgemeines} Diese Datei enthält alle Funktionen, welche die Tabelle mit Quellen von Interessenpunkten benutzen.
\begin{table}[H]
	\begin{tabular}{|c|p{11cm}|}
		\hline
		\textbf{Einbindungspunkt} & inc-db.php \\ \hline
		\textbf{Einbindungspunkt} & inc-db-sub.php \\ \hline
	\end{tabular}
\end{table}
Die Datei ist nicht direkt durch den Nutzer aufrufbar, dies wird durch folgenden Code-Ausschnitt sichergestellt:
\begin{lstlisting}[language=php]
	if (!defined('NICE_PROJECT')) {
		die('Permission denied.');
	}
\end{lstlisting}
Der Globale Wert {\glqq NICE\_PROJECT\grqq} wird durch für den Nutzer valide Aufrufpunkte festgelegt, z.B. {\glqq api.php\grqq}.
\newpage
\subsection{Funktionen}
\subsubsection{insertSourceOfPOI}
\paragraph{Parameter} Die Funktion besitzt folgende Parameter:
\begin{table}[H]
	\begin{tabular}{|c|p{11cm}|}
		\hline
		\textbf{Parametername} & \textbf{Parameterbeschreibung} \\ \hline
		\$type     & Identifikator des Typs der Quelle \\ \hline
		\$source   & Quellenangabe \\ \hline
		\$relation & Identifikator des Bezugs der Quelle \\ \hline
		\$poiid    & Identifikator eines Interessenpunktes \\ \hline
	\end{tabular}
\end{table}
\paragraph{Beschreibung} Die Funktion fügt eine neue Quelle einem Interessenpunkt hinzu. Die Funktion hat Auswirkungen auf folgende Quellen:
\begin{itemize}
	\item Tabelle mit Quellenangaben
\end{itemize}
Es findet bei dieser Funktion kein Abruf von Daten aus {\glqq COSP\grqq} statt. Es wird eine Antwort zurück gegeben.
\subsubsection{getSourcePoiAPI}
\paragraph{Parameter} Die Funktion besitzt folgende Parameter:
\begin{table}[H]
	\begin{tabular}{|c|p{11cm}|}
		\hline
		\textbf{Parametername} & \textbf{Parameterbeschreibung} \\ \hline
		\$poiid    & Identifikator eines Interessenpunktes \\ \hline
	\end{tabular}
\end{table}
\paragraph{Beschreibung} Die Funktion ruft alle Quellen eines Interessenpunktes ab. Die Funktion nutzt folgende Quellen:
\begin{itemize}
	\item Tabelle mit Quellenangaben
\end{itemize}
Es findet bei dieser Funktion kein Abruf von Daten aus {\glqq COSP\grqq} statt. Es wird eine Antwort zurück gegeben.
\subsubsection{updateSource}
\paragraph{Parameter} Die Funktion besitzt folgende Parameter:
\begin{table}[H]
	\begin{tabular}{|c|p{11cm}|}
		\hline
		\textbf{Parametername} & \textbf{Parameterbeschreibung} \\ \hline
		\$id       & Identifikator einer Quelle \\ \hline
		\$relation & Identifikator des Bezugs der Quelle \\ \hline
		\$source   & Quellenangabe \\ \hline
		\$type     & Identifikator des Typs der Quelle \\ \hline
	\end{tabular}
\end{table}
\paragraph{Beschreibung} Die Funktion ändert eine Quelle eines Interessenpunktes. Die Funktion hat Auswirkungen auf folgende Quellen:
\begin{itemize}
	\item Tabelle mit Quellenangaben
\end{itemize}
Es findet bei dieser Funktion kein Abruf von Daten aus {\glqq COSP\grqq} statt. Es wird eine Antwort zurück gegeben.
\subsubsection{getSource}
\paragraph{Parameter} Die Funktion besitzt folgende Parameter:
\begin{table}[H]
	\begin{tabular}{|c|p{11cm}|}
		\hline
		\textbf{Parametername} & \textbf{Parameterbeschreibung} \\ \hline
		\$id & Identifikator einer Quelle \\ \hline
	\end{tabular}
\end{table}
\paragraph{Beschreibung} Die Funktion ruft eine Quelle ab. Die Funktion nutzt folgende Quellen:
\begin{itemize}
	\item Tabelle mit Quellenangaben
\end{itemize}
Es findet bei dieser Funktion kein Abruf von Daten aus {\glqq COSP\grqq} statt. Es wird eine Antwort zurück gegeben.
\subsubsection{updateSourceDeletionState}
\paragraph{Parameter} Die Funktion besitzt folgende Parameter:
\begin{table}[H]
	\begin{tabular}{|c|p{11cm}|}
		\hline
		\textbf{Parametername} & \textbf{Parameterbeschreibung} \\ \hline
		\$id    & Identifikator einer Quelle \\ \hline
		\$state & Identifikator des Bezugs der Quelle \\ \hline
	\end{tabular}
\end{table}
\paragraph{Beschreibung} Die Funktion aktualisiert den Löschungszustand einer Quelle eines Interessenpunktes. Die Funktion hat Auswirkungen auf folgende Quellen:
\begin{itemize}
	\item Tabelle mit Quellenangaben
\end{itemize}
Es findet bei dieser Funktion kein Abruf von Daten aus {\glqq COSP\grqq} statt. Es wird eine Antwort zurück gegeben.
\subsubsection{deleteSource}
\paragraph{Parameter} Die Funktion besitzt folgende Parameter:
\begin{table}[H]
	\begin{tabular}{|c|p{11cm}|}
		\hline
		\textbf{Parametername} & \textbf{Parameterbeschreibung} \\ \hline
		\$id & Identifikator einer Quelle \\ \hline
	\end{tabular}
\end{table}
\paragraph{Beschreibung} Die Funktion löscht eine Quelle eines Interessenpunktes. Die Funktion hat Auswirkungen auf folgende Quellen:
\begin{itemize}
	\item Tabelle mit Quellenangaben
\end{itemize}
Es findet bei dieser Funktion kein Abruf von Daten aus {\glqq COSP\grqq} statt. Es wird eine Antwort zurück gegeben.
\newpage
\section{poi-story-db}
\subsection{Allgemeines} Diese Datei enthält alle Funktionen, welche die Tabelle mit Links zwischen Geschichten und Interessenpunkten benutzen.
\begin{table}[H]
	\begin{tabular}{|c|p{11cm}|}
		\hline
		\textbf{Einbindungspunkt} & inc-db.php \\ \hline
		\textbf{Einbindungspunkt} & inc-db-sub.php \\ \hline
	\end{tabular}
\end{table}
Die Datei ist nicht direkt durch den Nutzer aufrufbar, dies wird durch folgenden Code-Ausschnitt sichergestellt:
\begin{lstlisting}[language=php]
	if (!defined('NICE_PROJECT')) {
		die('Permission denied.');
	}
\end{lstlisting}
Der Globale Wert {\glqq NICE\_PROJECT\grqq} wird durch für den Nutzer valide Aufrufpunkte festgelegt, z.B. {\glqq api.php\grqq}.
\newpage
\subsection{Funktionen}
\subsubsection{insertPoiStory}
\paragraph{Parameter} Die Funktion besitzt folgende Parameter:
\begin{table}[H]
	\begin{tabular}{|c|p{11cm}|}
		\hline
		\textbf{Parametername} & \textbf{Parameterbeschreibung} \\ \hline
		\$poi\_id    & Identifikator eines Interessenpunktes \\ \hline
		\$storytoken & alphanumerischer Identifikator einer Geschichte \\ \hline
	\end{tabular}
\end{table}
\paragraph{Beschreibung} Die Funktion fügt einen neuen Link zwischen einem Interessenpunkt und einer Geschichte ein. Die Funktion hat Auswirkung auf folgende Quellen:
\begin{itemize}
	\item Tabelle mit Links zwischen Geschichten und Interessenpunkten
\end{itemize}
Es findet bei dieser Funktion kein Abruf von Daten aus {\glqq COSP\grqq} statt. Es gibt einen Rückgabewert.
\subsubsection{getPoiForStory}
\paragraph{Parameter} Die Funktion besitzt folgende Parameter:
\begin{table}[H]
	\begin{tabular}{|c|p{11cm}|}
		\hline
		\textbf{Parametername} & \textbf{Parameterbeschreibung} \\ \hline
		\$token    & alphanumerische Identifikator einer Geschichte \\ \hline
		\$override & Ignoriert Löschmarkierungen \\ \hline
		\$api      & Legt fest ob Aufrufer API ist \\ \hline
	\end{tabular}
\end{table}
\paragraph{Beschreibung} Die Funktion fragt alle Links zwischen einer Geschichte und Interessenpunkten ab. Die Funktion nutzt folgende Quellen:
\begin{itemize}
	\item Tabelle mit Links zwischen Geschichten und Interessenpunkten
\end{itemize}
Es findet bei dieser Funktion kein Abruf von Daten aus {\glqq COSP\grqq} statt. Die Antwort wird als strukturiertes Array an den Aufrufer zurückgegeben.
\subsubsection{getPoiForStoryByPoiId}
\paragraph{Parameter} Die Funktion besitzt folgende Parameter:
\begin{table}[H]
	\begin{tabular}{|c|p{11cm}|}
		\hline
		\textbf{Parametername} & \textbf{Parameterbeschreibung} \\ \hline
		\$poi\_id  & Identifikator eines Interessenpunktes \\ \hline
		\$override & Ignoriert Löschmarkierungen \\ \hline
	\end{tabular}
\end{table}
\paragraph{Beschreibung} Die Funktion fragt alle Links zwischen Geschichten und einem Interessenpunkt ab. Die Funktion nutzt folgende Quellen:
\begin{itemize}
	\item Tabelle mit Links zwischen Geschichten und Interessenpunkten
\end{itemize}
Es findet bei dieser Funktion kein Abruf von Daten aus {\glqq COSP\grqq} statt. Die Antwort wird als strukturiertes Array an den Aufrufer zurückgegeben.
\subsubsection{getPoiForStoryByPoiId}
\paragraph{Parameter} Die Funktion besitzt folgende Parameter:
\begin{table}[H]
	\begin{tabular}{|c|p{11cm}|}
		\hline
		\textbf{Parametername} & \textbf{Parameterbeschreibung} \\ \hline
		\$story\_poi\_id  & Identifikator eines Links zwischen einem Interessenpunkt und einer Geschichte \\ \hline
	\end{tabular}
\end{table}
\paragraph{Beschreibung} Die Funktion den Ersteller eines Links zwischen einer Geschichte und einem Interessenpunkt ab. Die Funktion nutzt folgende Quellen:
\begin{itemize}
	\item Tabelle mit Links zwischen Geschichten und Interessenpunkten
\end{itemize}
Es findet bei dieser Funktion kein Abruf von Daten aus {\glqq COSP\grqq} statt. Der Rückgabewert ist eine Zeichenkette.
\subsubsection{deletePoiStory}
\paragraph{Parameter} Die Funktion besitzt folgende Parameter:
\begin{table}[H]
	\begin{tabular}{|c|p{11cm}|}
		\hline
		\textbf{Parametername} & \textbf{Parameterbeschreibung} \\ \hline
		\$story\_poi\_id  & Identifikator eines Links zwischen einem Interessenpunkt und einer Geschichte \\ \hline
	\end{tabular}
\end{table}
\paragraph{Beschreibung} Die Funktion löscht einen Link zwischen einer Geschichte und einem Interessenpunkt ab. Die Funktion hat Auswirkungen auf folgende Quellen:
\begin{itemize}
	\item Tabelle mit Links zwischen Geschichten und Interessenpunkten
\end{itemize}
Es findet bei dieser Funktion kein Abruf von Daten aus {\glqq COSP\grqq} statt. Es gibt einen Rückgabewert.
\subsubsection{updateDeletionStateLinkPoiStoryByID}
\paragraph{Parameter} Die Funktion besitzt folgende Parameter:
\begin{table}[H]
	\begin{tabular}{|c|p{11cm}|}
		\hline
		\textbf{Parametername} & \textbf{Parameterbeschreibung} \\ \hline
		\$id    & Identifikator eines Links zwischen einem Interessenpunkt und einer Geschichte \\ \hline
		\$state & Status der Löschung \\ \hline
	\end{tabular}
\end{table}
\paragraph{Beschreibung} Die Funktion markiert einen Link zwischen einer Geschichte und einem Interessenpunkt als gelöscht. Die Funktion hat Auswirkungen auf folgende Quellen:
\begin{itemize}
	\item Tabelle mit Links zwischen Geschichten und Interessenpunkten
\end{itemize}
Es findet bei dieser Funktion kein Abruf von Daten aus {\glqq COSP\grqq} statt. Es gibt einen Rückgabewert.
\subsubsection{updateDeletionPoiStateLinkPoiStoryByID}
\paragraph{Parameter} Die Funktion besitzt folgende Parameter:
\begin{table}[H]
	\begin{tabular}{|c|p{11cm}|}
		\hline
		\textbf{Parametername} & \textbf{Parameterbeschreibung} \\ \hline
		\$id    & Identifikator eines Links zwischen einem Interessenpunkt und einer Geschichte \\ \hline
		\$state & Status der Löschung \\ \hline
	\end{tabular}
\end{table}
\paragraph{Beschreibung} Die Funktion markiert einen Link zwischen einer Geschichte und einem Interessenpunkt als gelöscht, aufgrund eines gelöschten Interessenpunktes. Die Funktion hat Auswirkungen auf folgende Quellen:
\begin{itemize}
	\item Tabelle mit Links zwischen Geschichten und Interessenpunkten
\end{itemize}
Es findet bei dieser Funktion kein Abruf von Daten aus {\glqq COSP\grqq} statt. Es gibt einen Rückgabewert.
\subsubsection{updateDeletionStoryStateLinkPoiStoryByID}
\paragraph{Parameter} Die Funktion besitzt folgende Parameter:
\begin{table}[H]
	\begin{tabular}{|c|p{11cm}|}
		\hline
		\textbf{Parametername} & \textbf{Parameterbeschreibung} \\ \hline
		\$id    & Identifikator eines Links zwischen einem Interessenpunkt und einer Geschichte \\ \hline
		\$state & Status der Löschung \\ \hline
	\end{tabular}
\end{table}
\paragraph{Beschreibung} Die Funktion markiert einen Link zwischen einer Geschichte und einem Interessenpunkt als gelöscht, aufgrund einer gelöschten Geschichte. Die Funktion hat Auswirkungen auf folgende Quellen:
\begin{itemize}
	\item Tabelle mit Links zwischen Geschichten und Interessenpunkten
\end{itemize}
Es findet bei dieser Funktion kein Abruf von Daten aus {\glqq COSP\grqq} statt. Es gibt einen Rückgabewert.
\subsubsection{LinkPoiStoryRestictedPOI}
\paragraph{Parameter} Die Funktion besitzt folgende Parameter:
\begin{table}[H]
	\begin{tabular}{|c|p{11cm}|}
		\hline
		\textbf{Parametername} & \textbf{Parameterbeschreibung} \\ \hline
		\$id    & Identifikator eines Links zwischen einer Geschichte und einem Interessenpunkt \\ \hline
	\end{tabular}
\end{table}
\paragraph{Beschreibung} Die Funktion fragt ab, ob eine Restriktion der Wiederherstellung aufgrund eines gelöschten Interessenpunktes besteht. Die Funktion nutzt folgende Quellen:
\begin{itemize}
	\item Tabelle mit Links zwischen Geschichten und Interessenpunkten
\end{itemize}
Es findet bei dieser Funktion kein Abruf von Daten aus {\glqq COSP\grqq} statt. Der Rückgabewert ist ein Boolean.
\subsubsection{LinkPoiStoryRestictedStory}
\paragraph{Parameter} Die Funktion besitzt folgende Parameter:
\begin{table}[H]
	\begin{tabular}{|c|p{11cm}|}
		\hline
		\textbf{Parametername} & \textbf{Parameterbeschreibung} \\ \hline
		\$id    & Identifikator eines Links zwischen einer Geschichte und einem Interessenpunkt \\ \hline
	\end{tabular}
\end{table}
\paragraph{Beschreibung} Die Funktion fragt ab, ob eine Restriktion der Wiederherstellung aufgrund eines gelöschten Bildes besteht. Die Funktion nutzt folgende Quellen:
\begin{itemize}
	\item Tabelle mit Links zwischen Geschichten und Interessenpunkten
\end{itemize}
Es findet bei dieser Funktion kein Abruf von Daten aus {\glqq COSP\grqq} statt. Der Rückgabewert ist ein Boolean.
\newpage
\section{poi-val}
\subsection{Allgemeines} Diese Datei enthält alle Funktionen, welche die Tabelle mit Validierungsinformationen zu Interessenpunkten benutzen.
\begin{table}[H]
	\begin{tabular}{|c|p{11cm}|}
		\hline
		\textbf{Einbindungspunkt} & inc-db.php \\ \hline
		\textbf{Einbindungspunkt} & inc-db-sub.php \\ \hline
	\end{tabular}
\end{table}
Die Datei ist nicht direkt durch den Nutzer aufrufbar, dies wird durch folgenden Code-Ausschnitt sichergestellt:
\begin{lstlisting}[language=php]
	if (!defined('NICE_PROJECT')) {
		die('Permission denied.');
	}
\end{lstlisting}
Der Globale Wert {\glqq NICE\_PROJECT\grqq} wird durch für den Nutzer valide Aufrufpunkte festgelegt, z.B. {\glqq api.php\grqq}.
\newpage
\subsection{Funktionen}
\subsubsection{getAllValidatedForPOI}
\paragraph{Parameter} Die Funktion besitzt keine Parameter.
\paragraph{Beschreibung} Die Funktion fragt alle Validierungsinformationen ab. Die Funktion nutzt folgende Quellen:
\begin{itemize}
	\item Tabelle mit Validierungsinformationen zu Interessenpunkten
\end{itemize}
Es findet bei dieser Funktion kein Abruf von Daten aus {\glqq COSP\grqq} statt. Die Antwort wird als strukturiertes Array an den Aufrufer zurückgegeben.
\subsubsection{insertValidateForPOI}
\paragraph{Parameter} Die Funktion besitzt folgende Parameter:
\begin{table}[H]
	\begin{tabular}{|c|p{11cm}|}
		\hline
		\textbf{Parametername} & \textbf{Parameterbeschreibung} \\ \hline
		\$poiid & Identifikator des Interessenpunktes \\ \hline
		\$value & Wert der Validierung \\ \hline
	\end{tabular}
\end{table}
\paragraph{Beschreibung} Die Funktion fügt eine Validierung einem Interessenpunkt hinzu. Die Funktion hat Auswirkungen auf folgende Quellen:
\begin{itemize}
	\item Tabelle mit Validierungsinformationen zu Interessenpunkten
\end{itemize}
Es findet bei dieser Funktion kein Abruf von Daten aus {\glqq COSP\grqq} statt. Es gibt einen Rückgabewert.
\subsubsection{deletevalidateByPOI}
\paragraph{Parameter} Die Funktion besitzt folgende Parameter:
\begin{table}[H]
	\begin{tabular}{|c|p{11cm}|}
		\hline
		\textbf{Parametername} & \textbf{Parameterbeschreibung} \\ \hline
		\$poiid & Identifikator des Interessenpunktes \\ \hline
	\end{tabular}
\end{table}
\paragraph{Beschreibung} Die Funktion löscht alle Validierungen zu einem Interessenpunkt. Die Funktion hat Auswirkungen auf folgende Quellen:
\begin{itemize}
	\item Tabelle mit Validierungsinformationen zu Interessenpunkten
\end{itemize}
Es findet bei dieser Funktion kein Abruf von Daten aus {\glqq COSP\grqq} statt. Es gibt einen Rückgabewert.
\subsubsection{getValidateSumForPOI}
\paragraph{Parameter} Die Funktion besitzt folgende Parameter:
\begin{table}[H]
	\begin{tabular}{|c|p{11cm}|}
		\hline
		\textbf{Parametername} & \textbf{Parameterbeschreibung} \\ \hline
		\$poiid & Identifikator des Interessenpunktes \\ \hline
	\end{tabular}
\end{table}
\paragraph{Beschreibung} Die Funktion fragt die Validierungssumme zu einem Interessenpunkt ab. Die Funktion nutzt folgende Quellen:
\begin{itemize}
	\item Tabelle mit Validierungsinformationen zu Interessenpunkten
\end{itemize}
Es findet bei dieser Funktion kein Abruf von Daten aus {\glqq COSP\grqq} statt. Die Antwort wird als strukturiertes Array an den Aufrufer zurückgegeben.
\newpage
\section{seats-db}
\subsection{Allgemeines} Diese Datei enthält alle Funktionen, welche die Sitzplatzanzahl-Tabelle benutzen.
\begin{table}[H]
	\begin{tabular}{|c|p{11cm}|}
		\hline
		\textbf{Einbindungspunkt} & inc-db.php \\ \hline
		\textbf{Einbindungspunkt} & inc-db-sub.php \\ \hline
	\end{tabular}
\end{table}
Die Datei ist nicht direkt durch den Nutzer aufrufbar, dies wird durch folgenden Code-Ausschnitt sichergestellt:
\begin{lstlisting}[language=php]
	if (!defined('NICE_PROJECT')) {
		die('Permission denied.');
	}
\end{lstlisting}
Der Globale Wert {\glqq NICE\_PROJECT\grqq} wird durch für den Nutzer valide Aufrufpunkte festgelegt, z.B. {\glqq api.php\grqq}.
\newpage
\subsection{Funktionen}
\subsubsection{insertSeatsOfPOI}
\paragraph{Parameter} Die Funktion besitzt folgende Parameter:
\begin{table}[H]
	\begin{tabular}{|c|p{11cm}|}
		\hline
		\textbf{Parametername} & \textbf{Parameterbeschreibung} \\ \hline
		\$poi\_id   & Identifikator eines Interessenpunktes \\ \hline
		\$start     & Startjahr \\ \hline
		\$end       & Endjahr \\ \hline
		\$seatcount & Sitzplatzanzahl \\ \hline
	\end{tabular}
\end{table}
\paragraph{Beschreibung} Die Funktion fügt eine neue Sitzplatzanzahl einem Interessenpunkt hinzu. Die Funktion hat Auswirkungen auf folgende Quellen:
\begin{itemize}
	\item Sitzplatzanzahl-Tabelle
\end{itemize}
Es findet bei dieser Funktion kein Abruf von Daten aus {\glqq COSP\grqq} statt. Es gibt einen Rückgabewert.
\subsubsection{getSeatsByPoiId}
\paragraph{Parameter} Die Funktion besitzt folgende Parameter:
\begin{table}[H]
	\begin{tabular}{|c|p{11cm}|}
		\hline
		\textbf{Parametername} & \textbf{Parameterbeschreibung} \\ \hline
		\$poiid & Identifikator eines Interessenpunktes \\ \hline
	\end{tabular}
\end{table}
\paragraph{Beschreibung} Die Funktion fragt alle Sitzplatzanzahlen eines Interessenpunktes ab. Die Funktion nutzt folgende Quellen:
\begin{itemize}
	\item Sitzplatzanzahl-Tabelle
\end{itemize}
Es findet bei dieser Funktion kein Abruf von Daten aus {\glqq COSP\grqq} statt. Die Antwort wird als strukturiertes Array an den Aufrufer zurückgegeben.
\subsubsection{getCreatorBySeatsID}
\paragraph{Parameter} Die Funktion besitzt folgende Parameter:
\begin{table}[H]
	\begin{tabular}{|c|p{11cm}|}
		\hline
		\textbf{Parametername} & \textbf{Parameterbeschreibung} \\ \hline
		\$seat\_id & Identifikator der Sitzplatzanzahl \\ \hline
	\end{tabular}
\end{table}
\paragraph{Beschreibung} Die Funktion fragt den Ersteller einer Sitzplatzanzahl ab. Die Funktion nutzt folgende Quellen:
\begin{itemize}
	\item Sitzplatzanzahl-Tabelle
\end{itemize}
Es findet bei dieser Funktion kein Abruf von Daten aus {\glqq COSP\grqq} statt. Der Rückgabewert ist eine Zeichenkette.
\subsubsection{deleteSeats}
\paragraph{Parameter} Die Funktion besitzt folgende Parameter:
\begin{table}[H]
	\begin{tabular}{|c|p{11cm}|}
		\hline
		\textbf{Parametername} & \textbf{Parameterbeschreibung} \\ \hline
		\$seat\_id & Identifikator der Sitzplatzanzahl \\ \hline
	\end{tabular}
\end{table}
\paragraph{Beschreibung} Die Funktion löscht eine Sitzplatzanzahl. Die Funktion hat Auswirkungen auf folgende Quellen:
\begin{itemize}
	\item Sitzplatzanzahl-Tabelle
\end{itemize}
Es findet bei dieser Funktion kein Abruf von Daten aus {\glqq COSP\grqq} statt. Es gibt einen Rückgabewert.
\subsubsection{updateSeats}
\paragraph{Parameter} Die Funktion besitzt folgende Parameter:
\begin{table}[H]
	\begin{tabular}{|c|p{11cm}|}
		\hline
		\textbf{Parametername} & \textbf{Parameterbeschreibung} \\ \hline
		\$id        & Identifikator der Sitzplatzanzahl \\ \hline
		\$seatCount & Sitzplatzanzahl \\ \hline
		\$start     & Startjahr \\ \hline
		\$end       & Endjahr \\ \hline
	\end{tabular}
\end{table}
\paragraph{Beschreibung} Die Funktion aktualisiert eine Sitzplatzanzahl. Die Funktion hat Auswirkungen auf folgende Quellen:
\begin{itemize}
	\item Sitzplatzanzahl-Tabelle
\end{itemize}
Es findet bei dieser Funktion kein Abruf von Daten aus {\glqq COSP\grqq} statt. Es gibt einen Rückgabewert.
\subsubsection{updateDeletionStateSeatsById}
\paragraph{Parameter} Die Funktion besitzt folgende Parameter:
\begin{table}[H]
	\begin{tabular}{|c|p{11cm}|}
		\hline
		\textbf{Parametername} & \textbf{Parameterbeschreibung} \\ \hline
		\$id        & Identifikator der Sitzplatzanzahl \\ \hline
		\$state & Status der Löschung \\ \hline
		\$start     & Startjahr \\ \hline
		\$end       & Endjahr \\ \hline
	\end{tabular}
\end{table}
\paragraph{Beschreibung} Die Funktion markiert eine Sitzplatzanzahl als gelöscht. Die Funktion hat Auswirkungen auf folgende Quellen:
\begin{itemize}
	\item Sitzplatzanzahl-Tabelle
\end{itemize}
Es findet bei dieser Funktion kein Abruf von Daten aus {\glqq COSP\grqq} statt. Es gibt einen Rückgabewert
\newpage
\section{source-relation-db}
\subsection{Allgemeines} Diese Datei enthält alle Funktionen, welche die Tabelle mit Bezugsangaben für Quellen benutzen.
\begin{table}[H]
	\begin{tabular}{|c|p{11cm}|}
		\hline
		\textbf{Einbindungspunkt} & inc-db.php \\ \hline
		\textbf{Einbindungspunkt} & inc-db-sub.php \\ \hline
	\end{tabular}
\end{table}
Die Datei ist nicht direkt durch den Nutzer aufrufbar, dies wird durch folgenden Code-Ausschnitt sichergestellt:
\begin{lstlisting}[language=php]
	if (!defined('NICE_PROJECT')) {
		die('Permission denied.');
	}
\end{lstlisting}
Der Globale Wert {\glqq NICE\_PROJECT\grqq} wird durch für den Nutzer valide Aufrufpunkte festgelegt, z.B. {\glqq api.php\grqq}.
\newpage
\subsection{Funktionen}
\subsubsection{getAllSourceRelations}
\paragraph{Parameter} Die Funktion besitzt keine Parameter.
\paragraph{Beschreibung} Die Funktion ruft alle Beziehungsmöglichkeiten von Quellen zu Informationen ab. Die Funktion nutzt folgende Quellen:
\begin{itemize}
	\item Tabelle mit Beziehungen von Quellen
\end{itemize}
Es findet bei dieser Funktion kein Abruf von Daten aus {\glqq COSP\grqq} statt. Es gibt einen Rückgabewert.
\newpage
\section{source-type-db}
\subsection{Allgemeines} Diese Datei enthält alle Funktionen, welche die Tabelle mit Typen von Quellen benutzen.
\begin{table}[H]
	\begin{tabular}{|c|p{11cm}|}
		\hline
		\textbf{Einbindungspunkt} & inc-db.php \\ \hline
		\textbf{Einbindungspunkt} & inc-db-sub.php \\ \hline
	\end{tabular}
\end{table}
Die Datei ist nicht direkt durch den Nutzer aufrufbar, dies wird durch folgenden Code-Ausschnitt sichergestellt:
\begin{lstlisting}[language=php]
	if (!defined('NICE_PROJECT')) {
		die('Permission denied.');
	}
\end{lstlisting}
Der Globale Wert {\glqq NICE\_PROJECT\grqq} wird durch für den Nutzer valide Aufrufpunkte festgelegt, z.B. {\glqq api.php\grqq}.
\newpage
\subsection{Funktionen}
\subsubsection{getAllSourceRelations}
\paragraph{Parameter} Die Funktion besitzt keine Parameter.
\paragraph{Beschreibung} Die Funktion ruft alle Typen von Quellen ab. Die Funktion nutzt folgende Quellen:
\begin{itemize}
	\item Tabelle mit Typen von Quellen
\end{itemize}
Es findet bei dieser Funktion kein Abruf von Daten aus {\glqq COSP\grqq} statt. Es gibt einen Rückgabewert.
\newpage
\section{statistics-basic-dbfunctions}
\subsection{Allgemeines} Diese Datei enthält alle grundlegenden Funktionen für statistische Datenbankabfragen.
\begin{table}[H]
	\begin{tabular}{|c|p{11cm}|}
		\hline
		\textbf{Einbindungspunkt} & inc-db.php \\ \hline
		\textbf{Einbindungspunkt} & inc-db-sub.php \\ \hline
	\end{tabular}
\end{table}
Die Datei ist nicht direkt durch den Nutzer aufrufbar, dies wird durch folgenden Code-Ausschnitt sichergestellt:
\begin{lstlisting}[language=php]
	if (!defined('NICE_PROJECT')) {
		die('Permission denied.');
	}
\end{lstlisting}
Der Globale Wert {\glqq NICE\_PROJECT\grqq} wird durch für den Nutzer valide Aufrufpunkte festgelegt, z.B. {\glqq api.php\grqq}.
\newpage
\subsection{Funktionen}
\subsubsection{ExecuteStatisticStatement}
\paragraph{Parameter} Die Funktion besitzt folgende Parameter:
\begin{table}[H]
	\begin{tabular}{|c|p{11cm}|}
		\hline
		\textbf{Parametername} & \textbf{Parameterbeschreibung} \\ \hline
		\$amount & Anzahl an Einheiten \\ \hline
	\end{tabular}
\end{table}
\paragraph{Beschreibung} Die Funktion fragt mithilfe der gegebenen vorbereiteten Abfrage statistische Daten ab. Es findet bei dieser Funktion kein Abruf von Daten aus {\glqq COSP\grqq} statt. Die Antwort wird als strukturiertes Array an den Aufrufer zurückgegeben.

\newpage
\section{statistics-comments-db}
\subsection{Allgemeines} Diese Datei enthält alle Funktionen für statistische Datenbankabfragen der Kommentar-Tabelle.
\begin{table}[H]
	\begin{tabular}{|c|p{11cm}|}
		\hline
		\textbf{Einbindungspunkt} & inc-db.php \\ \hline
		\textbf{Einbindungspunkt} & inc-db-sub.php \\ \hline
	\end{tabular}
\end{table}
Die Datei ist nicht direkt durch den Nutzer aufrufbar, dies wird durch folgenden Code-Ausschnitt sichergestellt:
\begin{lstlisting}[language=php]
	if (!defined('NICE_PROJECT')) {
		die('Permission denied.');
	}
\end{lstlisting}
Der Globale Wert {\glqq NICE\_PROJECT\grqq} wird durch für den Nutzer valide Aufrufpunkte festgelegt, z.B. {\glqq api.php\grqq}.
\newpage
\subsection{Funktionen}
\subsubsection{getCommentsCreateStatisticalDataLastWeeks}
\paragraph{Parameter} Die Funktion besitzt folgende Parameter:
\begin{table}[H]
	\begin{tabular}{|c|p{11cm}|}
		\hline
		\textbf{Parametername} & \textbf{Parameterbeschreibung} \\ \hline
		\$number & Anzahl der Wochen \\ \hline
	\end{tabular}
\end{table}
\paragraph{Beschreibung} Die Funktion fragt statistische Daten zu Kommentaren für den in Wochen angegeben Zeitraum ab. Die Funktion nutzt folgende Quellen:
\begin{itemize}
	\item Kommentar-Tabelle
\end{itemize}
Es findet bei dieser Funktion kein Abruf von Daten aus {\glqq COSP\grqq} statt. Die Antwort wird als strukturiertes Array an den Aufrufer zurückgegeben.
\subsubsection{getCommentsCreateStatisticalDataLastMonth}
\paragraph{Parameter} Die Funktion besitzt folgende Parameter:
\begin{table}[H]
	\begin{tabular}{|c|p{11cm}|}
		\hline
		\textbf{Parametername} & \textbf{Parameterbeschreibung} \\ \hline
		\$number & Anzahl der Monate \\ \hline
	\end{tabular}
\end{table}
\paragraph{Beschreibung} Die Funktion fragt statistische Daten zu Kommentaren für den in Monaten angegeben Zeitraum ab. Die Funktion nutzt folgende Quellen:
\begin{itemize}
	\item Kommentar-Tabelle
\end{itemize}
Es findet bei dieser Funktion kein Abruf von Daten aus {\glqq COSP\grqq} statt. Die Antwort wird als strukturiertes Array an den Aufrufer zurückgegeben.
\subsubsection{getCommentsCreateStatisticalDataLastYear}
\paragraph{Parameter} Die Funktion besitzt folgende Parameter:
\begin{table}[H]
	\begin{tabular}{|c|p{11cm}|}
		\hline
		\textbf{Parametername} & \textbf{Parameterbeschreibung} \\ \hline
		\$number & Anzahl der Jahre \\ \hline
	\end{tabular}
\end{table}
\paragraph{Beschreibung} Die Funktion fragt statistische Daten zu Kommentaren für den in Jahren angegeben Zeitraum ab. Die Funktion nutzt folgende Quellen:
\begin{itemize}
	\item Kommentar-Tabelle
\end{itemize}
Es findet bei dieser Funktion kein Abruf von Daten aus {\glqq COSP\grqq} statt. Die Antwort wird als strukturiertes Array an den Aufrufer zurückgegeben.
\subsubsection{getCommentsCreateStatisticalDataLastDays}
\paragraph{Parameter} Die Funktion besitzt folgende Parameter:
\begin{table}[H]
	\begin{tabular}{|c|p{11cm}|}
		\hline
		\textbf{Parametername} & \textbf{Parameterbeschreibung} \\ \hline
		\$number & Anzahl der Tage \\ \hline
	\end{tabular}
\end{table}
\paragraph{Beschreibung} Die Funktion fragt statistische Daten  zu Kommentaren für den in Tagen angegeben Zeitraum ab. Die Funktion nutzt folgende Quellen:
\begin{itemize}
	\item Kommentar-Tabelle
\end{itemize}
Es findet bei dieser Funktion kein Abruf von Daten aus {\glqq COSP\grqq} statt. Die Antwort wird als strukturiertes Array an den Aufrufer zurückgegeben.
\newpage
\section{statistics-poi-db}
\subsection{Allgemeines} Diese Datei enthält alle Funktionen für statistische Datenbankabfragen der Interessenpunkt-Tabelle.
\begin{table}[H]
	\begin{tabular}{|c|p{11cm}|}
		\hline
		\textbf{Einbindungspunkt} & inc-db.php \\ \hline
		\textbf{Einbindungspunkt} & inc-db-sub.php \\ \hline
	\end{tabular}
\end{table}
Die Datei ist nicht direkt durch den Nutzer aufrufbar, dies wird durch folgenden Code-Ausschnitt sichergestellt:
\begin{lstlisting}[language=php]
	if (!defined('NICE_PROJECT')) {
		die('Permission denied.');
	}
\end{lstlisting}
Der Globale Wert {\glqq NICE\_PROJECT\grqq} wird durch für den Nutzer valide Aufrufpunkte festgelegt, z.B. {\glqq api.php\grqq}.
\newpage
\subsection{Funktionen}
\subsubsection{getPoiCreateStatisticalDataLastWeeks}
\paragraph{Parameter} Die Funktion besitzt folgende Parameter:
\begin{table}[H]
	\begin{tabular}{|c|p{11cm}|}
		\hline
		\textbf{Parametername} & \textbf{Parameterbeschreibung} \\ \hline
		\$number & Anzahl der Wochen \\ \hline
	\end{tabular}
\end{table}
\paragraph{Beschreibung} Die Funktion fragt statistische Daten zu Interessenpunkten für den in Wochen angegeben Zeitraum ab. Die Funktion nutzt folgende Quellen:
\begin{itemize}
	\item Interessenpunkt-Tabelle
\end{itemize}
Es findet bei dieser Funktion kein Abruf von Daten aus {\glqq COSP\grqq} statt. Die Antwort wird als strukturiertes Array an den Aufrufer zurückgegeben.
\subsubsection{getPoiCreateStatisticalDataLastMonth}
\paragraph{Parameter} Die Funktion besitzt folgende Parameter:
\begin{table}[H]
	\begin{tabular}{|c|p{11cm}|}
		\hline
		\textbf{Parametername} & \textbf{Parameterbeschreibung} \\ \hline
		\$number & Anzahl der Monate \\ \hline
	\end{tabular}
\end{table}
\paragraph{Beschreibung} Die Funktion fragt statistische Daten zu Interessenpunkten für den in Monaten angegeben Zeitraum ab. Die Funktion nutzt folgende Quellen:
\begin{itemize}
	\item Interessenpunkt-Tabelle
\end{itemize}
Es findet bei dieser Funktion kein Abruf von Daten aus {\glqq COSP\grqq} statt. Die Antwort wird als strukturiertes Array an den Aufrufer zurückgegeben.
\subsubsection{getPoiCreateStatisticalDataLastYear}
\paragraph{Parameter} Die Funktion besitzt folgende Parameter:
\begin{table}[H]
	\begin{tabular}{|c|p{11cm}|}
		\hline
		\textbf{Parametername} & \textbf{Parameterbeschreibung} \\ \hline
		\$number & Anzahl der Jahre \\ \hline
	\end{tabular}
\end{table}
\paragraph{Beschreibung} Die Funktion fragt statistische Daten zu Interessenpunkten für den in Jahren angegeben Zeitraum ab. Die Funktion nutzt folgende Quellen:
\begin{itemize}
	\item Interessenpunkt-Tabelle
\end{itemize}
Es findet bei dieser Funktion kein Abruf von Daten aus {\glqq COSP\grqq} statt. Die Antwort wird als strukturiertes Array an den Aufrufer zurückgegeben.
\subsubsection{getPoiCreateStatisticalDataLastDays}
\paragraph{Parameter} Die Funktion besitzt folgende Parameter:
\begin{table}[H]
	\begin{tabular}{|c|p{11cm}|}
		\hline
		\textbf{Parametername} & \textbf{Parameterbeschreibung} \\ \hline
		\$number & Anzahl der Tage \\ \hline
	\end{tabular}
\end{table}
\paragraph{Beschreibung} Die Funktion fragt statistische Daten  zu Interessenpunkten für den in Tagen angegeben Zeitraum ab. Die Funktion nutzt folgende Quellen:
\begin{itemize}
	\item Interessenpunkt-Tabelle
\end{itemize}
Es findet bei dieser Funktion kein Abruf von Daten aus {\glqq COSP\grqq} statt. Die Antwort wird als strukturiertes Array an den Aufrufer zurückgegeben.
\newpage
\section{user}
\subsection{Allgemeines} Diese Datei enthält alle Funktionen für statistische Datenbankabfragen der Interessenpunkt-Tabelle.
\begin{table}[H]
	\begin{tabular}{|c|p{11cm}|}
		\hline
		\textbf{Einbindungspunkt} & inc-db.php \\ \hline
		\textbf{Einbindungspunkt} & inc-db-sub.php \\ \hline
	\end{tabular}
\end{table}
Die Datei ist nicht direkt durch den Nutzer aufrufbar, dies wird durch folgenden Code-Ausschnitt sichergestellt:
\begin{lstlisting}[language=php]
	if (!defined('NICE_PROJECT')) {
		die('Permission denied.');
	}
\end{lstlisting}
Der Globale Wert {\glqq NICE\_PROJECT\grqq} wird durch für den Nutzer valide Aufrufpunkte festgelegt, z.B. {\glqq api.php\grqq}.
\newpage
\subsection{Funktionen}
\subsubsection{addUser}
\paragraph{Parameter} Die Funktion besitzt folgende Parameter:
\begin{table}[H]
	\begin{tabular}{|c|p{11cm}|}
		\hline
		\textbf{Parametername} & \textbf{Parameterbeschreibung} \\ \hline
		\$name      & Nutzername \\ \hline
		\$pwd\_hash & Hash des Passwortes \\ \hline
		\$email     & E-Mailadresse des Nutzers \\ \hline
		\$firstname & Vorname des Nutzers (optional) \\ \hline
		\$lastname  & Nachname des Nutzers (optional) \\ \hline
	\end{tabular}
\end{table}
\paragraph{Beschreibung} Die Funktion fügt einen neuen Nutzer ein. Die Funktion hat Auswirkungen auf folgende Quellen:
\begin{itemize}
	\item Nutzerdaten-Tabelle
\end{itemize}
Es findet bei dieser Funktion kein Abruf von Daten aus {\glqq COSP\grqq} statt. Es gibt einen Rückgabewert.
\subsubsection{updateUser}
\paragraph{Parameter} Die Funktion besitzt folgende Parameter:
\begin{table}[H]
	\begin{tabular}{|c|p{11cm}|}
		\hline
		\textbf{Parametername} & \textbf{Parameterbeschreibung} \\ \hline
		\$name      & Nutzername \\ \hline
		\$pwd\_hash & Hash des Passwortes \\ \hline
		\$email     & E-Mailadresse des Nutzers \\ \hline
		\$firstname & Vorname des Nutzers (optional) \\ \hline
		\$lastname  & Nachname des Nutzers (optional) \\ \hline
	\end{tabular}
\end{table}
\paragraph{Beschreibung} Die Funktion aktualisiert einen Nutzer. Die Funktion hat Auswirkungen auf folgende Quellen:
\begin{itemize}
	\item Nutzerdaten-Tabelle
\end{itemize}
Es findet bei dieser Funktion kein Abruf von Daten aus {\glqq COSP\grqq} statt. Es gibt einen Rückgabewert.
\subsubsection{getUserData}
\paragraph{Parameter} Die Funktion besitzt folgende Parameter:
\begin{table}[H]
	\begin{tabular}{|c|p{11cm}|}
		\hline
		\textbf{Parametername} & \textbf{Parameterbeschreibung} \\ \hline
		\$name & Nutzername \\ \hline
	\end{tabular}
\end{table}
\paragraph{Beschreibung} Die Funktion fragt alle Daten eines Nutzers ab. Die Funktion nutzt folgende Quellen:
\begin{itemize}
	\item Nutzerdaten-Tabelle
\end{itemize}
Es findet bei dieser Funktion kein Abruf von Daten aus {\glqq COSP\grqq} statt. Die Antwort wird als strukturiertes Array an den Aufrufer zurückgegeben.
\subsubsection{getUserDataById}
\paragraph{Parameter} Die Funktion besitzt folgende Parameter:
\begin{table}[H]
	\begin{tabular}{|c|p{11cm}|}
		\hline
		\textbf{Parametername} & \textbf{Parameterbeschreibung} \\ \hline
		\$uid & Identifikator eines Nutzers \\ \hline
	\end{tabular}
\end{table}
\paragraph{Beschreibung} Die Funktion fragt alle Daten eines Nutzers ab. Die Funktion nutzt folgende Quellen:
\begin{itemize}
	\item Nutzerdaten-Tabelle
\end{itemize}
Es findet bei dieser Funktion kein Abruf von Daten aus {\glqq COSP\grqq} statt. Die Antwort wird als strukturiertes Array an den Aufrufer zurückgegeben.
\subsubsection{getAllUsernames}
\paragraph{Parameter} Die Funktion besitzt folgende Parameter:
\begin{table}[H]
	\begin{tabular}{|c|p{11cm}|}
		\hline
		\textbf{Parametername} & \textbf{Parameterbeschreibung} \\ \hline
		\$onlyLocal & Fragt nur lokale Nutzernamen ab \\ \hline
	\end{tabular}
\end{table}
\paragraph{Beschreibung} Die Funktion fragt Nutzernamen ab, auch die Nutzernamen von {\glqq COSP\grqq}. Die Funktion nutzt folgende Quellen:
\begin{itemize}
	\item Nutzerdaten-Tabelle
	\item COSP
\end{itemize}
Es findet bei dieser Funktion ein Abruf von Daten aus {\glqq COSP\grqq} statt. Die Antwort wird als strukturiertes Array an den Aufrufer zurückgegeben.
\subsubsection{updateDeaktivate}
\paragraph{Parameter} Die Funktion besitzt folgende Parameter:
\begin{table}[H]
	\begin{tabular}{|c|p{11cm}|}
		\hline
		\textbf{Parametername} & \textbf{Parameterbeschreibung} \\ \hline
		\$username      & Nutzername \\ \hline
		\$state         & Status der Deaktivierung \\ \hline
	\end{tabular}
\end{table}
\paragraph{Beschreibung} Die Funktion deaktiviert einen Nutzer. Die Funktion hat Auswirkungen auf folgende Quellen:
\begin{itemize}
	\item Nutzerdaten-Tabelle
\end{itemize}
Es findet bei dieser Funktion kein Abruf von Daten aus {\glqq COSP\grqq} statt. Es gibt einen Rückgabewert.
\newpage
\section{validate-cinemas}
\subsection{Allgemeines} Diese Datei enthält alle Funktionen, welche die Tabelle mit Validierungsinformationen zu Saalanzahlen benutzen.
\begin{table}[H]
	\begin{tabular}{|c|p{11cm}|}
		\hline
		\textbf{Einbindungspunkt} & inc-db.php \\ \hline
		\textbf{Einbindungspunkt} & inc-db-sub.php \\ \hline
	\end{tabular}
\end{table}
Die Datei ist nicht direkt durch den Nutzer aufrufbar, dies wird durch folgenden Code-Ausschnitt sichergestellt:
\begin{lstlisting}[language=php]
	if (!defined('NICE_PROJECT')) {
		die('Permission denied.');
	}
\end{lstlisting}
Der Globale Wert {\glqq NICE\_PROJECT\grqq} wird durch für den Nutzer valide Aufrufpunkte festgelegt, z.B. {\glqq api.php\grqq}.
\newpage
\subsection{Funktionen}
\subsubsection{getValidateSumCinemas}
\paragraph{Parameter} Die Funktion besitzt folgende Parameter:
\begin{table}[H]
	\begin{tabular}{|c|p{11cm}|}
		\hline
		\textbf{Parametername} & \textbf{Parameterbeschreibung} \\ \hline
		\$cinemas\_id & Identifikator einer Saalanzahl \\ \hline
	\end{tabular}
\end{table}
\paragraph{Beschreibung} Die Funktion ruft die Summe der Validierungen einer Saalanzahl ab. Die Funktion nutzt folgende Quellen:
\begin{itemize}
	\item Tabelle mit Validierungsinformationen zu Saalanzahlen
\end{itemize}
Es findet bei dieser Funktion kein Abruf von Daten aus {\glqq COSP\grqq} statt. Die Antwort wird als strukturiertes Array an den Aufrufer zurückgegeben.
\subsubsection{getAllValidatedForPoiCinemas}
\paragraph{Parameter} Die Funktion besitzt keine Parameter.
\paragraph{Beschreibung} Die Funktion ruft alle Validierungsinformationen zu Saalanzahlen ab. Die Funktion nutzt folgende Quellen:
\begin{itemize}
	\item Tabelle mit Validierungsinformationen zu Saalanzahlen
\end{itemize}
Es findet bei dieser Funktion kein Abruf von Daten aus {\glqq COSP\grqq} statt. Die Antwort wird als strukturiertes Array an den Aufrufer zurückgegeben.
\subsubsection{insertValidateCinemas}
\paragraph{Parameter} Die Funktion besitzt folgende Parameter:
\begin{table}[H]
	\begin{tabular}{|c|p{11cm}|}
		\hline
		\textbf{Parametername} & \textbf{Parameterbeschreibung} \\ \hline
		\$cinemas\_id & Identifikator einer Saalanzahl \\ \hline
		\$value       & Wert der Validierung \\ \hline
	\end{tabular}
\end{table}
\paragraph{Beschreibung} Die Funktion fügt einer Saalanzahl eine neue Validierung hinzu. Die Funktion hat Auswirkungen auf folgende Quellen
\begin{itemize}
	\item Tabelle mit Validierungsinformationen zu Saalanzahlen
\end{itemize}
Es findet bei dieser Funktion kein Abruf von Daten aus {\glqq COSP\grqq} statt. Es gibt einen Rückgabewert.
\subsubsection{deleteValidateCinema}
\paragraph{Parameter} Die Funktion besitzt folgende Parameter:
\begin{table}[H]
	\begin{tabular}{|c|p{11cm}|}
		\hline
		\textbf{Parametername} & \textbf{Parameterbeschreibung} \\ \hline
		\$cinemas\_id & Identifikator einer Saalanzahl \\ \hline
	\end{tabular}
\end{table}
\paragraph{Beschreibung} Die Funktion löscht alle Validierungen einer Saalanzahl. Die Funktion hat Auswirkungen auf folgende Quellen
\begin{itemize}
	\item Tabelle mit Validierungsinformationen zu Saalanzahlen
\end{itemize}
Es findet bei dieser Funktion kein Abruf von Daten aus {\glqq COSP\grqq} statt. Es gibt einen Rückgabewert.
\newpage
\section{validate-curr-adr}
\subsection{Allgemeines} Diese Datei enthält alle Funktionen, welche die Tabelle mit Validierungsinformationen zu aktuellen Adressen benutzen.
\begin{table}[H]
	\begin{tabular}{|c|p{11cm}|}
		\hline
		\textbf{Einbindungspunkt} & inc-db.php \\ \hline
		\textbf{Einbindungspunkt} & inc-db-sub.php \\ \hline
	\end{tabular}
\end{table}
Die Datei ist nicht direkt durch den Nutzer aufrufbar, dies wird durch folgenden Code-Ausschnitt sichergestellt:
\begin{lstlisting}[language=php]
	if (!defined('NICE_PROJECT')) {
		die('Permission denied.');
	}
\end{lstlisting}
Der Globale Wert {\glqq NICE\_PROJECT\grqq} wird durch für den Nutzer valide Aufrufpunkte festgelegt, z.B. {\glqq api.php\grqq}.
\newpage
\subsection{Funktionen}
\subsubsection{insertValidateCurrentAddress}
\paragraph{Parameter} Die Funktion besitzt folgende Parameter:
\begin{table}[H]
	\begin{tabular}{|c|p{11cm}|}
		\hline
		\textbf{Parametername} & \textbf{Parameterbeschreibung} \\ \hline
		\$poi\_id & Identifikator des zugehörigen Interessenpunktes \\ \hline
		\$value   & Wert der Validierung \\ \hline
	\end{tabular}
\end{table}
\paragraph{Beschreibung} Die Funktion fügt einem Interessenpunkt eine Validierung für die jeweilige aktuelle Adresse hinzu. Die Funktion hat Auswirkungen auf folgende Quellen:
\begin{itemize}
	\item Tabelle mit Validierungsinformationen zu aktuellen Adressen
\end{itemize}
Es findet bei dieser Funktion kein Abruf von Daten aus {\glqq COSP\grqq} statt. Es gibt einen Rückgabewert.
\subsubsection{deleteValidateCurAddress}
\paragraph{Parameter} Die Funktion besitzt folgende Parameter:
\begin{table}[H]
	\begin{tabular}{|c|p{11cm}|}
		\hline
		\textbf{Parametername} & \textbf{Parameterbeschreibung} \\ \hline
		\$poi\_id & Identifikator des zugehörigen Interessenpunktes \\ \hline
	\end{tabular}
\end{table}
\paragraph{Beschreibung} Die Funktion löscht Validierungen zu einer aktuellen Adresse des gegebenen Interessenpunktes. Die Funktion hat Auswirkungen auf folgende Quellen:
\begin{itemize}
	\item Tabelle mit Validierungsinformationen zu aktuellen Adressen
\end{itemize}
Es findet bei dieser Funktion kein Abruf von Daten aus {\glqq COSP\grqq} statt. Es gibt einen Rückgabewert.
\subsubsection{getValidateSumCurAddresse}
\paragraph{Parameter} Die Funktion besitzt folgende Parameter:
\begin{table}[H]
	\begin{tabular}{|c|p{11cm}|}
		\hline
		\textbf{Parametername} & \textbf{Parameterbeschreibung} \\ \hline
		\$poi\_id & Identifikator des zugehörigen Interessenpunktes \\ \hline
	\end{tabular}
\end{table}
\paragraph{Beschreibung} Die Funktion ermittelt den Validierungswert der aktuellen Adresse des gegebenen Interessenpunktes. Die Funktion nutzt folgende Quellen:
\begin{itemize}
	\item Tabelle mit Validierungsinformationen zu aktuellen Adressen
\end{itemize}
Es findet bei dieser Funktion kein Abruf von Daten aus {\glqq COSP\grqq} statt. Die Antwort wird als strukturiertes Array an den Aufrufer zurückgegeben.
\subsubsection{getValidateSumCurAddresse}
\paragraph{Parameter} Die Funktion besitzt keine Parameter.
\paragraph{Beschreibung} Die Funktion ruft alle Validierungswert zu aktuellen Adressen ab. Die Funktion nutzt folgende Quellen:
\begin{itemize}
	\item Tabelle mit Validierungsinformationen zu aktuellen Adressen
\end{itemize}
Es findet bei dieser Funktion kein Abruf von Daten aus {\glqq COSP\grqq} statt. Die Antwort wird als strukturiertes Array an den Aufrufer zurückgegeben.
\newpage
\section{validate-hist-adr}
\subsection{Allgemeines} Diese Datei enthält alle Funktionen, welche die Tabelle mit Validierungsinformationen zu historischen Adressen von Interessenpunkten benutzen.
\begin{table}[H]
	\begin{tabular}{|c|p{11cm}|}
		\hline
		\textbf{Einbindungspunkt} & inc-db.php \\ \hline
		\textbf{Einbindungspunkt} & inc-db-sub.php \\ \hline
	\end{tabular}
\end{table}
Die Datei ist nicht direkt durch den Nutzer aufrufbar, dies wird durch folgenden Code-Ausschnitt sichergestellt:
\begin{lstlisting}[language=php]
	if (!defined('NICE_PROJECT')) {
		die('Permission denied.');
	}
\end{lstlisting}
Der Globale Wert {\glqq NICE\_PROJECT\grqq} wird durch für den Nutzer valide Aufrufpunkte festgelegt, z.B. {\glqq api.php\grqq}.
\newpage
\subsection{Funktionen}
\subsubsection{insertValidateHistAddress}
\paragraph{Parameter} Die Funktion besitzt folgende Parameter:
\begin{table}[H]
	\begin{tabular}{|c|p{11cm}|}
		\hline
		\textbf{Parametername} & \textbf{Parameterbeschreibung} \\ \hline
		\$address\_id & Identifikator einer historischen Adresse \\ \hline
		\$value       & Wert der Validierung \\ \hline
	\end{tabular}
\end{table}
\paragraph{Beschreibung} Die Funktion fügt eine Validierung einer historischen Adresse hinzu. Die Funktion hat Auswirkungen auf folgende Quellen:
\begin{itemize}
	\item Tabelle mit Validierungsinformationen zu historischen Adressen
\end{itemize}
Es findet bei dieser Funktion kein Abruf von Daten aus {\glqq COSP\grqq} statt. Es gibt einen Rückgabewert.
\subsubsection{getValidateSumHistAddress}
\paragraph{Parameter} Die Funktion besitzt folgende Parameter:
\begin{table}[H]
	\begin{tabular}{|c|p{11cm}|}
		\hline
		\textbf{Parametername} & \textbf{Parameterbeschreibung} \\ \hline
		\$poi\_id & Identifikator eines Interessenpunktes \\ \hline
	\end{tabular}
\end{table}
\paragraph{Beschreibung} Die Funktion ruft den Validierungswert einer historischen Adresse ab. Die Funktion nutzt folgende Quellen:
\begin{itemize}
	\item Tabelle mit Validierungsinformationen zu historischen Adressen
\end{itemize}
Es findet bei dieser Funktion kein Abruf von Daten aus {\glqq COSP\grqq} statt. Die Antwort wird als strukturiertes Array an den Aufrufer zurückgegeben.
\subsubsection{deleteValidateHistAddress}
\paragraph{Parameter} Die Funktion besitzt folgende Parameter:
\begin{table}[H]
	\begin{tabular}{|c|p{11cm}|}
		\hline
		\textbf{Parametername} & \textbf{Parameterbeschreibung} \\ \hline
		\$poi\_id & Identifikator des zugehörigen Interessenpunktes \\ \hline
	\end{tabular}
\end{table}
\paragraph{Beschreibung} Die Funktion löscht alle Validierung einer historischen Adresse. Die Funktion hat Auswirkungen auf folgende Quellen:
\begin{itemize}
	\item Tabelle mit Validierungsinformationen zu historischen Adressen
\end{itemize}
Es findet bei dieser Funktion kein Abruf von Daten aus {\glqq COSP\grqq} statt. Es gibt einen Rückgabewert.
\subsubsection{getAllValidatedForPoiHistAddresses}
\paragraph{Parameter} Die Funktion besitzt keine Parameter.
\paragraph{Beschreibung} Die Funktion ruft alle Validierungsinformationen von historischen Adressen ab. Die Funktion nutzt folgende Quellen:
\begin{itemize}
	\item Tabelle mit Validierungsinformationen zu historischen Adressen
\end{itemize}
Es findet bei dieser Funktion kein Abruf von Daten aus {\glqq COSP\grqq} statt. Die Antwort wird als strukturiertes Array an den Aufrufer zurückgegeben.

\newpage
\section{validate-hist}
\subsection{Allgemeines} Diese Datei enthält alle Funktionen, welche die Tabelle mit Validierungsinformationen zu Geschichten von Interessenpunkten benutzen.
\begin{table}[H]
	\begin{tabular}{|c|p{11cm}|}
		\hline
		\textbf{Einbindungspunkt} & inc-db.php \\ \hline
		\textbf{Einbindungspunkt} & inc-db-sub.php \\ \hline
	\end{tabular}
\end{table}
Die Datei ist nicht direkt durch den Nutzer aufrufbar, dies wird durch folgenden Code-Ausschnitt sichergestellt:
\begin{lstlisting}[language=php]
	if (!defined('NICE_PROJECT')) {
		die('Permission denied.');
	}
\end{lstlisting}
Der Globale Wert {\glqq NICE\_PROJECT\grqq} wird durch für den Nutzer valide Aufrufpunkte festgelegt, z.B. {\glqq api.php\grqq}.
\newpage
\subsection{Funktionen}
\subsubsection{insertValidateHistory}
\paragraph{Parameter} Die Funktion besitzt folgende Parameter:
\begin{table}[H]
	\begin{tabular}{|c|p{11cm}|}
		\hline
		\textbf{Parametername} & \textbf{Parameterbeschreibung} \\ \hline
		\$poi\_id & Identifikator des zugehörigen Interessenpunktes \\ \hline
		\$value   & Wert der Validierung \\ \hline
	\end{tabular}
\end{table}
\paragraph{Beschreibung} Die Funktion fügt eine Validierung der Geschichte des gegebenen Interessenpunktes hinzu. Die Funktion hat Auswirkungen auf folgende Quellen:
\begin{itemize}
	\item Tabelle mit Validierungsinformationen zu Geschichten von Interessenpunkten
\end{itemize}
Es findet bei dieser Funktion kein Abruf von Daten aus {\glqq COSP\grqq} statt. Es gibt einen Rückgabewert.
\subsubsection{deleteValidateHistory}
\paragraph{Parameter} Die Funktion besitzt folgende Parameter:
\begin{table}[H]
	\begin{tabular}{|c|p{11cm}|}
		\hline
		\textbf{Parametername} & \textbf{Parameterbeschreibung} \\ \hline
		\$poi\_id & Identifikator des zugehörigen Interessenpunktes \\ \hline
	\end{tabular}
\end{table}
\paragraph{Beschreibung} Die Funktion löscht alle Validierung der Geschichte des gegebenen Interessenpunktes. Die Funktion hat Auswirkungen auf folgende Quellen:
\begin{itemize}
	\item Tabelle mit Validierungsinformationen zu Geschichten von Interessenpunkten
\end{itemize}
Es findet bei dieser Funktion kein Abruf von Daten aus {\glqq COSP\grqq} statt. Es gibt einen Rückgabewert.
\subsubsection{getValidateSumHist}
\paragraph{Parameter} Die Funktion besitzt folgende Parameter:
\begin{table}[H]
	\begin{tabular}{|c|p{11cm}|}
		\hline
		\textbf{Parametername} & \textbf{Parameterbeschreibung} \\ \hline
		\$poi\_id & Identifikator eines Interessenpunktes \\ \hline
	\end{tabular}
\end{table}
\paragraph{Beschreibung} Die Funktion ruft den Validierungswert der Geschichte eines Interessenpunktes ab. Die Funktion nutzt folgende Quellen:
\begin{itemize}
	\item Tabelle mit Validierungsinformationen zu Geschichten von Interessenpunkten
\end{itemize}
Es findet bei dieser Funktion kein Abruf von Daten aus {\glqq COSP\grqq} statt. Die Antwort wird als strukturiertes Array an den Aufrufer zurückgegeben.
\subsubsection{getAllValidatedForHistory}
\paragraph{Parameter} Die Funktion besitzt keine Parameter.
\paragraph{Beschreibung} Die Funktion ruft alle Validierungsinformationen von Geschichten der Interessenpunkte ab. Die Funktion nutzt folgende Quellen:
\begin{itemize}
	\item Tabelle mit Validierungsinformationen zu Geschichten von Interessenpunkten
\end{itemize}
Es findet bei dieser Funktion kein Abruf von Daten aus {\glqq COSP\grqq} statt. Die Antwort wird als strukturiertes Array an den Aufrufer zurückgegeben.

\newpage
\section{validate-name}
\subsection{Allgemeines} Diese Datei enthält alle Funktionen, welche die Tabelle mit Validierungsinformationen zu Namen benutzen.
\begin{table}[H]
	\begin{tabular}{|c|p{11cm}|}
		\hline
		\textbf{Einbindungspunkt} & inc-db.php \\ \hline
		\textbf{Einbindungspunkt} & inc-db-sub.php \\ \hline
	\end{tabular}
\end{table}
Die Datei ist nicht direkt durch den Nutzer aufrufbar, dies wird durch folgenden Code-Ausschnitt sichergestellt:
\begin{lstlisting}[language=php]
	if (!defined('NICE_PROJECT')) {
		die('Permission denied.');
	}
\end{lstlisting}
Der Globale Wert {\glqq NICE\_PROJECT\grqq} wird durch für den Nutzer valide Aufrufpunkte festgelegt, z.B. {\glqq api.php\grqq}.
\newpage
\subsection{Funktionen}
\subsubsection{insertValidateName}
\paragraph{Parameter} Die Funktion besitzt folgende Parameter:
\begin{table}[H]
	\begin{tabular}{|c|p{11cm}|}
		\hline
		\textbf{Parametername} & \textbf{Parameterbeschreibung} \\ \hline
		\$name\_id & Identifikator eines Namens \\ \hline
		\$value    & Wert der Validierung \\ \hline
	\end{tabular}
\end{table}
\paragraph{Beschreibung} Die Funktion fügt einem Namen eine neue Validierung hinzu. Die Funktion hat Auswirkungen auf folgende Quellen
\begin{itemize}
	\item Tabelle mit Validierungsinformationen zu Namen
\end{itemize}
Es findet bei dieser Funktion kein Abruf von Daten aus {\glqq COSP\grqq} statt. Es gibt einen Rückgabewert.
\subsubsection{getValidateSumName}
\paragraph{Parameter} Die Funktion besitzt folgende Parameter:
\begin{table}[H]
	\begin{tabular}{|c|p{11cm}|}
		\hline
		\textbf{Parametername} & \textbf{Parameterbeschreibung} \\ \hline
		\$name\_id & Identifikator eines Namens \\ \hline
	\end{tabular}
\end{table}
\paragraph{Beschreibung} Die Funktion ruft die Summe der Validierungen eines Namen ab. Die Funktion nutzt folgende Quellen:
\begin{itemize}
	\item Tabelle mit Validierungsinformationen zu Namen
\end{itemize}
Es findet bei dieser Funktion kein Abruf von Daten aus {\glqq COSP\grqq} statt. Die Antwort wird als strukturiertes Array an den Aufrufer zurückgegeben.
\subsubsection{deleteValidateName}
\paragraph{Parameter} Die Funktion besitzt folgende Parameter:
\begin{table}[H]
	\begin{tabular}{|c|p{11cm}|}
		\hline
		\textbf{Parametername} & \textbf{Parameterbeschreibung} \\ \hline
		\$name\_id & Identifikator eines Namens \\ \hline
	\end{tabular}
\end{table}
\paragraph{Beschreibung} Die Funktion löscht alle Validierungen eines Namen. Die Funktion hat Auswirkungen auf folgende Quellen
\begin{itemize}
	\item Tabelle mit Validierungsinformationen zu Namen
\end{itemize}
Es findet bei dieser Funktion kein Abruf von Daten aus {\glqq COSP\grqq} statt. Es gibt einen Rückgabewert.
\subsubsection{getAllValidatedForPoiNames}
\paragraph{Parameter} Die Funktion besitzt keine Parameter.
\paragraph{Beschreibung} Die Funktion ruft alle Validierungsinformationen zu Namen ab. Die Funktion nutzt folgende Quellen:
\begin{itemize}
	\item Tabelle mit Validierungsinformationen zu Namen
\end{itemize}
Es findet bei dieser Funktion kein Abruf von Daten aus {\glqq COSP\grqq} statt. Die Antwort wird als strukturiertes Array an den Aufrufer zurückgegeben.
\newpage
\section{validate-operator}
\subsection{Allgemeines} Diese Datei enthält alle Funktionen, welche die Tabelle mit Validierungsinformationen zu Betreibern benutzen.
\begin{table}[H]
	\begin{tabular}{|c|p{11cm}|}
		\hline
		\textbf{Einbindungspunkt} & inc-db.php \\ \hline
		\textbf{Einbindungspunkt} & inc-db-sub.php \\ \hline
	\end{tabular}
\end{table}
Die Datei ist nicht direkt durch den Nutzer aufrufbar, dies wird durch folgenden Code-Ausschnitt sichergestellt:
\begin{lstlisting}[language=php]
	if (!defined('NICE_PROJECT')) {
		die('Permission denied.');
	}
\end{lstlisting}
Der Globale Wert {\glqq NICE\_PROJECT\grqq} wird durch für den Nutzer valide Aufrufpunkte festgelegt, z.B. {\glqq api.php\grqq}.
\newpage
\subsection{Funktionen}
\subsubsection{insertValidateOperator}
\paragraph{Parameter} Die Funktion besitzt folgende Parameter:
\begin{table}[H]
	\begin{tabular}{|c|p{11cm}|}
		\hline
		\textbf{Parametername} & \textbf{Parameterbeschreibung} \\ \hline
		\$operator\_id & Identifikator eines Betreibers \\ \hline
		\$value        & Wert der Validierung \\ \hline
	\end{tabular}
\end{table}
\paragraph{Beschreibung} Die Funktion fügt einem Betreiber eine neue Validierung hinzu. Die Funktion hat Auswirkungen auf folgende Quellen
\begin{itemize}
	\item Tabelle mit Validierungsinformationen zu Betreibern
\end{itemize}
Es findet bei dieser Funktion kein Abruf von Daten aus {\glqq COSP\grqq} statt. Es gibt einen Rückgabewert.
\subsubsection{getValidateSumOperator}
\paragraph{Parameter} Die Funktion besitzt folgende Parameter:
\begin{table}[H]
	\begin{tabular}{|c|p{11cm}|}
		\hline
		\textbf{Parametername} & \textbf{Parameterbeschreibung} \\ \hline
		\$operator\_id & Identifikator eines Betreibers \\ \hline
	\end{tabular}
\end{table}
\paragraph{Beschreibung} Die Funktion ruft die Summe der Validierungen eines Betreibers ab. Die Funktion nutzt folgende Quellen:
\begin{itemize}
	\item Tabelle mit Validierungsinformationen zu Betreibern
\end{itemize}
Es findet bei dieser Funktion kein Abruf von Daten aus {\glqq COSP\grqq} statt. Die Antwort wird als strukturiertes Array an den Aufrufer zurückgegeben.
\subsubsection{deleteValidateOperator}
\paragraph{Parameter} Die Funktion besitzt folgende Parameter:
\begin{table}[H]
	\begin{tabular}{|c|p{11cm}|}
		\hline
		\textbf{Parametername} & \textbf{Parameterbeschreibung} \\ \hline
		\$operator\_id & Identifikator eines Betreibers \\ \hline
	\end{tabular}
\end{table}
\paragraph{Beschreibung} Die Funktion löscht alle Validierungen eines Betreibers. Die Funktion hat Auswirkungen auf folgende Quellen
\begin{itemize}
	\item Tabelle mit Validierungsinformationen zu Betreibern
\end{itemize}
Es findet bei dieser Funktion kein Abruf von Daten aus {\glqq COSP\grqq} statt. Es gibt einen Rückgabewert.
\subsubsection{getAllValidatedForPoiOperators}
\paragraph{Parameter} Die Funktion besitzt keine Parameter.
\paragraph{Beschreibung} Die Funktion ruft alle Validierungsinformationen zu Betreibern ab. Die Funktion nutzt folgende Quellen:
\begin{itemize}
	\item Tabelle mit Validierungsinformationen zu Betreibern
\end{itemize}
Es findet bei dieser Funktion kein Abruf von Daten aus {\glqq COSP\grqq} statt. Die Antwort wird als strukturiertes Array an den Aufrufer zurückgegeben.
\newpage
\section{validate-poi-picture}
\subsection{Allgemeines} Diese Datei enthält alle Funktionen, welche die Tabelle mit Validierungsinformationen zu Links zwischen Bildern und Interessenpunkten benutzen.
\begin{table}[H]
	\begin{tabular}{|c|p{11cm}|}
		\hline
		\textbf{Einbindungspunkt} & inc-db.php \\ \hline
		\textbf{Einbindungspunkt} & inc-db-sub.php \\ \hline
	\end{tabular}
\end{table}
Die Datei ist nicht direkt durch den Nutzer aufrufbar, dies wird durch folgenden Code-Ausschnitt sichergestellt:
\begin{lstlisting}[language=php]
	if (!defined('NICE_PROJECT')) {
		die('Permission denied.');
	}
\end{lstlisting}
Der Globale Wert {\glqq NICE\_PROJECT\grqq} wird durch für den Nutzer valide Aufrufpunkte festgelegt, z.B. {\glqq api.php\grqq}.
\newpage
\subsection{Funktionen}
\subsubsection{insertValidatePoiPicLink}
\paragraph{Parameter} Die Funktion besitzt folgende Parameter:
\begin{table}[H]
	\begin{tabular}{|c|p{11cm}|}
		\hline
		\textbf{Parametername} & \textbf{Parameterbeschreibung} \\ \hline
		\$pic\_poi\_id & Identifikator eines Links zwischen einem Bild und einem Interessenpunkt \\ \hline
		\$value        & Wert der Validierung \\ \hline
	\end{tabular}
\end{table}
\paragraph{Beschreibung} Die Funktion fügt einem Link zwischen einem Bild und einem Interessenpunkt eine neue Validierung hinzu. Die Funktion hat Auswirkungen auf folgende Quellen
\begin{itemize}
	\item Tabelle mit Validierungsinformationen zu Links zwischen Bildern und Interessenpunkten
\end{itemize}
Es findet bei dieser Funktion kein Abruf von Daten aus {\glqq COSP\grqq} statt. Es gibt einen Rückgabewert.
\subsubsection{getValidateSumPoiPic}
\paragraph{Parameter} Die Funktion besitzt folgende Parameter:
\begin{table}[H]
	\begin{tabular}{|c|p{11cm}|}
		\hline
		\textbf{Parametername} & \textbf{Parameterbeschreibung} \\ \hline
		\$pic\_poi\_id & Identifikator eines Links zwischen einem Bild und einem Interessenpunkt \\ \hline
	\end{tabular}
\end{table}
\paragraph{Beschreibung} Die Funktion ruft die Summe der Validierungen eines Links zwischen einem Bild und einem Interessenpunkt ab. Die Funktion nutzt folgende Quellen:
\begin{itemize}
	\item Tabelle mit Validierungsinformationen zu Links zwischen Bildern und Interessenpunkten
\end{itemize}
Es findet bei dieser Funktion kein Abruf von Daten aus {\glqq COSP\grqq} statt. Die Antwort wird als strukturiertes Array an den Aufrufer zurückgegeben.
\subsubsection{getAllValidatedForPoiPicLink}
\paragraph{Parameter} Die Funktion besitzt keine Parameter.
\paragraph{Beschreibung} Die Funktion ruft alle Validierungsinformationen zu Links zwischen Bildern und Interessenpunkten ab. Die Funktion nutzt folgende Quellen:
\begin{itemize}
	\item Tabelle mit Validierungsinformationen zu Links zwischen Bildern und Interessenpunkten
\end{itemize}
Es findet bei dieser Funktion kein Abruf von Daten aus {\glqq COSP\grqq} statt. Die Antwort wird als strukturiertes Array an den Aufrufer zurückgegeben.
\subsubsection{deleteValidationsForCertainPoiPicLink}
\paragraph{Parameter} Die Funktion besitzt folgende Parameter:
\begin{table}[H]
	\begin{tabular}{|c|p{11cm}|}
		\hline
		\textbf{Parametername} & \textbf{Parameterbeschreibung} \\ \hline
		\$lid & Identifikator eines Links zwischen einem Bild und einem Interessenpunkt \\ \hline
	\end{tabular}
\end{table}
\paragraph{Beschreibung} Die Funktion löscht alle Validierungen eines Links zwischen einem Bild und einem Interessenpunkt. Die Funktion hat Auswirkungen auf folgende Quellen
\begin{itemize}
	\item Tabelle mit Validierungsinformationen zu Links zwischen Bildern und Interessenpunkten
\end{itemize}
Es findet bei dieser Funktion kein Abruf von Daten aus {\glqq COSP\grqq} statt. Es gibt einen Rückgabewert.
\newpage
\section{validate-poi-story}
\subsection{Allgemeines} Diese Datei enthält alle Funktionen, welche die Tabelle mit Validierungsinformationen zu Links zwischen Geschichten und Interessenpunkten benutzen.
\begin{table}[H]
	\begin{tabular}{|c|p{11cm}|}
		\hline
		\textbf{Einbindungspunkt} & inc-db.php \\ \hline
		\textbf{Einbindungspunkt} & inc-db-sub.php \\ \hline
	\end{tabular}
\end{table}
Die Datei ist nicht direkt durch den Nutzer aufrufbar, dies wird durch folgenden Code-Ausschnitt sichergestellt:
\begin{lstlisting}[language=php]
	if (!defined('NICE_PROJECT')) {
		die('Permission denied.');
	}
\end{lstlisting}
Der Globale Wert {\glqq NICE\_PROJECT\grqq} wird durch für den Nutzer valide Aufrufpunkte festgelegt, z.B. {\glqq api.php\grqq}.
\newpage
\subsection{Funktionen}
\subsubsection{insertValidatePoiStory}
\paragraph{Parameter} Die Funktion besitzt folgende Parameter:
\begin{table}[H]
	\begin{tabular}{|c|p{11cm}|}
		\hline
		\textbf{Parametername} & \textbf{Parameterbeschreibung} \\ \hline
		\$story\_poi\_id & Identifikator eines Links zwischen einer Geschichte und einem Interessenpunkt \\ \hline
		\$value        & Wert der Validierung \\ \hline
	\end{tabular}
\end{table}
\paragraph{Beschreibung} Die Funktion fügt einem Link zwischen einer Geschichte und einem Interessenpunkt eine neue Validierung hinzu. Die Funktion hat Auswirkungen auf folgende Quellen
\begin{itemize}
	\item Tabelle mit Validierungsinformationen zu Links zwischen Geschichten und Interessenpunkten
\end{itemize}
Es findet bei dieser Funktion kein Abruf von Daten aus {\glqq COSP\grqq} statt. Es gibt einen Rückgabewert.
\subsubsection{deleteValidatePoiStory}
\paragraph{Parameter} Die Funktion besitzt folgende Parameter:
\begin{table}[H]
	\begin{tabular}{|c|p{11cm}|}
		\hline
		\textbf{Parametername} & \textbf{Parameterbeschreibung} \\ \hline
		\$story\_poi\_id & Identifikator eines Links zwischen einer Geschichte und einem Interessenpunkt \\ \hline
	\end{tabular}
\end{table}
\paragraph{Beschreibung} Die Funktion löscht alle Validierungen eines Links zwischen einer Geschichte und einem Interessenpunkt. Die Funktion hat Auswirkungen auf folgende Quellen
\begin{itemize}
	\item Tabelle mit Validierungsinformationen zu Links zwischen Geschichten und Interessenpunkten
\end{itemize}
Es findet bei dieser Funktion kein Abruf von Daten aus {\glqq COSP\grqq} statt. Es gibt einen Rückgabewert.
\subsubsection{getValidateSumPoiStory}
\paragraph{Parameter} Die Funktion besitzt folgende Parameter:
\begin{table}[H]
	\begin{tabular}{|c|p{11cm}|}
		\hline
		\textbf{Parametername} & \textbf{Parameterbeschreibung} \\ \hline
		\$story\_poi\_id & Identifikator eines Links zwischen einer Geschichte und einem Interessenpunkt \\ \hline
	\end{tabular}
\end{table}
\paragraph{Beschreibung} Die Funktion ruft die Summe der Validierungen eines Links zwischen einer Geschichte und einem Interessenpunkt ab. Die Funktion nutzt folgende Quellen:
\begin{itemize}
	\item Tabelle mit Validierungsinformationen zu Links zwischen Geschichten und Interessenpunkten
\end{itemize}
Es findet bei dieser Funktion kein Abruf von Daten aus {\glqq COSP\grqq} statt. Die Antwort wird als strukturiertes Array an den Aufrufer zurückgegeben.
\subsubsection{getAllValidatedForPoiStory}
\paragraph{Parameter} Die Funktion besitzt keine Parameter.
\paragraph{Beschreibung} Die Funktion ruft alle Validierungsinformationen zu Links zwischen Geschichten und Interessenpunkten ab. Die Funktion nutzt folgende Quellen:
\begin{itemize}
	\item Tabelle mit Validierungsinformationen zu Links zwischen Geschichten und Interessenpunkten
\end{itemize}
Es findet bei dieser Funktion kein Abruf von Daten aus {\glqq COSP\grqq} statt. Die Antwort wird als strukturiertes Array an den Aufrufer zurückgegeben.
\newpage
\section{validate-seats}
\subsection{Allgemeines} Diese Datei enthält alle Funktionen, welche die Tabelle mit Validierungsinformationen zu Sitzplatzanzahlen benutzen.
\begin{table}[H]
	\begin{tabular}{|c|p{11cm}|}
		\hline
		\textbf{Einbindungspunkt} & inc-db.php \\ \hline
		\textbf{Einbindungspunkt} & inc-db-sub.php \\ \hline
	\end{tabular}
\end{table}
Die Datei ist nicht direkt durch den Nutzer aufrufbar, dies wird durch folgenden Code-Ausschnitt sichergestellt:
\begin{lstlisting}[language=php]
	if (!defined('NICE_PROJECT')) {
		die('Permission denied.');
	}
\end{lstlisting}
Der Globale Wert {\glqq NICE\_PROJECT\grqq} wird durch für den Nutzer valide Aufrufpunkte festgelegt, z.B. {\glqq api.php\grqq}.
\newpage
\subsection{Funktionen}
\subsubsection{getValidateSumSeats}
\paragraph{Parameter} Die Funktion besitzt folgende Parameter:
\begin{table}[H]
	\begin{tabular}{|c|p{11cm}|}
		\hline
		\textbf{Parametername} & \textbf{Parameterbeschreibung} \\ \hline
		\$seats\_id & Identifikator einer Sitzplatzanzahl \\ \hline
	\end{tabular}
\end{table}
\paragraph{Beschreibung} Die Funktion ruft die Summe der Validierungen einer Sitzplatzanzahl ab. Die Funktion nutzt folgende Quellen:
\begin{itemize}
	\item Tabelle mit Validierungsinformationen zu Sitzplatzanzahlen
\end{itemize}
Es findet bei dieser Funktion kein Abruf von Daten aus {\glqq COSP\grqq} statt. Die Antwort wird als strukturiertes Array an den Aufrufer zurückgegeben.
\subsubsection{getAllValidatedForPoiSeats}
\paragraph{Parameter} Die Funktion besitzt keine Parameter.
\paragraph{Beschreibung} Die Funktion ruft alle Validierungsinformationen zu Sitzplatzanzahlen ab. Die Funktion nutzt folgende Quellen:
\begin{itemize}
	\item Tabelle mit Validierungsinformationen zu Sitzplatzanzahlen
\end{itemize}
Es findet bei dieser Funktion kein Abruf von Daten aus {\glqq COSP\grqq} statt. Die Antwort wird als strukturiertes Array an den Aufrufer zurückgegeben.
\subsubsection{insertValidateSeats}
\paragraph{Parameter} Die Funktion besitzt folgende Parameter:
\begin{table}[H]
	\begin{tabular}{|c|p{11cm}|}
		\hline
		\textbf{Parametername} & \textbf{Parameterbeschreibung} \\ \hline
		\$seat\_id & Identifikator einer Sitzplatzanzahl \\ \hline
		\$value    & Wert der Validierung \\ \hline
	\end{tabular}
\end{table}
\paragraph{Beschreibung} Die Funktion fügt einer Sitzplatzanzahl eine neue Validierung hinzu. Die Funktion hat Auswirkungen auf folgende Quellen
\begin{itemize}
	\item Tabelle mit Validierungsinformationen zu Sitzplatzanzahlen
\end{itemize}
Es findet bei dieser Funktion kein Abruf von Daten aus {\glqq COSP\grqq} statt. Es gibt einen Rückgabewert.
\subsubsection{deleteValidateSeats}
\paragraph{Parameter} Die Funktion besitzt folgende Parameter:
\begin{table}[H]
	\begin{tabular}{|c|p{11cm}|}
		\hline
		\textbf{Parametername} & \textbf{Parameterbeschreibung} \\ \hline
		\$seat\_id & Identifikator einer Sitzplatzanzahl \\ \hline
	\end{tabular}
\end{table}
\paragraph{Beschreibung} Die Funktion löscht alle Validierungen einer Sitzplatzanzahl. Die Funktion hat Auswirkungen auf folgende Quellen
\begin{itemize}
	\item Tabelle mit Validierungsinformationen zu Sitzplatzanzahlen
\end{itemize}
Es findet bei dieser Funktion kein Abruf von Daten aus {\glqq COSP\grqq} statt. Es gibt einen Rückgabewert.
\newpage
\section{validate-timespan}
\subsection{Allgemeines} Diese Datei enthält alle Funktionen, welche die Tabelle mit Validierungsinformationen zur Zeitspanne benutzen.
\begin{table}[H]
	\begin{tabular}{|c|p{11cm}|}
		\hline
		\textbf{Einbindungspunkt} & inc-db.php \\ \hline
		\textbf{Einbindungspunkt} & inc-db-sub.php \\ \hline
	\end{tabular}
\end{table}
Die Datei ist nicht direkt durch den Nutzer aufrufbar, dies wird durch folgenden Code-Ausschnitt sichergestellt:
\begin{lstlisting}[language=php]
	if (!defined('NICE_PROJECT')) {
		die('Permission denied.');
	}
\end{lstlisting}
Der Globale Wert {\glqq NICE\_PROJECT\grqq} wird durch für den Nutzer valide Aufrufpunkte festgelegt, z.B. {\glqq api.php\grqq}.
\newpage
\subsection{Funktionen}
\subsubsection{insertValidateTimespan}
\paragraph{Parameter} Die Funktion besitzt folgende Parameter:
\begin{table}[H]
	\begin{tabular}{|c|p{11cm}|}
		\hline
		\textbf{Parametername} & \textbf{Parameterbeschreibung} \\ \hline
		\$poi\_id & Identifikator des zugehörigen Interessenpunktes \\ \hline
		\$value   & Wert der Validierung \\ \hline
	\end{tabular}
\end{table}
\paragraph{Beschreibung} Die Funktion fügt einem Interessenpunkt eine neue Validierung für seine Zeitspanne hinzu. Die Funktion hat Auswirkungen auf folgende Quellen
\begin{itemize}
	\item Tabelle mit Validierungsinformationen zur Zeitspanne
\end{itemize}
Es findet bei dieser Funktion kein Abruf von Daten aus {\glqq COSP\grqq} statt. Es gibt einen Rückgabewert.
\subsubsection{deleteValidateTimeSpan}
\paragraph{Parameter} Die Funktion besitzt folgende Parameter:
\begin{table}[H]
	\begin{tabular}{|c|p{11cm}|}
		\hline
		\textbf{Parametername} & \textbf{Parameterbeschreibung} \\ \hline
		\$poi\_id & Identifikator des zugehörigen Interessenpunktes \\ \hline
	\end{tabular}
\end{table}
\paragraph{Beschreibung} Die Funktion löscht alle Validierungen der Zeitspanne eines Interessenpunktes. Die Funktion hat Auswirkungen auf folgende Quellen
\begin{itemize}
	\item Tabelle mit Validierungsinformationen zur Zeitspanne
\end{itemize}
Es findet bei dieser Funktion kein Abruf von Daten aus {\glqq COSP\grqq} statt. Es gibt einen Rückgabewert.
\subsubsection{getValidateSumTimespan}
\paragraph{Parameter} Die Funktion besitzt folgende Parameter:
\begin{table}[H]
	\begin{tabular}{|c|p{11cm}|}
		\hline
		\textbf{Parametername} & \textbf{Parameterbeschreibung} \\ \hline
		\$poi\_id & Identifikator des zugehörigen Interessenpunktes \\ \hline
	\end{tabular}
\end{table}
\paragraph{Beschreibung} Die Funktion ruft die Summe der Validierungen der Zeitspanne eines Interessenpunktes ab. Die Funktion nutzt folgende Quellen:
\begin{itemize}
	\item Tabelle mit Validierungsinformationen zur Zeitspanne
\end{itemize}
Es findet bei dieser Funktion kein Abruf von Daten aus {\glqq COSP\grqq} statt. Die Antwort wird als strukturiertes Array an den Aufrufer zurückgegeben.
\subsubsection{getAllValidatedForTimeSpan}
\paragraph{Parameter} Die Funktion besitzt keine Parameter.
\paragraph{Beschreibung} Die Funktion ruft alle Validierungsinformationen zur Zeitspanne aller Interessenpunkte ab. Die Funktion nutzt folgende Quellen:
\begin{itemize}
	\item Tabelle mit Validierungsinformationen zur Zeitspanne
\end{itemize}
Es findet bei dieser Funktion kein Abruf von Daten aus {\glqq COSP\grqq} statt. Die Antwort wird als strukturiertes Array an den Aufrufer zurückgegeben.
\newpage
\section{validate-type}
\subsection{Allgemeines} Diese Datei enthält alle Funktionen, welche die Tabelle mit Validierungsinformationen zum Typ benutzen.
\begin{table}[H]
	\begin{tabular}{|c|p{11cm}|}
		\hline
		\textbf{Einbindungspunkt} & inc-db.php \\ \hline
		\textbf{Einbindungspunkt} & inc-db-sub.php \\ \hline
	\end{tabular}
\end{table}
Die Datei ist nicht direkt durch den Nutzer aufrufbar, dies wird durch folgenden Code-Ausschnitt sichergestellt:
\begin{lstlisting}[language=php]
	if (!defined('NICE_PROJECT')) {
		die('Permission denied.');
	}
\end{lstlisting}
Der Globale Wert {\glqq NICE\_PROJECT\grqq} wird durch für den Nutzer valide Aufrufpunkte festgelegt, z.B. {\glqq api.php\grqq}.
\newpage
\subsection{Funktionen}
\subsubsection{insertValidateType}
\paragraph{Parameter} Die Funktion besitzt folgende Parameter:
\begin{table}[H]
	\begin{tabular}{|c|p{11cm}|}
		\hline
		\textbf{Parametername} & \textbf{Parameterbeschreibung} \\ \hline
		\$poi\_id & Identifikator des zugehörigen Interessenpunktes \\ \hline
		\$value   & Wert der Validierung \\ \hline
	\end{tabular}
\end{table}
\paragraph{Beschreibung} Die Funktion fügt einem Interessenpunkt eine neue Validierung für seinen Typ hinzu. Die Funktion hat Auswirkungen auf folgende Quellen
\begin{itemize}
	\item Tabelle mit Validierungsinformationen zum Typ
\end{itemize}
Es findet bei dieser Funktion kein Abruf von Daten aus {\glqq COSP\grqq} statt. Es gibt einen Rückgabewert.
\subsubsection{getValidateSumType}
\paragraph{Parameter} Die Funktion besitzt folgende Parameter:
\begin{table}[H]
	\begin{tabular}{|c|p{11cm}|}
		\hline
		\textbf{Parametername} & \textbf{Parameterbeschreibung} \\ \hline
		\$poi\_id & Identifikator des zugehörigen Interessenpunktes \\ \hline
	\end{tabular}
\end{table}
\paragraph{Beschreibung} Die Funktion ruft die Summe der Validierungen des Typs eines Interessenpunktes ab. Die Funktion nutzt folgende Quellen:
\begin{itemize}
	\item Tabelle mit Validierungsinformationen zum Typ
\end{itemize}
Es findet bei dieser Funktion kein Abruf von Daten aus {\glqq COSP\grqq} statt. Die Antwort wird als strukturiertes Array an den Aufrufer zurückgegeben.
\subsubsection{getAllValidatedForCinemaTypes}
\paragraph{Parameter} Die Funktion besitzt keine Parameter.
\paragraph{Beschreibung} Die Funktion ruft alle Validierungsinformationen zu Typen aller Interessenpunkte ab. Die Funktion nutzt folgende Quellen:
\begin{itemize}
	\item Tabelle mit Validierungsinformationen zum Typ
\end{itemize}
Es findet bei dieser Funktion kein Abruf von Daten aus {\glqq COSP\grqq} statt. Die Antwort wird als strukturiertes Array an den Aufrufer zurückgegeben.
\subsubsection{deleteValidateType}
\paragraph{Parameter} Die Funktion besitzt folgende Parameter:
\begin{table}[H]
	\begin{tabular}{|c|p{11cm}|}
		\hline
		\textbf{Parametername} & \textbf{Parameterbeschreibung} \\ \hline
		\$poi\_id & Identifikator des zugehörigen Interessenpunktes \\ \hline
	\end{tabular}
\end{table}
\paragraph{Beschreibung} Die Funktion löscht alle Validierungen des Typs eines Interessenpunktes. Die Funktion hat Auswirkungen auf folgende Quellen
\begin{itemize}
	\item Tabelle mit Validierungsinformationen zum Typ
\end{itemize}
Es findet bei dieser Funktion kein Abruf von Daten aus {\glqq COSP\grqq} statt. Es gibt einen Rückgabewert.
\newpage
\section{de}
\label{lang:de}
\subsection{Allgemeines} Diese Datei enthält ein Array mit Werten um eine Mehrsprachlichkeit in Teilen zu ermöglichen. Speziell beinhaltet diese Datei die Deutsche Version
\begin{table}[H]
	\begin{tabular}{|c|p{11cm}|}
		\hline
		\textbf{Einbindungspunkt} & inc.php \\ \hline
		\textbf{Einbindungspunkt} & inc-sub.php \\ \hline
	\end{tabular}
\end{table}
Die Datei ist nicht direkt durch den Nutzer aufrufbar, dies wird durch folgenden Code-Ausschnitt sichergestellt:
\begin{lstlisting}[language=php]
if (!defined('NICE_PROJECT')) {
	die('Permission denied.');
}
\end{lstlisting}
Der Globale Wert {\glqq NICE\_PROJECT\grqq} wird durch für den Nutzer valide Aufrufpunkte festgelegt, z.B. {\glqq api.php\grqq}.
\newpage
\subsection{Aufbau}
\subsubsection{Array}
Das Nachfolgend zu sehende Array beinhaltet alle verwendeten Schlüssel um eine Mehrsprachlichkeit zu ermöglichen. Viele der Texte sind jedoch nicht durch einen Schlüssel hinterlegt.
\begin{lstlisting}[language=php]
$lang = array(
	"Lehrer"=>"Lehrer",
	"NeuesBenutzerkonto"=>"Neues Benutzerkonto",
	"Erstellen"=>"Erstellen",
	"Klassenname"=>"Klassenname",
	"NeueKlasse"=>"Neue Klasse",
	"NeuesPasswort"=>"Neues Passwort",
	"Rolle"=>"Rolle",
	"Benutzername"=>"Benutzername",
	"Vorname"=>"Vorname",
	"Nachname"=>"Nachname",
	"Löschen"=>"Löschen",
	"Klassdel"=>"Klasse löschen",
	"CSV"=>"CSV-Datei hochladen",
	"Benutzerkontenverwalten"=>"Benutzerkonten verwalten",
	"Kommentare"=>"Persönlicher Bereich",
	"Speichern"=>"Speichern",
	"Spielstätte"=>"Spielstätte",
	"Koordinaten"=>"Koordinaten",
	"Auswahl"=>"Wähle ein Bild (*.png oder *.jpeg) aus.",
	"FotoVideo"=>"Foto oder Video",
	"Geschichte"=>"Hintergrundinformationen zum Kino",
	"Besonderheiten"=>"Besonderheiten",
	"Hist_Adresse"=>"Historische Adresse",
	"Akt_Adresse"=>"Aktuelle Adresse *",
	"Längengrad"=>"Längengrad",
	"Breitengrad"=>"Breitengrad",
	"Name"=>"Name (Pflichtfeld)",
	"Poi"=>"Interessenpunkt hinzufügen",
	"Abmelden"=>"Abmelden",
	"Suche"=>"Kino-Suche",
	"Klasse"=>"Klasse wählen",
	"Sprachen"=>"Sprachen",
	"Benutzerkonten"=>"Benutzerkonten",
	"infos" => "Persönlicher Bereich",
	"lang_en" => "Englisch",
	"lang_de" => "Deutsch",
	"Betreiber" => "Betreiber",
	"betrieb_von" => "Betrieb von",
	"betrieb_bis" => "Betrieb bis",
	"Verw" => "Kartenfunktionen",
	"eintragsverwaltung" => "Eintragsverwaltung"
);
\end{lstlisting}
\subsubsection{Besonderheiten}
Aktuell werden keine neuen Schlüssel hinzugefügt und die Ermöglichung einer Mehrsprachlichkeit nicht verfolgt.
\newpage
\section{deletions}
\subsection{Allgemeines} Diese Datei enthält alle Funktionen zum Löschen von Daten beziehungsweise um Daten als gelöscht zu markieren.
\begin{table}[H]
	\begin{tabular}{|c|p{11cm}|}
		\hline
		\textbf{Einbindungspunkt} & inc.php \\ \hline
		\textbf{Einbindungspunkt} & inc-sub.php \\ \hline
	\end{tabular}
\end{table}
Die Datei ist nicht direkt durch den Nutzer aufrufbar, dies wird durch folgenden Code-Ausschnitt sichergestellt:
\begin{lstlisting}[language=php]
if (!defined('NICE_PROJECT')) {
	die('Permission denied.');
}
\end{lstlisting}
Der Globale Wert {\glqq NICE\_PROJECT\grqq} wird durch für den Nutzer valide Aufrufpunkte festgelegt, z.B. {\glqq api.php\grqq}.
\newpage
\subsection{Funktionen}
\subsubsection{deleteSeatsDBWrap}
\paragraph{Parameter} Die Funktion besitzt folgende Parameter:
\begin{table}[H]
	\begin{tabular}{|c|p{11cm}|}
		\hline
		\textbf{Parametername} & \textbf{Parameterbeschreibung} \\ \hline
		\$seatID    & Identifikator einer Sitzplatzanzahl \\ \hline
		\$overwrite & ermöglicht direktes Löschen \\ \hline
	\end{tabular}
\end{table}
\paragraph{Beschreibung} Die Funktion löscht oder markiert eine Sitzalplatzanzahl als gelöscht.
\begin{itemize}
	\item Sitzplatzanzahl-Tabelle
	\item Tabelle mit Validierungsinformationen zu Sitzplatzanzahlen
\end{itemize}
Es findet bei dieser Funktion kein Abruf von Daten aus {\glqq COSP\grqq} statt. Die Antwort wird als strukturiertes Array an den Aufrufer zurückgegeben.
\subsubsection{deleteNamesDBWrap}
\paragraph{Parameter} Die Funktion besitzt folgende Parameter:
\begin{table}[H]
	\begin{tabular}{|c|p{11cm}|}
		\hline
		\textbf{Parametername} & \textbf{Parameterbeschreibung} \\ \hline
		\$nameID    & Identifikator eines Namens \\ \hline
		\$overwrite & ermöglicht direktes Löschen \\ \hline
	\end{tabular}
\end{table}
\paragraph{Beschreibung} Die Funktion löscht oder markiert einen Namen als gelöscht.
\begin{itemize}
	\item Namen-Tabelle
	\item Tabelle mit Validierungsinformationen zu Namen
\end{itemize}
Es findet bei dieser Funktion kein Abruf von Daten aus {\glqq COSP\grqq} statt. Die Antwort wird als strukturiertes Array an den Aufrufer zurückgegeben.
\subsubsection{deleteCinemasDBWrap}
\paragraph{Parameter} Die Funktion besitzt folgende Parameter:
\begin{table}[H]
	\begin{tabular}{|c|p{11cm}|}
		\hline
		\textbf{Parametername} & \textbf{Parameterbeschreibung} \\ \hline
		\$cinemaID  & Identifikator einer Saalanzahl \\ \hline
		\$overwrite & ermöglicht direktes Löschen \\ \hline
	\end{tabular}
\end{table}
\paragraph{Beschreibung} Die Funktion löscht oder markiert eine Saalanzahl als gelöscht.
\begin{itemize}
	\item Saalanzahl-Tabelle
	\item Tabelle mit Validierungsinformationen zu Saalanzahlen
\end{itemize}
Es findet bei dieser Funktion kein Abruf von Daten aus {\glqq COSP\grqq} statt. Die Antwort wird als strukturiertes Array an den Aufrufer zurückgegeben.
\subsubsection{deleteOperatorsDBWrap}
\paragraph{Parameter} Die Funktion besitzt folgende Parameter:
\begin{table}[H]
	\begin{tabular}{|c|p{11cm}|}
		\hline
		\textbf{Parametername} & \textbf{Parameterbeschreibung} \\ \hline
		\$operatorID & Identifikator eines Betreibers \\ \hline
		\$overwrite  & ermöglicht direktes Löschen \\ \hline
	\end{tabular}
\end{table}
\paragraph{Beschreibung} Die Funktion löscht oder markiert einen Betreiber als gelöscht.
\begin{itemize}
	\item Betreiber-Tabelle
	\item Tabelle mit Validierungsinformationen zu Betreibern
\end{itemize}
Es findet bei dieser Funktion kein Abruf von Daten aus {\glqq COSP\grqq} statt. Die Antwort wird als strukturiertes Array an den Aufrufer zurückgegeben.
\subsubsection{deleteHistAddressDBWrap}
\paragraph{Parameter} Die Funktion besitzt folgende Parameter:
\begin{table}[H]
	\begin{tabular}{|c|p{11cm}|}
		\hline
		\textbf{Parametername} & \textbf{Parameterbeschreibung} \\ \hline
		\$histAddressID & Identifikator einer historischen Adresse \\ \hline
		\$overwrite     & ermöglicht direktes Löschen \\ \hline
	\end{tabular}
\end{table}
\paragraph{Beschreibung} Die Funktion löscht oder markiert eine historische Adresse als gelöscht.
\begin{itemize}
	\item Tabelle mit historischen Adressen
	\item Tabelle mit Validierungsinformationen zu historischen Adressen
\end{itemize}
Es findet bei dieser Funktion kein Abruf von Daten aus {\glqq COSP\grqq} statt. Die Antwort wird als strukturiertes Array an den Aufrufer zurückgegeben.
\subsubsection{deleteCommentsDBWrap}
\paragraph{Parameter} Die Funktion besitzt folgende Parameter:
\begin{table}[H]
	\begin{tabular}{|c|p{11cm}|}
		\hline
		\textbf{Parametername} & \textbf{Parameterbeschreibung} \\ \hline
		\$commentID & Identifikator eines Kommentars \\ \hline
		\$overwrite & ermöglicht direktes Löschen \\ \hline
	\end{tabular}
\end{table}
\paragraph{Beschreibung} Die Funktion löscht oder markiert einen Kommentar als gelöscht.
\begin{itemize}
	\item Kommentar-Tabelle
\end{itemize}
Es findet bei dieser Funktion kein Abruf von Daten aus {\glqq COSP\grqq} statt. Die Antwort wird als strukturiertes Array an den Aufrufer zurückgegeben.
\subsubsection{deleteCommentsByPoiidDBWrap}
\paragraph{Parameter} Die Funktion besitzt folgende Parameter:
\begin{table}[H]
	\begin{tabular}{|c|p{11cm}|}
		\hline
		\textbf{Parametername} & \textbf{Parameterbeschreibung} \\ \hline
		\$POIID     & Identifikator eines Interessenpunktes \\ \hline
		\$overwrite & ermöglicht direktes Löschen \\ \hline
	\end{tabular}
\end{table}
\paragraph{Beschreibung} Die Funktion löscht oder markiert alle Kommentare eines Interessenpunktes als gelöscht.
\begin{itemize}
	\item Kommentar-Tabelle
\end{itemize}
Es findet bei dieser Funktion kein Abruf von Daten aus {\glqq COSP\grqq} statt. Die Antwort wird als strukturiertes Array an den Aufrufer zurückgegeben.
\subsubsection{deletePoiDBWrap}
\paragraph{Parameter} Die Funktion besitzt folgende Parameter:
\begin{table}[H]
	\begin{tabular}{|c|p{11cm}|}
		\hline
		\textbf{Parametername} & \textbf{Parameterbeschreibung} \\ \hline
		\$POIID     & Identifikator eines Interessenpunktes \\ \hline
		\$overwrite & ermöglicht direktes Löschen \\ \hline
	\end{tabular}
\end{table}
\paragraph{Beschreibung} Die Funktion löscht oder markiert einen Betreiber als gelöscht.
\begin{itemize}
	\item Interessenpunkt-Tabelle
	\item Kommentar-Tabelle
	\item Saalanzahl-Tabelle
	\item Sitzplatzanzahl-Tabelle
	\item Namen-Tabelle
	\item Betreiber-Tabelle
	\item Tabelle mit historischen Adressen
	\item Tabelle mit Links zwischen Interessenpunkten und Bildern
	\item Tabelle mit Links zwischen Interessenpunkten und Geschichten
	\item Tabelle mit Validierungsinformationen zu Interessenpunkten
	\item Tabelle mit Validierungsinformationen zu Saalanzahlen
	\item Tabelle mit Validierungsinformationen zu Sitzplatzanzahlen
	\item Tabelle mit Validierungsinformationen zu Namen
	\item Tabelle mit Validierungsinformationen zu historischen Adressen
	\item Tabelle mit Validierungsinformationen zu Links zwischen Interessenpunkten und Bildern
	\item Tabelle mit Validierungsinformationen zu Links zwischen Interessenpunkten und Geschichten
\end{itemize}
Es findet bei dieser Funktion kein Abruf von Daten aus {\glqq COSP\grqq} statt. Die Antwort wird als strukturiertes Array an den Aufrufer zurückgegeben.
\subsubsection{deletePoiPicLinkByIDDBWrap}
\paragraph{Parameter} Die Funktion besitzt folgende Parameter:
\begin{table}[H]
	\begin{tabular}{|c|p{11cm}|}
		\hline
		\textbf{Parametername} & \textbf{Parameterbeschreibung} \\ \hline
		\$LinkID    & Identifikator eines Links zwischen einem Interessenpunkt und einem Bild \\ \hline
		\$overwrite & ermöglicht direktes Löschen \\ \hline
	\end{tabular}
\end{table}
\paragraph{Beschreibung} Die Funktion löscht oder markiert einen Link zwischen einem Interessenpunkt und einem Bild als gelöscht.
\begin{itemize}
	\item Tabelle mit Links zwischen Interessenpunkten und Bildern
	\item Tabelle mit Validierungsinformationen zu Links zwischen Interessenpunkten und Bildern
\end{itemize}
Es findet bei dieser Funktion kein Abruf von Daten aus {\glqq COSP\grqq} statt. Die Antwort wird als strukturiertes Array an den Aufrufer zurückgegeben.
\subsubsection{deletePoiStoryLinkByIDDBWrap}
\paragraph{Parameter} Die Funktion besitzt folgende Parameter:
\begin{table}[H]
	\begin{tabular}{|c|p{11cm}|}
		\hline
		\textbf{Parametername} & \textbf{Parameterbeschreibung} \\ \hline
		\$LinkID    & Identifikator eines Links zwischen einem Interessenpunkt und einem Geschichte \\ \hline
		\$overwrite & ermöglicht direktes Löschen \\ \hline
	\end{tabular}
\end{table}
\paragraph{Beschreibung} Die Funktion löscht oder markiert einen Link zwischen einem Interessenpunkt und einer Geschichte als gelöscht.
\begin{itemize}
	\item Tabelle mit Links zwischen Interessenpunkten und Geschichten
	\item Tabelle mit Validierungsinformationen zu Links zwischen Interessenpunkten und Geschichten
\end{itemize}
Es findet bei dieser Funktion kein Abruf von Daten aus {\glqq COSP\grqq} statt. Die Antwort wird als strukturiertes Array an den Aufrufer zurückgegeben.
\subsubsection{deletePoiMainPicDBWrap}
\paragraph{Parameter} Die Funktion besitzt folgende Parameter:
\begin{table}[H]
	\begin{tabular}{|c|p{11cm}|}
		\hline
		\textbf{Parametername} & \textbf{Parameterbeschreibung} \\ \hline
		\$POIID     & Identifikator eines Interessenpunktes \\ \hline
		\$overwrite & ermöglicht direktes Löschen \\ \hline
	\end{tabular}
\end{table}
\paragraph{Beschreibung} Die Funktion löscht oder markiert das Hauptbild eines Interessenpunktes als gelöscht.
\begin{itemize}
	\item Interessenpunkt-Tabelle
	\item Tabelle mit Validierungsinformationen zu Interessenpunkten
\end{itemize}
Es findet bei dieser Funktion kein Abruf von Daten aus {\glqq COSP\grqq} statt. Die Antwort wird als strukturiertes Array an den Aufrufer zurückgegeben.
\subsubsection{deleteSourceDBWrap}
\paragraph{Parameter} Die Funktion besitzt folgende Parameter:
\begin{table}[H]
	\begin{tabular}{|c|p{11cm}|}
		\hline
		\textbf{Parametername} & \textbf{Parameterbeschreibung} \\ \hline
		\$sid       & Identifikator einer Quelle \\ \hline
		\$overwrite & ermöglicht direktes Löschen \\ \hline
	\end{tabular}
\end{table}
\paragraph{Beschreibung} Die Funktion löscht oder markiert eine Quelle als gelöscht.
\begin{itemize}
	\item Interessenpunkt-Tabelle
	\item Tabelle mit Validierungsinformationen zu Interessenpunkten
\end{itemize}
Es findet bei dieser Funktion kein Abruf von Daten aus {\glqq COSP\grqq} statt. Die Antwort wird als strukturiertes Array an den Aufrufer zurückgegeben.
\newpage
\section{en}
\label{lang:en}
\subsection{Allgemeines} Diese Datei enthält ein Array mit Werten um eine Mehrsprachlichkeit in Teilen zu ermöglichen. Speziell beinhaltet diese Datei die Deutsche Version
\begin{table}[H]
	\begin{tabular}{|c|p{11cm}|}
		\hline
		\textbf{Einbindungspunkt} & keiner \\ \hline
	\end{tabular}
\end{table}
Die Datei ist nicht direkt durch den Nutzer aufrufbar, dies wird durch folgenden Code-Ausschnitt sichergestellt:
\begin{lstlisting}[language=php]
if (!defined('NICE_PROJECT')) {
	die('Permission denied.');
}
\end{lstlisting}
Der Globale Wert {\glqq NICE\_PROJECT\grqq} wird durch für den Nutzer valide Aufrufpunkte festgelegt, z.B. {\glqq api.php\grqq}.
\newpage
\subsection{Aufbau}
\subsubsection{Array}
Das Nachfolgend zu sehende Array beinhaltet alle verwendeten Schlüssel um eine Mehrsprachlichkeit zu ermöglichen. Viele der Texte sind jedoch nicht durch einen Schlüssel hinterlegt. Diese Datei verwendet nicht alle Schlüssel und ist somit momentan nicht nutzbar.
\begin{lstlisting}[language=php]
$lang = array(
	"Lehrer"=>"Teacher",
	"NeuesBenutzerkonto"=>"New User",
	"Erstellen"=>"Create",
	"Klassenname"=>"Class Name",
	"NeueKlasse"=>"New Class",
	"NeuesPasswort"=>"New Password",
	"Rolle"=>"Role",
	"Benutzername"=>"User Name",
	"Vorname"=>"Forename",
	"Nachname"=>"Surname",
	"Löschen"=>"Delete",
	"Klassdel"=>"Delete Class",
	"CSV"=>"CSV Upload",
	"NeuesBenutzerkonto"=>"New User",
	"Benutzerkontenverwalten"=>"Manage User Accounts",
	"Kommentare"=>"My Comments",
	"Speichern"=>"Save",
	"Spielstätte"=>"Cinema",
	"Koordinaten"=>"Coordinates",
	"Auswahl"=>"Choose an image (* .png or * .jpeg).",
	"FotoVideo"=>"Picture or Video",
	"Biographie"=>"Biography",
	"Besonderheiten"=>"Particularity",
	"Epoche"=>"Epoch",
	"Adresse"=>"Adress",
	"Längengrad"=>"Longitude",
	"Breitengrad"=>"Latitude",
	"Todesdatum"=>"Date of death",
	"Name"=>"Name",
	"Geburtsdatum"=>"Date of birth",
	"Poi"=>"Add POI",
	"Abmelden"=>"Logout",
	"Suche"=>"Search",
	"Klasse"=>"Class",
	"Sprachen"=>"Language",
	"Benutzerkonten"=>"User accounts",
	"infos" => "Personal space",
	"lang_en" => "English",
	"lang_de" => "German"
);
\end{lstlisting}
\subsubsection{Besonderheiten}
Aktuell werden keine neuen Schlüssel hinzugefügt und die Ermöglichung einer Mehrsprachlichkeit nicht verfolgt.
\newpage
\section{functionLib}
\subsection{Allgemeines} Diese Datei enthält alle Funktionen, welche an multiplen Stellen verwendet werden und daher nicht klar zugeordnet werden können.
\begin{table}[H]
	\begin{tabular}{|c|p{11cm}|}
		\hline
		\textbf{Einbindungspunkt} & inc.php \\ \hline
		\textbf{Einbindungspunkt} & inc-sub.php \\ \hline
	\end{tabular}
\end{table}
Die Datei ist nicht direkt durch den Nutzer aufrufbar, dies wird durch folgenden Code-Ausschnitt sichergestellt:
\begin{lstlisting}[language=php]
if (!defined('NICE_PROJECT')) {
	die('Permission denied.');
}
\end{lstlisting}
Der Globale Wert {\glqq NICE\_PROJECT\grqq} wird durch für den Nutzer valide Aufrufpunkte festgelegt, z.B. {\glqq api.php\grqq}.
\newpage
\subsection{Funktionen}
\subsubsection{generateHeader}
\paragraph{Parameter} Die Funktion besitzt folgende Parameter:
\begin{table}[H]
	\begin{tabular}{|c|p{11cm}|}
		\hline
		\textbf{Parametername} & \textbf{Parameterbeschreibung} \\ \hline
		\$Login     & Gibt an, ob Nutzer eingeloggt ist. \\ \hline
		\$lg        & Array für Mehrsprachigkeit (siehe \autoref{lang:de} oder \autoref{lang:en}) \\ \hline
		\$map       & Gibt an, ob einbindende Seite Karte ist \\ \hline
		\$loginpage & Gibt an, ob einbindende Seite Login- beziehungsweise Logout-Seite ist \\ \hline
	\end{tabular}
\end{table}
\paragraph{Beschreibung} Die Funktion erzeugt die Navbar sowie mehrfach verwendete Modals. Die Funktion nutzt folgende Quellen:
\begin{itemize}
	\item COSP
\end{itemize}
Es findet bei dieser Funktion ein Abruf von Daten aus {\glqq COSP\grqq} statt. Die Antwort wird direkt Ausgegeben.
\subsubsection{dump}
\paragraph{Parameter} Die Funktion besitzt folgende Parameter:
\begin{table}[H]
	\begin{tabular}{|c|p{11cm}|}
		\hline
		\textbf{Parametername} & \textbf{Parameterbeschreibung} \\ \hline
		\$data  & Daten für var\_dump \\ \hline
		\$level & Optionale Angabe des Debuglevels, wird mit Angabe aus Konfiguration verglichen , siehe \autoref{config:debug-level} \\ \hline
	\end{tabular}
\end{table}
\paragraph{Beschreibung} Die Funktion dient der Ausgabe von Ergebnissen von Funktionen zu Entwicklungszwecken. Die Funktion nutzt folgende Quellen:
\begin{itemize}
	\item Konfigurationsdatei
\end{itemize}
Es findet bei dieser Funktion kein Abruf von Daten aus {\glqq COSP\grqq} statt. Die Antwort wird direkt Ausgegeben.
\subsubsection{Redirect}
\paragraph{Parameter} Die Funktion besitzt folgende Parameter:
\begin{table}[H]
	\begin{tabular}{|c|p{11cm}|}
		\hline
		\textbf{Parametername} & \textbf{Parameterbeschreibung} \\ \hline
		\$url       & Leitet die Anfrage auf eine andere Seite weiter \\ \hline
		\$permanent & Legt fest, ob Weiterleitung permanent ist. \\ \hline
	\end{tabular}
\end{table}
\paragraph{Beschreibung} Die Funktion leitet den Aufrufer auf eine andere Seite weiter und beendet die Ausführung des aktuellen PHP-Scriptes. Es findet bei dieser Funktion kein Abruf von Daten aus {\glqq COSP\grqq} statt. Die Antwort wird direkt Ausgegeben.
\subsubsection{permissionDenied}
\paragraph{Parameter} Die Funktion besitzt folgende Parameter:
\begin{table}[H]
	\begin{tabular}{|c|p{11cm}|}
		\hline
		\textbf{Parametername} & \textbf{Parameterbeschreibung} \\ \hline
		\$string & Optionale Angab des Grundes \\ \hline
	\end{tabular}
\end{table}
\paragraph{Beschreibung} Die Funktion verhindert ein weiteres Ausführen des PHP-Scriptes und gibt einen entsprechenden HTTP-Statuscode zurück. Es findet bei dieser Funktion kein Abruf von Daten aus {\glqq COSP\grqq} statt. Die Antwort wird direkt Ausgegeben.
\subsubsection{generateHeaderTags}
\paragraph{Parameter} Die Funktion besitzt folgende Parameter:
\begin{table}[H]
	\begin{tabular}{|c|p{11cm}|}
		\hline
		\textbf{Parametername} & \textbf{Parameterbeschreibung} \\ \hline
		\$additional & Optional ein zu bindende Daten \\ \hline
	\end{tabular}
\end{table}
\subparagraph{\$additional}Das Array enthält Einträge mit folgenden Elementen:
\begin{table}[H]
	\begin{tabular}{|c|p{11cm}|}
		\hline
		\textbf{Parametername} & \textbf{Parameterbeschreibung} \\ \hline
		type    & Typ der Datei ({\glqq link\grqq} für zum Beispiel CSS-Files oder {\glqq script\grqq} für zum Beispiel javaScript-Files ) \\ \hline
		rel     & Gibt den Rel-Tag eines HTML-Link Elements an \\ \hline
		href    & Gibt die Position der Datei an \\ \hline
		hrefmin & Minimierte Version der Datei \\ \hline
		typeval & Gibt den Typ der Datei an (zum Beispiel: {\glqq text/javascript\grqq}) \\ \hline
	\end{tabular}
\end{table}
\paragraph{Beschreibung} Die Funktion dient der Generierung von HTML-Head Elementen und der zentralen Pflege der eingebundenen Dateien. Es findet bei dieser Funktion kein Abruf von Daten aus {\glqq COSP\grqq} statt. Die Antwort wird direkt Ausgegeben.
\paragraph{Besonderheiten} Die Einbindung zusätzlicher Dateien erfolgt in der im Array angegebenen Reihenfolge mittels:
\begin{lstlisting}[language=php]
foreach ($additional as $line) {
	switch ($line['type']) {
		case 'link':
			if (isset($line['typeval']) === false || $line['typeval'] === "") {
				echo '<link rel="' . $line['rel'] . '" href="' . $line['href'] . '" >';
			} else {
				echo '<link rel="' . $line['rel'] . '" type="' . $line['typeval'] . '" href="' . $line['href'] . '" >';
			}
			break;
		case 'script':
			echo '<script type="' . $line['typeval'] . '" src="' . $line['href'] . '" ></script>';
			break;
	}
}
\end{lstlisting}
\subsubsection{decode\_json}
\paragraph{Parameter} Die Funktion besitzt folgende Parameter:
\begin{table}[H]
	\begin{tabular}{|c|p{11cm}|}
		\hline
		\textbf{Parametername} & \textbf{Parameterbeschreibung} \\ \hline
		\$string & Eingabe des JSON als Zeichenkette \\ \hline
	\end{tabular}
\end{table}
\paragraph{Beschreibung} Die Funktion decodiert Daten im JSON-Format in PHP-Arrays. Es findet bei dieser Funktion kein Abruf von Daten aus {\glqq COSP\grqq} statt. Die Antwort wird als strukturiertes Array an den Aufrufer zurückgegeben.
\subsubsection{checkPoiExists}
\paragraph{Parameter} Die Funktion besitzt folgende Parameter:
\begin{table}[H]
	\begin{tabular}{|c|p{11cm}|}
		\hline
		\textbf{Parametername} & \textbf{Parameterbeschreibung} \\ \hline
		\$POIID & Identifikator eines Interessenpunktes \\ \hline
	\end{tabular}
\end{table}
\paragraph{Beschreibung} Die Funktion prüft die Existenz eines Interessenpunktes. Die Funktion nutzt folgende Quellen:
\begin{itemize}
	\item Interessenpunkt-Tabelle
\end{itemize}
Es findet bei dieser Funktion kein Abruf von Daten aus {\glqq COSP\grqq} statt. Die Antwort wird als Boolean an den Aufrufer zurückgegeben.
\subsubsection{remoteLogin}
\paragraph{Parameter} Die Funktion besitzt folgende Parameter:
\begin{table}[H]
	\begin{tabular}{|c|p{11cm}|}
		\hline
		\textbf{Parametername} & \textbf{Parameterbeschreibung} \\ \hline
		\$name & Nutzername, dessen Daten abgefragt werden sollen \\ \hline
	\end{tabular}
\end{table}
\paragraph{Beschreibung} Die Funktion fragt die vollständigen Nutzerdaten aus {\glqq COSP\grqq} ab. Die Funktion nutzt folgende Quellen:
\begin{itemize}
	\item COSP
\end{itemize}
Es findet bei dieser Funktion ein Abruf von Daten aus {\glqq COSP\grqq} statt. Die Antwort wird als strukturiertes Array an den Aufrufer zurückgegeben.
\subsubsection{remoteRole}
\paragraph{Parameter} Die Funktion besitzt folgende Parameter:
\begin{table}[H]
	\begin{tabular}{|c|p{11cm}|}
		\hline
		\textbf{Parametername} & \textbf{Parameterbeschreibung} \\ \hline
		\$name              & Nutzername, dessen Rolle abgefragt werden sollen \\ \hline
		\$ignoreDeaktivatet & Ignoriert Freischaltungsstatus des Nutzers \\ \hline
	\end{tabular}
\end{table}
\paragraph{Beschreibung} Die Funktion fragt die Rangdaten eines Nutzers aus {\glqq COSP\grqq} ab. Die Funktion nutzt folgende Quellen:
\begin{itemize}
	\item COSP
\end{itemize}
Es findet bei dieser Funktion ein Abruf von Daten aus {\glqq COSP\grqq} statt. Die Antwort wird als strukturiertes Array an den Aufrufer zurückgegeben.
\subsubsection{addRemoteUser}
\paragraph{Parameter} Die Funktion besitzt folgende Parameter:
\begin{table}[H]
	\begin{tabular}{|c|p{11cm}|}
		\hline
		\textbf{Parametername} & \textbf{Parameterbeschreibung} \\ \hline
		\$Username  & Nutzername des neuen Nutzers\\ \hline
		\$EMail     & E-Mailadresse des neuen Nutzers \\ \hline
		\$pwd       & Passwort als Hash \\ \hline
		\$firstname & Vorname des neuen Nutzers \\ \hline
		\$lastname  & Nachname des neuen Nutzers \\ \hline
	\end{tabular}
\end{table}
\paragraph{Beschreibung} Die Funktion fügt einen neuen Nutzer zu {\glqq COSP\grqq} hinzu. Die Funktion hat Auswirkungen auf folgende Quellen:
\begin{itemize}
	\item COSP
\end{itemize}
Es findet bei dieser Funktion kein Abruf von Daten aus {\glqq COSP\grqq} statt. Es werden jedoch Daten an {\glqq COSP\grqq} gesendet. Die Antwort wird als strukturiertes Array an den Aufrufer zurückgegeben.
\subsubsection{getRemoteAllUsernames}
\paragraph{Parameter} Die Funktion besitzt keine Parameter.
\paragraph{Beschreibung} Die Funktion fragt alle Nutzernamen aus {\glqq COSP\grqq} ab. Die Funktion nutzt folgende Quellen:
\begin{itemize}
	\item 
\end{itemize}
Es findet bei dieser Funktion ein Abruf von Daten aus {\glqq COSP\grqq} statt. Die Antwort wird als strukturiertes Array an den Aufrufer zurückgegeben.
\subsubsection{UploadPicture}
\paragraph{Parameter} Die Funktion besitzt folgende Parameter:
\begin{table}[H]
	\begin{tabular}{|c|p{11cm}|}
		\hline
		\textbf{Parametername} & \textbf{Parameterbeschreibung} \\ \hline
		\$title      & Titel des Bildes \\ \hline
		\$desc       & Beschreibung des Bildes \\ \hline
		\$filepath   & Lokaler Pfad zum Bild \\ \hline
		\$ftype      & Gibt den Dateityp des Bildes an \\ \hline
		\$username   & Nutzername des hochladenden Nutzers \\ \hline
		\$source     & Quellenangabe (optional) \\ \hline
		\$sourceType & Identifikator des Typs der Quelle (optional) \\ \hline
	\end{tabular}
\end{table}
\paragraph{Beschreibung} Die Funktion lädt ein Bild auf {\glqq COSP\grqq} hoch. Die Funktion hat Auswirkungen auf folgende Quellen:
\begin{itemize}
	\item COSP
\end{itemize}
Es findet bei dieser Funktion kein Abruf von Daten aus {\glqq COSP\grqq} statt. Es werden jedoch Daten an {\glqq COSP\grqq} gesendet. Die Antwort wird als strukturiertes Array an den Aufrufer zurückgegeben.
\subsubsection{getRemoteSeccode}
\paragraph{Parameter} Die Funktion besitzt folgende Parameter:
\begin{table}[H]
	\begin{tabular}{|c|p{11cm}|}
		\hline
		\textbf{Parametername} & \textbf{Parameterbeschreibung} \\ \hline
		\$pictureToken & alphanumerischer Identifikator eines Bildes \\ \hline
	\end{tabular}
\end{table}
\paragraph{Beschreibung} Die Funktion fragt Daten zum Laden eines bestimmten Bildes bei {\glqq COSP\grqq} an. Die Funktion nutzt folgende Quellen:
\begin{itemize}
	\item COSP
\end{itemize}
Es findet bei dieser Funktion ein Abruf von Daten aus {\glqq COSP\grqq} statt. Die Antwort wird als strukturiertes Array an den Aufrufer zurückgegeben.
\subsubsection{getRemotePictureList}
\paragraph{Parameter} Die Funktion besitzt keine Parameter.
\paragraph{Beschreibung} Die Funktion fragt Daten zum Laden aller Bilder bei {\glqq COSP\grqq} an. Die Funktion nutzt folgende Quellen:
\begin{itemize}
	\item COSP
\end{itemize}
Es findet bei dieser Funktion ein Abruf von Daten aus {\glqq COSP\grqq} statt. Die Antwort wird als strukturiertes Array an den Aufrufer zurückgegeben.
\subsubsection{ApiCall}
\paragraph{Parameter} Die Funktion besitzt folgende Parameter:
\begin{table}[H]
	\begin{tabular}{|c|p{11cm}|}
		\hline
		\textbf{Parametername} & \textbf{Parameterbeschreibung} \\ \hline
		\$params       & Array mit Parametern des API-Aufrufs \\ \hline
		\$type         & Typ des API-Aufrufs \\ \hline
		\$file\_upload & Gibt an, ob eine Datei hochgeladen werden soll \\ \hline
	\end{tabular}
\end{table}
\subparagraph{\$params}Das Array besteht aus Key-Value-Paaren.
\paragraph{Beschreibung} Die Funktion führt einen API-Aufruf an {\glqq COSP\grqq} aus. Die Funktion nutzt folgende Quellen:
\begin{itemize}
	\item Konfigurationsdatei
\end{itemize}
Es findet bei dieser Funktion ein Abruf von Daten aus {\glqq COSP\grqq} statt. Es können auch Daten an {\glqq COSP\grqq} gesendet werden. Die Antwort wird als strukturiertes Array an den Aufrufer zurückgegeben.
\subsubsection{checkMailAddress}
\paragraph{Parameter} Die Funktion besitzt folgende Parameter:
\begin{table}[H]
	\begin{tabular}{|c|p{11cm}|}
		\hline
		\textbf{Parametername} & \textbf{Parameterbeschreibung} \\ \hline
		\$email & zu prüfende E-Mailadresse \\ \hline
	\end{tabular}
\end{table}
\paragraph{Beschreibung} Die Funktion prüft, ob die angegebene Zeichenkette eine E-Mailadresse darstellt. Es findet bei dieser Funktion kein Abruf von Daten aus {\glqq COSP\grqq} statt. Die Antwort wird als strukturiertes Array an den Aufrufer zurückgegeben.
\subsubsection{inspectPassword}
\paragraph{Parameter} Die Funktion besitzt folgende Parameter:
\begin{table}[H]
	\begin{tabular}{|c|p{11cm}|}
		\hline
		\textbf{Parametername} & \textbf{Parameterbeschreibung} \\ \hline
		\$PasswordField1Val & Inhalt des ersten Passwortfeldes \\ \hline
		\$PasswordField2Val & Inhalt des zweiten Passwortfeldes \\ \hline
	\end{tabular}
\end{table}
\paragraph{Beschreibung} Die Funktion prüft, ob die angegebenen Inhalt der Passworteingabe Felder übereinstimmen und den geforderten Kriterien entsprechen. Es findet bei dieser Funktion kein Abruf von Daten aus {\glqq COSP\grqq} statt. Die Antwort wird als strukturiertes Array an den Aufrufer zurückgegeben.\\
\subsubsection{checkPermission}
\paragraph{Parameter} Die Funktion besitzt folgende Parameter:
\begin{table}[H]
	\begin{tabular}{|c|p{11cm}|}
		\hline
		\textbf{Parametername} & \textbf{Parameterbeschreibung} \\ \hline
		\$requiredPermission & erforderlicher Freigabewert \\ \hline
	\end{tabular}
\end{table}
\paragraph{Beschreibung} Die Funktion prüft, ob ein Nutzer eine angegebene Berechtigungsstufe besitzt. Sollte der Nutzer nicht die geforderte Berechtigungsstufe besitzen so wird die Ausführung des PHP-Scriptes beendet. Es findet bei dieser Funktion kein Abruf von Daten aus {\glqq COSP\grqq} statt. Die Antwort wird als strukturiertes Array an den Aufrufer zurückgegeben.
\subsubsection{getRemoteRank}
\paragraph{Parameter} Die Funktion besitzt folgende Parameter:
\begin{table}[H]
	\begin{tabular}{|c|p{11cm}|}
		\hline
		\textbf{Parametername} & \textbf{Parameterbeschreibung} \\ \hline
		\$username & Nutzername, dessen Rang abgefragt werden soll \\ \hline
	\end{tabular}
\end{table}
\paragraph{Beschreibung} Die Funktion fragt den Rang eines Nutzers in {\glqq COSP\grqq} ab. Die Funktion nutzt folgende Quellen:
\begin{itemize}
	\item COSP
\end{itemize}
Es findet bei dieser Funktion ein Abruf von Daten aus {\glqq COSP\grqq} statt. Die Antwort wird als strukturiertes Array an den Aufrufer zurückgegeben.
\subsubsection{getRanktypes}
\paragraph{Parameter} Die Funktion besitzt keine Parameter.
\paragraph{Beschreibung} Die Funktion fragt alle verfügbaren Ränge in {\glqq COSP\grqq} ab. Die Funktion nutzt folgende Quellen:
\begin{itemize}
	\item COSP
\end{itemize}
Es findet bei dieser Funktion ein Abruf von Daten aus {\glqq COSP\grqq} statt. Die Antwort wird als strukturiertes Array an den Aufrufer zurückgegeben.
\subsubsection{getRanklist}
\paragraph{Parameter} Die Funktion besitzt keine Parameter.
\paragraph{Beschreibung} Die Funktion fragt eine Rangliste für dieses Modul in {\glqq COSP\grqq} ab. Die Funktion nutzt folgende Quellen:
\begin{itemize}
	\item COSP
\end{itemize}
Es findet bei dieser Funktion ein Abruf von Daten aus {\glqq COSP\grqq} statt. Die Antwort wird als strukturiertes Array an den Aufrufer zurückgegeben.
\subsubsection{addRankPoints}
\paragraph{Parameter} Die Funktion besitzt folgende Parameter:
\begin{table}[H]
	\begin{tabular}{|c|p{11cm}|}
		\hline
		\textbf{Parametername} & \textbf{Parameterbeschreibung} \\ \hline
		\$username & Nutzername zu welchem Punkte hinzugefügt werden sollen \\ \hline
		\$points   & hinzuzufügende Punkte \\ \hline
		\$reason   & Begründung der Punkte \\ \hline
	\end{tabular}
\end{table}
\paragraph{Beschreibung} Die Funktion fügt einem Nutzer Rangpunkte hinzu. Die Funktion hat Auswirkungen auf folgende Quellen:
\begin{itemize}
	\item COSP
\end{itemize}
Es findet bei dieser Funktion kein Abruf von Daten aus {\glqq COSP\grqq} statt. Es werden jedoch Daten an {\glqq COSP\grqq} gesendet. Die Antwort wird als strukturiertes Array an den Aufrufer zurückgegeben.
\subsubsection{getValidationValue}
\paragraph{Parameter} Die Funktion besitzt folgende Parameter:
\begin{table}[H]
	\begin{tabular}{|c|p{11cm}|}
		\hline
		\textbf{Parametername} & \textbf{Parameterbeschreibung} \\ \hline
		\$name & Nutzername zu welchem Rolle abgefragt werden soll \\ \hline
	\end{tabular}
\end{table}
\paragraph{Beschreibung} Die Funktion bestimmt den Validierungswert des aktuellen Nutzers. Die Funktion nutzt folgende Quellen:
\begin{itemize}
	\item COSP
	\item Konfigurationsdatei
\end{itemize}
Es findet bei dieser Funktion ein Abruf von Daten aus {\glqq COSP\grqq} statt. Die Antwort wird als strukturiertes Array an den Aufrufer zurückgegeben.
\subsubsection{validatePoi}
\paragraph{Parameter} Die Funktion besitzt folgende Parameter:
\begin{table}[H]
	\begin{tabular}{|c|p{11cm}|}
		\hline
		\textbf{Parametername} & \textbf{Parameterbeschreibung} \\ \hline
		\$poiid & Identifikator eines Interessenpunktes \\ \hline
	\end{tabular}
\end{table}
\paragraph{Beschreibung} Die Funktion fügt eine Validierung einem Interessenpunkt hinzu und vergibt Rangpunkte. Die Funktion hat Auswirkungen auf folgende Quellen:
\begin{itemize}
	\item Tabelle mit Validierungsdaten zu Interessenpunkten
	\item COSP
\end{itemize}
Es findet bei dieser Funktion kein Abruf von Daten aus {\glqq COSP\grqq} statt. Es werden jedoch Daten an {\glqq COSP\grqq} gesendet. Die Antwort wird als strukturiertes Array an den Aufrufer zurückgegeben.
\subsubsection{validateTimeSpanPoi}
\paragraph{Parameter} Die Funktion besitzt folgende Parameter:
\begin{table}[H]
	\begin{tabular}{|c|p{11cm}|}
		\hline
		\textbf{Parametername} & \textbf{Parameterbeschreibung} \\ \hline
		\$poiid & Identifikator eines Interessenpunktes \\ \hline
	\end{tabular}
\end{table}
\paragraph{Beschreibung} Die Funktion fügt eine Validierung zur Zeitspanne eines Interessenpunkt hinzu und vergibt Rangpunkte. Die Funktion hat Auswirkungen auf folgende Quellen:
\begin{itemize}
	\item Tabelle mit Validierungsdaten zur Zeitspanne von Interessenpunkten
	\item COSP
\end{itemize}
Es findet bei dieser Funktion kein Abruf von Daten aus {\glqq COSP\grqq} statt. Es werden jedoch Daten an {\glqq COSP\grqq} gesendet. Die Antwort wird als strukturiertes Array an den Aufrufer zurückgegeben.
\subsubsection{validateCurrentAddressPoi}
\paragraph{Parameter} Die Funktion besitzt folgende Parameter:
\begin{table}[H]
	\begin{tabular}{|c|p{11cm}|}
		\hline
		\textbf{Parametername} & \textbf{Parameterbeschreibung} \\ \hline
		\$poiid & Identifikator eines Interessenpunktes \\ \hline
	\end{tabular}
\end{table}
\paragraph{Beschreibung} Die Funktion fügt eine Validierung zur aktuellen Adresse eines Interessenpunkt hinzu und vergibt Rangpunkte. Die Funktion hat Auswirkungen auf folgende Quellen:
\begin{itemize}
	\item Tabelle mit Validierungsdaten zur aktuellen Adresse von Interessenpunkten
	\item COSP
\end{itemize}
Es findet bei dieser Funktion kein Abruf von Daten aus {\glqq COSP\grqq} statt. Es werden jedoch Daten an {\glqq COSP\grqq} gesendet. Die Antwort wird als strukturiertes Array an den Aufrufer zurückgegeben.
\subsubsection{validateHistoryPoi}
\paragraph{Parameter} Die Funktion besitzt folgende Parameter:
\begin{table}[H]
	\begin{tabular}{|c|p{11cm}|}
		\hline
		\textbf{Parametername} & \textbf{Parameterbeschreibung} \\ \hline
		\$poiid & Identifikator eines Interessenpunktes \\ \hline
	\end{tabular}
\end{table}
\paragraph{Beschreibung} Die Funktion fügt eine Validierung zur Geschichte eines Interessenpunkt hinzu und vergibt Rangpunkte. Die Funktion hat Auswirkungen auf folgende Quellen:
\begin{itemize}
	\item Tabelle mit Validierungsdaten zur Geschichte von Interessenpunkten
	\item COSP
\end{itemize}
Es findet bei dieser Funktion kein Abruf von Daten aus {\glqq COSP\grqq} statt. Es werden jedoch Daten an {\glqq COSP\grqq} gesendet. Die Antwort wird als strukturiertes Array an den Aufrufer zurückgegeben.
\subsubsection{validatePoiName}
\paragraph{Parameter} Die Funktion besitzt folgende Parameter:
\begin{table}[H]
	\begin{tabular}{|c|p{11cm}|}
		\hline
		\textbf{Parametername} & \textbf{Parameterbeschreibung} \\ \hline
		\$nameid & Identifikator eines Namens \\ \hline
	\end{tabular}
\end{table}
\paragraph{Beschreibung} Die Funktion fügt eine Validierung zu einem Namen hinzu und vergibt Rangpunkte. Die Funktion hat Auswirkungen auf folgende Quellen:
\begin{itemize}
	\item Tabelle mit Validierungsdaten zu Namen
	\item COSP
\end{itemize}
Es findet bei dieser Funktion kein Abruf von Daten aus {\glqq COSP\grqq} statt. Es werden jedoch Daten an {\glqq COSP\grqq} gesendet. Die Antwort wird als strukturiertes Array an den Aufrufer zurückgegeben.
\subsubsection{validatePoiOperator}
\paragraph{Parameter} Die Funktion besitzt folgende Parameter:
\begin{table}[H]
	\begin{tabular}{|c|p{11cm}|}
		\hline
		\textbf{Parametername} & \textbf{Parameterbeschreibung} \\ \hline
		\$opertorid & Identifikator eines Betreibers \\ \hline
	\end{tabular}
\end{table}
\paragraph{Beschreibung} Die Funktion fügt eine Validierung zu einem Betreiber hinzu und vergibt Rangpunkte. Die Funktion hat Auswirkungen auf folgende Quellen:
\begin{itemize}
	\item Tabelle mit Validierungsdaten zu Betreibern
	\item COSP
\end{itemize}
Es findet bei dieser Funktion kein Abruf von Daten aus {\glqq COSP\grqq} statt. Es werden jedoch Daten an {\glqq COSP\grqq} gesendet. Die Antwort wird als strukturiertes Array an den Aufrufer zurückgegeben.
\subsubsection{validatePoiHistAddress}
\paragraph{Parameter} Die Funktion besitzt folgende Parameter:
\begin{table}[H]
	\begin{tabular}{|c|p{11cm}|}
		\hline
		\textbf{Parametername} & \textbf{Parameterbeschreibung} \\ \hline
		\$histAddrId & Identifikator einer historischen Adresse \\ \hline
	\end{tabular}
\end{table}
\paragraph{Beschreibung} Die Funktion fügt eine Validierung zu einer historischen Adresse hinzu und vergibt Rangpunkte. Die Funktion hat Auswirkungen auf folgende Quellen:
\begin{itemize}
	\item Tabelle mit Validierungsdaten zu historischen Adressen
	\item COSP
\end{itemize}
Es findet bei dieser Funktion kein Abruf von Daten aus {\glqq COSP\grqq} statt. Es werden jedoch Daten an {\glqq COSP\grqq} gesendet. Die Antwort wird als strukturiertes Array an den Aufrufer zurückgegeben.
\subsubsection{validatePoiStory}
\paragraph{Parameter} Die Funktion besitzt folgende Parameter:
\begin{table}[H]
	\begin{tabular}{|c|p{11cm}|}
		\hline
		\textbf{Parametername} & \textbf{Parameterbeschreibung} \\ \hline
		\$story\_poi\_id & Identifikator eines Links zwischen einer Geschichte und einem Interessenpunkt \\ \hline
	\end{tabular}
\end{table}
\paragraph{Beschreibung} Die Funktion fügt eine Validierung zu einem Link zwischen einem Interessenpunkt und einer Geschichte hinzu und vergibt Rangpunkte. Die Funktion hat Auswirkungen auf folgende Quellen:
\begin{itemize}
	\item Tabelle mit Validierungsdaten zu Links zwischen Geschichten und Interessenpunkten
	\item COSP
\end{itemize}
Es findet bei dieser Funktion kein Abruf von Daten aus {\glqq COSP\grqq} statt. Es werden jedoch Daten an {\glqq COSP\grqq} gesendet. Die Antwort wird als strukturiertes Array an den Aufrufer zurückgegeben.
\subsubsection{getValidatedByPOI}
\paragraph{Parameter} Die Funktion besitzt keine Parameter.
\paragraph{Beschreibung} Die Funktion fragt alle Validierungswerte für alle Interessenpunkte ab. Die Funktion nutzt folgende Quellen:
\begin{itemize}
	\item Interessenpunkt-Tabelle
\end{itemize}
Es findet bei dieser Funktion kein Abruf von Daten aus {\glqq COSP\grqq} statt. Die Antwort wird als strukturiertes Array an den Aufrufer zurückgegeben.
\subsubsection{getPoisForUser}
\paragraph{Parameter} Die Funktion besitzt keine Parameter.
\paragraph{Beschreibung} Die Funktion fragt alle Interessenpunkte, welche für den Nutzer freigegeben sind, ab. Die Funktion nutzt folgende Quellen:
\begin{itemize}
	\item Interessenpunkt-Tabelle
	\item Tabelle mit Validierungsdaten zu Intressenpunkten
\end{itemize}
Es findet bei dieser Funktion kein Abruf von Daten aus {\glqq COSP\grqq} statt. Die Antwort wird als strukturiertes Array an den Aufrufer zurückgegeben.
\subsubsection{getPoisForUser}
\paragraph{Parameter} Die Funktion besitzt keine Parameter.
\paragraph{Beschreibung} Die Funktion fragt die Titel aller Interessenpunkte, welche für den Nutzer freigegeben sind, ab. Die Funktion nutzt folgende Quellen:
\begin{itemize}
	\item Interessenpunkt-Tabelle
	\item Tabelle mit Validierungsdaten zu Intressenpunkten
\end{itemize}
Es findet bei dieser Funktion kein Abruf von Daten aus {\glqq COSP\grqq} statt. Die Antwort wird als strukturiertes Array an den Aufrufer zurückgegeben.
\subsubsection{getValidationsByUserForPOI}
\paragraph{Parameter} Die Funktion besitzt folgende Parameter:
\begin{table}[H]
	\begin{tabular}{|c|p{11cm}|}
		\hline
		\textbf{Parametername} & \textbf{Parameterbeschreibung} \\ \hline
		\$uid & Angabe eines Nutzeridentifikators \\ \hline
	\end{tabular}
\end{table}
\paragraph{Beschreibung} Die Funktion liefert alle Identifikatoren von Interessenpunkten, welche bereits durch den Nutzer validiert wurden. Die Funktion nutzt folgende Quellen:
\begin{itemize}
	\item Tabelle mit Validierungsdaten zu Intressenpunkten
\end{itemize}
Es findet bei dieser Funktion kein Abruf von Daten aus {\glqq COSP\grqq} statt. Die Antwort wird als strukturiertes Array an den Aufrufer zurückgegeben.
\subsubsection{getValidationsByUserForTimeSpans}
\paragraph{Parameter} Die Funktion besitzt folgende Parameter:
\begin{table}[H]
	\begin{tabular}{|c|p{11cm}|}
		\hline
		\textbf{Parametername} & \textbf{Parameterbeschreibung} \\ \hline
		\$uid & Angabe eines Nutzeridentifikators \\ \hline
	\end{tabular}
\end{table}
\paragraph{Beschreibung} Die Funktion liefert alle Identifikatoren von Interessenpunkten, bei welchen bereits durch den Nutzer die Zeitspanne validiert wurde. Die Funktion nutzt folgende Quellen:
\begin{itemize}
	\item Tabelle mit Validierungsdaten zu Zeitspannen von Interessenpunkten
\end{itemize}
Es findet bei dieser Funktion kein Abruf von Daten aus {\glqq COSP\grqq} statt. Die Antwort wird als strukturiertes Array an den Aufrufer zurückgegeben.
\subsubsection{getValidationsByUserForCurrentAddress}
\paragraph{Parameter} Die Funktion besitzt folgende Parameter:
\begin{table}[H]
	\begin{tabular}{|c|p{11cm}|}
		\hline
		\textbf{Parametername} & \textbf{Parameterbeschreibung} \\ \hline
		\$uid & Angabe eines Nutzeridentifikators \\ \hline
	\end{tabular}
\end{table}
\paragraph{Beschreibung} Die Funktion liefert alle Identifikatoren von Interessenpunkten, bei welchen bereits durch den Nutzer die aktuelle Adresse validiert wurde. Die Funktion nutzt folgende Quellen:
\begin{itemize}
	\item Tabelle mit Validierungsdaten zu aktuellen Adressen von Interessenpunkten
\end{itemize}
Es findet bei dieser Funktion kein Abruf von Daten aus {\glqq COSP\grqq} statt. Die Antwort wird als strukturiertes Array an den Aufrufer zurückgegeben.
\subsubsection{getValidationsByUserForHistory}
\paragraph{Parameter} Die Funktion besitzt folgende Parameter:
\begin{table}[H]
	\begin{tabular}{|c|p{11cm}|}
		\hline
		\textbf{Parametername} & \textbf{Parameterbeschreibung} \\ \hline
		\$uid & Angabe eines Nutzeridentifikators \\ \hline
	\end{tabular}
\end{table}
\paragraph{Beschreibung} Die Funktion liefert alle Identifikatoren von Interessenpunkten, bei welchen bereits durch den Nutzer die Geschichte validiert wurde. Die Funktion nutzt folgende Quellen:
\begin{itemize}
	\item Tabelle mit Validierungsdaten zur Geschichte von Interessenpunkten
\end{itemize}
Es findet bei dieser Funktion kein Abruf von Daten aus {\glqq COSP\grqq} statt. Die Antwort wird als strukturiertes Array an den Aufrufer zurückgegeben.
\subsubsection{getValidationsByUserForPoiNames}
\paragraph{Parameter} Die Funktion besitzt folgende Parameter:
\begin{table}[H]
	\begin{tabular}{|c|p{11cm}|}
		\hline
		\textbf{Parametername} & \textbf{Parameterbeschreibung} \\ \hline
		\$uid & Angabe eines Nutzeridentifikators \\ \hline
	\end{tabular}
\end{table}
\paragraph{Beschreibung} Die Funktion liefert alle Identifikatoren von Namen, welche bereits durch den Nutzer die validiert wurden. Die Funktion nutzt folgende Quellen:
\begin{itemize}
	\item Tabelle mit Validierungsdaten zu Namen
\end{itemize}
Es findet bei dieser Funktion kein Abruf von Daten aus {\glqq COSP\grqq} statt. Die Antwort wird als strukturiertes Array an den Aufrufer zurückgegeben.
\subsubsection{getValidationsByUserForPoiOperators}
\paragraph{Parameter} Die Funktion besitzt folgende Parameter:
\begin{table}[H]
	\begin{tabular}{|c|p{11cm}|}
		\hline
		\textbf{Parametername} & \textbf{Parameterbeschreibung} \\ \hline
		\$uid & Angabe eines Nutzeridentifikators \\ \hline
	\end{tabular}
\end{table}
\paragraph{Beschreibung} Die Funktion liefert alle Identifikatoren von Betreibern, welche bereits durch den Nutzer die validiert wurden. Die Funktion nutzt folgende Quellen:
\begin{itemize}
	\item Tabelle mit Validierungsdaten zu Betreibern
\end{itemize}
Es findet bei dieser Funktion kein Abruf von Daten aus {\glqq COSP\grqq} statt. Die Antwort wird als strukturiertes Array an den Aufrufer zurückgegeben.
\subsubsection{getValidationsByUserForPoiHistAddresses}
\paragraph{Parameter} Die Funktion besitzt folgende Parameter:
\begin{table}[H]
	\begin{tabular}{|c|p{11cm}|}
		\hline
		\textbf{Parametername} & \textbf{Parameterbeschreibung} \\ \hline
		\$uid & Angabe eines Nutzeridentifikators \\ \hline
	\end{tabular}
\end{table}
\paragraph{Beschreibung} Die Funktion liefert alle Identifikatoren von historischen Adressen, welche bereits durch den Nutzer die validiert wurden. Die Funktion nutzt folgende Quellen:
\begin{itemize}
	\item Tabelle mit Validierungsdaten zu historischen Adressen
\end{itemize}
Es findet bei dieser Funktion kein Abruf von Daten aus {\glqq COSP\grqq} statt. Die Antwort wird als strukturiertes Array an den Aufrufer zurückgegeben.
\subsubsection{getValidationsByUserForLinkPoiStory}
\paragraph{Parameter} Die Funktion besitzt folgende Parameter:
\begin{table}[H]
	\begin{tabular}{|c|p{11cm}|}
		\hline
		\textbf{Parametername} & \textbf{Parameterbeschreibung} \\ \hline
		\$uid & Angabe eines Nutzeridentifikators \\ \hline
	\end{tabular}
\end{table}
\paragraph{Beschreibung} Die Funktion liefert alle Identifikatoren von Links zwischen Interessenpunkten und Geschichten, welche bereits durch den Nutzer die validiert wurden. Die Funktion nutzt folgende Quellen:
\begin{itemize}
	\item Tabelle mit Validierungsdaten zu Links zwischen Interessenpunkten und Geschichten
\end{itemize}
Es findet bei dieser Funktion kein Abruf von Daten aus {\glqq COSP\grqq} statt. Die Antwort wird als strukturiertes Array an den Aufrufer zurückgegeben.
\subsubsection{getAllStoriesData}
\paragraph{Parameter} Die Funktion besitzt keine Parameter.
\paragraph{Beschreibung} Die Funktion fragt Daten für das abrufen aller für das Modul verfügbaren Geschichten ab. Die Funktion nutzt folgende Quellen:
\begin{itemize}
	\item COSP
\end{itemize}
Es findet bei dieser Funktion ein Abruf von Daten aus {\glqq COSP\grqq} statt. Die Antwort wird als strukturiertes Array an den Aufrufer zurückgegeben.
\subsubsection{addUserStoryRemote}
\paragraph{Parameter} Die Funktion besitzt folgende Parameter:
\begin{table}[H]
	\begin{tabular}{|c|p{11cm}|}
		\hline
		\textbf{Parametername} & \textbf{Parameterbeschreibung} \\ \hline
		\$json & Daten der Geschichte \\ \hline
	\end{tabular}
\end{table}
\subparagraph{\$json}Das Array enthält Elemente nach \autoref{api:NewStoryAdd} und \autoref{api-functions:addUserStory}.
\paragraph{Beschreibung} Die Funktion fügt eine neue Geschichte zu {\glqq COSP\grqq} hinzu. Die Funktion hat Auswirkungen auf folgende Quellen:
\begin{itemize}
	\item COSP
\end{itemize}
Es findet bei dieser Funktion kein Abruf von Daten aus {\glqq COSP\grqq} statt. Es werden jedoch Daten an {\glqq COSP\grqq} gesendet. Die Antwort wird als strukturiertes Array an den Aufrufer zurückgegeben.
\subsubsection{deletePOIComplete}
\paragraph{Parameter} Die Funktion besitzt folgende Parameter:
\begin{table}[H]
	\begin{tabular}{|c|p{11cm}|}
		\hline
		\textbf{Parametername} & \textbf{Parameterbeschreibung} \\ \hline
		\$poiid     & Identifikator eines Interessenpunktes \\ \hline
		\$overwrite & Legt fest, ob Daten direkt gelöscht werden sollen \\ \hline
	\end{tabular}
\end{table}
\paragraph{Beschreibung} Die Funktion löscht einen Interessenpunkt mitsamt aller zugehöriger Daten. Die Funktion hat Auswirkungen auf:
\begin{itemize}
	\item Interessenpunkt-Tabelle
	\item Kommentar-Tabelle
	\item Saalanzahl-Tabelle
	\item Sitzplatzanzahl-Tabelle
	\item Namen-Tabelle
	\item Betreiber-Tabelle
	\item Tabelle mit historischen Adressen
	\item Tabelle mit Links zwischen Interessenpunkten und Bildern
	\item Tabelle mit Links zwischen Interessenpunkten und Geschichten
	\item Tabelle mit Validierungsinformationen zu Interessenpunkten
	\item Tabelle mit Validierungsinformationen zu Saalanzahlen
	\item Tabelle mit Validierungsinformationen zu Sitzplatzanzahlen
	\item Tabelle mit Validierungsinformationen zu Namen
	\item Tabelle mit Validierungsinformationen zu historischen Adressen
	\item Tabelle mit Validierungsinformationen zu Links zwischen Interessenpunkten und Bildern
	\item Tabelle mit Validierungsinformationen zu Links zwischen Interessenpunkten und Geschichten
\end{itemize}
Es findet bei dieser Funktion kein Abruf von Daten aus {\glqq COSP\grqq} statt. Die Antwort wird als strukturiertes Array an den Aufrufer zurückgegeben.
\subsubsection{restorePOI}
\paragraph{Parameter} Die Funktion besitzt folgende Parameter:
\begin{table}[H]
	\begin{tabular}{|c|p{11cm}|}
		\hline
		\textbf{Parametername} & \textbf{Parameterbeschreibung} \\ \hline
		\$poiid     & Identifikator eines Interessenpunktes \\ \hline
	\end{tabular}
\end{table}
\paragraph{Beschreibung} Die Funktion stellt einen als gelöscht markierten Interessenpunkt mitsamt aller zugehöriger Daten wieder her. Die Funktion hat Auswirkungen auf:
\begin{itemize}
	\item Interessenpunkt-Tabelle
	\item Kommentar-Tabelle
	\item Saalanzahl-Tabelle
	\item Sitzplatzanzahl-Tabelle
	\item Namen-Tabelle
	\item Betreiber-Tabelle
	\item Tabelle mit historischen Adressen
	\item Tabelle mit Links zwischen Interessenpunkten und Bildern
	\item Tabelle mit Links zwischen Interessenpunkten und Geschichten
\end{itemize}
Es findet bei dieser Funktion kein Abruf von Daten aus {\glqq COSP\grqq} statt. Die Antwort wird als strukturiertes Array an den Aufrufer zurückgegeben.
\subsubsection{deleteSourceByPoi}
\paragraph{Parameter} Die Funktion besitzt folgende Parameter:
\begin{table}[H]
	\begin{tabular}{|c|p{11cm}|}
		\hline
		\textbf{Parametername} & \textbf{Parameterbeschreibung} \\ \hline
		\$poiid     & Identifikator eines Interessenpunktes \\ \hline
		\$overwrite & Legt fest, ob Daten direkt gelöscht werden sollen \\ \hline
	\end{tabular}
\end{table}
\paragraph{Beschreibung} Die Funktion löscht alle Quellen eines Interessenpunktes oder markiert diese als gelöscht. Die Funktion hat Auswirkung auf folgende Quellen:
\begin{itemize}
	\item Tabelle mit Quellenangaben
\end{itemize}
Es findet bei dieser Funktion kein Abruf von Daten aus {\glqq COSP\grqq} statt. Die Antwort wird als strukturiertes Array an den Aufrufer zurückgegeben.
\subsubsection{restoreSourceByPoi}
\paragraph{Parameter} Die Funktion besitzt folgende Parameter:
\begin{table}[H]
	\begin{tabular}{|c|p{11cm}|}
		\hline
		\textbf{Parametername} & \textbf{Parameterbeschreibung} \\ \hline
		\$poiid     & Identifikator eines Interessenpunktes \\ \hline
	\end{tabular}
\end{table}
\paragraph{Beschreibung} Die Funktion stellt alle Quellen eines Interessenpunktes, welche als gelöscht markiert wurden, wieder her. Die Funktion hat Auswirkung auf folgende Quellen:
\begin{itemize}
	\item Tabelle mit Quellenangaben
\end{itemize}
Es findet bei dieser Funktion kein Abruf von Daten aus {\glqq COSP\grqq} statt. Die Antwort wird als strukturiertes Array an den Aufrufer zurückgegeben.
\subsubsection{deleteCinemasByPoi}
\paragraph{Parameter} Die Funktion besitzt folgende Parameter:
\begin{table}[H]
	\begin{tabular}{|c|p{11cm}|}
		\hline
		\textbf{Parametername} & \textbf{Parameterbeschreibung} \\ \hline
		\$poiid     & Identifikator eines Interessenpunktes \\ \hline
		\$overwrite & Legt fest, ob Daten direkt gelöscht werden sollen \\ \hline
	\end{tabular}
\end{table}
\paragraph{Beschreibung} Die Funktion löscht alle Saalanzahlen eines Interessenpunktes oder markiert diese als gelöscht. Die Funktion hat Auswirkung auf folgende Quellen:
\begin{itemize}
	\item Saalanzahl-Tabelle
\end{itemize}
Es findet bei dieser Funktion kein Abruf von Daten aus {\glqq COSP\grqq} statt. Die Antwort wird als strukturiertes Array an den Aufrufer zurückgegeben.
\subsubsection{restoreCinemasByPoi}
\paragraph{Parameter} Die Funktion besitzt folgende Parameter:
\begin{table}[H]
	\begin{tabular}{|c|p{11cm}|}
		\hline
		\textbf{Parametername} & \textbf{Parameterbeschreibung} \\ \hline
		\$poiid     & Identifikator eines Interessenpunktes \\ \hline
	\end{tabular}
\end{table}
\paragraph{Beschreibung} Die Funktion stellt alle Saalanzahlen eines Interessenpunktes, welche als gelöscht markiert wurden, wieder her. Die Funktion hat Auswirkung auf folgende Quellen:
\begin{itemize}
	\item Saalanzahl-Tabelle
\end{itemize}
Es findet bei dieser Funktion kein Abruf von Daten aus {\glqq COSP\grqq} statt. Die Antwort wird als strukturiertes Array an den Aufrufer zurückgegeben.
\subsubsection{deleteNamesByPoi}
\paragraph{Parameter} Die Funktion besitzt folgende Parameter:
\begin{table}[H]
	\begin{tabular}{|c|p{11cm}|}
		\hline
		\textbf{Parametername} & \textbf{Parameterbeschreibung} \\ \hline
		\$poiid     & Identifikator eines Interessenpunktes \\ \hline
		\$overwrite & Legt fest, ob Daten direkt gelöscht werden sollen \\ \hline
	\end{tabular}
\end{table}
\paragraph{Beschreibung} Die Funktion löscht alle Namen eines Interessenpunktes oder markiert diese als gelöscht. Die Funktion hat Auswirkung auf folgende Quellen:
\begin{itemize}
	\item Namen-Tabelle
\end{itemize}
Es findet bei dieser Funktion kein Abruf von Daten aus {\glqq COSP\grqq} statt. Die Antwort wird als strukturiertes Array an den Aufrufer zurückgegeben.
\subsubsection{restoreNamesByPoi}
\paragraph{Parameter} Die Funktion besitzt folgende Parameter:
\begin{table}[H]
	\begin{tabular}{|c|p{11cm}|}
		\hline
		\textbf{Parametername} & \textbf{Parameterbeschreibung} \\ \hline
		\$poiid     & Identifikator eines Interessenpunktes \\ \hline
	\end{tabular}
\end{table}
\paragraph{Beschreibung} Die Funktion stellt alle Namen eines Interessenpunktes, welche als gelöscht markiert wurden, wieder her. Die Funktion hat Auswirkung auf folgende Quellen:
\begin{itemize}
	\item Namen-Tabelle
\end{itemize}
Es findet bei dieser Funktion kein Abruf von Daten aus {\glqq COSP\grqq} statt. Die Antwort wird als strukturiertes Array an den Aufrufer zurückgegeben.
\subsubsection{deletePoiStoryByPoi}
\paragraph{Parameter} Die Funktion besitzt folgende Parameter:
\begin{table}[H]
	\begin{tabular}{|c|p{11cm}|}
		\hline
		\textbf{Parametername} & \textbf{Parameterbeschreibung} \\ \hline
		\$poiid     & Identifikator eines Interessenpunktes \\ \hline
		\$overwrite & Legt fest, ob Daten direkt gelöscht werden sollen \\ \hline
	\end{tabular}
\end{table}
\paragraph{Beschreibung} Die Funktion löscht alle Links eines Interessenpunktes mit Geschichten oder markiert diese als gelöscht. Die Funktion hat Auswirkung auf folgende Quellen:
\begin{itemize}
	\item Tabelle mit Links von Interessenpunkten und Geschichten
\end{itemize}
Es findet bei dieser Funktion kein Abruf von Daten aus {\glqq COSP\grqq} statt. Die Antwort wird als strukturiertes Array an den Aufrufer zurückgegeben.
\subsubsection{restorePoiStoryByPoi}
\paragraph{Parameter} Die Funktion besitzt folgende Parameter:
\begin{table}[H]
	\begin{tabular}{|c|p{11cm}|}
		\hline
		\textbf{Parametername} & \textbf{Parameterbeschreibung} \\ \hline
		\$poiid     & Identifikator eines Interessenpunktes \\ \hline
	\end{tabular}
\end{table}
\paragraph{Beschreibung} Die Funktion stellt alle Links eines Interessenpunktes mit Geschichten, welche als gelöscht markiert wurden, wieder her. Die Funktion hat Auswirkung auf folgende Quellen:
\begin{itemize}
	\item Tabelle mit Links von Interessenpunkten und Geschichten
\end{itemize}
Es findet bei dieser Funktion kein Abruf von Daten aus {\glqq COSP\grqq} statt. Die Antwort wird als strukturiertes Array an den Aufrufer zurückgegeben.
\subsubsection{deleteSeatsByPoi}
\paragraph{Parameter} Die Funktion besitzt folgende Parameter:
\begin{table}[H]
	\begin{tabular}{|c|p{11cm}|}
		\hline
		\textbf{Parametername} & \textbf{Parameterbeschreibung} \\ \hline
		\$poiid     & Identifikator eines Interessenpunktes \\ \hline
		\$overwrite & Legt fest, ob Daten direkt gelöscht werden sollen \\ \hline
	\end{tabular}
\end{table}
\paragraph{Beschreibung} Die Funktion löscht alle Sitzplatzanzahlen eines Interessenpunktes oder markiert diese als gelöscht. Die Funktion hat Auswirkung auf folgende Quellen:
\begin{itemize}
	\item Sitzplatzanzahl-Tabelle
\end{itemize}
Es findet bei dieser Funktion kein Abruf von Daten aus {\glqq COSP\grqq} statt. Die Antwort wird als strukturiertes Array an den Aufrufer zurückgegeben.
\subsubsection{restoreSeatsByPoi}
\paragraph{Parameter} Die Funktion besitzt folgende Parameter:
\begin{table}[H]
	\begin{tabular}{|c|p{11cm}|}
		\hline
		\textbf{Parametername} & \textbf{Parameterbeschreibung} \\ \hline
		\$poiid     & Identifikator eines Interessenpunktes \\ \hline
	\end{tabular}
\end{table}
\paragraph{Beschreibung} Die Funktion stellt alle Sitzplatzanzahlen eines Interessenpunktes, welche als gelöscht markiert wurden, wieder her. Die Funktion hat Auswirkung auf folgende Quellen:
\begin{itemize}
	\item Sitzplatzanzahl-Tabelle
\end{itemize}
Es findet bei dieser Funktion kein Abruf von Daten aus {\glqq COSP\grqq} statt. Die Antwort wird als strukturiertes Array an den Aufrufer zurückgegeben.
\subsubsection{deleteHistAddressByPoi}
\paragraph{Parameter} Die Funktion besitzt folgende Parameter:
\begin{table}[H]
	\begin{tabular}{|c|p{11cm}|}
		\hline
		\textbf{Parametername} & \textbf{Parameterbeschreibung} \\ \hline
		\$poiid     & Identifikator eines Interessenpunktes \\ \hline
		\$overwrite & Legt fest, ob Daten direkt gelöscht werden sollen \\ \hline
	\end{tabular}
\end{table}
\paragraph{Beschreibung} Die Funktion löscht alle historischen Adressen eines Interessenpunktes oder markiert diese als gelöscht. Die Funktion hat Auswirkung auf folgende Quellen:
\begin{itemize}
	\item Tabelle mit historischen Adressen
\end{itemize}
Es findet bei dieser Funktion kein Abruf von Daten aus {\glqq COSP\grqq} statt. Die Antwort wird als strukturiertes Array an den Aufrufer zurückgegeben.
\subsubsection{restoreHistAddressByPoi}
\paragraph{Parameter} Die Funktion besitzt folgende Parameter:
\begin{table}[H]
	\begin{tabular}{|c|p{11cm}|}
		\hline
		\textbf{Parametername} & \textbf{Parameterbeschreibung} \\ \hline
		\$poiid     & Identifikator eines Interessenpunktes \\ \hline
	\end{tabular}
\end{table}
\paragraph{Beschreibung} Die Funktion stellt alle historischen Adressen eines Interessenpunktes, welche als gelöscht markiert wurden, wieder her. Die Funktion hat Auswirkung auf folgende Quellen:
\begin{itemize}
	\item Tabelle mit historischen Adressen
\end{itemize}
Es findet bei dieser Funktion kein Abruf von Daten aus {\glqq COSP\grqq} statt. Die Antwort wird als strukturiertes Array an den Aufrufer zurückgegeben.
\subsubsection{deleteOperatorsByPoi}
\paragraph{Parameter} Die Funktion besitzt folgende Parameter:
\begin{table}[H]
	\begin{tabular}{|c|p{11cm}|}
		\hline
		\textbf{Parametername} & \textbf{Parameterbeschreibung} \\ \hline
		\$poiid     & Identifikator eines Interessenpunktes \\ \hline
		\$overwrite & Legt fest, ob Daten direkt gelöscht werden sollen \\ \hline
	\end{tabular}
\end{table}
\paragraph{Beschreibung} Die Funktion löscht alle Betreiber eines Interessenpunktes oder markiert diese als gelöscht. Die Funktion hat Auswirkung auf folgende Quellen:
\begin{itemize}
	\item Betreiber-Tabelle
\end{itemize}
Es findet bei dieser Funktion kein Abruf von Daten aus {\glqq COSP\grqq} statt. Die Antwort wird als strukturiertes Array an den Aufrufer zurückgegeben.
\subsubsection{restoreOperatorsByPoi}
\paragraph{Parameter} Die Funktion besitzt folgende Parameter:
\begin{table}[H]
	\begin{tabular}{|c|p{11cm}|}
		\hline
		\textbf{Parametername} & \textbf{Parameterbeschreibung} \\ \hline
		\$poiid     & Identifikator eines Interessenpunktes \\ \hline
	\end{tabular}
\end{table}
\paragraph{Beschreibung} Die Funktion stellt alle Betreiber eines Interessenpunktes, welche als gelöscht markiert wurden, wieder her. Die Funktion hat Auswirkung auf folgende Quellen:
\begin{itemize}
	\item Betreiber-Tabelle
\end{itemize}
Es findet bei dieser Funktion kein Abruf von Daten aus {\glqq COSP\grqq} statt. Die Antwort wird als strukturiertes Array an den Aufrufer zurückgegeben.
\subsubsection{deletePoiPicLinkByPoi}
\paragraph{Parameter} Die Funktion besitzt folgende Parameter:
\begin{table}[H]
	\begin{tabular}{|c|p{11cm}|}
		\hline
		\textbf{Parametername} & \textbf{Parameterbeschreibung} \\ \hline
		\$poiid     & Identifikator eines Interessenpunktes \\ \hline
		\$overwrite & Legt fest, ob Daten direkt gelöscht werden sollen \\ \hline
	\end{tabular}
\end{table}
\paragraph{Beschreibung} Die Funktion löscht alle Links eines Interessenpunktes mit Bildern oder markiert diese als gelöscht. Die Funktion hat Auswirkung auf folgende Quellen:
\begin{itemize}
	\item Tabelle mit Links von Interessenpunkten und Bildern
\end{itemize}
Es findet bei dieser Funktion kein Abruf von Daten aus {\glqq COSP\grqq} statt. Die Antwort wird als strukturiertes Array an den Aufrufer zurückgegeben.
\subsubsection{restorePoiPicLinkByPoi}
\paragraph{Parameter} Die Funktion besitzt folgende Parameter:
\begin{table}[H]
	\begin{tabular}{|c|p{11cm}|}
		\hline
		\textbf{Parametername} & \textbf{Parameterbeschreibung} \\ \hline
		\$poiid     & Identifikator eines Interessenpunktes \\ \hline
	\end{tabular}
\end{table}
\paragraph{Beschreibung} Die Funktion stellt alle Links eines Interessenpunktes mit Bildern, welche als gelöscht markiert wurden, wieder her. Die Funktion hat Auswirkung auf folgende Quellen:
\begin{itemize}
	\item Tabelle mit Links von Interessenpunkten und Bildern
\end{itemize}
Es findet bei dieser Funktion kein Abruf von Daten aus {\glqq COSP\grqq} statt. Die Antwort wird als strukturiertes Array an den Aufrufer zurückgegeben.
\subsubsection{GetDataForSingleMaterial}
\paragraph{Parameter} Die Funktion besitzt folgende Parameter:
\begin{table}[H]
	\begin{tabular}{|c|p{11cm}|}
		\hline
		\textbf{Parametername} & \textbf{Parameterbeschreibung} \\ \hline
		\$token & alphanumerischer Identifikator eines Bildes \\ \hline
	\end{tabular}
\end{table}
\paragraph{Beschreibung} Die Funktion fragt alle zum Daten zum Abrufen eines einzelnen Bildes ab. Die Funktion nutzt folgende Quellen:
\begin{itemize}
	\item COSP
\end{itemize}
Es findet bei dieser Funktion ein Abruf von Daten aus {\glqq COSP\grqq} statt. Die Antwort wird als strukturiertes Array an den Aufrufer zurückgegeben.
\subsubsection{SaveDataForSingleMaterial}
\paragraph{Parameter} Die Funktion besitzt folgende Parameter:
\begin{table}[H]
	\begin{tabular}{|c|p{11cm}|}
		\hline
		\textbf{Parametername} & \textbf{Parameterbeschreibung} \\ \hline
		\$token       & alphanumerischer Identifikator eines Bildes \\ \hline
		\$title       & Titel des Bildes \\ \hline
		\$description & Beschreibung des Bildes \\ \hline
		\$source      & Quellenangabe (optional) \\ \hline
		\$sourceType  & Identifikator des Typs der Quelle \\ \hline
	\end{tabular}
\end{table}
\paragraph{Beschreibung} Die Funktion aktualisiert den Titel und/oder die Beschreibung eines Bildes in {\glqq COSP\grqq}. Die Funktion hat Auswirkungen auf folgende Quellen:
\begin{itemize}
	\item COSP
\end{itemize}
Es findet bei dieser Funktion kein Abruf von Daten aus {\glqq COSP\grqq} statt. Es werden jedoch Daten an {\glqq COSP\grqq} gesendet. Die Antwort wird als strukturiertes Array an den Aufrufer zurückgegeben.
\subsubsection{saveStoryEditedDataToCOSP}
\paragraph{Parameter} Die Funktion besitzt folgende Parameter:
\begin{table}[H]
	\begin{tabular}{|c|p{11cm}|}
		\hline
		\textbf{Parametername} & \textbf{Parameterbeschreibung} \\ \hline
		\$token & alphanumerischer Identifikator einer Geschichte \\ \hline
		\$title & Titel der Geschichte \\ \hline
		\$story & Inhalt der Geschichte \\ \hline
	\end{tabular}
\end{table}
\paragraph{Beschreibung} Die Funktion aktualisiert den Titel und/oder den Inhalt einer Geschichte in {\glqq COSP\grqq}. Die Funktion hat Auswirkungen auf folgende Quellen:
\begin{itemize}
	\item COSP
\end{itemize}
Es findet bei dieser Funktion kein Abruf von Daten aus {\glqq COSP\grqq} statt. Es werden jedoch Daten an {\glqq COSP\grqq} gesendet. Die Antwort wird als strukturiertes Array an den Aufrufer zurückgegeben.
\subsubsection{resetUserPassword}
\paragraph{Parameter} Die Funktion besitzt folgende Parameter:
\begin{table}[H]
	\begin{tabular}{|c|p{11cm}|}
		\hline
		\textbf{Parametername} & \textbf{Parameterbeschreibung} \\ \hline
		\$username & Nutzername \\ \hline
	\end{tabular}
\end{table}
\paragraph{Beschreibung} Die Funktion fordert ein zurücksetzen des Passwortes in {\glqq COSP\grqq} an. Die Funktion hat Auswirkungen auf folgende Quellen:
\begin{itemize}
	\item COSP
\end{itemize}
Es findet bei dieser Funktion kein Abruf von Daten aus {\glqq COSP\grqq} statt. Es werden jedoch Daten an {\glqq COSP\grqq} gesendet. Die Antwort wird als strukturiertes Array an den Aufrufer zurückgegeben.
\subsubsection{getCompleteInformationOfPoiNames}
\paragraph{Parameter} Die Funktion besitzt folgende Parameter:
\begin{table}[H]
	\begin{tabular}{|c|p{11cm}|}
		\hline
		\textbf{Parametername} & \textbf{Parameterbeschreibung} \\ \hline
		\$poiid      & Identifikator eines Interessenpunktes \\ \hline
		\$main\_name & Name des Interessenpunktes aus der Interessenpunkt-Tabelle \\ \hline
	\end{tabular}
\end{table}
\paragraph{Beschreibung} Die Funktion fragt alle Namen eines Interessenpunktes mit Validierungswerten ab. Die Funktion nutzt folgende Quellen:
\begin{itemize}
	\item Namen-Tabelle
	\item Tabelle mit Validierungsinformationen zu Namen
\end{itemize}
Es findet bei dieser Funktion kein Abruf von Daten aus {\glqq COSP\grqq} statt. Die Antwort wird als strukturiertes Array an den Aufrufer zurückgegeben.
\subsubsection{getCompleteInformationOfPoiOperators}
\paragraph{Parameter} Die Funktion besitzt folgende Parameter:
\begin{table}[H]
	\begin{tabular}{|c|p{11cm}|}
		\hline
		\textbf{Parametername} & \textbf{Parameterbeschreibung} \\ \hline
		\$poiid      & Identifikator eines Interessenpunktes \\ \hline
	\end{tabular}
\end{table}
\paragraph{Beschreibung} Die Funktion fragt alle Betreiber eines Interessenpunktes mit Validierungswerten ab. Die Funktion nutzt folgende Quellen:
\begin{itemize}
	\item Betreiber-Tabelle
	\item Tabelle mit Validierungsinformationen zu Betreibern
\end{itemize}
Es findet bei dieser Funktion kein Abruf von Daten aus {\glqq COSP\grqq} statt. Die Antwort wird als strukturiertes Array an den Aufrufer zurückgegeben.
\subsubsection{getCompleteInformationOfPoiHistAddress}
\paragraph{Parameter} Die Funktion besitzt folgende Parameter:
\begin{table}[H]
	\begin{tabular}{|c|p{11cm}|}
		\hline
		\textbf{Parametername} & \textbf{Parameterbeschreibung} \\ \hline
		\$poiid      & Identifikator eines Interessenpunktes \\ \hline
	\end{tabular}
\end{table}
\paragraph{Beschreibung} Die Funktion fragt alle historischen Adressen eines Interessenpunktes mit Validierungswerten ab. Die Funktion nutzt folgende Quellen:
\begin{itemize}
	\item Tabelle mit historischen Adressen
	\item Tabelle mit Validierungsinformationen zu historischen Adressen
\end{itemize}
Es findet bei dieser Funktion kein Abruf von Daten aus {\glqq COSP\grqq} statt. Die Antwort wird als strukturiertes Array an den Aufrufer zurückgegeben.
\subsubsection{deleteMaterial}
\paragraph{Parameter} Die Funktion besitzt folgende Parameter:
\begin{table}[H]
	\begin{tabular}{|c|p{11cm}|}
		\hline
		\textbf{Parametername} & \textbf{Parameterbeschreibung} \\ \hline
		\$picToken      & alphanumerischer Identifikator eines Bildes \\ \hline
	\end{tabular}
\end{table}
\paragraph{Beschreibung} Die Funktion fragt die Löschung eines Bildes in {\glqq COSP\grqq} an. Die Funktion hat Auswirkungen auf folgende Quellen:
\begin{itemize}
	\item COSP
\end{itemize}
Es findet bei dieser Funktion kein Abruf von Daten aus {\glqq COSP\grqq} statt. Es werden jedoch Daten an {\glqq COSP\grqq} gesendet. Die Antwort wird als strukturiertes Array an den Aufrufer zurückgegeben.
\subsubsection{getStoriesAsListFromCOSP}
\paragraph{Parameter} Die Funktion besitzt folgende Parameter:
\begin{table}[H]
	\begin{tabular}{|c|p{11cm}|}
		\hline
		\textbf{Parametername} & \textbf{Parameterbeschreibung} \\ \hline
		\$tokenList      & Array mit alphanumerischen Identifikatoren von Geschichten \\ \hline
	\end{tabular}
\end{table}
\paragraph{Beschreibung} Die Funktion fragt alle Daten zum Abrufen mehrerer bestimmter Geschichten aus {\glqq COSP\grqq} an. Die Funktion nutzt folgende Quellen:
\begin{itemize}
	\item COSP
\end{itemize}
Es findet bei dieser Funktion ein Abruf von Daten aus {\glqq COSP\grqq} statt. Die Antwort wird als strukturiertes Array an den Aufrufer zurückgegeben.
\subsubsection{deleteMaterial}
\paragraph{Parameter} Die Funktion besitzt folgende Parameter:
\begin{table}[H]
	\begin{tabular}{|c|p{11cm}|}
		\hline
		\textbf{Parametername} & \textbf{Parameterbeschreibung} \\ \hline
		\$data      & alphanumerischer Identifikator einer Geschichte \\ \hline
	\end{tabular}
\end{table}
\paragraph{Beschreibung} Die Funktion fragt die Löschung einer Geschichte in {\glqq COSP\grqq} an. Die Funktion hat Auswirkungen auf folgende Quellen:
\begin{itemize}
	\item COSP
\end{itemize}
Es findet bei dieser Funktion kein Abruf von Daten aus {\glqq COSP\grqq} statt. Es werden jedoch Daten an {\glqq COSP\grqq} gesendet. Die Antwort wird als strukturiertes Array an den Aufrufer zurückgegeben.
\subsubsection{GetPoiPicLinkValidators}
\paragraph{Parameter} Die Funktion besitzt folgende Parameter:
\begin{table}[H]
	\begin{tabular}{|c|p{11cm}|}
		\hline
		\textbf{Parametername} & \textbf{Parameterbeschreibung} \\ \hline
		\$lid      & Identifikatoren eines Links zwischen einem Bild und einer Geschichte \\ \hline
	\end{tabular}
\end{table}
\paragraph{Beschreibung} Die Funktion fragt alle Validatoren eines Links zwischen einer Geschichte und einem Bild in {\glqq COSP\grqq} ab. Die Funktion nutzt folgende Quellen:
\begin{itemize}
	\item COSP
\end{itemize}
Es findet bei dieser Funktion ein Abruf von Daten aus {\glqq COSP\grqq} statt. Die Antwort wird als strukturiertes Array an den Aufrufer zurückgegeben.
\subsubsection{getCompleteInformationOfPoiSeats}
\paragraph{Parameter} Die Funktion besitzt folgende Parameter:
\begin{table}[H]
	\begin{tabular}{|c|p{11cm}|}
		\hline
		\textbf{Parametername} & \textbf{Parameterbeschreibung} \\ \hline
		\$poiid      & Identifikator eines Interessenpunktes \\ \hline
	\end{tabular}
\end{table}
\paragraph{Beschreibung} Die Funktion fragt alle Sitzplatzanzahlen eines Interessenpunktes mit Validierungswerten ab. Die Funktion nutzt folgende Quellen:
\begin{itemize}
	\item Sitzplatzanzahl-Tabelle
	\item Tabelle mit Validierungswerten zu Sitzplatzanzahlen
\end{itemize}
Es findet bei dieser Funktion kein Abruf von Daten aus {\glqq COSP\grqq} statt. Die Antwort wird als strukturiertes Array an den Aufrufer zurückgegeben.
\subsubsection{getValidationsByUserForPoiSeats}
\paragraph{Parameter} Die Funktion besitzt folgende Parameter:
\begin{table}[H]
	\begin{tabular}{|c|p{11cm}|}
		\hline
		\textbf{Parametername} & \textbf{Parameterbeschreibung} \\ \hline
		\$uid & Angabe eines Nutzeridentifikators \\ \hline
	\end{tabular}
\end{table}
\paragraph{Beschreibung} Die Funktion liefert alle Identifikatoren von Sitzplatzanzahlen, welche bereits durch den Nutzer validiert wurden. Die Funktion nutzt folgende Quellen:
\begin{itemize}
	\item Tabelle mit Validierungsdaten zu Sitzplatzanzahlen
\end{itemize}
Es findet bei dieser Funktion kein Abruf von Daten aus {\glqq COSP\grqq} statt. Die Antwort wird als strukturiertes Array an den Aufrufer zurückgegeben.
\subsubsection{validatePoiSeats}
\paragraph{Parameter} Die Funktion besitzt folgende Parameter:
\begin{table}[H]
	\begin{tabular}{|c|p{11cm}|}
		\hline
		\textbf{Parametername} & \textbf{Parameterbeschreibung} \\ \hline
		\$seatid & Identifikator der Sitzplatzanzahl \\ \hline
	\end{tabular}
\end{table}
\paragraph{Beschreibung} Die Funktion fügt einer Sitzplatzanzahl eine Validierung hinzu. Die Funktion hat Auswirkungen auf folgende Quellen:
\begin{itemize}
	\item Tabelle mit Validierungsdaten zu Sitzplatzanzahlen
\end{itemize}
Es findet bei dieser Funktion kein Abruf von Daten aus {\glqq COSP\grqq} statt. Die Antwort wird als strukturiertes Array an den Aufrufer zurückgegeben.
\subsubsection{getCompleteInformationOfPoiCinemas}
\paragraph{Parameter} Die Funktion besitzt folgende Parameter:
\begin{table}[H]
	\begin{tabular}{|c|p{11cm}|}
		\hline
		\textbf{Parametername} & \textbf{Parameterbeschreibung} \\ \hline
		\$poiid      & Identifikator eines Interessenpunktes \\ \hline
	\end{tabular}
\end{table}
\paragraph{Beschreibung} Die Funktion fragt alle Saalanzahlen eines Interessenpunktes mit Validierungswerten ab. Die Funktion nutzt folgende Quellen:
\begin{itemize}
	\item Saalanzahl-Tabelle
	\item Tabelle mit Validierungswerten zu Saalanzahlen
\end{itemize}
Es findet bei dieser Funktion kein Abruf von Daten aus {\glqq COSP\grqq} statt. Die Antwort wird als strukturiertes Array an den Aufrufer zurückgegeben.
\subsubsection{getValidationsByUserForPoiCinemas}
\paragraph{Parameter} Die Funktion besitzt folgende Parameter:
\begin{table}[H]
	\begin{tabular}{|c|p{11cm}|}
		\hline
		\textbf{Parametername} & \textbf{Parameterbeschreibung} \\ \hline
		\$uid & Angabe eines Nutzeridentifikators \\ \hline
	\end{tabular}
\end{table}
\paragraph{Beschreibung} Die Funktion liefert alle Identifikatoren von Sitzplatzanzahlen, welche bereits durch den Nutzer validiert wurden. Die Funktion nutzt folgende Quellen:
\begin{itemize}
	\item Tabelle mit Validierungsdaten zu Saalanzahlen
\end{itemize}
Es findet bei dieser Funktion kein Abruf von Daten aus {\glqq COSP\grqq} statt. Die Antwort wird als strukturiertes Array an den Aufrufer zurückgegeben.
\subsubsection{validatePoiCinemas}
\paragraph{Parameter} Die Funktion besitzt folgende Parameter:
\begin{table}[H]
	\begin{tabular}{|c|p{11cm}|}
		\hline
		\textbf{Parametername} & \textbf{Parameterbeschreibung} \\ \hline
		\$cinemasid & Identifikator einer Saalanzahl \\ \hline
	\end{tabular}
\end{table}
\paragraph{Beschreibung} Die Funktion fügt einer Sitzplatzanzahl eine Validierung hinzu. Die Funktion hat Auswirkungen auf folgende Quellen:
\begin{itemize}
	\item Tabelle mit Validierungsdaten zu Saalanzahlen
\end{itemize}
Es findet bei dieser Funktion kein Abruf von Daten aus {\glqq COSP\grqq} statt. Die Antwort wird als strukturiertes Array an den Aufrufer zurückgegeben.
\subsubsection{getValidationsByUserForCinemaType}
\paragraph{Parameter} Die Funktion besitzt folgende Parameter:
\begin{table}[H]
	\begin{tabular}{|c|p{11cm}|}
		\hline
		\textbf{Parametername} & \textbf{Parameterbeschreibung} \\ \hline
		\$uid & Angabe eines Nutzeridentifikators \\ \hline
	\end{tabular}
\end{table}
\paragraph{Beschreibung} Die Funktion liefert alle Identifikatoren von Interessenpunkten, bei welchen der Typ bereits durch den Nutzer validiert wurden. Die Funktion nutzt folgende Quellen:
\begin{itemize}
	\item Tabelle mit Validierungsdaten zum Typ eines Interessenpunktes
\end{itemize}
Es findet bei dieser Funktion kein Abruf von Daten aus {\glqq COSP\grqq} statt. Die Antwort wird als strukturiertes Array an den Aufrufer zurückgegeben.
\subsubsection{validateTypePoi}
\paragraph{Parameter} Die Funktion besitzt folgende Parameter:
\begin{table}[H]
	\begin{tabular}{|c|p{11cm}|}
		\hline
		\textbf{Parametername} & \textbf{Parameterbeschreibung} \\ \hline
		\$poiid & Identifikator eines Interessenpunktes \\ \hline
	\end{tabular}
\end{table}
\paragraph{Beschreibung} Die Funktion fügt einem Interessenpunkt eine Validierung für seinen Typ hinzu. Die Funktion hat Auswirkungen auf folgende Quellen:
\begin{itemize}
	\item Tabelle mit Validierungsdaten zum Typ eines Interessenpunktes
\end{itemize}
Es findet bei dieser Funktion kein Abruf von Daten aus {\glqq COSP\grqq} statt. Es gibt keinen Rückgabewert.
\subsubsection{RedirectMainBetaIndex}
\paragraph{Parameter} Die Funktion besitzt keine Parameter.
\paragraph{Beschreibung} Die Funktion leitet den bereits im Wartungs- beziehungsweise Beta-Modus angemeldeten Nutzern wider auf die Startseite ohne die URL nochmals eingeben zu müssen. Es findet bei dieser Funktion kein Abruf von Daten aus {\glqq COSP\grqq} statt. Die Antwort wird als strukturiertes Array an den Aufrufer zurückgegeben.
\subsubsection{SendStoryApprovalChange}
\paragraph{Parameter} Die Funktion besitzt folgende Parameter:
\begin{table}[H]
	\begin{tabular}{|c|p{11cm}|}
		\hline
		\textbf{Parametername} & \textbf{Parameterbeschreibung} \\ \hline
		\$story\_token & alphanumerischer Identifikator einer Geschichte \\ \hline
		\$state        & Status der Freischaltung der Geschichte \\ \hline
	\end{tabular}
\end{table}
\paragraph{Beschreibung} Die Funktion schaltet oder sperrt eine Geschichte in {\glqq COSP\grqq}. Die Funktion hat Auswirkungen auf folgende Quellen:
\begin{itemize}
	\item COSP
\end{itemize}
Es findet bei dieser Funktion kein Abruf von Daten aus {\glqq COSP\grqq} statt. Es werden jedoch Daten an {\glqq COSP\grqq} gesendet. Es gibt keine Antwort.
\subsubsection{GetCaptchaFromCOSP}
\paragraph{Parameter} Die Funktion besitzt keine Parameter.
\paragraph{Beschreibung} Die Funktion ruft ein Captcha von {\glqq COSP\grqq} ab. Die Funktion nutzt folgende Quellen:
\begin{itemize}
	\item Konfigurationsdatei
\end{itemize}
Es findet bei dieser Funktion ein Abruf von Daten aus {\glqq COSP\grqq} statt. Die Antwort wird als strukturiertes Array an den Aufrufer zurückgegeben.
\subsubsection{sendContactMail}
\paragraph{Parameter} Die Funktion besitzt folgende Parameter:
\begin{table}[H]
	\begin{tabular}{|c|p{11cm}|}
		\hline
		\textbf{Parametername} & \textbf{Parameterbeschreibung} \\ \hline
		\$mail    & Mailadresse des Senders \\ \hline
		\$subject & Betreff der Mail \\ \hline
		\$msg     & Inhalt der Mail \\ \hline
	\end{tabular}
\end{table}
\paragraph{Beschreibung} Die Funktion fordert das senden einer Kontakmail durch {\glqq COSP\grqq} an. Die Funktion nutzt folgende Quellen:
\begin{itemize}
	\item Konfigurationsdatei
\end{itemize}
Es findet bei dieser Funktion kein Abruf von Daten aus {\glqq COSP\grqq} statt. Es werden jedoch Daten an {\glqq COSP\grqq} gesendet. Die Antwort wird als strukturiertes Array an den Aufrufer zurückgegeben.
\subsubsection{getAllSourceTypesCOSP}
\paragraph{Parameter} Die Funktion besitzt keine Parameter.
\paragraph{Beschreibung} Die Funktion ruft alle Quelltypen von COSP ab. Es findet bei dieser Funktion ein Abruf von Daten aus {\glqq COSP\grqq} statt. Die Antwort wird als strukturiertes Array an den Aufrufer zurückgegeben.
\subsubsection{getUserIp}
\paragraph{Parameter} Die Funktion besitzt keine Parameter.
\paragraph{Beschreibung} Die Funktion bestimmt die IP-Adresse des Aufrufenden Nutzers. Es findet bei dieser Funktion ein Abruf von Daten aus {\glqq COSP\grqq} statt. Die Antwort wird als strukturiertes Array an den Aufrufer zurückgegeben.
\newpage
\section{inc-sub}
\subsection{Allgemeines} Diese Datei Einbindungen aller benötigten Dateien für die Ausführung der Anwendung d.
\begin{table}[H]
	\begin{tabular}{|c|p{11cm}|}
		\hline
		\textbf{Einbindungspunkt} & inc-sub.php \\ \hline
	\end{tabular}
\end{table}
Die Datei ist nicht direkt durch den Nutzer aufrufbar, dies wird durch folgenden Code-Ausschnitt sichergestellt:
\begin{lstlisting}[language=php]
if (!defined('NICE_PROJECT')) {
	die('Permission denied.');
}
\end{lstlisting}
Der Globale Wert {\glqq NICE\_PROJECT\grqq} wird durch für den Nutzer valide Aufrufpunkte festgelegt, z.B. {\glqq api.php\grqq}.
\newpage
\subsection{Einbindungen}
\subsubsection{Grundlegendes}
Zu Anfangs wird zunächst der HTTP-Content-Typ festgelegt:
\begin{lstlisting}[language=php]
header('Content-Type: text/html; charset=utf-8');
\end{lstlisting}
Nachfolgend zu sehender Code-Block bindet alle benötigten Dateien in korrekter Reihenfolge ein. Beim Einbinden neuer Dateien, sind diese stets an das Ende zu schreiben, außer die Dateien sind Umstrukturierungen bereits existenten Dateien.
\begin{lstlisting}[language=php]
require_once '../bin/config.php';
require_once '../bin/settings.php';
require_once '../bin/database/inc-db-sub.php';
require_once '../bin/deletions.php';
require_once '../bin/api-functions.php';
require_once '../bin/functionLib.php';
require_once '../bin/authSystem.php';
require_once '../bin/session.php';
require_once '../bin/api-functions.php';
require_once '../bin/rapi-functions.php';
require_once '../bin/statistic-calc.php';
\end{lstlisting}
Des weiteren wird hier die für den Nutzer sichtbare Fehlerausgabe anhand des Debug-Konfigurationsparameters festgelegt:
\begin{lstlisting}[language=php]
if (config::$DEBUG === true) {
	error_reporting(E_ALL);
	ini_set('display_errors', 1);
	ini_set('display_startup_errors', 1);
}
\end{lstlisting}
Ebenfalls wird auch die Sprache des Systems einheitlich auf Deutsch festgelegt und die entsprechende Datei eingebunden:
\begin{lstlisting}[language=php]
if (!isset($_SESSION['lang'])) {
	$_SESSION['lang'] = "de";
}
require_once "../bin/" . $_SESSION['lang'] . ".php";
\end{lstlisting}
\subsubsection{Besonderheit}
Die Einbindungen sind immer mit {\glqq ../\grqq} anzufangen, da Sie für Subordner des Hauptordners gedacht sind.
\newpage
\section{inc}
\subsection{Allgemeines} Diese Datei Einbindungen aller benötigten Dateien für die Ausführung der Anwendung d.
\begin{table}[H]
	\begin{tabular}{|c|p{11cm}|}
		\hline
	\end{tabular}
\end{table}
Die Datei ist nicht direkt durch den Nutzer aufrufbar, dies wird durch folgenden Code-Ausschnitt sichergestellt:
\begin{lstlisting}[language=php]
if (!defined('NICE_PROJECT')) {
	die('Permission denied.');
}
\end{lstlisting}
Der Globale Wert {\glqq NICE\_PROJECT\grqq} wird durch für den Nutzer valide Aufrufpunkte festgelegt, z.B. {\glqq api.php\grqq}.
\newpage
\subsection{Einbindungen}
\subsubsection{Grundlegendes}
Zu Anfangs wird zunächst der HTTP-Content-Typ festgelegt:
\begin{lstlisting}[language=php]
header('Content-Type: text/html; charset=utf-8');
\end{lstlisting}
Nachfolgend zu sehender Code-Block bindet alle benötigten Dateien in korrekter Reihenfolge ein. Beim Einbinden neuer Dateien, sind diese stets an das Ende zu schreiben, außer die Dateien sind Umstrukturierungen bereits existenten Dateien.
\begin{lstlisting}[language=php]
require_once 'bin/config.php';
require_once 'bin/settings.php';
require_once 'bin/database/inc-db.php';
require_once 'bin/deletions.php';
require_once 'bin/functionLib.php';
require_once 'bin/authSystem.php';
require_once 'bin/session.php';
require_once 'bin/api-functions.php';
require_once 'bin/rapi-functions.php';
require_once 'bin/statistic-calc.php';
\end{lstlisting}
Des weiteren wird hier die für den Nutzer sichtbare Fehlerausgabe anhand des Debug-Konfigurationsparameters festgelegt:
\begin{lstlisting}[language=php]
if (config::$DEBUG === true) {
	error_reporting(E_ALL);
	ini_set('display_errors', 1);
	ini_set('display_startup_errors', 1);
}
\end{lstlisting}
Ebenfalls wird auch die Sprache des Systems einheitlich auf Deutsch festgelegt und die entsprechende Datei eingebunden:
\begin{lstlisting}[language=php]
if (!isset($_SESSION['lang'])) {
	$_SESSION['lang'] = "de";
}
require_once "bin/" . $_SESSION['lang'] . ".php";
\end{lstlisting}
\subsubsection{Besonderheit}
Die Einbindungen sind immer vom Hauptordner aus zu erreichen und auch relativ zu diesem anzugeben.
\newpage
\section{rapi-functions}
\subsection{Allgemeines} Diese Datei enthält alle durch die Reverse-API des Frontends aufgerufene Funktionen sowie zusätzliche Funktionen, um die Datenstrukturierung der Antworten zu vereinheitlichen.
\begin{table}[H]
	\begin{tabular}{|c|p{11cm}|}
		\hline
		\textbf{Einbindungspunkt} & inc-db.php \\ \hline
		\textbf{Einbindungspunkt} & inc-db-sub.php \\ \hline
	\end{tabular}
\end{table}
Die Datei ist nicht direkt durch den Nutzer aufrufbar, dies wird durch folgenden Code-Ausschnitt sichergestellt:
\begin{lstlisting}[language=php]
if (!defined('NICE_PROJECT')) {
	die('Permission denied.');
}
\end{lstlisting}
Der Globale Wert {\glqq NICE\_PROJECT\grqq} wird durch für den Nutzer valide Aufrufpunkte festgelegt, z.B. {\glqq api.php\grqq}.
\newpage
\subsection{Funktionen}
\subsubsection{checkApiToken}
\paragraph{Parameter} Die Funktion besitzt folgende Parameter:
\begin{table}[H]
	\begin{tabular}{|c|p{11cm}|}
		\hline
		\textbf{Parametername} & \textbf{Parameterbeschreibung} \\ \hline
		\$token & Authentifikationstoken des Requests \\ \hline
	\end{tabular}
\end{table}
\paragraph{Beschreibung} Die Funktion prüft, ob ein Authentifikationstoken mit dem Authentifikationstoken des Moduls übereinstimmt. Die Funktion nutzt folgende Quellen:
\begin{itemize}
	\item Konfigurationsdatei
\end{itemize}
Es findet bei dieser Funktion kein Abruf von Daten aus {\glqq COSP\grqq} statt. Die Antwort wird als strukturiertes Array an den Aufrufer zurückgegeben.
\subsubsection{successfullRequest}
\paragraph{Parameter} Die Funktion besitzt folgende Parameter:
\begin{table}[H]
	\begin{tabular}{|c|p{11cm}|}
		\hline
		\textbf{Parametername} & \textbf{Parameterbeschreibung} \\ \hline
		\$input & Zusätzliche Daten (Optional)\\ \hline
	\end{tabular}
\end{table}
\paragraph{Beschreibung} Die Funktion erzeugt dein Array, welches als Antwort auf eine erfolgreiche Anfrage zurückgesendet wird. Es findet bei dieser Funktion kein Abruf von Daten aus {\glqq COSP\grqq} statt. Die Antwort wird als strukturiertes Array an den Aufrufer zurückgegeben.
\subsubsection{failedRequest}
\paragraph{Parameter} Die Funktion besitzt folgende Parameter:
\begin{table}[H]
	\begin{tabular}{|c|p{11cm}|}
		\hline
		\textbf{Parametername} & \textbf{Parameterbeschreibung} \\ \hline
		\$input & Zusätzliche Daten (Optional)\\ \hline
	\end{tabular}
\end{table}
\paragraph{Beschreibung} Die Funktion erzeugt dein Array, welches als Antwort auf eine erfolglose/falsche Anfrage zurückgesendet wird. Es findet bei dieser Funktion kein Abruf von Daten aus {\glqq COSP\grqq} statt. Die Antwort wird als strukturiertes Array an den Aufrufer zurückgegeben.
\subsubsection{checkForUsername}
\paragraph{Parameter} Die Funktion besitzt folgende Parameter:
\begin{table}[H]
	\begin{tabular}{|c|p{11cm}|}
		\hline
		\textbf{Parametername} & \textbf{Parameterbeschreibung} \\ \hline
		\$username & Nutzername \\ \hline
	\end{tabular}
\end{table}
\paragraph{Beschreibung} Die Funktion prüft, ob der gegebene Nutzername in der lokalen Datenbank vorhanden ist. Die Funktion nutzt folgende Quellen: Es findet bei dieser Funktion kein Abruf von Daten aus {\glqq COSP\grqq} statt. Die Antwort wird als strukturiertes Array an den Aufrufer zurückgegeben.
\subsubsection{AktivateUserRapi}
\paragraph{Parameter} Die Funktion besitzt folgende Parameter:
\begin{table}[H]
	\begin{tabular}{|c|p{11cm}|}
		\hline
		\textbf{Parametername} & \textbf{Parameterbeschreibung} \\ \hline
		\$json  & Array mit Daten \\ \hline
		\$value & Aktivierungsstatus \\ \hline
	\end{tabular}
\end{table}
\subparagraph{\$json}Das Array enthält folgende Elemente:
\begin{table}[H]
	\begin{tabular}{|c|p{11cm}|}
		\hline
		\textbf{Parametername} & \textbf{Parameterbeschreibung} \\ \hline
		username & Nutzername des zu aktivierenden Nutzers \\ \hline
	\end{tabular}
\end{table}
\paragraph{Beschreibung} Die Funktion aktiviert oder deaktiviert einen Nutzer. Die Funktion hat Auswirkungen auf folgende Quellen:
\begin{itemize}
	\item Nutzerdaten-Tabelle
\end{itemize}
Es findet bei dieser Funktion kein Abruf von Daten aus {\glqq COSP\grqq} statt. Die Antwort wird als strukturiertes Array an den Aufrufer zurückgegeben.
\subsubsection{removePictureTokenRevApi}
\paragraph{Parameter} Die Funktion besitzt folgende Parameter:
\begin{table}[H]
	\begin{tabular}{|c|p{11cm}|}
		\hline
		\textbf{Parametername} & \textbf{Parameterbeschreibung} \\ \hline
		\$json & Array mit Parametern der Anfrage \\ \hline
	\end{tabular}
\end{table}
\subparagraph{\$json}Das Array enthält folgende Elemente:
\begin{table}[H]
	\begin{tabular}{|c|p{11cm}|}
		\hline
		\textbf{Parametername} & \textbf{Parameterbeschreibung} \\ \hline
		picToken & alphanumerischer Identifikator eines Bildes \\ \hline
		override & schaltet direktes Löschen ein \\ \hline
	\end{tabular}
\end{table}
\paragraph{Beschreibung} Die Funktion löscht Referenzen auf Bilder oder markiert diese als gelöscht. Die Funktion hat Auswirkungen auf folgende Quellen:
\begin{itemize}
	\item Tabelle mit Links zwischen Bildern und Interessenpunkten
	\item Tabelle mit Interessenpunkten
	\item Tabelle mit Validierungsinformationen zu Interessenpunkten
	\item Tabelle mit Validierungsinformationen zu Links zwischen Bildern und Interessenpunkten
\end{itemize}
Es findet bei dieser Funktion kein Abruf von Daten aus {\glqq COSP\grqq} statt. Die Antwort wird als strukturiertes Array an den Aufrufer zurückgegeben.
\subsubsection{deleteStoryReference}
\paragraph{Parameter} Die Funktion besitzt folgende Parameter:
\begin{table}[H]
	\begin{tabular}{|c|p{11cm}|}
		\hline
		\textbf{Parametername} & \textbf{Parameterbeschreibung} \\ \hline
		\$json & Array mit Parametern der Anfrage \\ \hline
	\end{tabular}
\end{table}
\subparagraph{\$json}Das Array enthält folgende Elemente:
\begin{table}[H]
	\begin{tabular}{|c|p{11cm}|}
		\hline
		\textbf{Parametername} & \textbf{Parameterbeschreibung} \\ \hline
		StoryToken & alphanumerischer Identifikator einer Geschichte \\ \hline
		override   & schaltet direktes Löschen ein \\ \hline
	\end{tabular}
\end{table}
\paragraph{Beschreibung} Die Funktion löscht Referenzen auf Geschichten oder markiert diese als gelöscht. Die Funktion hat Auswirkungen auf folgende Quellen:
\begin{itemize}
	\item Tabelle mit Links zwischen Geschichten und Interessenpunkten
	\item Tabelle mit Validierungsinformationen zu Links zwischen Geschichten und Interessenpunkten
\end{itemize}
Es findet bei dieser Funktion kein Abruf von Daten aus {\glqq COSP\grqq} statt. Die Antwort wird als strukturiertes Array an den Aufrufer zurückgegeben.
\subsubsection{restoreStoryRapi}
\paragraph{Parameter} Die Funktion besitzt folgende Parameter:
\begin{table}[H]
	\begin{tabular}{|c|p{11cm}|}
		\hline
		\textbf{Parametername} & \textbf{Parameterbeschreibung} \\ \hline
		\$json & Array mit Parametern der Anfrage \\ \hline
	\end{tabular}
\end{table}
\subparagraph{\$json}Das Array enthält folgende Elemente:
\begin{table}[H]
	\begin{tabular}{|c|p{11cm}|}
		\hline
		\textbf{Parametername} & \textbf{Parameterbeschreibung} \\ \hline
		StoryToken & alphanumerischer Identifikator einer Geschichte \\ \hline
	\end{tabular}
\end{table}
\paragraph{Beschreibung} Die Funktion stellt als gelöscht markierte Referenzen auf Geschichten wieder her. Die Funktion hat Auswirkungen auf folgende Quellen:
\begin{itemize}
	\item Tabelle mit Links zwischen Geschichten und Interessenpunkten
\end{itemize}
Es findet bei dieser Funktion kein Abruf von Daten aus {\glqq COSP\grqq} statt. Die Antwort wird als strukturiertes Array an den Aufrufer zurückgegeben.
\subsubsection{restorePictureRapi}
\paragraph{Parameter} Die Funktion besitzt folgende Parameter:
\begin{table}[H]
	\begin{tabular}{|c|p{11cm}|}
		\hline
		\textbf{Parametername} & \textbf{Parameterbeschreibung} \\ \hline
		\$json & Array mit Parametern der Anfrage \\ \hline
	\end{tabular}
\end{table}
\subparagraph{\$json}Das Array enthält folgende Elemente:
\begin{table}[H]
	\begin{tabular}{|c|p{11cm}|}
		\hline
		\textbf{Parametername} & \textbf{Parameterbeschreibung} \\ \hline
		picToken & alphanumerischer Identifikator eines Bildes \\ \hline
	\end{tabular}
\end{table}
\paragraph{Beschreibung} Die Funktion stellt als gelöscht markierte Referenzen auf Bilder wieder her. Die Funktion hat Auswirkungen auf folgende Quellen:
\begin{itemize}
	\item Tabelle mit Links zwischen Bildern und Interessenpunkten
\end{itemize}
Es findet bei dieser Funktion kein Abruf von Daten aus {\glqq COSP\grqq} statt. Die Antwort wird als strukturiertes Array an den Aufrufer zurückgegeben.
\newpage
\section{session}
\subsection{Allgemeines} Diese Datei startet die PHP-Session.
\begin{table}[H]
	\begin{tabular}{|c|p{11cm}|}
		\hline
		\textbf{Einbindungspunkt} & inc.php \\ \hline
		\textbf{Einbindungspunkt} & inc-sub.php \\ \hline
	\end{tabular}
\end{table}
Die Datei ist nicht direkt durch den Nutzer aufrufbar, dies wird durch folgenden Code-Ausschnitt sichergestellt:
\begin{lstlisting}[language=php]
if (!defined('NICE_PROJECT')) {
	die('Permission denied.');
}
\end{lstlisting}
Der Globale Wert {\glqq NICE\_PROJECT\grqq} wird durch für den Nutzer valide Aufrufpunkte festgelegt, z.B. {\glqq api.php\grqq}.
\newpage
\subsection{Allgemeines}
Der Sessionstart erfolgt mittels nachfolgendem Code:
\begin{lstlisting}[language=php]
session_start();
\end{lstlisting}
\subsection{Besonderheiten}
Hier ist es in Zukunft geplant die Sessiondaten in eine Datenbank zu schreiben.
\newpage
\section{settings}
\subsection{Allgemeines} Diese Datei stellt entsprechende PHp-Initialwerte ein.
\begin{table}[H]
	\begin{tabular}{|c|p{11cm}|}
		\hline
		\textbf{Einbindungspunkt} & inc.php \\ \hline
		\textbf{Einbindungspunkt} & inc-sub.php \\ \hline
	\end{tabular}
\end{table}
Die Datei ist nicht direkt durch den Nutzer aufrufbar, dies wird durch folgenden Code-Ausschnitt sichergestellt:
\begin{lstlisting}[language=php]
if (!defined('NICE_PROJECT')) {
	die('Permission denied.');
}
\end{lstlisting}
Der Globale Wert {\glqq NICE\_PROJECT\grqq} wird durch für den Nutzer valide Aufrufpunkte festgelegt, z.B. {\glqq api.php\grqq}.
\newpage
\subsection{Allgemeines}
Momentan wird die Lebenszeit der Session auf 24h gesetzt. Es erfolgt bei fast jedem Zugriff eine Bereinigung alter Sessions.
\begin{lstlisting}[language=php]
ini_set('session.gc_maxlifetime', 86400);
ini_set('session.gc_probability', 1);
ini_set('session.gc_divisor', 100);
\end{lstlisting}
\subsection{Besonderheiten}
Hier sind alle Projektweiten PHP-Einstellungen vorzunehmen.
\newpage
\section{statistic-calc}
\subsection{Allgemeines} Diese Datei enthält Funktionen, welche zur Darstellung statistischer Daten verwendet werden.
\begin{table}[H]
	\begin{tabular}{|c|p{11cm}|}
		\hline
		\textbf{Einbindungspunkt} & inc.php \\ \hline
		\textbf{Einbindungspunkt} & inc-sub.php \\ \hline
	\end{tabular}
\end{table}
Die Datei ist nicht direkt durch den Nutzer aufrufbar, dies wird durch folgenden Code-Ausschnitt sichergestellt:
\begin{lstlisting}[language=php]
if (!defined('NICE_PROJECT')) {
	die('Permission denied.');
}
\end{lstlisting}
Der Globale Wert {\glqq NICE\_PROJECT\grqq} wird durch für den Nutzer valide Aufrufpunkte festgelegt, z.B. {\glqq api.php\grqq}.
\newpage
\subsection{Funktionen}
\subsubsection{loginStatistics}
\paragraph{Parameter} Die Funktion besitzt folgende Parameter:
\begin{table}[H]
	\begin{tabular}{|c|p{11cm}|}
		\hline
		\textbf{Parametername} & \textbf{Parameterbeschreibung} \\ \hline
		\$Input & Eingabedaten als Array \\ \hline
	\end{tabular}
\end{table}
\subparagraph{\$json}Das Array enthält Einträge mit folgenden Elemente:
\begin{table}[H]
	\begin{tabular}{|c|p{11cm}|}
		\hline
		\textbf{Parametername} & \textbf{Parameterbeschreibung} \\ \hline
		Amount & Anzahl an Zeiteinheiten \\ \hline
		type   & Typ der Zeiteinheit \\ \hline
	\end{tabular}
\end{table}
\paragraph{Beschreibung} Die Funktion erstellt ein Datenarray, welches zur Anzeige der Nutzungsstatistiken verwendet wird. Die Funktion nutzt folgende Quellen:
\begin{itemize}
	\item Tabelle mit statistischen Nutzungsdaten
\end{itemize}
Es findet bei dieser Funktion kein Abruf von Daten aus {\glqq COSP\grqq} statt. Die Antwort wird als strukturiertes Array an den Aufrufer zurückgegeben.
\subsubsection{fillUnknownData}
\paragraph{Parameter} Die Funktion besitzt folgende Parameter:
\begin{table}[H]
	\begin{tabular}{|c|p{11cm}|}
		\hline
		\textbf{Parametername} & \textbf{Parameterbeschreibung} \\ \hline
		\$statisticalData & Array mit statistischen Daten \\ \hline
		\$periodeAmount   & Anzahl an Zeiteinheiten \\ \hline
		\$type            & Typ der Zeiteinheit \\ \hline
	\end{tabular}
\end{table}
\paragraph{Beschreibung} Die Funktion füllt fehlende Einträge in den statistischen Daten mit dem Wert {\glqq 0\grqq} auf. Es findet bei dieser Funktion kein Abruf von Daten aus {\glqq COSP\grqq} statt. Die Antwort wird als strukturiertes Array an den Aufrufer zurückgegeben.
\subsubsection{createGraph}
\paragraph{Parameter} Die Funktion besitzt folgende Parameter:
\begin{table}[H]
	\begin{tabular}{|c|p{11cm}|}
		\hline
		\textbf{Parametername} & \textbf{Parameterbeschreibung} \\ \hline
		\$data      & statistische Daten in Array-Form \\ \hline
		\$colorBg   & Füllfarbe \\ \hline
		\$colorFont & Schriftfarbe \\ \hline
		\$label     & Name des Datensatzes \\ \hline
		\$fill      & Gibt an, ob Graph gefüllt werden soll \\ \hline
	\end{tabular}
\end{table}
\paragraph{Beschreibung} Die Funktion generiert einen durch das Chart.js anzeigbaren Datensatz. Es findet bei dieser Funktion kein Abruf von Daten aus {\glqq COSP\grqq} statt. Die Antwort wird als strukturiertes Array an den Aufrufer zurückgegeben.
\subsubsection{CreateStatistics}
\paragraph{Parameter} Die Funktion besitzt folgende Parameter:
\begin{table}[H]
	\begin{tabular}{|c|p{11cm}|}
		\hline
		\textbf{Parametername} & \textbf{Parameterbeschreibung} \\ \hline
		\$Input  & Eingabedaten als Array \\ \hline
		\$source & Quelle der statistischen Daten \\ \hline
	\end{tabular}
\end{table}
\subparagraph{\$json}Das Array enthält Einträge mit folgenden Elemente:
\begin{table}[H]
	\begin{tabular}{|c|p{11cm}|}
		\hline
		\textbf{Parametername} & \textbf{Parameterbeschreibung} \\ \hline
		Amount & Anzahl an Zeiteinheiten \\ \hline
		type   & Typ der Zeiteinheit \\ \hline
	\end{tabular}
\end{table}
\paragraph{Beschreibung} Die Funktion erstellt ein Datenarray, welches zur Anzeige der Statistik zu neuen/geänderten Interessenpunkten oder Kommentaren verwendet wird. Die Funktion nutzt folgende Quellen:
\begin{itemize}
	\item Interessenpunkt-Tabelle
	\item Kommentar-Tabelle
\end{itemize}
Es findet bei dieser Funktion kein Abruf von Daten aus {\glqq COSP\grqq} statt. Die Antwort wird als strukturiertes Array an den Aufrufer zurückgegeben.
\subsubsection{correctData}
\paragraph{Parameter} Die Funktion besitzt folgende Parameter:
\begin{table}[H]
	\begin{tabular}{|c|p{11cm}|}
		\hline
		\textbf{Parametername} & \textbf{Parameterbeschreibung} \\ \hline
		\$inputArray  & Eingabedaten als Array \\ \hline
	\end{tabular}
\end{table}
\paragraph{Beschreibung} Die Funktion korrigiert fehlerhafte statistische Daten. Die Antwort wird als strukturiertes Array an den Aufrufer zurückgegeben.
\newpage
\section{announcement}
\subsection{Allgemeines} Diese Datei enhält alle separat für die Ankündigungsseite verwendete Funktionen.
Die Ausführung des Codes findet im Browser statt.
\subsection{Funktionen}
\subsubsection{addAnnouncement}
\paragraph{Parameter} Die Funktion besitzt keine Parameter.
\paragraph{Beschreibung} Die Funktion legt eine neue Ankündigung an. Es findet bei dieser Funktion kein Abruf von Daten aus {\glqq COSP\grqq} statt.
\subsubsection{setFaultAnnouncement}
\paragraph{Parameter} Die Funktion besitzt folgende Parameter:
\begin{table}[H]
	\begin{tabular}{|c|p{11cm}|}
		\hline
		\textbf{Parametername} & \textbf{Parameterbeschreibung} \\ \hline
		fieldName & Identifikator eines Eingabefeldes \\ \hline
	\end{tabular}
\end{table}
\paragraph{Beschreibung} Die Funktion markiert ein Eingabefeld für den Nutzer als falsch ausgefüllt. Es findet bei dieser Funktion kein Abruf von Daten aus {\glqq COSP\grqq} statt.
\subsubsection{resetFaultAnnouncement}
\paragraph{Parameter} Die Funktion besitzt folgende Parameter:
\begin{table}[H]
	\begin{tabular}{|c|p{11cm}|}
		\hline
		\textbf{Parametername} & \textbf{Parameterbeschreibung} \\ \hline
		fieldName & Identifikator eines Eingabefeldes \\ \hline
	\end{tabular}
\end{table}
\paragraph{Beschreibung} Die Funktion entfernt eine Fehlermarkierung von einem Eingabefeld, das Ergebnis ist für den Nutzer sichtbar. Es findet bei dieser Funktion kein Abruf von Daten aus {\glqq COSP\grqq} statt.
\subsubsection{openEditAnnouncement}
\paragraph{Parameter} Die Funktion besitzt folgende Parameter:
\begin{table}[H]
	\begin{tabular}{|c|p{11cm}|}
		\hline
		\textbf{Parametername} & \textbf{Parameterbeschreibung} \\ \hline
		id & Identifikator einer Ankündigung \\ \hline
	\end{tabular}
\end{table}
\paragraph{Beschreibung} Die Funktion ruft alle Daten zum Bearbeiten einer Ankündigung ab und öffnet das entsprechende Modal. Es findet bei dieser Funktion kein Abruf von Daten aus {\glqq COSP\grqq} statt.
\subsubsection{saveEditAnnouncement}
\paragraph{Parameter} Die Funktion besitzt keine Parameter.
\paragraph{Beschreibung} Die Funktion ruft speichert eine bearbeitet Ankündigung und lädt die Seite neu. Es findet bei dieser Funktion kein Abruf von Daten aus {\glqq COSP\grqq} statt.
\subsubsection{addPreview}
\paragraph{Parameter} Die Funktion besitzt keine Parameter.
\paragraph{Beschreibung} Die Funktion öffnet die Vorschau des Inhaltes der Ankündigung beim Hinzufügen einer neuen Ankündigung. Es findet bei dieser Funktion kein Abruf von Daten aus {\glqq COSP\grqq} statt.
\subsubsection{editPreview}
\paragraph{Parameter} Die Funktion besitzt keine Parameter.
\paragraph{Beschreibung} Die Funktion öffnet die Vorschau des Inhaltes der Ankündigung beim Ändern einer Ankündigung. Es findet bei dieser Funktion kein Abruf von Daten aus {\glqq COSP\grqq} statt.
\subsubsection{setAktivionStateAnnouncement}
\paragraph{Parameter} Die Funktion besitzt folgende Parameter:
\begin{table}[H]
	\begin{tabular}{|c|p{11cm}|}
		\hline
		\textbf{Parametername} & \textbf{Parameterbeschreibung} \\ \hline
		id    & Identifikator einer Ankündigung \\ \hline
		state & Aktivierungsstatus einer Ankündigung \\ \hline
	\end{tabular}
\end{table}
\paragraph{Beschreibung} Die Funktion setzt den Aktivierungsstatus einer Ankündigung. Es findet bei dieser Funktion kein Abruf von Daten aus {\glqq COSP\grqq} statt.
\subsubsection{aktivateAnnouncement}
\paragraph{Parameter} Die Funktion besitzt folgende Parameter:
\begin{table}[H]
	\begin{tabular}{|c|p{11cm}|}
		\hline
		\textbf{Parametername} & \textbf{Parameterbeschreibung} \\ \hline
		id    & Identifikator einer Ankündigung \\ \hline
	\end{tabular}
\end{table}
\paragraph{Beschreibung} Die Funktion aktiviert eine Ankündigung. Es findet bei dieser Funktion kein Abruf von Daten aus {\glqq COSP\grqq} statt.
\subsubsection{deaktivateAnnouncement}
\paragraph{Parameter} Die Funktion besitzt folgende Parameter:
\begin{table}[H]
	\begin{tabular}{|c|p{11cm}|}
		\hline
		\textbf{Parametername} & \textbf{Parameterbeschreibung} \\ \hline
		id    & Identifikator einer Ankündigung \\ \hline
	\end{tabular}
\end{table}
\paragraph{Beschreibung} Die Funktion deaktiviert eine Ankündigung. Es findet bei dieser Funktion kein Abruf von Daten aus {\glqq COSP\grqq} statt.
\subsubsection{openAnnouncementPreview}
\paragraph{Parameter} Die Funktion besitzt folgende Parameter:
\begin{table}[H]
	\begin{tabular}{|c|p{11cm}|}
		\hline
		\textbf{Parametername} & \textbf{Parameterbeschreibung} \\ \hline
		id    & Identifikator einer Ankündigung \\ \hline
	\end{tabular}
\end{table}
\paragraph{Beschreibung} Die Funktion öffnet die Previewansicht einer Ankündigung. Es findet bei dieser Funktion kein Abruf von Daten aus {\glqq COSP\grqq} statt.
\newpage
\section{contact}
\subsection{Allgemeines} Diese Datei stellt ein Kontaktformular dar.
Die Datei ist direkt durch den Nutzer aufrufbar. Sie setzt auch die entsprechende Konstante und bindet alle notwendigen Dateien ein:
\begin{lstlisting}[language=php]
define('NICE_PROJECT', true);
require_once "bin/inc.php";
\end{lstlisting}
\subsection{Allgemeines}
Die Seite verfügt über Eingabefelder eines Kontaktformulars.
\subsection{Besonderheiten}
Diese Seite dient auch zum Berichten auftretender Fehler.
\newpage
\section{editPoi}
\subsection{Allgemeines} Diese Datei dient dem ändern eines bereits bestehenden Interessenpunktes.
Die Datei ist direkt durch den Nutzer aufrufbar. Sie setzt auch die entsprechende Konstante und bindet alle notwendigen Dateien ein:
\begin{lstlisting}[language=php]
define('NICE_PROJECT', true);
require_once "bin/inc.php";
\end{lstlisting}
\subsection{Allgemeines}
Die Seite verfügt über alle notwendigen Eingabefelder für eine Bearbeitung bereits bestehender Interessenpunkte. Es ist ebenfalls ein verschieben der Position eines Interessenpunktes möglich. Es wird die Verarbeitung der entsprechenden Daten nach der Eingabe ebenfalls vorgenommen.
\subsection{Besonderheiten}
Diese Seite bindet eine Karte mit einem beweglichen Marker ein. Die hierfür benötigten Scripte werden zusätzlich zu allen überall notwendigen Scripten geladen.
\newpage
\section{api}
\subsection{Allgemeines} Diese Datei ist der Endpunkt der Frontend-API.
Die Datei ist direkt durch den Nutzer aufrufbar. Sie setzt auch die entsprechende Konstante und bindet alle notwendigen Dateien ein:
\begin{lstlisting}[language=php]
define('NICE_PROJECT', true);
require_once "bin/inc.php";
\end{lstlisting}
\subsection{Allgemeines}
Diese Seite verteilt die API-Anfragen auf die verschiedenen Funktionen auf. Für eine Liste aller möglichen API-Befehle siehe \autoref{api}.
\subsection{Besonderheiten}
Diese Seite bietet keine Graphische Nutzeroberfläche an. Alle Antworten auf Anfragen sind im JSON-Format.
\newpage
\section{rapi}
\subsection{Allgemeines} Diese Datei ist der Endpunkt der Reverse-API.
Die Datei ist direkt durch den Nutzer aufrufbar. Sie setzt auch die entsprechende Konstante und bindet alle notwendigen Dateien ein:
\begin{lstlisting}[language=php]
define('NICE_PROJECT', true);
require_once "bin/inc.php";
\end{lstlisting}
\subsection{Allgemeines}
Diese Seite verteilt die Reverse-API-Anfragen auf die verschiedenen Funktionen auf. Für eine Liste aller möglichen Reverse-API-Befehle siehe \autoref{rapi}.
\subsection{Besonderheiten}
Diese Seite bietet keine Graphische Nutzeroberfläche an. Alle Antworten auf Anfragen sind im JSON-Format.
\newpage
\section{hub}
\subsection{Allgemeines} Diese Datei enhält alle separat für den Hub verwendete Funktionen.
Die Ausführung des Codes findet im Browser statt.
\subsection{Funktionen}
\subsubsection{closeAnnouncement}
\paragraph{Parameter} Die Funktion besitzt keine Parameter.
\paragraph{Beschreibung} Die Funktion schließt das Ankündigungsmodal und setzt eine Markierung, dass die Ankündigung nicht erneut angezeigt wird. Es findet bei dieser Funktion kein Abruf von Daten aus {\glqq COSP\grqq} statt.
\subsubsection{loadAnnouncement}
\paragraph{Parameter} Die Funktion besitzt keine Parameter.
\paragraph{Beschreibung} Die Funktion lädt das Ankündigungsmodal mit entsprechenden Daten. Es findet bei dieser Funktion kein Abruf von Daten aus {\glqq COSP\grqq} statt.
\newpage
\section{impressum}
\subsection{Allgemeines} Diese Datei zeigt das Impressum an.
Die Datei ist direkt durch den Nutzer aufrufbar. Sie setzt auch die entsprechende Konstante und bindet alle notwendigen Dateien ein:
\begin{lstlisting}[language=php]
	define('NICE_PROJECT', true);
	require_once "bin/inc.php";
\end{lstlisting}
\subsection{Allgemeines}
Hier steht für jeden sichtbar das Impressum. Die Seite unterstützt keine Funktionen.
\subsection{Besonderheiten}
Die Seite ist im Debug-Modus nicht für alle Sichtbar. Es soll hier auch die Adresse, des Verantwortlichen mittels Konfigurationsdatei geändert werden können. Hierfür sind die Konfigurationsflags \autoref{config:impressum-name}, \autoref{config:impressum-street} und \autoref{config:impressum-city} gedacht.
\newpage
\section{index}
\subsection{Allgemeines} Diese Datei dient dem Login und ist die erste Seite auf welcher der Nutzer landet.
Die Datei ist direkt durch den Nutzer aufrufbar. Sie setzt auch die entsprechende Konstante und bindet alle notwendigen Dateien ein:
\begin{lstlisting}[language=php]
	define('NICE_PROJECT', true);
	require_once "bin/inc.php";
\end{lstlisting}
\subsection{Allgemeines}
Auf dieser Seite kann sich der Nutzer authentifizieren oder sich als Gast anmelden. Alternativ kann hat er die Option über den Registrieren-Button zur Selbstregistrierung zu gelangen. Des Weiteren wird der Nutzer auch auf die Verwendung von Cookies aufmerksam gemacht.
\subsection{Besonderheiten}
Die Seite besitzt eine vereinfachte Navbar, da zu diesem Zeitpunkt nicht alle Funktionen der Navbar zur Verfügung stehen.
\newpage
\section{ListMaterial}
\subsection{Allgemeines} Diese Datei zeigt eine Liste aller hochgeladenen Bilder an.
Die Datei ist direkt durch den Nutzer aufrufbar. Sie setzt auch die entsprechende Konstante und bindet alle notwendigen Dateien ein:
\begin{lstlisting}[language=php]
	define('NICE_PROJECT', true);
	require_once "bin/inc.php";
\end{lstlisting}
\subsection{Allgemeines}
Auf dieser Seite besteht die Option ein Bild zu validieren, zu löschen oder zu die Metadaten zu bearbeiten. Des Weiteren ist es möglich ein Bild mit einem oder mehreren Interessenpunkten zu verknüpfen. Die Seite besteht aus einer Tabelle aller Bilder in nicht sortierter Reihenfolge. Es werden tabellarisch zu jedem Eintrag ein Vorschaubild, der Titel, die Beschreibung und der hochladende Nutzer angezeigt.
\subsection{Besonderheiten}
Die Seite ist auch als Gast verfügbar, jedoch nur im angemeldeten Zustand sichtbar.
\newpage
\section{logoutpage}
\subsection{Allgemeines} Diese Datei zeigt den Logout-Dialog an.
Die Datei ist direkt durch den Nutzer aufrufbar. Sie setzt auch die entsprechende Konstante und bindet alle notwendigen Dateien ein:
\begin{lstlisting}[language=php]
	define('NICE_PROJECT', true);
	require_once "bin/inc.php";
\end{lstlisting}
\subsection{Allgemeines}
Diese Seite führt den gewollten Logout durch und leitet, wenn gewollt, auf den Login weiter. 

\newpage
\section{map}
\subsection{Allgemeines} Diese Datei zeigt eine Karte mit allen für den Nutzer sichtbaren eingetragenen Interessenpunkten an.
Die Datei ist direkt durch den Nutzer aufrufbar. Sie setzt auch die entsprechende Konstante und bindet alle notwendigen Dateien ein:
\begin{lstlisting}[language=php]
	define('NICE_PROJECT', true);
	require_once "bin/inc.php";
\end{lstlisting}
\subsection{Allgemeines}
Die Seite besteht aus einer Karte, auf welcher Interessenpunkte mit entsprechenden Daten vorhanden sind. Es besteht auch die Möglichkeit, neue Interessenpunkte ein zu tragen. Zu bereits bestehenden Interessenpunkten können auch zusätzliche Informationen, wie zum Beispiel die Anzahl der Sitzplätze in einem bestimmten Zeitraum, die Anzahl der Kinosäle in einem bestimmten Zeitraum oder auch die Betreiber, hinzugefügt werden.
\subsection{Besonderheiten}
Die Seite lädt auch das Hauptbild eines Interessenpunktes auf {\glqq COSP\grqq} hoch.
\newpage
\section{MaterialUpload}
\subsection{Allgemeines} Diese Datei dient dem Upload neuer Bilder.
Die Datei ist direkt durch den Nutzer aufrufbar. Sie setzt auch die entsprechende Konstante und bindet alle notwendigen Dateien ein:
\begin{lstlisting}[language=php]
	define('NICE_PROJECT', true);
	require_once "bin/inc.php";
\end{lstlisting}
\subsection{Allgemeines}
Auf dieser Seite kann der Nutzer ein neues Bild hochladen, hierzu ist die Angabe eines Titels erforderlich. Eine Beschreibung des Bildes kann optional ebenfalls erfolgen. Des Weiteren muss er auch Bestätigen, dass er die Rechte am Bild besitzt.
\subsection{Besonderheiten}
Die Seite lädt ein Bild auf {\glqq COSP\grqq} hoch.
\newpage
\section{poimgmt}
\subsection{Allgemeines} Diese Datei zeigt eine Auflistung aller Interessenpunkte an.
Die Datei ist direkt durch den Nutzer aufrufbar. Sie setzt auch die entsprechende Konstante und bindet alle notwendigen Dateien ein:
\begin{lstlisting}[language=php]
	define('NICE_PROJECT', true);
	require_once "bin/inc.php";
\end{lstlisting}
\subsection{Allgemeines}
Auf dieser Seite kann der Nutzer eine Liste aller Interessenpunkte einsehen. Des Weiteren kann er auch die Interessenpunkte validieren, löschen und bearbeiten. Die Interessenpunkte sind tabellarisch mit Titel, Identifikator, aktueller Adresse und Ersteller gelistet.
\subsection{Besonderheiten}
Die Seite nutzt zum Ausführen der Funktionen teilweise JavaScript.
\newpage
\section{privacy-policy}
\subsection{Allgemeines} Diese Datei zeigt die Datenschutzerklärung an.
Die Datei ist direkt durch den Nutzer aufrufbar. Sie setzt auch die entsprechende Konstante und bindet alle notwendigen Dateien ein:
\begin{lstlisting}[language=php]
	define('NICE_PROJECT', true);
	require_once "bin/inc.php";
\end{lstlisting}
\subsection{Allgemeines}
Auf dieser Seite kann der Nutzer Datenschutzerklärung lesen. Diese Seite ist für jeden Sichtbar.
\subsection{Besonderheiten}
Die Seite wird im Debug-Modus nicht für jeden sichtbar angezeigt. Der Name des Verantwortlichen kann mittels der Konfigurationsparameter \autoref{config:privacy-comp-name}, \autoref{config:privacy-comp-street}, \autoref{config:privacy-comp-city}, \autoref{config:privacy-comp-fon}, \autoref{config:privacy-comp-fax} und \autoref{config:privacy-comp-mail}. Auch kann der Datenschutzverantwortliche mittels der Konfigurationsparameter \autoref{config:privacy-rep-name}, \autoref{config:privacy-rep-pos}, \autoref{config:privacy-rep-street}, \autoref{config:privacy-rep-city}, \autoref{config:privacy-rep-fon}, \autoref{config:privacy-rep-fax} und \autoref{config:privacy-rep-mail} gesetzt werden.
\newpage
\section{ranklist}
\subsection{Allgemeines} Diese Datei zeigt eine Reihenfolge der besten und aktivsten Nutzer an.
Die Datei ist direkt durch den Nutzer aufrufbar. Sie setzt auch die entsprechende Konstante und bindet alle notwendigen Dateien ein:
\begin{lstlisting}[language=php]
	define('NICE_PROJECT', true);
	require_once "bin/inc.php";
\end{lstlisting}
\subsection{Allgemeines}
Auf dieser Seite kann der Nutzer eine Liste aller Nutzer sehen, welche nach den Verdienten Rangpunkten geordnet ist. Die Seite hat keine weitere Funktion.
\subsection{Besonderheiten}
Die Seite ruft die benötigten Informationen aus {\glqq COSP\grqq} ab.
\newpage
\section{registration}
\subsection{Allgemeines} Diese Datei zeigt ein Formular zur Selbstregistrierung an.
Die Datei ist direkt durch den Nutzer aufrufbar. Sie setzt auch die entsprechende Konstante und bindet alle notwendigen Dateien ein:
\begin{lstlisting}[language=php]
	define('NICE_PROJECT', true);
	require_once "bin/inc.php";
\end{lstlisting}
\subsection{Allgemeines}
Auf dieser Seite kann sich Nutzer selbst registrieren.
\subsection{Besonderheiten}
Die Seite sendet die für eine Registrierung benötigten Informationen an {\glqq COSP\grqq}.
\newpage
\section{statistics}
\subsection{Allgemeines} Diese Datei zeigt statistische Grafen an.
Die Datei ist direkt durch den Nutzer aufrufbar. Sie setzt auch die entsprechende Konstante und bindet alle notwendigen Dateien ein:
\begin{lstlisting}[language=php]
define('NICE_PROJECT', true);
require_once "bin/inc.php";
\end{lstlisting}
\subsection{Allgemeines}
Auf dieser Seite sind mittels {\glqq Chart.js\grqq} erzeugte Diagramme für statistische Daten wie neue oder geänderte Interessenpunkte beziehungsweise Kommentare zu sehen.
\subsection{Besonderheiten}
Diese Seite kann nur durch Mitarbeiter und Administratoren eingesehen werden.
\newpage
\section{StoryUpload}
\subsection{Allgemeines} Diese Datei dient enthält alle Funktionen, welche für die Geschichten-Seite zusätzlich benötigt werden.
Die Ausführung des Codes findet im Browser statt. Hier wird eine Variable für die Sortierung der Geschichten gesetzt:
\begin{lstlisting}[language=JavaScript]
var sortdown = true;
\end{lstlisting} 
Des Weiteren wird auch das initiale Laden der Daten hier ausgelöst:
\begin{lstlisting}[language=JavaScript]
window.onload = function () {
	getAllStories();
};
\end{lstlisting} 
\newpage
\subsection{Funktionen}
\subsubsection{updateSortType}
\paragraph{Parameter} Die Funktion besitzt folgende Parameter:
\begin{table}[H]
	\begin{tabular}{|c|p{11cm}|}
		\hline
		\textbf{Parametername} & \textbf{Parameterbeschreibung} \\ \hline
		sortDownState & Wahr, wenn Absteigend sortiert werden soll \\ \hline
	\end{tabular}
\end{table}
\paragraph{Beschreibung} Die Funktion updated die Variable, welche die Sortierung der Geschichten angibt und löst eine erneute Sortierung aus, bei welcher anschließend auch die Anzeige aktualisiert wird. Es findet bei dieser Funktion kein Abruf von Daten aus {\glqq COSP\grqq} statt.
\subsubsection{FilterStorys}
\paragraph{Parameter} Die Funktion besitzt keine Parameter.
\paragraph{Beschreibung} Die Funktion löst eine erneute Sortierung und Filterung der Geschichten aus, bei welcher anschließend auch die Anzeige aktualisiert wird. Es findet bei dieser Funktion kein Abruf von Daten aus {\glqq COSP\grqq} statt.
\subsubsection{saveLinkedPoi}
\paragraph{Parameter} Die Funktion besitzt keine Parameter.
\paragraph{Beschreibung} Die Funktion speichert einen Link zwischen einer Geschichte und einem Interessenpunkt. Die Funktion hat Auswirkungen auf folgende Quellen:
\begin{itemize}
	\item Frontend-API
\end{itemize}
Es findet bei dieser Funktion kein Abruf von Daten aus {\glqq COSP\grqq} statt.
\newpage
\section{announcement}
\subsection{Allgemeines} Diese Datei enhält alle separat für die Ankündigungsseite verwendete Funktionen.
Die Ausführung des Codes findet im Browser statt.
\subsection{Funktionen}
\subsubsection{addAnnouncement}
\paragraph{Parameter} Die Funktion besitzt keine Parameter.
\paragraph{Beschreibung} Die Funktion legt eine neue Ankündigung an. Es findet bei dieser Funktion kein Abruf von Daten aus {\glqq COSP\grqq} statt.
\subsubsection{setFaultAnnouncement}
\paragraph{Parameter} Die Funktion besitzt folgende Parameter:
\begin{table}[H]
	\begin{tabular}{|c|p{11cm}|}
		\hline
		\textbf{Parametername} & \textbf{Parameterbeschreibung} \\ \hline
		fieldName & Identifikator eines Eingabefeldes \\ \hline
	\end{tabular}
\end{table}
\paragraph{Beschreibung} Die Funktion markiert ein Eingabefeld für den Nutzer als falsch ausgefüllt. Es findet bei dieser Funktion kein Abruf von Daten aus {\glqq COSP\grqq} statt.
\subsubsection{resetFaultAnnouncement}
\paragraph{Parameter} Die Funktion besitzt folgende Parameter:
\begin{table}[H]
	\begin{tabular}{|c|p{11cm}|}
		\hline
		\textbf{Parametername} & \textbf{Parameterbeschreibung} \\ \hline
		fieldName & Identifikator eines Eingabefeldes \\ \hline
	\end{tabular}
\end{table}
\paragraph{Beschreibung} Die Funktion entfernt eine Fehlermarkierung von einem Eingabefeld, das Ergebnis ist für den Nutzer sichtbar. Es findet bei dieser Funktion kein Abruf von Daten aus {\glqq COSP\grqq} statt.
\subsubsection{openEditAnnouncement}
\paragraph{Parameter} Die Funktion besitzt folgende Parameter:
\begin{table}[H]
	\begin{tabular}{|c|p{11cm}|}
		\hline
		\textbf{Parametername} & \textbf{Parameterbeschreibung} \\ \hline
		id & Identifikator einer Ankündigung \\ \hline
	\end{tabular}
\end{table}
\paragraph{Beschreibung} Die Funktion ruft alle Daten zum Bearbeiten einer Ankündigung ab und öffnet das entsprechende Modal. Es findet bei dieser Funktion kein Abruf von Daten aus {\glqq COSP\grqq} statt.
\subsubsection{saveEditAnnouncement}
\paragraph{Parameter} Die Funktion besitzt keine Parameter.
\paragraph{Beschreibung} Die Funktion ruft speichert eine bearbeitet Ankündigung und lädt die Seite neu. Es findet bei dieser Funktion kein Abruf von Daten aus {\glqq COSP\grqq} statt.
\subsubsection{addPreview}
\paragraph{Parameter} Die Funktion besitzt keine Parameter.
\paragraph{Beschreibung} Die Funktion öffnet die Vorschau des Inhaltes der Ankündigung beim Hinzufügen einer neuen Ankündigung. Es findet bei dieser Funktion kein Abruf von Daten aus {\glqq COSP\grqq} statt.
\subsubsection{editPreview}
\paragraph{Parameter} Die Funktion besitzt keine Parameter.
\paragraph{Beschreibung} Die Funktion öffnet die Vorschau des Inhaltes der Ankündigung beim Ändern einer Ankündigung. Es findet bei dieser Funktion kein Abruf von Daten aus {\glqq COSP\grqq} statt.
\subsubsection{setAktivionStateAnnouncement}
\paragraph{Parameter} Die Funktion besitzt folgende Parameter:
\begin{table}[H]
	\begin{tabular}{|c|p{11cm}|}
		\hline
		\textbf{Parametername} & \textbf{Parameterbeschreibung} \\ \hline
		id    & Identifikator einer Ankündigung \\ \hline
		state & Aktivierungsstatus einer Ankündigung \\ \hline
	\end{tabular}
\end{table}
\paragraph{Beschreibung} Die Funktion setzt den Aktivierungsstatus einer Ankündigung. Es findet bei dieser Funktion kein Abruf von Daten aus {\glqq COSP\grqq} statt.
\subsubsection{aktivateAnnouncement}
\paragraph{Parameter} Die Funktion besitzt folgende Parameter:
\begin{table}[H]
	\begin{tabular}{|c|p{11cm}|}
		\hline
		\textbf{Parametername} & \textbf{Parameterbeschreibung} \\ \hline
		id    & Identifikator einer Ankündigung \\ \hline
	\end{tabular}
\end{table}
\paragraph{Beschreibung} Die Funktion aktiviert eine Ankündigung. Es findet bei dieser Funktion kein Abruf von Daten aus {\glqq COSP\grqq} statt.
\subsubsection{deaktivateAnnouncement}
\paragraph{Parameter} Die Funktion besitzt folgende Parameter:
\begin{table}[H]
	\begin{tabular}{|c|p{11cm}|}
		\hline
		\textbf{Parametername} & \textbf{Parameterbeschreibung} \\ \hline
		id    & Identifikator einer Ankündigung \\ \hline
	\end{tabular}
\end{table}
\paragraph{Beschreibung} Die Funktion deaktiviert eine Ankündigung. Es findet bei dieser Funktion kein Abruf von Daten aus {\glqq COSP\grqq} statt.
\subsubsection{openAnnouncementPreview}
\paragraph{Parameter} Die Funktion besitzt folgende Parameter:
\begin{table}[H]
	\begin{tabular}{|c|p{11cm}|}
		\hline
		\textbf{Parametername} & \textbf{Parameterbeschreibung} \\ \hline
		id    & Identifikator einer Ankündigung \\ \hline
	\end{tabular}
\end{table}
\paragraph{Beschreibung} Die Funktion öffnet die Previewansicht einer Ankündigung. Es findet bei dieser Funktion kein Abruf von Daten aus {\glqq COSP\grqq} statt.
\newpage
\section{Control.Geocoder}
\subsection{Allgemeines} Diese Datei enthält alle Funktionen, welche für eine Adresssuche auf der Karte benötigt werden. Hier beschrieben wird nur eine geänderte Funktionalität.
Die Ausführung des Codes findet im Browser statt.
\subsection{Funktionen}
\subsubsection{Geocoder.markGeocode}
\paragraph{Parameter} Die Funktion besitzt keine Parameter.
\paragraph{Beschreibung} Die Funktion ermittelt die Koordinaten der gewünschten Adresse und platziert mit einer Funktion dort den Kartenmarker. Es findet bei dieser Funktion kein Abruf von Daten aus {\glqq COSP\grqq} statt.

\newpage
\section{hub}
\subsection{Allgemeines} Diese Datei enhält alle separat für den Hub verwendete Funktionen.
Die Ausführung des Codes findet im Browser statt.
\subsection{Funktionen}
\subsubsection{closeAnnouncement}
\paragraph{Parameter} Die Funktion besitzt keine Parameter.
\paragraph{Beschreibung} Die Funktion schließt das Ankündigungsmodal und setzt eine Markierung, dass die Ankündigung nicht erneut angezeigt wird. Es findet bei dieser Funktion kein Abruf von Daten aus {\glqq COSP\grqq} statt.
\subsubsection{loadAnnouncement}
\paragraph{Parameter} Die Funktion besitzt keine Parameter.
\paragraph{Beschreibung} Die Funktion lädt das Ankündigungsmodal mit entsprechenden Daten. Es findet bei dieser Funktion kein Abruf von Daten aus {\glqq COSP\grqq} statt.
\newpage
\section{editmap}
\subsection{Allgemeines} Diese Datei initialisiert die Karte auf der {\glqq Interessenpunkt bearbeiten\grqq}-Seite.
Die Ausführung des Codes findet im Browser statt. Das Script ruft zuerst den Längen- und Breitengrad aus den entsprechenden Formularfeldern ab und setzt anschließend einen Marker mit diesen Informationen auf der Karte.
\subsection{Besonderheiten}
Der Marker bekommt veränderte onDrag- und onClick-Funktionen, um eine Intuitivere Bedienung zu gewährleisten.
\newpage
\section{kinoMainLib}
\lstset{
	language=JavaScript,
	extendedchars=true,
	basicstyle= \small\ttfamily,
	showstringspaces=true,
	showspaces=false,
	tabsize=2,
	breaklines=true,
	showtabs=false,
	captionpos=b,
	showlines=true,
	xleftmargin=4.0ex,
	extendedchars=true,
	literate={ä}{{\"a}}1 {ö}{{\"o}}1 {ü}{{\"u}}1 {Ä}{{\"A}}1 {Ö}{{\"O}}1 {Ü}{{\"U}}1,
	breaklines=true,
	postbreak=\mbox{\textcolor{red}{$\hookrightarrow$}\space},
}
\subsection{Allgemeines} Diese Datei enthält alle grundlegenden JavaScript-Funktionen.
Die Ausführung des Codes findet im Browser statt. Auch werden hier einige in verschiedenen anderen Javascripts verwendeten Variablen festgelegt.
\begin{lstlisting}[language=JavaScript]
var stories = {};
var comments = {};
var storiesMap = {};
var guest = false;
var approver = false;
var admin = false;
\end{lstlisting}
Ebenfalls werden mittels dieser Bibliothek die Tooltipps von Elementen aktiviert:
\begin{lstlisting}[language=JavaScript]
$(document).ready(function () {
	$('[data-toggle="tooltip"]').tooltip();
});
\end{lstlisting}
\newpage
\subsection{Funktionen}
\subsubsection{sendApiRequest}
\paragraph{Parameter} Die Funktion besitzt folgende Parameter:
\begin{table}[H]
	\begin{tabular}{|c|p{11cm}|}
		\hline
		\textbf{Parametername} & \textbf{Parameterbeschreibung} \\ \hline
		json   & Strukturierte Daten der Anfrage \\ \hline
		reload & Legt Neuladen der Seite fest \\ \hline
	\end{tabular}
\end{table}
\paragraph{Beschreibung} Sendet Daten an Frontend-API-Endpunkt und gibt Ergebnis zurück. Es findet bei dieser Funktion kein Abruf von Daten aus {\glqq COSP\grqq} statt. Die Antwort wird als strukturiertes Array an den Aufrufer zurückgegeben.
\subsubsection{sendCOSPRequest}
\paragraph{Parameter} Die Funktion besitzt folgende Parameter:
\begin{table}[H]
	\begin{tabular}{|c|p{11cm}|}
		\hline
		\textbf{Parametername} & \textbf{Parameterbeschreibung} \\ \hline
		url & URI der Anfrage an {\glqq COSP\grqq} \\ \hline
	\end{tabular}
\end{table}
\paragraph{Beschreibung} Die Funktion sendet eine HTTP-Get Anfrage an {\glqq COSP\grqq} und empfängt die übermittelten Daten. Es findet bei dieser Funktion ein Abruf von Daten aus {\glqq COSP\grqq} statt oder es werden Daten an {\glqq COSP\grqq} gesendet. Die Antwort wird als strukturiertes Array an den Aufrufer zurückgegeben.
\subsubsection{sendCOSPRequestPost}
\paragraph{Parameter} Die Funktion besitzt folgende Parameter:
\begin{table}[H]
	\begin{tabular}{|c|p{11cm}|}
		\hline
		\textbf{Parametername} & \textbf{Parameterbeschreibung} \\ \hline
		url     & URI der Anfrage an {\glqq COSP\grqq} \\ \hline
		content & Inhalt der Anfrage an {\glqq COSP\grqq} \\ \hline
	\end{tabular}
\end{table}
\paragraph{Beschreibung} Die Funktion sendet eine HTTP-Post Anfrage an {\glqq COSP\grqq} und empfängt die übermittelten Daten. Es findet bei dieser Funktion ein Abruf von Daten aus {\glqq COSP\grqq} statt oder es werden Daten an {\glqq COSP\grqq} gesendet. Die Antwort wird als strukturiertes Array an den Aufrufer zurückgegeben.
\subsubsection{sendApiRequest}
\paragraph{Parameter} Die Funktion besitzt keine Parameter.
\paragraph{Beschreibung} Die Funktion wählt eine entsprechende Sortierfunktion zum Sortieren der von Benutzern hochgeladenen Geschichten. Es findet bei dieser Funktion kein Abruf von Daten aus {\glqq COSP\grqq} statt.
\subsubsection{SortStoriesByDateDown}
\paragraph{Parameter} Die Funktion besitzt keine Parameter.
\paragraph{Beschreibung} Die Funktion sortiert durch Nutzer hochgeladene Geschichten nach Datum absteigend. Die Funktion ist eine Implementierung von Bubble-Sort. Es findet bei dieser Funktion kein Abruf von Daten aus {\glqq COSP\grqq} statt.
\subsubsection{SortStoriesByDateUp}
\paragraph{Parameter} Die Funktion besitzt keine Parameter.
\paragraph{Beschreibung} Die Funktion sortiert durch Nutzer hochgeladene Geschichten nach Datum aufsteigend. Die Funktion ist eine Implementierung von Bubble-Sort. Es findet bei dieser Funktion kein Abruf von Daten aus {\glqq COSP\grqq} statt.
\subsubsection{getAllStories}
\paragraph{Parameter} Die Funktion besitzt keine Parameter.
\paragraph{Beschreibung} Die Funktion fragt alle durch Nutzer hochgeladenen Geschichten aus {\glqq COSP\grqq} ab. Die Funktion nutzt folgende Quellen:
\begin{itemize}
	\item COSP
\end{itemize}
Es findet bei dieser Funktion kein Abruf von Daten aus {\glqq COSP\grqq} statt.
\subsubsection{sortAndDisplay}
\paragraph{Parameter} Die Funktion besitzt keine Parameter.
\paragraph{Beschreibung} Die Funktion weist die Sortierung der Nutzergeschichten an und lässt die Geschichten anschließend anzeigen. Es findet bei dieser Funktion kein Abruf von Daten aus {\glqq COSP\grqq} statt.
\subsubsection{sendApiRequest}
\paragraph{Parameter} Die Funktion besitzt folgende Parameter:
\begin{table}[H]
	\begin{tabular}{|c|p{11cm}|}
		\hline
		\textbf{Parametername} & \textbf{Parameterbeschreibung} \\ \hline
		Link       & URI zum Abrufen der Daten in {\glqq COSP\grqq} \\ \hline
		IntCounter & Speicherplatz im {\glqq stories\grqq}-Array \\ \hline
	\end{tabular}
\end{table}
\paragraph{Beschreibung} Die Funktion lädt eine einzelne Geschichte aus {\glqq COSP\grqq}. Die Funktion nutzt folgende Quellen:
\begin{itemize}
	\item COSP
\end{itemize}
Es findet bei dieser Funktion ein Abruf von Daten aus {\glqq COSP\grqq} statt.
\subsubsection{loadStory}
\paragraph{Parameter} Die Funktion besitzt folgende Parameter:
\begin{table}[H]
	\begin{tabular}{|c|p{11cm}|}
		\hline
		\textbf{Parametername} & \textbf{Parameterbeschreibung} \\ \hline
		Story      & Array mit Daten einer Geschichte \\ \hline
		IntCounter & Speicherstelle im Array \\ \hline
	\end{tabular}
\end{table}
\paragraph{Beschreibung} Die Funktion speichert die gegebenen Daten der Geschichte an die gegebene Stelle im {\glqq stories\grqq}-Array. Es findet bei dieser Funktion kein Abruf von Daten aus {\glqq COSP\grqq} statt.
\subsubsection{loadStoryText}
\paragraph{Parameter} Die Funktion besitzt folgende Parameter:
\begin{table}[H]
	\begin{tabular}{|c|p{11cm}|}
		\hline
		\textbf{Parametername} & \textbf{Parameterbeschreibung} \\ \hline
		Story      & Array mit Daten einer Geschichte \\ \hline
		IntCounter & Speicherstelle im Array \\ \hline
	\end{tabular}
\end{table}
\paragraph{Beschreibung} Die Funktion zeigt die Geschichte im dafür Entsprechenden <div>-Elements der {\glqq Upload-Story.php\grqq}-Seite an. . Es findet bei dieser Funktion kein Abruf von Daten aus {\glqq COSP\grqq} statt.
\subsubsection{finalDeleteStory}
\paragraph{Parameter} Die Funktion besitzt folgende Parameter:
\begin{table}[H]
	\begin{tabular}{|c|p{11cm}|}
		\hline
		\textbf{Parametername} & \textbf{Parameterbeschreibung} \\ \hline
		intCounter & Speicherstelle der Geschichte im Array \\ \hline
	\end{tabular}
\end{table}
\paragraph{Beschreibung} Die Funktion löscht eine Geschichte mittels API endgültig. Die Funktion hat Auswirkungen auf folgende Quellen:
\begin{itemize}
	\item COSP
\end{itemize}
Es findet bei dieser Funktion kein Abruf von Daten aus {\glqq COSP\grqq} statt. Es werden jedoch Daten an {\glqq COSP\grqq} gesendet.
\subsubsection{restoreStory}
\paragraph{Parameter} Die Funktion besitzt folgende Parameter:
\begin{table}[H]
	\begin{tabular}{|c|p{11cm}|}
		\hline
		\textbf{Parametername} & \textbf{Parameterbeschreibung} \\ \hline
		intCounter & Speicherstelle der Geschichte im Array \\ \hline
	\end{tabular}
\end{table}
\paragraph{Beschreibung} Die Funktion stellt eine Geschichte mittels API wieder her. Die Funktion hat Auswirkungen auf folgende Quellen:
\begin{itemize}
	\item COSP
\end{itemize}
Es findet bei dieser Funktion kein Abruf von Daten aus {\glqq COSP\grqq} statt. Es werden jedoch Daten an {\glqq COSP\grqq} gesendet.
\subsubsection{ApproveStory}
\paragraph{Parameter} Die Funktion besitzt folgende Parameter:
\begin{table}[H]
	\begin{tabular}{|c|p{11cm}|}
		\hline
		\textbf{Parametername} & \textbf{Parameterbeschreibung} \\ \hline
		intCounter & Speicherstelle der Geschichte im Array \\ \hline
	\end{tabular}
\end{table}
\paragraph{Beschreibung} Die Funktion gibt eine Geschichte mittels API frei. Die Funktion hat Auswirkungen auf folgende Quellen:
\begin{itemize}
	\item COSP
\end{itemize}
Es findet bei dieser Funktion kein Abruf von Daten aus {\glqq COSP\grqq} statt. Es werden jedoch Daten an {\glqq COSP\grqq} gesendet.
\subsubsection{DisapproveStory}
\paragraph{Parameter} Die Funktion besitzt folgende Parameter:
\begin{table}[H]
	\begin{tabular}{|c|p{11cm}|}
		\hline
		\textbf{Parametername} & \textbf{Parameterbeschreibung} \\ \hline
		intCounter & Speicherstelle der Geschichte im Array \\ \hline
	\end{tabular}
\end{table}
\paragraph{Beschreibung} Die Funktion sperrt eine Geschichte mittels API. Die Funktion hat Auswirkungen auf folgende Quellen:
\begin{itemize}
	\item COSP
\end{itemize}
Es findet bei dieser Funktion kein Abruf von Daten aus {\glqq COSP\grqq} statt. Es werden jedoch Daten an {\glqq COSP\grqq} gesendet.
\subsubsection{showMoreStory}
\paragraph{Parameter} Die Funktion besitzt folgende Parameter:
\begin{table}[H]
	\begin{tabular}{|c|p{11cm}|}
		\hline
		\textbf{Parametername} & \textbf{Parameterbeschreibung} \\ \hline
		intCounter & Speicherstelle der Geschichte im Array \\ \hline
	\end{tabular}
\end{table}
\paragraph{Beschreibung} Die Funktion bespielt das {\glqq vollständige Geschichte anzeigen\grqq}-Modal mit allen benötigten Daten. Es findet bei dieser Funktion kein Abruf von Daten aus {\glqq COSP\grqq} statt. Die Antwort wird als strukturiertes Array an den Aufrufer zurückgegeben.
\subsubsection{showLinksLongStory}
\paragraph{Parameter} Die Funktion besitzt folgende Parameter:
\begin{table}[H]
	\begin{tabular}{|c|p{11cm}|}
		\hline
		\textbf{Parametername} & \textbf{Parameterbeschreibung} \\ \hline
		intCounter & Speicherstelle der Geschichte im Array \\ \hline
	\end{tabular}
\end{table}
\paragraph{Beschreibung} Die Funktion schließt das {\glqq vollständige Geschichte anzeigen\grqq}-Modal und öffnet die Anzeige von Verknüpfungen zwischen Interessenpunkten und dieser Geschichte. Es findet bei dieser Funktion kein Abruf von Daten aus {\glqq COSP\grqq} statt.
\subsubsection{checkIfGuest}
\paragraph{Parameter} Die Funktion besitzt keine Parameter.
\paragraph{Beschreibung} Die Funktion prüft, ob der aktuelle Nutzer ein Gast ist. Es findet bei dieser Funktion kein Abruf von Daten aus {\glqq COSP\grqq} statt.
\subsubsection{openEditStorie}
\paragraph{Parameter} Die Funktion besitzt folgende Parameter:
\begin{table}[H]
	\begin{tabular}{|c|p{11cm}|}
		\hline
		\textbf{Parametername} & \textbf{Parameterbeschreibung} \\ \hline
		intCounter & Speicherstelle der Geschichte im Array \\ \hline
	\end{tabular}
\end{table}
\paragraph{Beschreibung} Die Funktion lädt alle benötigten Daten zum bearbeiten einer Geschichte in das entsprechende Modal und zeigt dieses an. Es findet bei dieser Funktion kein Abruf von Daten aus {\glqq COSP\grqq} statt.
\subsubsection{saveEditStorie}
\paragraph{Parameter} Die Funktion besitzt keine Parameter.
\paragraph{Beschreibung} Die Funktion speichert eine bearbeitete Geschichte. Die Funktion hat Auswirkungen auf folgende Quellen:
\begin{itemize}
	\item COSP
	\item Frontend-API
\end{itemize}
Es findet bei dieser Funktion kein Abruf von Daten aus {\glqq COSP\grqq} statt. Es werden jedoch Daten an {\glqq COSP\grqq} gesendet.
\subsubsection{showPoiLinks}
\paragraph{Parameter} Die Funktion besitzt folgende Parameter:
\begin{table}[H]
	\begin{tabular}{|c|p{11cm}|}
		\hline
		\textbf{Parametername} & \textbf{Parameterbeschreibung} \\ \hline
		intCounter & Speicherstelle der Geschichte im Array \\ \hline
	\end{tabular}
\end{table}
\paragraph{Beschreibung} Die Funktion zeigt alle Links zwischen Interessenpunkten und der gegebenen Geschichte an. Die Funktion nutzt folgende Quellen:
\begin{itemize}
	\item Frontend-API
\end{itemize}
Es findet bei dieser Funktion kein Abruf von Daten aus {\glqq COSP\grqq} statt.
\subsubsection{FinalDeletePoiStoryLink}
\paragraph{Parameter} Die Funktion besitzt folgende Parameter:
\begin{table}[H]
	\begin{tabular}{|c|p{11cm}|}
		\hline
		\textbf{Parametername} & \textbf{Parameterbeschreibung} \\ \hline
		id         & Identifikator des Links zwischen der Geschichte und dem Interessenpunkt \\ \hline
		intCounter & Speicherstelle der Geschichte im Array \\ \hline
	\end{tabular}
\end{table}
\paragraph{Beschreibung} Die Funktion löscht einen Link zwischen einer Geschichte und einem Interessenpunkt mittels API endgültig. Anschließend wird das Modal neu geladen. Die Funktion hat Auswirkungen auf folgende Quellen:
\begin{itemize}
	\item Frontend-API
\end{itemize}
Es findet bei dieser Funktion kein Abruf von Daten aus {\glqq COSP\grqq} statt. Es werden jedoch Daten an {\glqq COSP\grqq} gesendet.
\subsubsection{RestorePoiStoryLink}
\paragraph{Parameter} Die Funktion besitzt folgende Parameter:
\begin{table}[H]
	\begin{tabular}{|c|p{11cm}|}
		\hline
		\textbf{Parametername} & \textbf{Parameterbeschreibung} \\ \hline
		id         & Identifikator des Links zwischen der Geschichte und dem Interessenpunkt \\ \hline
		intCounter & Speicherstelle der Geschichte im Array \\ \hline
	\end{tabular}
\end{table}
\paragraph{Beschreibung} Die Funktion stellt einen Link zwischen einer Geschichte und einem Interessenpunkt mittels API wieder her. Anschließend wird das Modal neu geladen. Die Funktion hat Auswirkungen auf folgende Quellen:
\begin{itemize}
	\item Frontend-API
\end{itemize}
Es findet bei dieser Funktion kein Abruf von Daten aus {\glqq COSP\grqq} statt. Es werden jedoch Daten an {\glqq COSP\grqq} gesendet.
\subsubsection{deletePoiStoryLink}
\paragraph{Parameter} Die Funktion besitzt folgende Parameter:
\begin{table}[H]
	\begin{tabular}{|c|p{11cm}|}
		\hline
		\textbf{Parametername} & \textbf{Parameterbeschreibung} \\ \hline
		IdPoiStory & Identifikator des Links zwischen der Geschichte und dem Interessenpunkt \\ \hline
		intCounter & Speicherstelle der Geschichte im Array \\ \hline
	\end{tabular}
\end{table}
\paragraph{Beschreibung} Die Funktion löscht einen Link zwischen einer Geschichte und einem Interessenpunkt mittels API oder markiert den Link als gelöscht. Anschließend wird das Modal neu geladen. Die Funktion hat Auswirkungen auf folgende Quellen:
\begin{itemize}
	\item Frontend-API
\end{itemize}
Es findet bei dieser Funktion kein Abruf von Daten aus {\glqq COSP\grqq} statt. Es werden jedoch Daten an {\glqq COSP\grqq} gesendet.
\subsubsection{ApiRequestDeletePoiStoryLink}
\paragraph{Parameter} Die Funktion besitzt folgende Parameter:
\begin{table}[H]
	\begin{tabular}{|c|p{11cm}|}
		\hline
		\textbf{Parametername} & \textbf{Parameterbeschreibung} \\ \hline
		IdPoiStory & Identifikator des Links zwischen der Geschichte und dem Interessenpunkt \\ \hline
		intCounter & Speicherstelle der Geschichte im Array \\ \hline
	\end{tabular}
\end{table}
\paragraph{Beschreibung} Die Funktion sendet eine Löschanfrage für einen Link zwischen einem Interessenpunkt und einer Geschichte an die Frontend-API. Die Funktion hat Auswirkungen auf folgende Quellen:
\begin{itemize}
	\item Frontend-API
\end{itemize}
Es findet bei dieser Funktion kein Abruf von Daten aus {\glqq COSP\grqq} statt. Es werden jedoch Daten an {\glqq COSP\grqq} gesendet.
\subsubsection{validatePoiStoryLink}
\paragraph{Parameter} Die Funktion besitzt folgende Parameter:
\begin{table}[H]
	\begin{tabular}{|c|p{11cm}|}
		\hline
		\textbf{Parametername} & \textbf{Parameterbeschreibung} \\ \hline
		IdPoiStory & Identifikator des Links zwischen der Geschichte und dem Interessenpunkt \\ \hline
		intCounter & Speicherstelle der Geschichte im Array \\ \hline
	\end{tabular}
\end{table}
\paragraph{Beschreibung} Die Funktion validiert einen Link zwischen einem Interessenpunkt und einer Geschichte. Die Funktion lädt anschließend das entsprechende Modal neu. Die Funktion hat Auswirkungen auf folgende Quellen:
\begin{itemize}
	\item Frontend-API
\end{itemize}
Es findet bei dieser Funktion kein Abruf von Daten aus {\glqq COSP\grqq} statt.
\subsubsection{ApiRequestValidatePoiStoryLink}
\paragraph{Parameter} Die Funktion besitzt folgende Parameter:
\begin{table}[H]
	\begin{tabular}{|c|p{11cm}|}
		\hline
		\textbf{Parametername} & \textbf{Parameterbeschreibung} \\ \hline
		IdPoiStory & Identifikator des Links zwischen der Geschichte und dem Interessenpunkt \\ \hline
	\end{tabular}
\end{table}
\paragraph{Beschreibung} Die Funktion sendet eine Validierungsanfrage an die Frontend-API um einen Link zwischen einer Geschichte und einem Interessenpunkt zu validieren. Die Funktion hat Auswirkungen auf folgende Quellen:
\begin{itemize}
	\item Frontend-API
\end{itemize}
Es findet bei dieser Funktion kein Abruf von Daten aus {\glqq COSP\grqq} statt.
\subsubsection{getPoiTitle}
\paragraph{Parameter} Die Funktion besitzt folgende Parameter:
\begin{table}[H]
	\begin{tabular}{|c|p{11cm}|}
		\hline
		\textbf{Parametername} & \textbf{Parameterbeschreibung} \\ \hline
		token  & alphanumerischer Identifikator einer Geschichte \\ \hline
	\end{tabular}
\end{table}
\paragraph{Beschreibung} Die Funktion fragt alle Titel der nicht mit der angegebenen Geschichten verlinkten Interessenpunkte ab. Die Funktion nutzt folgende Quellen:
\begin{itemize}
	\item Frontend-API
\end{itemize}
Es findet bei dieser Funktion kein Abruf von Daten aus {\glqq COSP\grqq} statt.
\subsubsection{loadKnownStoriesPoiLinks}
\paragraph{Parameter} Die Funktion besitzt folgende Parameter:
\begin{table}[H]
	\begin{tabular}{|c|p{11cm}|}
		\hline
		\textbf{Parametername} & \textbf{Parameterbeschreibung} \\ \hline
		token  & alphanumerischer Identifikator einer Geschichte \\ \hline
	\end{tabular}
\end{table}
\paragraph{Beschreibung} Die Funktion fragt alle Verknüpfungen zwischen der angegebenen Geschichte und Interessenpunkten ab. Die Funktion nutzt folgende Quellen:
\begin{itemize}
	\item Frontend-API
\end{itemize}
Es findet bei dieser Funktion kein Abruf von Daten aus {\glqq COSP\grqq} statt.
\subsubsection{addStory}
\paragraph{Parameter} Die Funktion besitzt keine Parameter.
\paragraph{Beschreibung} Die Funktion fragt das Anlegen einer neuen Geschichte bei der Frontend-API an. Die Funktion hat Auswirkungen auf folgende Quellen:
\begin{itemize}
	\item COSP
\end{itemize}
Es findet bei dieser Funktion kein Abruf von Daten aus {\glqq COSP\grqq} statt. Es werden jedoch Daten an {\glqq COSP\grqq} gesendet.
\subsubsection{validateStory}
\paragraph{Parameter} Die Funktion besitzt folgende Parameter:
\begin{table}[H]
	\begin{tabular}{|c|p{11cm}|}
		\hline
		\textbf{Parametername} & \textbf{Parameterbeschreibung} \\ \hline
		intCounter & Speicherstelle der Geschichte im Array \\ \hline
	\end{tabular}
\end{table}
\paragraph{Beschreibung} Die Funktion validiert eine Geschichte und lädt anschließend die Seite neu. Die Funktion hat Auswirkungen auf folgende Quellen:
\begin{itemize}
	\item COSP
\end{itemize}
Es findet bei dieser Funktion kein Abruf von Daten aus {\glqq COSP\grqq} statt. Es werden jedoch Daten an {\glqq COSP\grqq} gesendet.
\subsubsection{deleteStory}
\paragraph{Parameter} Die Funktion besitzt folgende Parameter:
\begin{table}[H]
	\begin{tabular}{|c|p{11cm}|}
		\hline
		\textbf{Parametername} & \textbf{Parameterbeschreibung} \\ \hline
		intCounter & Speicherstelle der Geschichte im Array \\ \hline
	\end{tabular}
\end{table}
\paragraph{Beschreibung} Die Funktion fragt das löschen einer Geschichte an oder fragt die Markierung einer Geschichte als gelöscht an. Die Funktion hat Auswirkungen auf folgende Quellen:
\begin{itemize}
	\item COSP
	\item Frontend-API
\end{itemize}
Es findet bei dieser Funktion kein Abruf von Daten aus {\glqq COSP\grqq} statt. Es werden jedoch Daten an {\glqq COSP\grqq} gesendet.
\subsubsection{validatePicture}
\paragraph{Parameter} Die Funktion besitzt folgende Parameter:
\begin{table}[H]
	\begin{tabular}{|c|p{11cm}|}
		\hline
		\textbf{Parametername} & \textbf{Parameterbeschreibung} \\ \hline
		url & URI zum validieren eines Bildes \\ \hline
	\end{tabular}
\end{table}
\paragraph{Beschreibung} Die Funktion validiert ein Bild und lädt anschließend die Seite neu. Die Funktion hat Auswirkungen auf folgende Quellen:
\begin{itemize}
	\item COSP
\end{itemize}
Es findet bei dieser Funktion kein Abruf von Daten aus {\glqq COSP\grqq} statt. Es werden jedoch Daten an {\glqq COSP\grqq} gesendet.
\subsubsection{deletePOI}
\paragraph{Parameter} Die Funktion besitzt folgende Parameter:
\begin{table}[H]
	\begin{tabular}{|c|p{11cm}|}
		\hline
		\textbf{Parametername} & \textbf{Parameterbeschreibung} \\ \hline
		poiid & Identifikator eines Interessenpunktes \\ \hline
	\end{tabular}
\end{table}
\paragraph{Beschreibung} Die Funktion fragt das löschen eines Interessenpunktes an oder fragt die Markierung eines Interessenpunktes als gelöscht an. Die Funktion hat Auswirkungen auf folgende Quellen:
\begin{itemize}
	\item Frontend-API
\end{itemize}
Es findet bei dieser Funktion kein Abruf von Daten aus {\glqq COSP\grqq} statt.
\subsubsection{deleteComment}
\paragraph{Parameter} Die Funktion besitzt folgende Parameter:
\begin{table}[H]
	\begin{tabular}{|c|p{11cm}|}
		\hline
		\textbf{Parametername} & \textbf{Parameterbeschreibung} \\ \hline
		commentid    & Identifikator eines Kommentars \\ \hline
		personalArea & Gibt an, ob Funktion aus persönlichem Bereich ausgeführt wird \\ \hline
	\end{tabular}
\end{table}
\paragraph{Beschreibung} Die Funktion fragt das löschen eines Kommentars an oder fragt die Markierung eines Kommentars als gelöscht an. Die Funktion hat Auswirkungen auf folgende Quellen:
\begin{itemize}
	\item Frontend-API
\end{itemize}
Es findet bei dieser Funktion kein Abruf von Daten aus {\glqq COSP\grqq} statt.
\subsubsection{deleteCommentMap}
\paragraph{Parameter} Die Funktion besitzt folgende Parameter:
\begin{table}[H]
	\begin{tabular}{|c|p{11cm}|}
		\hline
		\textbf{Parametername} & \textbf{Parameterbeschreibung} \\ \hline
		commentid & Identifikator eines Kommentars \\ \hline
		poi\_id   & Identifikator des zugehörigen Interessenpunktes \\ \hline
	\end{tabular}
\end{table}
\paragraph{Beschreibung} Die Funktion löscht eines Kommentars und wird von {\glqq Mehr Anzeigen\grqq}-Modal aus aufgerufen. Nach löschen lädt es den Kommentarteil des {\glqq Mehr Anzeigen\grqq}-Modal neu. Die Funktion hat Auswirkungen auf folgende Quellen:
\begin{itemize}
	\item Frontend-API
\end{itemize}
Es findet bei dieser Funktion kein Abruf von Daten aus {\glqq COSP\grqq} statt.
\subsubsection{getCommentByID}
\paragraph{Parameter} Die Funktion besitzt folgende Parameter:
\begin{table}[H]
	\begin{tabular}{|c|p{11cm}|}
		\hline
		\textbf{Parametername} & \textbf{Parameterbeschreibung} \\ \hline
		commentid & Identifikator eines Kommentars \\ \hline
	\end{tabular}
\end{table}
\paragraph{Beschreibung} Die Funktion fragt die Daten eines bestimmten Kommentars an. Die Funktion nutzt folgende Quellen:
\begin{itemize}
	\item Frontend-API
\end{itemize}
Es findet bei dieser Funktion kein Abruf von Daten aus {\glqq COSP\grqq} statt.
\subsubsection{saveCommentByID}
\paragraph{Parameter} Die Funktion besitzt folgende Parameter:
\begin{table}[H]
	\begin{tabular}{|c|p{11cm}|}
		\hline
		\textbf{Parametername} & \textbf{Parameterbeschreibung} \\ \hline
		commentid      & Identifikator eines Kommentars \\ \hline
		commentContent & Inhalt des Kommentars \\ \hline
	\end{tabular}
\end{table}
\paragraph{Beschreibung} Die Funktion speichert einen geänderten Kommentar. Die Funktion hat Auswirkungen auf folgende Quellen:
\begin{itemize}
	\item Frontend-API
\end{itemize}
Es findet bei dieser Funktion kein Abruf von Daten aus {\glqq COSP\grqq} statt.
\subsubsection{AddCommentAPI}
\paragraph{Parameter} Die Funktion besitzt folgende Parameter:
\begin{table}[H]
	\begin{tabular}{|c|p{11cm}|}
		\hline
		\textbf{Parametername} & \textbf{Parameterbeschreibung} \\ \hline
		comment & Inhalt des Kommentars \\ \hline
		poiid   & Zugehöriger Interessenpunkt \\ \hline
	\end{tabular}
\end{table}
\paragraph{Beschreibung} Die Funktion fügt einem Interessenpunkt einen neuen Kommentar hinzu. Die Funktion hat Auswirkungen auf folgende Quellen:
\begin{itemize}
	\item Frontend-API
\end{itemize}
Es findet bei dieser Funktion kein Abruf von Daten aus {\glqq COSP\grqq} statt.
\subsubsection{LoadSingleMaterialData}
\paragraph{Parameter} Die Funktion besitzt folgende Parameter:
\begin{table}[H]
	\begin{tabular}{|c|p{11cm}|}
		\hline
		\textbf{Parametername} & \textbf{Parameterbeschreibung} \\ \hline
		token & alphanumerischer Identifikator eines Bildes \\ \hline
	\end{tabular}
\end{table}
\paragraph{Beschreibung} Die Funktion fordert alle zum Abrufen eines einzelnen Bildes benötigten Daten mittels Frontend-API aus {\glqq COSP\grqq} an. Die Funktion nutzt folgende Quellen:
\begin{itemize}
	\item Frontend-API
\end{itemize}
Es findet bei dieser Funktion ein Abruf von Daten aus {\glqq COSP\grqq} statt.
\subsubsection{sendEditedMaterialData}
\paragraph{Parameter} Die Funktion besitzt folgende Parameter:
\begin{table}[H]
	\begin{tabular}{|c|p{11cm}|}
		\hline
		\textbf{Parametername} & \textbf{Parameterbeschreibung} \\ \hline
		title       & Titel des Bildes \\ \hline
		description & Beschreibung des Bildes \\ \hline
		token       & alphanumerischer Identifikator des Bildes \\ \hline
	\end{tabular}
\end{table}
\paragraph{Beschreibung} Die Funktion speichert veränderte Metadaten eines Bildes. Die Funktion hat Auswirkungen auf folgende Quellen:
\begin{itemize}
	\item Frontend-API
\end{itemize}
Es findet bei dieser Funktion kein Abruf von Daten aus {\glqq COSP\grqq} statt. Es werden jedoch Daten an {\glqq COSP\grqq} gesendet.

Es findet bei dieser Funktion ein Abruf von Daten aus {\glqq COSP\grqq} statt.
\subsubsection{sendEditedMaterialDataSource}
\paragraph{Parameter} Die Funktion besitzt folgende Parameter:
\begin{table}[H]
	\begin{tabular}{|c|p{11cm}|}
		\hline
		\textbf{Parametername} & \textbf{Parameterbeschreibung} \\ \hline
		title       & Titel des Bildes \\ \hline
		description & Beschreibung des Bildes \\ \hline
		token       & alphanumerischer Identifikator des Bildes \\ \hline
		source      & Quellenangabe \\ \hline
		sourcetype  & Identifikator des Typs der Quelle \\ \hline
	\end{tabular}
\end{table}
\paragraph{Beschreibung} Die Funktion speichert veränderte Metadaten eines Bildes. Die Funktion hat Auswirkungen auf folgende Quellen:
\begin{itemize}
	\item Frontend-API
\end{itemize}
Es findet bei dieser Funktion kein Abruf von Daten aus {\glqq COSP\grqq} statt. Es werden jedoch Daten an {\glqq COSP\grqq} gesendet.

\subsubsection{AddNameOfPoiAPI}
\paragraph{Parameter} Die Funktion besitzt folgende Parameter:
\begin{table}[H]
	\begin{tabular}{|c|p{11cm}|}
		\hline
		\textbf{Parametername} & \textbf{Parameterbeschreibung} \\ \hline
		from  & Startjahr \\ \hline
		till  & Endjahr \\ \hline
		name  & Name \\ \hline
		poiid & Identifikator eines Interessenpunktes \\ \hline
	\end{tabular}
\end{table}
\paragraph{Beschreibung} Die Funktion fügt einem Interessenpunkt einen neuen Namen hinzu. Die Funktion hat Auswirkungen auf folgende Quellen:
\begin{itemize}
	\item Frontend-API
\end{itemize}
Es findet bei dieser Funktion kein Abruf von Daten aus {\glqq COSP\grqq} statt.
\subsubsection{deletePoiNameFromList}
\paragraph{Parameter} Die Funktion besitzt folgende Parameter:
\begin{table}[H]
	\begin{tabular}{|c|p{11cm}|}
		\hline
		\textbf{Parametername} & \textbf{Parameterbeschreibung} \\ \hline
		ID    & Identifikator eines Namens \\ \hline
		POiid & Identifikator eines Interessenpunktes \\ \hline
	\end{tabular}
\end{table}
\paragraph{Beschreibung} Die Funktion löscht einen Namen oder markiert diesen als gelöscht und lädt anschließend die Namensliste erneut. Die Funktion hat Auswirkungen auf folgende Quellen:
\begin{itemize}
	\item Frontend-API
\end{itemize}
Es findet bei dieser Funktion kein Abruf von Daten aus {\glqq COSP\grqq} statt.
\subsubsection{deletePoiOperatorFromList}
\paragraph{Parameter} Die Funktion besitzt folgende Parameter:
\begin{table}[H]
	\begin{tabular}{|c|p{11cm}|}
		\hline
		\textbf{Parametername} & \textbf{Parameterbeschreibung} \\ \hline
		ID    & Identifikator eines Betreibers \\ \hline
		POiid & Identifikator eines Interessenpunktes \\ \hline
	\end{tabular}
\end{table}
\paragraph{Beschreibung} Die Funktion löscht einen Betreiber oder markiert diesen als gelöscht und lädt anschließend die Liste der Betreiber erneut. Die Funktion hat Auswirkungen auf folgende Quellen:
\begin{itemize}
	\item Frontend-API
\end{itemize}
Es findet bei dieser Funktion kein Abruf von Daten aus {\glqq COSP\grqq} statt.
\subsubsection{deletePoiAddressFromList}
\paragraph{Parameter} Die Funktion besitzt folgende Parameter:
\begin{table}[H]
	\begin{tabular}{|c|p{11cm}|}
		\hline
		\textbf{Parametername} & \textbf{Parameterbeschreibung} \\ \hline
		ID    & Identifikator einer historischen Adresse \\ \hline
		POiid & Identifikator eines Interessenpunktes \\ \hline
	\end{tabular}
\end{table}
\paragraph{Beschreibung} Die Funktion löscht eine historische Adresse oder markiert diese als gelöscht und lädt anschließend die Liste der historischen Adressen erneut. Die Funktion hat Auswirkungen auf folgende Quellen:
\begin{itemize}
	\item Frontend-API
\end{itemize}
Es findet bei dieser Funktion kein Abruf von Daten aus {\glqq COSP\grqq} statt.
\subsubsection{validateTimeSpanPOI}
\paragraph{Parameter} Die Funktion besitzt folgende Parameter:
\begin{table}[H]
	\begin{tabular}{|c|p{11cm}|}
		\hline
		\textbf{Parametername} & \textbf{Parameterbeschreibung} \\ \hline
		POi\_id & Identifikator eines Interessenpunktes \\ \hline
	\end{tabular}
\end{table}
\paragraph{Beschreibung} Die Funktion validiert die Zeitspanne eines Interessenpunktes und lädt anschließend das entsprechende Modal neu. Die Funktion hat Auswirkungen auf folgende Quellen:
\begin{itemize}
	\item Frontend-API
\end{itemize}
Es findet bei dieser Funktion kein Abruf von Daten aus {\glqq COSP\grqq} statt.
\subsubsection{validateCurrentAddressPOI}
\paragraph{Parameter} Die Funktion besitzt folgende Parameter:
\begin{table}[H]
	\begin{tabular}{|c|p{11cm}|}
		\hline
		\textbf{Parametername} & \textbf{Parameterbeschreibung} \\ \hline
		POi\_id & Identifikator eines Interessenpunktes \\ \hline
	\end{tabular}
\end{table}
\paragraph{Beschreibung} Die Funktion validiert die aktuelle Adresse eines Interessenpunktes und lädt anschließend das entsprechende Modal neu. Die Funktion hat Auswirkungen auf folgende Quellen:
\begin{itemize}
	\item Frontend-API
\end{itemize}
Es findet bei dieser Funktion kein Abruf von Daten aus {\glqq COSP\grqq} statt.
\subsubsection{validateHistoryPOI}
\paragraph{Parameter} Die Funktion besitzt folgende Parameter:
\begin{table}[H]
	\begin{tabular}{|c|p{11cm}|}
		\hline
		\textbf{Parametername} & \textbf{Parameterbeschreibung} \\ \hline
		POi\_id & Identifikator eines Interessenpunktes \\ \hline
	\end{tabular}
\end{table}
\paragraph{Beschreibung} Die Funktion validiert die Geschichte eines Interessenpunktes und lädt anschließend das entsprechende Modal neu. Die Funktion hat Auswirkungen auf folgende Quellen:
\begin{itemize}
	\item Frontend-API
\end{itemize}
Es findet bei dieser Funktion kein Abruf von Daten aus {\glqq COSP\grqq} statt.
\subsubsection{validateTypePOI}
\paragraph{Parameter} Die Funktion besitzt folgende Parameter:
\begin{table}[H]
	\begin{tabular}{|c|p{11cm}|}
		\hline
		\textbf{Parametername} & \textbf{Parameterbeschreibung} \\ \hline
		POi\_id & Identifikator eines Interessenpunktes \\ \hline
	\end{tabular}
\end{table}
\paragraph{Beschreibung} Die Funktion validiert den Typ eines Interessenpunktes und lädt anschließend das entsprechende Modal neu. Die Funktion hat Auswirkungen auf folgende Quellen:
\begin{itemize}
	\item Frontend-API
\end{itemize}
Es findet bei dieser Funktion kein Abruf von Daten aus {\glqq COSP\grqq} statt.\\
\subsubsection{validatePoiName}
\paragraph{Parameter} Die Funktion besitzt folgende Parameter:
\begin{table}[H]
	\begin{tabular}{|c|p{11cm}|}
		\hline
		\textbf{Parametername} & \textbf{Parameterbeschreibung} \\ \hline
		name\_id & Identifikator eines Namen \\ \hline
		POi\_id  & Identifikator eines Interessenpunktes \\ \hline
	\end{tabular}
\end{table}
\paragraph{Beschreibung} Die Funktion validiert einen Namen und lädt anschließend die Namensliste neu. Die Funktion hat Auswirkungen auf folgende Quellen:
\begin{itemize}
	\item Frontend-API
\end{itemize}
Es findet bei dieser Funktion kein Abruf von Daten aus {\glqq COSP\grqq} statt.
\subsubsection{validatePoiOperator}
\paragraph{Parameter} Die Funktion besitzt folgende Parameter:
\begin{table}[H]
	\begin{tabular}{|c|p{11cm}|}
		\hline
		\textbf{Parametername} & \textbf{Parameterbeschreibung} \\ \hline
		operator\_id & Identifikator eines Namen \\ \hline
		POi\_id      & Identifikator eines Interessenpunktes \\ \hline
	\end{tabular}
\end{table}
\paragraph{Beschreibung} Die Funktion validiert einen Betreiber und lädt anschließend die Liste der Betreiber neu. Die Funktion hat Auswirkungen auf folgende Quellen:
\begin{itemize}
	\item Frontend-API
\end{itemize}
Es findet bei dieser Funktion kein Abruf von Daten aus {\glqq COSP\grqq} statt.
\subsubsection{validatePoiAddress}
\paragraph{Parameter} Die Funktion besitzt folgende Parameter:
\begin{table}[H]
	\begin{tabular}{|c|p{11cm}|}
		\hline
		\textbf{Parametername} & \textbf{Parameterbeschreibung} \\ \hline
		Address\_id & Identifikator einer historischen Adresse \\ \hline
		POi\_id     & Identifikator eines Interessenpunktes \\ \hline
	\end{tabular}
\end{table}
\paragraph{Beschreibung} Die Funktion validiert eine historische Adresse und lädt anschließend die Liste der historischen Adressen neu. Die Funktion hat Auswirkungen auf folgende Quellen:
\begin{itemize}
	\item Frontend-API
\end{itemize}
Es findet bei dieser Funktion kein Abruf von Daten aus {\glqq COSP\grqq} statt.
\subsubsection{showLongComment}
\paragraph{Parameter} Die Funktion besitzt folgende Parameter:
\begin{table}[H]
	\begin{tabular}{|c|p{11cm}|}
		\hline
		\textbf{Parametername} & \textbf{Parameterbeschreibung} \\ \hline
		commentID & Identifikator eines Kommentars \\ \hline
	\end{tabular}
\end{table}
\paragraph{Beschreibung} Die Funktion zeigt einen langen Kommentar im entsprechenden Modal an. Es findet bei dieser Funktion kein Abruf von Daten aus {\glqq COSP\grqq} statt.
\subsubsection{closeLongComment}
\paragraph{Parameter} Die Funktion besitzt folgende Parameter:
\begin{table}[H]
	\begin{tabular}{|c|p{11cm}|}
		\hline
		\textbf{Parametername} & \textbf{Parameterbeschreibung} \\ \hline
		id & Identifikator eines Interessenpunktes \\ \hline
	\end{tabular}
\end{table}
\paragraph{Beschreibung} Die Funktion schließt das Modal eines langen Kommentars und öffnet das {\glqq Mehr-Anzeigen\grqq}-Modal des zum Kommentar zugehörigen Interessenpunktes. Es findet bei dieser Funktion kein Abruf von Daten aus {\glqq COSP\grqq} statt.
\subsubsection{deletePoiStoryLinkOnPoi}
\paragraph{Parameter} Die Funktion besitzt folgende Parameter:
\begin{table}[H]
	\begin{tabular}{|c|p{11cm}|}
		\hline
		\textbf{Parametername} & \textbf{Parameterbeschreibung} \\ \hline
		IdPoiStory & Identifikator eines Links zwischen einem Interessenpunkt und einer Geschichte\\ \hline
		PoiId      & Identifikator eines Interessenpunktes \\ \hline
	\end{tabular}
\end{table}
\paragraph{Beschreibung} Die Funktion löscht einen Link zwischen einem Interessenpunkt und einer Geschichte oder markiert diesen als gelöscht und lädt die entsprechende Anzeige im Anschluss neu. Die Funktion hat Auswirkungen auf folgende Quellen:
\begin{itemize}
	\item Frontend-API
\end{itemize}
Es findet bei dieser Funktion kein Abruf von Daten aus {\glqq COSP\grqq} statt.
\subsubsection{validatePoiStoryLinkOnPoi}
\paragraph{Parameter} Die Funktion besitzt folgende Parameter:
\begin{table}[H]
	\begin{tabular}{|c|p{11cm}|}
		\hline
		\textbf{Parametername} & \textbf{Parameterbeschreibung} \\ \hline
		IdPoiStory & Identifikator eines Links zwischen einem Interessenpunkt und einer Geschichte\\ \hline
		PoiId      & Identifikator eines Interessenpunktes \\ \hline
	\end{tabular}
\end{table}
\paragraph{Beschreibung} Die Funktion validiert einen Link zwischen einem Interessenpunkt und einer Geschichte. Anschließend lädt die Funktion die entsprechende Anzeige neu. Die Funktion hat Auswirkungen auf folgende Quellen:
\begin{itemize}
	\item Frontend-API
\end{itemize}
Es findet bei dieser Funktion kein Abruf von Daten aus {\glqq COSP\grqq} statt.
\subsubsection{checkAddressExists}
\paragraph{Parameter} Die Funktion besitzt folgende Parameter:
\begin{table}[H]
	\begin{tabular}{|c|p{11cm}|}
		\hline
		\textbf{Parametername} & \textbf{Parameterbeschreibung} \\ \hline
		Streetname  & Straßenname einer Adresse \\ \hline
		Housenumber & Hausnummer einer Adresse \\ \hline
		City        & Ortsname einer Adresse \\ \hline
		Postalcode  & Postleitzahl einer Adresse \\ \hline
	\end{tabular}
\end{table}
\paragraph{Beschreibung} Die Funktion prüft ob eine Adresse bereits benutzt wird. Die Funktion nutzt folgende Quellen:
\begin{itemize}
	\item Frontend-API
\end{itemize}
Es findet bei dieser Funktion kein Abruf von Daten aus {\glqq COSP\grqq} statt. Die Antwort ist ein Boolean.
\subsubsection{getPicturesAsList}
\paragraph{Parameter} Die Funktion besitzt keine Parameter.
\paragraph{Beschreibung} Die Funktion fragt eine Liste aller Bilder des Moduls ab. Die Funktion nutzt folgende Quellen:
\begin{itemize}
	\item Frontend-API
\end{itemize}
Es findet bei dieser Funktion ein Abruf von Daten aus {\glqq COSP\grqq} statt.
\subsubsection{showSinglePicSelect}
\paragraph{Parameter} Die Funktion besitzt keine Parameter.
\paragraph{Beschreibung} Die Funktion lädt das Bildauswahl-modal. Es findet bei dieser Funktion ein Abruf von Daten aus {\glqq COSP\grqq} statt.
\subsubsection{onclick\_picture}
\paragraph{Parameter} Die Funktion besitzt folgende Parameter:
\begin{table}[H]
	\begin{tabular}{|c|p{11cm}|}
		\hline
		\textbf{Parametername} & \textbf{Parameterbeschreibung} \\ \hline
		e & alphanumerischer Identifikator eines Bildes \\ \hline
	\end{tabular}
\end{table}
\paragraph{Beschreibung} Die Funktion stellt den alphanumerischen Identifikator des/der ausgewählten Bildes/Bilder zur späteren Speicherung zur Verfügung. Es findet bei dieser Funktion kein Abruf von Daten aus {\glqq COSP\grqq} statt.
\subsubsection{setPictureSelect\_SingleSelect}
\paragraph{Parameter} Die Funktion besitzt keine Parameter.
\paragraph{Beschreibung} Die Funktion setzt den Modus des Bildauswahl-Modals in Einzelauswahl. Es findet bei dieser Funktion kein Abruf von Daten aus {\glqq COSP\grqq} statt.
\subsubsection{setPictureSelect\_MultiSelect}
\paragraph{Parameter} Die Funktion besitzt keine Parameter.
\paragraph{Beschreibung} Die Funktion setzt den Modus des Bildauswahl-Modals in Mehrfachauswahl. Es findet bei dieser Funktion kein Abruf von Daten aus {\glqq COSP\grqq} statt.
\subsubsection{verifyPicPoiLink}
\paragraph{Parameter} Die Funktion besitzt folgende Parameter:
\begin{table}[H]
	\begin{tabular}{|c|p{11cm}|}
		\hline
		\textbf{Parametername} & \textbf{Parameterbeschreibung} \\ \hline
		id    & Identifikator eines Links zwischen einem Bild und einem Interessenpunkt \\ \hline
		poiid & Identifikator eines Interessenpunktes \\ \hline
	\end{tabular}
\end{table}
\paragraph{Beschreibung} Die Funktion validiert einen Link zwischen einem Interessenpunkt und einem Bild. Anschließend wird die Anzeige neu geladen. Die Funktion hat Auswirkungen auf folgende Quellen:
\begin{itemize}
	\item Frontend-API
\end{itemize}
Es findet bei dieser Funktion kein Abruf von Daten aus {\glqq COSP\grqq} statt.
\subsubsection{deletePicPoiLink}
\paragraph{Parameter} Die Funktion besitzt folgende Parameter:
\begin{table}[H]
	\begin{tabular}{|c|p{11cm}|}
		\hline
		\textbf{Parametername} & \textbf{Parameterbeschreibung} \\ \hline
		id    & Identifikator eines Links zwischen einem Bild und einem Interessenpunkt \\ \hline
		poiid & Identifikator eines Interessenpunktes \\ \hline
	\end{tabular}
\end{table}
\paragraph{Beschreibung} Die Funktion löscht einen Link zwischen einem Interessenpunkt und einem Bild oder markiert diesen als gelöscht. Anschließend wird die Anzeige neu geladen. Die Funktion hat Auswirkungen auf folgende Quellen:
\begin{itemize}
	\item Frontend-API
\end{itemize}
Es findet bei dieser Funktion kein Abruf von Daten aus {\glqq COSP\grqq} statt.
\subsubsection{showPoiPicLinks}
\paragraph{Parameter} Die Funktion besitzt folgende Parameter:
\begin{table}[H]
	\begin{tabular}{|c|p{11cm}|}
		\hline
		\textbf{Parametername} & \textbf{Parameterbeschreibung} \\ \hline
		picToken & alphanumerischer Identifikator eines Bildes \\ \hline
		title    & Titel des Bildes \\ \hline
	\end{tabular}
\end{table}
\paragraph{Beschreibung} Die Funktion öffnet das Modal zur Anzeige der Verknüpften Interessenpunkte eines Bildes. Die Funktion nutzt folgende Quellen:
\begin{itemize}
	\item Frontend-API
\end{itemize}
Es findet bei dieser Funktion kein Abruf von Daten aus {\glqq COSP\grqq} statt.
\subsubsection{ListMaterialWrapperFinalDeletePictureLink}
\paragraph{Parameter} Die Funktion besitzt folgende Parameter:
\begin{table}[H]
	\begin{tabular}{|c|p{11cm}|}
		\hline
		\textbf{Parametername} & \textbf{Parameterbeschreibung} \\ \hline
		id       & Identifikator eines Links zwischen einem Bild und einem Interessenpunkt \\ \hline
		pictoken & alphanumerischer Identifikator eines Bildes \\ \hline
		title    & Titel des Bildes \\ \hline
	\end{tabular}
\end{table}
\paragraph{Beschreibung} Die Funktion löscht einen Link zwischen einem Bild und einem Interessenpunkt endgültig. Anschließend wird die entsprechende Anzeige neu geladen. Die Funktion hat Auswirkungen auf folgende Quellen:
\begin{itemize}
	\item Frontend-API
\end{itemize}
Es findet bei dieser Funktion kein Abruf von Daten aus {\glqq COSP\grqq} statt.
\subsubsection{ListMaterialWrapperRestorePictureLink}
\paragraph{Parameter} Die Funktion besitzt folgende Parameter:
\begin{table}[H]
	\begin{tabular}{|c|p{11cm}|}
		\hline
		\textbf{Parametername} & \textbf{Parameterbeschreibung} \\ \hline
		id       & Identifikator eines Links zwischen einem Bild und einem Interessenpunkt \\ \hline
		pictoken & alphanumerischer Identifikator eines Bildes \\ \hline
		title    & Titel des Bildes \\ \hline
	\end{tabular}
\end{table}
\paragraph{Beschreibung} Die Funktion stellt einen Link zwischen einem Bild und einem Interessenpunkt wieder her. Anschließend wird die entsprechende Anzeige neu geladen. Die Funktion hat Auswirkungen auf folgende Quellen:
\begin{itemize}
	\item Frontend-API
\end{itemize}
Es findet bei dieser Funktion kein Abruf von Daten aus {\glqq COSP\grqq} statt.
\subsubsection{getPoisForPicture}
\paragraph{Parameter} Die Funktion besitzt folgende Parameter:
\begin{table}[H]
	\begin{tabular}{|c|p{11cm}|}
		\hline
		\textbf{Parametername} & \textbf{Parameterbeschreibung} \\ \hline
		pictoken & alphanumerischer Identifikator eines Bildes \\ \hline
	\end{tabular}
\end{table}
\paragraph{Beschreibung} Die Funktion fragt alle mit einem Bild verknüpften Interessenpunkte ab. Die Funktion nutzt folgende Quellen:
\begin{itemize}
	\item Frontend-API
\end{itemize}
Es findet bei dieser Funktion kein Abruf von Daten aus {\glqq COSP\grqq} statt.
\subsubsection{deletePicPoiListMaterialLink}
\paragraph{Parameter} Die Funktion besitzt folgende Parameter:
\begin{table}[H]
	\begin{tabular}{|c|p{11cm}|}
		\hline
		\textbf{Parametername} & \textbf{Parameterbeschreibung} \\ \hline
		id       & Identifikator eines Links zwischen einem Bild und einem Interessenpunkt \\ \hline
		pictoken & alphanumerischer Identifikator eines Bildes \\ \hline
		title    & Titel des Bildes \\ \hline
	\end{tabular}
\end{table}
\paragraph{Beschreibung} Die Funktion löscht einen Link zwischen einem Bild und einem Interessenpunkt oder markiert diesen als gelöscht. Anschließend wird die entsprechende Anzeige neu geladen. Die Funktion hat Auswirkungen auf folgende Quellen:
\begin{itemize}
	\item Frontend-API
\end{itemize}
Es findet bei dieser Funktion kein Abruf von Daten aus {\glqq COSP\grqq} statt.
\subsubsection{verifyPicPoiLinkMaterial}
\paragraph{Parameter} Die Funktion besitzt folgende Parameter:
\begin{table}[H]
	\begin{tabular}{|c|p{11cm}|}
		\hline
		\textbf{Parametername} & \textbf{Parameterbeschreibung} \\ \hline
		id       & Identifikator eines Links zwischen einem Bild und einem Interessenpunkt \\ \hline
		pictoken & alphanumerischer Identifikator eines Bildes \\ \hline
		title    & Titel des Bildes \\ \hline
	\end{tabular}
\end{table}
\paragraph{Beschreibung} Die Funktion validiert einen Link zwischen einem Bild und einem Interessenpunkt. Anschließend wird die entsprechende Anzeige neu geladen. Die Funktion hat Auswirkungen auf folgende Quellen:
\begin{itemize}
	\item Frontend-API
\end{itemize}
Es findet bei dieser Funktion kein Abruf von Daten aus {\glqq COSP\grqq} statt.
\subsubsection{addPicPoiLinkMaterialList}
\paragraph{Parameter} Die Funktion besitzt folgende Parameter:
\begin{table}[H]
	\begin{tabular}{|c|p{11cm}|}
		\hline
		\textbf{Parametername} & \textbf{Parameterbeschreibung} \\ \hline
		pictoken & alphanumerischer Identifikator eines Bildes \\ \hline
		title    & Titel des Bildes \\ \hline
	\end{tabular}
\end{table}
\paragraph{Beschreibung} Die Funktion fügt einem Bild einen neuen Link zu einem Interessenpunkt hinzu. Die Funktion hat Auswirkungen auf folgende Quellen:
\begin{itemize}
	\item Frontend-API
\end{itemize}
Es findet bei dieser Funktion kein Abruf von Daten aus {\glqq COSP\grqq} statt.
\subsubsection{validatePoiSeats}
\paragraph{Parameter} Die Funktion besitzt folgende Parameter:
\begin{table}[H]
	\begin{tabular}{|c|p{11cm}|}
		\hline
		\textbf{Parametername} & \textbf{Parameterbeschreibung} \\ \hline
		seat\_id & Identifikator einer Sitzplatzanzahl \\ \hline
		POi\_id  & Identifikator eines Interessenpunktes \\ \hline
	\end{tabular}
\end{table}
\paragraph{Beschreibung} Die Funktion validiert eine Sitzplatzanzahl und lädt die Anzeige anschließend neu. Die Funktion hat Auswirkungen auf folgende Quellen:
\begin{itemize}
	\item Frontend-API
\end{itemize}
Es findet bei dieser Funktion kein Abruf von Daten aus {\glqq COSP\grqq} statt.
\subsubsection{deletePoiSeatsFromList}
\paragraph{Parameter} Die Funktion besitzt folgende Parameter:
\begin{table}[H]
	\begin{tabular}{|c|p{11cm}|}
		\hline
		\textbf{Parametername} & \textbf{Parameterbeschreibung} \\ \hline
		ID     & Identifikator einer Sitzplatzanzahl \\ \hline
		POiid  & Identifikator eines Interessenpunktes \\ \hline
	\end{tabular}
\end{table}
\paragraph{Beschreibung} Die Funktion löscht eine Sitzplatzanzahl oder markiert diese als gelöscht und lädt die Anzeige anschließend neu. Die Funktion hat Auswirkungen auf folgende Quellen:
\begin{itemize}
	\item Frontend-API
\end{itemize}
Es findet bei dieser Funktion kein Abruf von Daten aus {\glqq COSP\grqq} statt.
\subsubsection{validatePoiCinemas}
\paragraph{Parameter} Die Funktion besitzt folgende Parameter:
\begin{table}[H]
	\begin{tabular}{|c|p{11cm}|}
		\hline
		\textbf{Parametername} & \textbf{Parameterbeschreibung} \\ \hline
		seat\_id & Identifikator einer Saalanzahl \\ \hline
		POi\_id  & Identifikator eines Interessenpunktes \\ \hline
	\end{tabular}
\end{table}
\paragraph{Beschreibung} Die Funktion validiert eine Saalanzahl und lädt die Anzeige anschließend neu. Die Funktion hat Auswirkungen auf folgende Quellen:
\begin{itemize}
	\item Frontend-API
\end{itemize}
Es findet bei dieser Funktion kein Abruf von Daten aus {\glqq COSP\grqq} statt.
\subsubsection{deletePoiCinemasFromList}
\paragraph{Parameter} Die Funktion besitzt folgende Parameter:
\begin{table}[H]
	\begin{tabular}{|c|p{11cm}|}
		\hline
		\textbf{Parametername} & \textbf{Parameterbeschreibung} \\ \hline
		ID     & Identifikator einer Saalanzahl \\ \hline
		POiid  & Identifikator eines Interessenpunktes \\ \hline
	\end{tabular}
\end{table}
\paragraph{Beschreibung} Die Funktion löscht eine Saalanzahl oder markiert diese als gelöscht und lädt die Anzeige anschließend neu. Die Funktion hat Auswirkungen auf folgende Quellen:
\begin{itemize}
	\item Frontend-API
\end{itemize}
Es findet bei dieser Funktion kein Abruf von Daten aus {\glqq COSP\grqq} statt.
\subsubsection{loadCaptchaContact}
\paragraph{Parameter} Die Funktion besitzt keine Parameter.
\paragraph{Beschreibung} Die Funktion lädt ein Captcha. Die Funktion nutzt folgende Quellen:
\begin{itemize}
	\item Frontend-API
\end{itemize}
Es findet bei dieser Funktion ein Abruf von Daten aus {\glqq COSP\grqq} statt.
\subsubsection{submitContact}
\paragraph{Parameter} Die Funktion besitzt keine Parameter.
\paragraph{Beschreibung} Die Funktion fragt das senden einer Kontak-E-Mail an. Die Funktion hat Auswirkungen auf folgende Quellen:
\begin{itemize}
	\item Frontend-API
\end{itemize}
Es findet bei dieser Funktion ein Abruf von Daten aus {\glqq COSP\grqq} statt. Es werden jedoch Daten an {\glqq COSP\grqq} gesendet.
\subsubsection{setErrorOnInputContact}
\paragraph{Parameter} Die Funktion besitzt folgende Parameter:
\begin{table}[H]
	\begin{tabular}{|c|p{11cm}|}
		\hline
		\textbf{Parametername} & \textbf{Parameterbeschreibung} \\ \hline
		elementid & Identifikator des HTML-Elementes\\ \hline
		state & Status der gesetzt werden soll\\ \hline
		tootip & Tooltipp der angezeigt werden soll\\ \hline
	\end{tabular}
\end{table}
\paragraph{Beschreibung} Die Funktion setzt die Darstellung eines Fehler-Status.
\subsubsection{finalDeleteLinkPoiPic}
\paragraph{Parameter} Die Funktion besitzt folgende Parameter:
\begin{table}[H]
	\begin{tabular}{|c|p{11cm}|}
		\hline
		\textbf{Parametername} & \textbf{Parameterbeschreibung} \\ \hline
		id & Identifikator eines Links zwischen einem Bild und einem Interessenpunkt \\ \hline
	\end{tabular}
\end{table}
\paragraph{Beschreibung} Die Funktion löscht einen Link zwischen einem Interessenpunkt und einem Bild endgültig. Die Funktion hat Auswirkungen auf folgende Quellen:
\begin{itemize}
	\item Frontend-API
\end{itemize}
Es findet bei dieser Funktion kein Abruf von Daten aus {\glqq COSP\grqq} statt.
\subsubsection{finalDeleteLinkPoiPic}
\paragraph{Parameter} Die Funktion besitzt folgende Parameter:
\begin{table}[H]
	\begin{tabular}{|c|p{11cm}|}
		\hline
		\textbf{Parametername} & \textbf{Parameterbeschreibung} \\ \hline
		id & Identifikator eines Links zwischen einem Bild und einem Interessenpunkt \\ \hline
	\end{tabular}
\end{table}
\paragraph{Beschreibung} Die Funktion stellt einen Link zwischen einem Interessenpunkt und einem Bild wieder her. Die Funktion hat Auswirkungen auf folgende Quellen:
\begin{itemize}
	\item Frontend-API
\end{itemize}
Es findet bei dieser Funktion kein Abruf von Daten aus {\glqq COSP\grqq} statt.
\subsubsection{finalDeleteLinkPoiStory}
\paragraph{Parameter} Die Funktion besitzt folgende Parameter:
\begin{table}[H]
	\begin{tabular}{|c|p{11cm}|}
		\hline
		\textbf{Parametername} & \textbf{Parameterbeschreibung} \\ \hline
		id & Identifikator eines Links zwischen einer Geschichte und einem Interessenpunkt \\ \hline
	\end{tabular}
\end{table}
\paragraph{Beschreibung} Die Funktion löscht einen Link zwischen einem Interessenpunkt und einer Geschichte endgültig. Die Funktion hat Auswirkungen auf folgende Quellen:
\begin{itemize}
	\item Frontend-API
\end{itemize}
Es findet bei dieser Funktion kein Abruf von Daten aus {\glqq COSP\grqq} statt.
\subsubsection{RestoreLinkPoiStory}
\paragraph{Parameter} Die Funktion besitzt folgende Parameter:
\begin{table}[H]
	\begin{tabular}{|c|p{11cm}|}
		\hline
		\textbf{Parametername} & \textbf{Parameterbeschreibung} \\ \hline
		id & Identifikator eines Links zwischen einer Geschichte und einem Interessenpunkt \\ \hline
	\end{tabular}
\end{table}
\paragraph{Beschreibung} Die Funktion stellt einen Link zwischen einem Interessenpunkt und einer Geschichte wieder her. Die Funktion hat Auswirkungen auf folgende Quellen:
\begin{itemize}
	\item Frontend-API
\end{itemize}
Es findet bei dieser Funktion kein Abruf von Daten aus {\glqq COSP\grqq} statt.
\subsubsection{finalDeletePicture}
\paragraph{Parameter} Die Funktion besitzt folgende Parameter:
\begin{table}[H]
	\begin{tabular}{|c|p{11cm}|}
		\hline
		\textbf{Parametername} & \textbf{Parameterbeschreibung} \\ \hline
		token & alphanumerischer Identifikator eines Bildes \\ \hline
	\end{tabular}
\end{table}
\paragraph{Beschreibung} Die Funktion löscht ein Bild endgültig. Die Funktion hat Auswirkungen auf folgende Quellen:
\begin{itemize}
	\item Frontend-API
	\item COSP
\end{itemize}
Es findet bei dieser Funktion kein Abruf von Daten aus {\glqq COSP\grqq} statt. Es werden jedoch Daten an {\glqq COSP\grqq} gesendet.
\subsubsection{restorePicture}
\paragraph{Parameter} Die Funktion besitzt folgende Parameter:
\begin{table}[H]
	\begin{tabular}{|c|p{11cm}|}
		\hline
		\textbf{Parametername} & \textbf{Parameterbeschreibung} \\ \hline
		token & alphanumerischer Identifikator eines Bildes \\ \hline
	\end{tabular}
\end{table}
\paragraph{Beschreibung} Die Funktion stellt ein Bild wieder her. Die Funktion hat Auswirkungen auf folgende Quellen:
\begin{itemize}
	\item Frontend-API
	\item COSP
\end{itemize}
Es findet bei dieser Funktion kein Abruf von Daten aus {\glqq COSP\grqq} statt. Es werden jedoch Daten an {\glqq COSP\grqq} gesendet.

\subsubsection{getCookie}
\paragraph{Parameter} Die Funktion besitzt folgende Parameter:
\begin{table}[H]
	\begin{tabular}{|c|p{11cm}|}
		\hline
		\textbf{Parametername} & \textbf{Parameterbeschreibung} \\ \hline
		name & Name des Cookies \\ \hline
	\end{tabular}
\end{table}
\paragraph{Beschreibung} Die Funktion lädt den Wert des Cookies mit dem angegebenen Namen. Es findet bei dieser Funktion kein Abruf von Daten aus {\glqq COSP\grqq} statt.
\subsubsection{testCookie}
\paragraph{Parameter} Die Funktion besitzt folgende Parameter:
\begin{table}[H]
	\begin{tabular}{|c|p{11cm}|}
		\hline
		\textbf{Parametername} & \textbf{Parameterbeschreibung} \\ \hline
		name & Name des Cookies \\ \hline
	\end{tabular}
\end{table}
\paragraph{Beschreibung} Die Funktion prüft, ob ein Cookie mit dem angegebenen Namen existiert. Es findet bei dieser Funktion kein Abruf von Daten aus {\glqq COSP\grqq} statt.
\subsubsection{setCookie}
\paragraph{Parameter} Die Funktion besitzt folgende Parameter:
\begin{table}[H]
	\begin{tabular}{|c|p{11cm}|}
		\hline
		\textbf{Parametername} & \textbf{Parameterbeschreibung} \\ \hline
		name & Name des Cookies \\ \hline
		value & Inhalt des Cookies \\ \hline
		exdays & Tage bis Ablauf des Cookies \\ \hline
	\end{tabular}
\end{table}
\paragraph{Beschreibung} Die Funktion setzt einen Cookie mit dem angegebenen Wert und Namen, welcher nach den angegebenen Tagen abläuft. Es findet bei dieser Funktion kein Abruf von Daten aus {\glqq COSP\grqq} statt.
\subsubsection{deleteCookie}
\paragraph{Parameter} Die Funktion besitzt folgende Parameter:
\begin{table}[H]
	\begin{tabular}{|c|p{11cm}|}
		\hline
		\textbf{Parametername} & \textbf{Parameterbeschreibung} \\ \hline
		name & Name des Cookies \\ \hline
	\end{tabular}
\end{table}
\paragraph{Beschreibung} Die Funktion löscht den Cookie mit dem angegebenen Namen. Es findet bei dieser Funktion kein Abruf von Daten aus {\glqq COSP\grqq} statt.
\subsubsection{CheckAddress}
\paragraph{Parameter} Die Funktion besitzt keine Parameter.
\paragraph{Beschreibung} Die Funktion prüft ob eine eingegebene Adresse bereits existiert. Die Funktion nutzt folgende Quellen:
\begin{itemize}
	\item Frontend-API
\end{itemize}
Es findet bei dieser Funktion kein Abruf von Daten aus {\glqq COSP\grqq} statt.
\newpage
\section{loadCaptcha}
\subsection{Allgemeines} Diese Datei lädt den Captcha-Code auf der Kontakt-Seite.
Die Ausführung des Codes findet im Browser statt. Das Laden des Captchas wird mittels folgendem Code initialisiert:
\begin{lstlisting}[language=JavaScript]
window.onload = function () {
	loadCaptchaContact();
}
\end{lstlisting}
\newpage
\section{loadCookie}
\subsection{Allgemeines} Diese Datei öffnet das Cookie-Modal auf der Index-Seite, sollte dieses nicht bereits bestätigt worden sein.
Die Ausführung des Codes findet im Browser statt. Das öffnen des Modals erfolgt mittels folgendem Code:
\begin{lstlisting}[language=JavaScript]
window.onload = function (){
	if (testCookie('CookieAccept')){
		if (getCookie('CookieAccept') == 'true')
		{
			setCookie('CookieAccept', true, 3650);
			return;
		}
	}
	$('#CookieBannerModal').modal({
		keyboard: false,
		backdrop: 'static'
	});
}
\end{lstlisting}
Dabei wird auch das schließen des Modals mittels Tastendruck beziehungsweise daneben Clickens verhindert.
\subsection{Funktionen}
\subsubsection{AcceptCookies}
\paragraph{Parameter} Die Funktion besitzt keine Parameter.
\paragraph{Beschreibung} Die Funktion setzt einen Cookie, das der Nutzer Cookies akzeptiert und schließt das Modal. Es findet bei dieser Funktion kein Abruf von Daten aus {\glqq COSP\grqq} statt.
\newpage
\section{mapfunctions}
\subsection{Allgemeines} Diese Datei enthält alle grundlegenden JavaScript-Funktionen.
Die Ausführung des Codes findet im Browser statt. Auch werden hier einige in verschiedenen anderen Javascripts verwendeten Variablen festgelegt.
\begin{lstlisting}[language=JavaScript]
var operators = {};
var Seats = {};
var Cinemas = {};
var names = {};
var histAddress = {};
var Karte;
var Karte2;
var Sources = {};
var SourceRelations = sendApiRequest({type: 'grs'}, false).data;
var SourceTypes = sendApiRequest({type: 'gts'}, false).data;

var color = redIcon;
var mark;

var Spielstaette = L.layerGroup();
var latlng = [0, 0];

var mark2;
var data = sendApiRequest({type: "gpu"}, false).data;
var minimap = false;
var guestmode = sendApiRequest({type: "gue"}, false).data;;
var deletedPOI = false;
\end{lstlisting}
\newpage
\subsection{Funktionen}
\subsubsection{loadMap}
\paragraph{Parameter} Die Funktion besitzt keine Parameter.
\paragraph{Beschreibung} Die Funktion lädt die Karte mitsamt Daten. Es findet bei dieser Funktion kein Abruf von Daten aus {\glqq COSP\grqq} statt.
\subsubsection{loadMinimap}
\paragraph{Parameter} Die Funktion besitzt keine Parameter.
\paragraph{Beschreibung} Die Funktion lädt die Vorschaukarte beim Anlegen eines neuen Interessenpunktes. Es findet bei dieser Funktion kein Abruf von Daten aus {\glqq COSP\grqq} statt.
\subsubsection{Style}
\paragraph{Parameter} Die Funktion besitzt keine Parameter.
\paragraph{Beschreibung} Die Funktion passt das aussehen der Karte an. Es findet bei dieser Funktion kein Abruf von Daten aus {\glqq COSP\grqq} statt.
\subsubsection{loadData}
\paragraph{Parameter} Die Funktion besitzt keine Parameter.
\paragraph{Beschreibung} Die Funktion lädt alle Marker und fügt diese anschließend der Karte hinzu. Es findet bei dieser Funktion kein Abruf von Daten aus {\glqq COSP\grqq} statt.
\subsubsection{onClick}
\paragraph{Parameter} Die Funktion besitzt folgende Parameter:
\begin{table}[H]
	\begin{tabular}{|c|p{11cm}|}
		\hline
		\textbf{Parametername} & \textbf{Parameterbeschreibung} \\ \hline
		e & Position des Clicks auf der Karte \\ \hline
	\end{tabular}
\end{table}
\paragraph{Beschreibung} Die Funktion fügt an der geklickten Stelle einen Marker auf der Karte ein. Dieser dient dem hinzufügen neuer Interessenpunkte. Es findet bei dieser Funktion kein Abruf von Daten aus {\glqq COSP\grqq} statt.
\subsubsection{addMarker}
\paragraph{Parameter} Die Funktion besitzt folgende Parameter:
\begin{table}[H]
	\begin{tabular}{|c|p{11cm}|}
		\hline
		\textbf{Parametername} & \textbf{Parameterbeschreibung} \\ \hline
		data & Array mit Daten eines Interessenpunktes \\ \hline
	\end{tabular}
\end{table}
\subparagraph{\$data}Das Array enthält folgende Elemente:
\begin{table}[H]
	\begin{tabular}{|c|p{11cm}|}
		\hline
		\textbf{Parametername} & \textbf{Parameterbeschreibung} \\ \hline
		lat              & Breitengrad \\ \hline
		lng              & Längengrad \\ \hline
		validated        & Validierungswert des Interessenpunktes \\ \hline
		validatedByUser  & Gibt an, ob aktueller Nutzer Interessenpunkt bereits validiert hat \\ \hline
		deleted          & Gibt an, ob Interessenpunkt als gelöscht markiert ist \\ \hline
		category         & Kategorie des Interessenpunktes \\ \hline
		current\_address & aktuelle Adresse des Interessenpunktes \\ \hline
		start            & Startjahr \\ \hline
		end              & Endjahr \\ \hline
		name             & Name des Interessenpunktes \\ \hline
		poi\_id          & Identifikator des Interessenpunktes \\ \hline
	\end{tabular}
\end{table}
\paragraph{Beschreibung} Die Funktion fügt einen Marker mit Popup für einen Interessenpunkt einem Kartenlayer hinzu. Es findet bei dieser Funktion kein Abruf von Daten aus {\glqq COSP\grqq} statt.
\subsubsection{refreshMap}
\paragraph{Parameter} Die Funktion besitzt keine Parameter.
\paragraph{Beschreibung} Die Funktion lädt alle Marker der Karte neu. Die Funktion nutzt folgende Quellen:
\begin{itemize}
	\item Frontend-API
\end{itemize}
Es findet bei dieser Funktion kein Abruf von Daten aus {\glqq COSP\grqq} statt.
\subsubsection{showMorePOI}
\paragraph{Parameter} Die Funktion besitzt folgende Parameter:
\begin{table}[H]
	\begin{tabular}{|c|p{11cm}|}
		\hline
		\textbf{Parametername} & \textbf{Parameterbeschreibung} \\ \hline
		poi\_id & Identifikator eines Interessenpunktes \\ \hline
	\end{tabular}
\end{table}
\paragraph{Beschreibung} Die Funktion lädt das {\glqq Mehr Anzeigen\grqq}-Modal eines Interessenpunktes. Die Funktion nutzt folgende Quellen:
\begin{itemize}
	\item Frontend-API
	\item COSP
\end{itemize}
Es findet bei dieser Funktion ein Abruf von Daten aus {\glqq COSP\grqq} statt.
\subsubsection{loadBasePoi}
\paragraph{Parameter} Die Funktion besitzt folgende Parameter:
\begin{table}[H]
	\begin{tabular}{|c|p{11cm}|}
		\hline
		\textbf{Parametername} & \textbf{Parameterbeschreibung} \\ \hline
		poiid & Identifikator eines Interessenpunktes \\ \hline
	\end{tabular}
\end{table}
\paragraph{Beschreibung} Die Funktion lädt alle Basisdaten eines Interessenpunktes, wie zum Beispiel die aktuelle Adresse, den Betriebszeitraum oder auch den Typ den Interessenpunktes. Anschließend werden diese Angezeigt. Die Funktion nutzt folgende Quellen:
\begin{itemize}
	\item Frontend-API
\end{itemize}
Es findet bei dieser Funktion kein Abruf von Daten aus {\glqq COSP\grqq} statt.
\subsubsection{loadComments}
\paragraph{Parameter} Die Funktion besitzt folgende Parameter:
\begin{table}[H]
	\begin{tabular}{|c|p{11cm}|}
		\hline
		\textbf{Parametername} & \textbf{Parameterbeschreibung} \\ \hline
		poiid & Identifikator eines Interessenpunktes \\ \hline
	\end{tabular}
\end{table}
\paragraph{Beschreibung} Die Funktion lädt alle Kommentare eines Interessenpunktes. Anschließend werden diese Angezeigt. Die Funktion nutzt folgende Quellen:
\begin{itemize}
	\item Frontend-API
\end{itemize}
Es findet bei dieser Funktion kein Abruf von Daten aus {\glqq COSP\grqq} statt.
\subsubsection{finalDeleteCommentPOI}
\paragraph{Parameter} Die Funktion besitzt folgende Parameter:
\begin{table}[H]
	\begin{tabular}{|c|p{11cm}|}
		\hline
		\textbf{Parametername} & \textbf{Parameterbeschreibung} \\ \hline
		id    & Identifikator eines Kommentares \\ \hline
		poiid & Identifikator eines Interessenpunktes \\ \hline
	\end{tabular}
\end{table}
\paragraph{Beschreibung} Die Funktion löscht einen Kommentare eines Interessenpunktes endgültig. Anschließend wird die Anzeige der Kommentare neu geladen. Die Funktion nutzt folgende Quellen:
\begin{itemize}
	\item Frontend-API
\end{itemize}
Es findet bei dieser Funktion kein Abruf von Daten aus {\glqq COSP\grqq} statt.
\subsubsection{RestoreCommentPOI}
\paragraph{Parameter} Die Funktion besitzt folgende Parameter:
\begin{table}[H]
	\begin{tabular}{|c|p{11cm}|}
		\hline
		\textbf{Parametername} & \textbf{Parameterbeschreibung} \\ \hline
		id    & Identifikator eines Kommentares \\ \hline
		poiid & Identifikator eines Interessenpunktes \\ \hline
	\end{tabular}
\end{table}
\paragraph{Beschreibung} Die Funktion stellt einen Kommentare eines Interessenpunktes wieder her. Anschließend wird die Anzeige der Kommentare neu geladen. Die Funktion nutzt folgende Quellen:
\begin{itemize}
	\item Frontend-API
\end{itemize}
Es findet bei dieser Funktion kein Abruf von Daten aus {\glqq COSP\grqq} statt.
\subsubsection{ShowMoreAdditionalPictures}
\paragraph{Parameter} Die Funktion besitzt folgende Parameter:
\begin{table}[H]
	\begin{tabular}{|c|p{11cm}|}
		\hline
		\textbf{Parametername} & \textbf{Parameterbeschreibung} \\ \hline
		poiid & Identifikator eines Interessenpunktes \\ \hline
	\end{tabular}
\end{table}
\paragraph{Beschreibung} Die Funktion lädt alle zusätzlichen Bilder eines Interessenpunktes. Anschließend werden diese Angezeigt. Die Funktion nutzt folgende Quellen:
\begin{itemize}
	\item Frontend-API
	\item COSP
\end{itemize}
Es findet bei dieser Funktion ein Abruf von Daten aus {\glqq COSP\grqq} statt.
\subsubsection{finalDeleteLinkPoiPicMapWrapper}
\paragraph{Parameter} Die Funktion besitzt folgende Parameter:
\begin{table}[H]
	\begin{tabular}{|c|p{11cm}|}
		\hline
		\textbf{Parametername} & \textbf{Parameterbeschreibung} \\ \hline
		id    & Identifikator eines Links zwischen einem Bild und einem Interessenpunkt \\ \hline
		poiid & Identifikator eines Interessenpunktes \\ \hline
	\end{tabular}
\end{table}
\paragraph{Beschreibung} Die Funktion löscht einen Link zwischen einem Bild und einem Interessenpunkt endgültig. Anschließend wird die Anzeige neu geladen. Die Funktion nutzt folgende Quellen:
\begin{itemize}
	\item Frontend-API
\end{itemize}
Es findet bei dieser Funktion kein Abruf von Daten aus {\glqq COSP\grqq} statt.
\subsubsection{RestoreLinkPoiPicMapWrapper}
\paragraph{Parameter} Die Funktion besitzt folgende Parameter:
\begin{table}[H]
	\begin{tabular}{|c|p{11cm}|}
		\hline
		\textbf{Parametername} & \textbf{Parameterbeschreibung} \\ \hline
		id    & Identifikator eines Links zwischen einem Bild und einem Interessenpunkt \\ \hline
		poiid & Identifikator eines Interessenpunktes \\ \hline
	\end{tabular}
\end{table}
\paragraph{Beschreibung} Die Funktion stellt einen Link zwischen einem Bild und einem Interessenpunkt wieder her. Anschließend wird die Anzeige neu geladen. Die Funktion nutzt folgende Quellen:
\begin{itemize}
	\item Frontend-API
\end{itemize}
Es findet bei dieser Funktion kein Abruf von Daten aus {\glqq COSP\grqq} statt.
\subsubsection{ShowMoreMainPic}
\paragraph{Parameter} Die Funktion besitzt folgende Parameter:
\begin{table}[H]
	\begin{tabular}{|c|p{11cm}|}
		\hline
		\textbf{Parametername} & \textbf{Parameterbeschreibung} \\ \hline
		poiid & Identifikator eines Interessenpunktes \\ \hline
	\end{tabular}
\end{table}
\paragraph{Beschreibung} Die Funktion lädt das Hauptbild eines Interessenpunktes. Anschließend werden diese Angezeigt. Die Funktion nutzt folgende Quellen:
\begin{itemize}
	\item Frontend-API
	\item COSP
\end{itemize}
Es findet bei dieser Funktion ein Abruf von Daten aus {\glqq COSP\grqq} statt.
\subsubsection{ShowMoreStories}
\paragraph{Parameter} Die Funktion besitzt folgende Parameter:
\begin{table}[H]
	\begin{tabular}{|c|p{11cm}|}
		\hline
		\textbf{Parametername} & \textbf{Parameterbeschreibung} \\ \hline
		poiid & Identifikator eines Interessenpunktes \\ \hline
	\end{tabular}
\end{table}
\paragraph{Beschreibung} Die Funktion lädt Geschichten zu einem Interessenpunkt. Anschließend werden diese Angezeigt. Die Funktion nutzt folgende Quellen:
\begin{itemize}
	\item Frontend-API
	\item COSP
\end{itemize}
Es findet bei dieser Funktion ein Abruf von Daten aus {\glqq COSP\grqq} statt.
\subsubsection{finalDeleteLinkPoiStoryMapWrapper}
\paragraph{Parameter} Die Funktion besitzt folgende Parameter:
\begin{table}[H]
	\begin{tabular}{|c|p{11cm}|}
		\hline
		\textbf{Parametername} & \textbf{Parameterbeschreibung} \\ \hline
		id    & Identifikator eines Links zwischen einer Geschichte und einem Interessenpunkt \\ \hline
		poiid & Identifikator eines Interessenpunktes \\ \hline
	\end{tabular}
\end{table}
\paragraph{Beschreibung} Die Funktion löscht einen Link zwischen einer Geschichte und einem Interessenpunkt endgültig. Anschließend wird die Anzeige neu geladen. Die Funktion nutzt folgende Quellen:
\begin{itemize}
	\item Frontend-API
\end{itemize}
Es findet bei dieser Funktion kein Abruf von Daten aus {\glqq COSP\grqq} statt.
\subsubsection{RestoreLinkPoiStoryMapWrapper}
\paragraph{Parameter} Die Funktion besitzt folgende Parameter:
\begin{table}[H]
	\begin{tabular}{|c|p{11cm}|}
		\hline
		\textbf{Parametername} & \textbf{Parameterbeschreibung} \\ \hline
		id    & Identifikator eines Links zwischen einer Geschichte und einem Interessenpunkt \\ \hline
		poiid & Identifikator eines Interessenpunktes \\ \hline
	\end{tabular}
\end{table}
\paragraph{Beschreibung} Die Funktion stellt einen Link zwischen einer Geschichte und einem Interessenpunkt wieder her. Anschließend wird die Anzeige neu geladen. Die Funktion nutzt folgende Quellen:
\begin{itemize}
	\item Frontend-API
\end{itemize}
Es findet bei dieser Funktion kein Abruf von Daten aus {\glqq COSP\grqq} statt.
\subsubsection{SetShowMoreStoryLinkOptions}
\paragraph{Parameter} Die Funktion besitzt folgende Parameter:
\begin{table}[H]
	\begin{tabular}{|c|p{11cm}|}
		\hline
		\textbf{Parametername} & \textbf{Parameterbeschreibung} \\ \hline
		poiid & Identifikator eines Interessenpunktes \\ \hline
	\end{tabular}
\end{table}
\paragraph{Beschreibung} Die Funktion lädt Optionen um diesen Interessenpunkt mit weiteren Geschichten zu verknüpfen. Die Funktion nutzt folgende Quellen:
\begin{itemize}
	\item Frontend-API
	\item COSP
\end{itemize}
Es findet bei dieser Funktion kein Abruf von Daten aus {\glqq COSP\grqq} statt.
\subsubsection{ShowMoreSeats}
\paragraph{Parameter} Die Funktion besitzt folgende Parameter:
\begin{table}[H]
	\begin{tabular}{|c|p{11cm}|}
		\hline
		\textbf{Parametername} & \textbf{Parameterbeschreibung} \\ \hline
		poiid & Identifikator eines Interessenpunktes \\ \hline
	\end{tabular}
\end{table}
\paragraph{Beschreibung} Die Funktion lädt alle Sitzplatzanzahlen eines Interessenpunktes. Anschließend werden diese Angezeigt. Die Funktion nutzt folgende Quellen:
\begin{itemize}
	\item Frontend-API
\end{itemize}
Es findet bei dieser Funktion kein Abruf von Daten aus {\glqq COSP\grqq} statt.
\subsubsection{finalDeleteSeatsPOI}
\paragraph{Parameter} Die Funktion besitzt folgende Parameter:
\begin{table}[H]
	\begin{tabular}{|c|p{11cm}|}
		\hline
		\textbf{Parametername} & \textbf{Parameterbeschreibung} \\ \hline
		id    & Identifikator einer Sitzplatzanzahl \\ \hline
		poiid & Identifikator eines Interessenpunktes \\ \hline
	\end{tabular}
\end{table}
\paragraph{Beschreibung} Die Funktion löscht eine Sitzplatzanzahl eines Interessenpunktes endgültig. Anschließend wird die Anzeige neu geladen. Die Funktion nutzt folgende Quellen:
\begin{itemize}
	\item Frontend-API
\end{itemize}
Es findet bei dieser Funktion kein Abruf von Daten aus {\glqq COSP\grqq} statt.
\subsubsection{RestoreSeatsPOI}
\paragraph{Parameter} Die Funktion besitzt folgende Parameter:
\begin{table}[H]
	\begin{tabular}{|c|p{11cm}|}
		\hline
		\textbf{Parametername} & \textbf{Parameterbeschreibung} \\ \hline
		id    & Identifikator einer Sitzplatzanzahl \\ \hline
		poiid & Identifikator eines Interessenpunktes \\ \hline
	\end{tabular}
\end{table}
\paragraph{Beschreibung} Die Funktion stellt eine Sitzplatzanzahl eines Interessenpunktes wieder her. Anschließend wird die Anzeige neu geladen. Die Funktion nutzt folgende Quellen:
\begin{itemize}
	\item Frontend-API
\end{itemize}
Es findet bei dieser Funktion kein Abruf von Daten aus {\glqq COSP\grqq} statt.
\subsubsection{ShowMoreCinemas}
\paragraph{Parameter} Die Funktion besitzt folgende Parameter:
\begin{table}[H]
	\begin{tabular}{|c|p{11cm}|}
		\hline
		\textbf{Parametername} & \textbf{Parameterbeschreibung} \\ \hline
		poiid & Identifikator eines Interessenpunktes \\ \hline
	\end{tabular}
\end{table}
\paragraph{Beschreibung} Die Funktion lädt alle Saalanzahlen eines Interessenpunktes. Anschließend werden diese Angezeigt. Die Funktion nutzt folgende Quellen:
\begin{itemize}
	\item Frontend-API
\end{itemize}
Es findet bei dieser Funktion kein Abruf von Daten aus {\glqq COSP\grqq} statt.
\subsubsection{finalDeleteCinemasPOI}
\paragraph{Parameter} Die Funktion besitzt folgende Parameter:
\begin{table}[H]
	\begin{tabular}{|c|p{11cm}|}
		\hline
		\textbf{Parametername} & \textbf{Parameterbeschreibung} \\ \hline
		id    & Identifikator einer Saalanzahl \\ \hline
		poiid & Identifikator eines Interessenpunktes \\ \hline
	\end{tabular}
\end{table}
\paragraph{Beschreibung} Die Funktion löscht eine Saalanzahl eines Interessenpunktes endgültig. Anschließend wird die Anzeige neu geladen. Die Funktion nutzt folgende Quellen:
\begin{itemize}
	\item Frontend-API
\end{itemize}
Es findet bei dieser Funktion kein Abruf von Daten aus {\glqq COSP\grqq} statt.
\subsubsection{RestoreCinemasPOI}
\paragraph{Parameter} Die Funktion besitzt folgende Parameter:
\begin{table}[H]
	\begin{tabular}{|c|p{11cm}|}
		\hline
		\textbf{Parametername} & \textbf{Parameterbeschreibung} \\ \hline
		id    & Identifikator einer Saalanzahl \\ \hline
		poiid & Identifikator eines Interessenpunktes \\ \hline
	\end{tabular}
\end{table}
\paragraph{Beschreibung} Die Funktion stellt eine Saalanzahl eines Interessenpunktes wieder her. Anschließend wird die Anzeige neu geladen. Die Funktion nutzt folgende Quellen:
\begin{itemize}
	\item Frontend-API
\end{itemize}
Es findet bei dieser Funktion kein Abruf von Daten aus {\glqq COSP\grqq} statt.
\subsubsection{ShowMoreHistoricalAddresses}
\paragraph{Parameter} Die Funktion besitzt folgende Parameter:
\begin{table}[H]
	\begin{tabular}{|c|p{11cm}|}
		\hline
		\textbf{Parametername} & \textbf{Parameterbeschreibung} \\ \hline
		poiid & Identifikator eines Interessenpunktes \\ \hline
	\end{tabular}
\end{table}
\paragraph{Beschreibung} Die Funktion lädt alle historischen Adressen eines Interessenpunktes. Anschließend werden diese Angezeigt. Die Funktion nutzt folgende Quellen:
\begin{itemize}
	\item Frontend-API
\end{itemize}
Es findet bei dieser Funktion kein Abruf von Daten aus {\glqq COSP\grqq} statt.
\subsubsection{finalDeleteHistAddrPOI}
\paragraph{Parameter} Die Funktion besitzt folgende Parameter:
\begin{table}[H]
	\begin{tabular}{|c|p{11cm}|}
		\hline
		\textbf{Parametername} & \textbf{Parameterbeschreibung} \\ \hline
		id    & Identifikator einer historischen Adresse \\ \hline
		poiid & Identifikator eines Interessenpunktes \\ \hline
	\end{tabular}
\end{table}
\paragraph{Beschreibung} Die Funktion löscht eine historische Adresse eines Interessenpunktes endgültig. Anschließend wird die Anzeige neu geladen. Die Funktion nutzt folgende Quellen:
\begin{itemize}
	\item Frontend-API
\end{itemize}
Es findet bei dieser Funktion kein Abruf von Daten aus {\glqq COSP\grqq} statt.
\subsubsection{RestoreHistAddrPOI}
\paragraph{Parameter} Die Funktion besitzt folgende Parameter:
\begin{table}[H]
	\begin{tabular}{|c|p{11cm}|}
		\hline
		\textbf{Parametername} & \textbf{Parameterbeschreibung} \\ \hline
		id    & Identifikator einer historischen Adresse \\ \hline
		poiid & Identifikator eines Interessenpunktes \\ \hline
	\end{tabular}
\end{table}
\paragraph{Beschreibung} Die Funktion stellt eine historische Adresse eines Interessenpunktes wieder her. Anschließend wird die Anzeige neu geladen. Die Funktion nutzt folgende Quellen:
\begin{itemize}
	\item Frontend-API
\end{itemize}
Es findet bei dieser Funktion kein Abruf von Daten aus {\glqq COSP\grqq} statt.
\subsubsection{ShowMoreOperators}
\paragraph{Parameter} Die Funktion besitzt folgende Parameter:
\begin{table}[H]
	\begin{tabular}{|c|p{11cm}|}
		\hline
		\textbf{Parametername} & \textbf{Parameterbeschreibung} \\ \hline
		poiid & Identifikator eines Interessenpunktes \\ \hline
	\end{tabular}
\end{table}
\paragraph{Beschreibung} Die Funktion lädt alle Betreiber eines Interessenpunktes. Anschließend werden diese Angezeigt. Die Funktion nutzt folgende Quellen:
\begin{itemize}
	\item Frontend-API
\end{itemize}
Es findet bei dieser Funktion kein Abruf von Daten aus {\glqq COSP\grqq} statt.
\subsubsection{finalDeleteOperatorPOI}
\paragraph{Parameter} Die Funktion besitzt folgende Parameter:
\begin{table}[H]
	\begin{tabular}{|c|p{11cm}|}
		\hline
		\textbf{Parametername} & \textbf{Parameterbeschreibung} \\ \hline
		id    & Identifikator eines Betreibers \\ \hline
		poiid & Identifikator eines Interessenpunktes \\ \hline
	\end{tabular}
\end{table}
\paragraph{Beschreibung} Die Funktion löscht einen Betreiber eines Interessenpunktes endgültig. Anschließend wird die Anzeige neu geladen. Die Funktion nutzt folgende Quellen:
\begin{itemize}
	\item Frontend-API
\end{itemize}
Es findet bei dieser Funktion kein Abruf von Daten aus {\glqq COSP\grqq} statt.
\subsubsection{RestoreOperatorPOI}
\paragraph{Parameter} Die Funktion besitzt folgende Parameter:
\begin{table}[H]
	\begin{tabular}{|c|p{11cm}|}
		\hline
		\textbf{Parametername} & \textbf{Parameterbeschreibung} \\ \hline
		id    & Identifikator eines Betreibers \\ \hline
		poiid & Identifikator eines Interessenpunktes \\ \hline
	\end{tabular}
\end{table}
\paragraph{Beschreibung} Die Funktion stellt einen Betreiber eines Interessenpunktes wieder her. Anschließend wird die Anzeige neu geladen. Die Funktion nutzt folgende Quellen:
\begin{itemize}
	\item Frontend-API
\end{itemize}
Es findet bei dieser Funktion kein Abruf von Daten aus {\glqq COSP\grqq} statt.
\subsubsection{ShowMoreNames}
\paragraph{Parameter} Die Funktion besitzt folgende Parameter:
\begin{table}[H]
	\begin{tabular}{|c|p{11cm}|}
		\hline
		\textbf{Parametername} & \textbf{Parameterbeschreibung} \\ \hline
		poiid & Identifikator eines Interessenpunktes \\ \hline
	\end{tabular}
\end{table}
\paragraph{Beschreibung} Die Funktion lädt alle Namen eines Interessenpunktes. Anschließend werden diese Angezeigt. Die Funktion nutzt folgende Quellen:
\begin{itemize}
	\item Frontend-API
\end{itemize}
Es findet bei dieser Funktion kein Abruf von Daten aus {\glqq COSP\grqq} statt.
\subsubsection{finalDeleteNamePOI}
\paragraph{Parameter} Die Funktion besitzt folgende Parameter:
\begin{table}[H]
	\begin{tabular}{|c|p{11cm}|}
		\hline
		\textbf{Parametername} & \textbf{Parameterbeschreibung} \\ \hline
		id    & Identifikator eines Namen \\ \hline
		poiid & Identifikator eines Interessenpunktes \\ \hline
	\end{tabular}
\end{table}
\paragraph{Beschreibung} Die Funktion löscht einen Namen eines Interessenpunktes endgültig. Anschließend wird die Anzeige neu geladen. Die Funktion nutzt folgende Quellen:
\begin{itemize}
	\item Frontend-API
\end{itemize}
Es findet bei dieser Funktion kein Abruf von Daten aus {\glqq COSP\grqq} statt.
\subsubsection{restoreNamePOI}
\paragraph{Parameter} Die Funktion besitzt folgende Parameter:
\begin{table}[H]
	\begin{tabular}{|c|p{11cm}|}
		\hline
		\textbf{Parametername} & \textbf{Parameterbeschreibung} \\ \hline
		id    & Identifikator eines Namen \\ \hline
		poiid & Identifikator eines Interessenpunktes \\ \hline
	\end{tabular}
\end{table}
\paragraph{Beschreibung} Die Funktion stellt einen Namen eines Interessenpunktes wieder her. Anschließend wird die Anzeige neu geladen. Die Funktion nutzt folgende Quellen:
\begin{itemize}
	\item Frontend-API
\end{itemize}
Es findet bei dieser Funktion kein Abruf von Daten aus {\glqq COSP\grqq} statt.
\subsubsection{disableCardLinks}
\paragraph{Parameter} Die Funktion besitzt folgende Parameter:
\begin{table}[H]
	\begin{tabular}{|c|p{11cm}|}
		\hline
		\textbf{Parametername} & \textbf{Parameterbeschreibung} \\ \hline
		identifer & Identifikator des Links der Bildkachel \\ \hline
	\end{tabular}
\end{table}
\paragraph{Beschreibung} Die Funktion entfernt die Lightbox-Vollbild Anzeige eines Bildes beim Hovern über einem Button. Es findet bei dieser Funktion kein Abruf von Daten aus {\glqq COSP\grqq} statt.
\subsubsection{enableCardLinks}
\paragraph{Parameter} Die Funktion besitzt folgende Parameter:
\begin{table}[H]
	\begin{tabular}{|c|p{11cm}|}
		\hline
		\textbf{Parametername} & \textbf{Parameterbeschreibung} \\ \hline
		identifer & Identifikator des Links der Bildkachel \\ \hline
	\end{tabular}
\end{table}
\paragraph{Beschreibung} Die Funktion aktiviert die Lightbox-Vollbild Anzeige eines Bildes sobald nicht mehr über einem Button gehovert wird. Es findet bei dieser Funktion kein Abruf von Daten aus {\glqq COSP\grqq} statt.
\subsubsection{blurButton}
\paragraph{Parameter} Die Funktion besitzt folgende Parameter:
\begin{table}[H]
	\begin{tabular}{|c|p{11cm}|}
		\hline
		\textbf{Parametername} & \textbf{Parameterbeschreibung} \\ \hline
		identifer & Identifikator des geklickten Buttons \\ \hline
	\end{tabular}
\end{table}
\paragraph{Beschreibung} Die Funktion entfernt den Focus und versteckt den Tooltip eines geklickten Buttons. Es findet bei dieser Funktion kein Abruf von Daten aus {\glqq COSP\grqq} statt.
\subsubsection{saveLinkedPoiMap}
\paragraph{Parameter} Die Funktion besitzt keine Parameter.
\paragraph{Beschreibung} Die Funktion speichert einen neuen Link zwischen einer Geschichte und einem Interessenpunkt. Anschließend wird die Anzeige neu geladen. Die Funktion hat Auswirkungen auf folgende Quellen:
\begin{itemize}
	\item Frontend-API
\end{itemize}
Es findet bei dieser Funktion kein Abruf von Daten aus {\glqq COSP\grqq} statt.
\subsubsection{showMoreStoryMap}
\paragraph{Parameter} Die Funktion besitzt folgende Parameter:
\begin{table}[H]
	\begin{tabular}{|c|p{11cm}|}
		\hline
		\textbf{Parametername} & \textbf{Parameterbeschreibung} \\ \hline
		IntCounter & Position der Geschichte im {\glqq storiesMap\grqq}-Array \\ \hline
		poi\_id    & Identifikator eines Interessenpunktes \\ \hline
	\end{tabular}
\end{table}
\paragraph{Beschreibung} Die Funktion zeigt eine Vollständige Geschichte an. Es findet bei dieser Funktion kein Abruf von Daten aus {\glqq COSP\grqq} statt.
\subsubsection{closeStoryModalShowMorePoi}
\paragraph{Parameter} Die Funktion besitzt folgende Parameter:
\begin{table}[H]
	\begin{tabular}{|c|p{11cm}|}
		\hline
		\textbf{Parametername} & \textbf{Parameterbeschreibung} \\ \hline
		poi\_id & Identifikator eines Interessenpunktes \\ \hline
	\end{tabular}
\end{table}
\paragraph{Beschreibung} Die Funktion schließt die Anzeige der vollständigen Geschichte und lädt das {\glqq Mehr Anzeigen\grqq}-Modal des davor angezeigten Interessenpunktes. Es findet bei dieser Funktion kein Abruf von Daten aus {\glqq COSP\grqq} statt.
\subsubsection{setfocus2}
\paragraph{Parameter} Die Funktion besitzt folgende Parameter:
\begin{table}[H]
	\begin{tabular}{|c|p{11cm}|}
		\hline
		\textbf{Parametername} & \textbf{Parameterbeschreibung} \\ \hline
		lat & Breitengrad \\ \hline
		lng & Längengrad \\ \hline
	\end{tabular}
\end{table}
\paragraph{Beschreibung} Die Funktion setzt den Focus der Karte auf die entsprechenden Koordinaten. Es findet bei dieser Funktion kein Abruf von Daten aus {\glqq COSP\grqq} statt.
\subsubsection{toggleAddPOIButton}
\paragraph{Parameter} Die Funktion besitzt folgende Parameter:
\begin{table}[H]
	\begin{tabular}{|c|p{11cm}|}
		\hline
		\textbf{Parametername} & \textbf{Parameterbeschreibung} \\ \hline
		enabled & Aktivierungsstatus \\ \hline
	\end{tabular}
\end{table}
\paragraph{Beschreibung} Die Funktion aktiviert oder deaktiviert den Button zum hinzufügen eines neuen Interessenpunktes. Es findet bei dieser Funktion kein Abruf von Daten aus {\glqq COSP\grqq} statt.
\subsubsection{getCommentFromFormular}
\paragraph{Parameter} Die Funktion besitzt keine Parameter.
\paragraph{Beschreibung} Die Funktion fügt einen neuen Kommentar einem Interessenpunkt hinzu und lädt alle Kommentare anschließend neu. Die Funktion hat Auswirkungen auf folgende Quellen:
\begin{itemize}
	\item Frontend-API
\end{itemize}
Es findet bei dieser Funktion kein Abruf von Daten aus {\glqq COSP\grqq} statt.
\subsubsection{saveNameShowMore}
\paragraph{Parameter} Die Funktion besitzt keine Parameter.
\paragraph{Beschreibung} Die Funktion fügt einen neuen Namen einem Interessenpunkt hinzu und lädt alle Namen anschließend neu. Die Funktion hat Auswirkungen auf folgende Quellen:
\begin{itemize}
	\item Frontend-API
\end{itemize}
Es findet bei dieser Funktion kein Abruf von Daten aus {\glqq COSP\grqq} statt.
\subsubsection{updateNameShowMore}
\paragraph{Parameter} Die Funktion besitzt folgende Parameter:
\begin{table}[H]
	\begin{tabular}{|c|p{11cm}|}
		\hline
		\textbf{Parametername} & \textbf{Parameterbeschreibung} \\ \hline
		nameId & Identifikator eines Namen \\ \hline
	\end{tabular}
\end{table}
\paragraph{Beschreibung} Die Funktion aktualisiert einen Namenseintrag. Anschließend wird die Anzeige neu geladen. Die Funktion hat Auswirkungen auf folgende Quellen:
\begin{itemize}
	\item Frontend-API
\end{itemize}
Es findet bei dieser Funktion kein Abruf von Daten aus {\glqq COSP\grqq} statt.
\subsubsection{updateNameShowMore}
\paragraph{Parameter} Die Funktion besitzt folgende Parameter:
\begin{table}[H]
	\begin{tabular}{|c|p{11cm}|}
		\hline
		\textbf{Parametername} & \textbf{Parameterbeschreibung} \\ \hline
		id & Identifikator eines Namen \\ \hline
	\end{tabular}
\end{table}
\paragraph{Beschreibung} Die Funktion schaltet das Änderungsfeld eines Namen frei. Es findet bei dieser Funktion kein Abruf von Daten aus {\glqq COSP\grqq} statt.
\subsubsection{saveOperatorShowMore}
\paragraph{Parameter} Die Funktion besitzt keine Parameter.
\paragraph{Beschreibung} Die Funktion fügt einen neuen Betreiber einem Interessenpunkt hinzu und lädt alle Betreiber anschließend neu. Die Funktion hat Auswirkungen auf folgende Quellen:
\begin{itemize}
	\item Frontend-API
\end{itemize}
Es findet bei dieser Funktion kein Abruf von Daten aus {\glqq COSP\grqq} statt.
\subsubsection{updateOperatorShowMore}
\paragraph{Parameter} Die Funktion besitzt folgende Parameter:
\begin{table}[H]
	\begin{tabular}{|c|p{11cm}|}
		\hline
		\textbf{Parametername} & \textbf{Parameterbeschreibung} \\ \hline
		operatorId & Identifikator eines Betreibers \\ \hline
	\end{tabular}
\end{table}
\paragraph{Beschreibung} Die Funktion aktualisiert einen Betreibereintrag. Anschließend wird die Anzeige neu geladen. Die Funktion hat Auswirkungen auf folgende Quellen:
\begin{itemize}
	\item Frontend-API
\end{itemize}
Es findet bei dieser Funktion kein Abruf von Daten aus {\glqq COSP\grqq} statt.
\subsubsection{editOperator}
\paragraph{Parameter} Die Funktion besitzt folgende Parameter:
\begin{table}[H]
	\begin{tabular}{|c|p{11cm}|}
		\hline
		\textbf{Parametername} & \textbf{Parameterbeschreibung} \\ \hline
		id & Identifikator eines Betreibers \\ \hline
	\end{tabular}
\end{table}
\paragraph{Beschreibung} Die Funktion schaltet das Änderungsfeld eines Betreibers frei. Es findet bei dieser Funktion kein Abruf von Daten aus {\glqq COSP\grqq} statt.
\subsubsection{saveHistoricalShowMore}
\paragraph{Parameter} Die Funktion besitzt keine Parameter.
\paragraph{Beschreibung} Die Funktion fügt eine neue historische Adresse einem Interessenpunkt hinzu und lädt alle Namen anschließend neu. Die Funktion hat Auswirkungen auf folgende Quellen:
\begin{itemize}
	\item Frontend-API
\end{itemize}
Es findet bei dieser Funktion kein Abruf von Daten aus {\glqq COSP\grqq} statt.
\subsubsection{updateHistoricalShowMore}
\paragraph{Parameter} Die Funktion besitzt folgende Parameter:
\begin{table}[H]
	\begin{tabular}{|c|p{11cm}|}
		\hline
		\textbf{Parametername} & \textbf{Parameterbeschreibung} \\ \hline
		nameId & Identifikator einer historischen Adresse \\ \hline
	\end{tabular}
\end{table}
\paragraph{Beschreibung} Die Funktion aktualisiert einen Eintrag einer historischen Adresse. Anschließend wird die Anzeige neu geladen. Die Funktion hat Auswirkungen auf folgende Quellen:
\begin{itemize}
	\item Frontend-API
\end{itemize}
Es findet bei dieser Funktion kein Abruf von Daten aus {\glqq COSP\grqq} statt.
\subsubsection{editHistAddress}
\paragraph{Parameter} Die Funktion besitzt folgende Parameter:
\begin{table}[H]
	\begin{tabular}{|c|p{11cm}|}
		\hline
		\textbf{Parametername} & \textbf{Parameterbeschreibung} \\ \hline
		id & Identifikator einer historischen Adresse \\ \hline
	\end{tabular}
\end{table}
\paragraph{Beschreibung} Die Funktion schaltet das Änderungsfeld einer historischen Adresse frei. Es findet bei dieser Funktion kein Abruf von Daten aus {\glqq COSP\grqq} statt.
\subsubsection{checkNStep}
\paragraph{Parameter} Die Funktion besitzt keine Parameter.
\paragraph{Beschreibung} Die Funktion öffnet das {\glqq nächste Schritte\grqq}-Modal, wenn eine entsprechende Question-URL gesetzt wurde. Es findet bei dieser Funktion kein Abruf von Daten aus {\glqq COSP\grqq} statt.
\subsubsection{CheckCommentShow}
\paragraph{Parameter} Die Funktion besitzt keine Parameter.
\paragraph{Beschreibung} Die Funktion prüft ob ein Modal geöffnet werden soll, öffnet dieses gegebenenfalls und entfernt den entsprechenden Cookies. Es findet bei dieser Funktion kein Abruf von Daten aus {\glqq COSP\grqq} statt.
\subsubsection{CheckCommentShow}
\paragraph{Parameter} Die Funktion besitzt keine Parameter.
\paragraph{Beschreibung} Die Funktion prüft ob der Focus auf bestimmte Koordinaten gelegt werden soll, legt diesen gegebenenfalls entsprechend und entfernt den entsprechenden Cookies. Es findet bei dieser Funktion kein Abruf von Daten aus {\glqq COSP\grqq} statt.
\subsubsection{openSelectMorePicturesOnMap}
\paragraph{Parameter} Die Funktion besitzt folgende Parameter:
\begin{table}[H]
	\begin{tabular}{|c|p{11cm}|}
		\hline
		\textbf{Parametername} & \textbf{Parameterbeschreibung} \\ \hline
		poiid & Identifikator eines Interessenpunktes \\ \hline
	\end{tabular}
\end{table}
\paragraph{Beschreibung} Die Funktion öffnet die Bilderauswahl für einen bestimmten Interessenpunkt um Bilder für Links zwischen einem Interessenpunkt und Bildern an zu legen. Es findet bei dieser Funktion kein Abruf von Daten aus {\glqq COSP\grqq} statt.
\subsubsection{saveSelectMorePicturesOnMap}
\paragraph{Parameter} Die Funktion besitzt folgende Parameter:
\begin{table}[H]
	\begin{tabular}{|c|p{11cm}|}
		\hline
		\textbf{Parametername} & \textbf{Parameterbeschreibung} \\ \hline
		poiid & Identifikator eines Interessenpunktes \\ \hline
	\end{tabular}
\end{table}
\paragraph{Beschreibung} Die Funktion speichert neue Links zwischen Bildern und einem Interessenpunkt. Anschließend wird wieder das {\glqq Mehr Anzeigen\grqq}-Modal angezeigt. Die Funktion hat Auswirkungen auf folgende Quellen:
\begin{itemize}
	\item Frontend-API
\end{itemize}
Es findet bei dieser Funktion kein Abruf von Daten aus {\glqq COSP\grqq} statt.
\subsubsection{abortSelectMorePicturesOnMap}
\paragraph{Parameter} Die Funktion besitzt folgende Parameter:
\begin{table}[H]
	\begin{tabular}{|c|p{11cm}|}
		\hline
		\textbf{Parametername} & \textbf{Parameterbeschreibung} \\ \hline
		poiid & Identifikator eines Interessenpunktes \\ \hline
	\end{tabular}
\end{table}
\paragraph{Beschreibung} Die Funktion bricht die Bilderauswahl ab und öffnet anschließend das {\glqq Mehr Anzeigen\grqq}-Modal des zuvor gewählten Interessenpunktes. Es findet bei dieser Funktion kein Abruf von Daten aus {\glqq COSP\grqq} statt.
\subsubsection{saveSeatCount}
\paragraph{Parameter} Die Funktion besitzt keine Parameter.
\paragraph{Beschreibung} Die Funktion fügt eine neue Sitzplatzanzahl einem Interessenpunkt hinzu und lädt diese anschließend neu. Die Funktion hat Auswirkungen auf folgende Quellen:
\begin{itemize}
	\item Frontend-API
\end{itemize}
Es findet bei dieser Funktion kein Abruf von Daten aus {\glqq COSP\grqq} statt.
\subsubsection{updateSeatCount}
\paragraph{Parameter} Die Funktion besitzt folgende Parameter:
\begin{table}[H]
	\begin{tabular}{|c|p{11cm}|}
		\hline
		\textbf{Parametername} & \textbf{Parameterbeschreibung} \\ \hline
		seatId & Identifikator einer Sitzplatzanzahl \\ \hline
	\end{tabular}
\end{table}
\paragraph{Beschreibung} Die Funktion aktualisiert einen Eintrag einer Sitzplatzanzahl. Anschließend wird die Anzeige neu geladen. Die Funktion hat Auswirkungen auf folgende Quellen:
\begin{itemize}
	\item Frontend-API
\end{itemize}
Es findet bei dieser Funktion kein Abruf von Daten aus {\glqq COSP\grqq} statt.
\subsubsection{editSeats}
\paragraph{Parameter} Die Funktion besitzt folgende Parameter:
\begin{table}[H]
	\begin{tabular}{|c|p{11cm}|}
		\hline
		\textbf{Parametername} & \textbf{Parameterbeschreibung} \\ \hline
		id & Identifikator einer Sitzplatzanzahl \\ \hline
	\end{tabular}
\end{table}
\paragraph{Beschreibung} Die Funktion schaltet das Änderungsfeld einer Sitzplatzanzahl frei. Es findet bei dieser Funktion kein Abruf von Daten aus {\glqq COSP\grqq} statt.
\subsubsection{saveCinemaCount}
\paragraph{Parameter} Die Funktion besitzt keine Parameter.
\paragraph{Beschreibung} Die Funktion fügt eine neue Saalanzahl einem Interessenpunkt hinzu und lädt diese anschließend neu. Die Funktion hat Auswirkungen auf folgende Quellen:
\begin{itemize}
	\item Frontend-API
\end{itemize}
Es findet bei dieser Funktion kein Abruf von Daten aus {\glqq COSP\grqq} statt.
\subsubsection{updateCinemaCount}
\paragraph{Parameter} Die Funktion besitzt folgende Parameter:
\begin{table}[H]
	\begin{tabular}{|c|p{11cm}|}
		\hline
		\textbf{Parametername} & \textbf{Parameterbeschreibung} \\ \hline
		cinemaId & Identifikator einer Saalanzahl \\ \hline
	\end{tabular}
\end{table}
\paragraph{Beschreibung} Die Funktion aktualisiert einen Eintrag einer Saalanzahl. Anschließend wird die Anzeige neu geladen. Die Funktion hat Auswirkungen auf folgende Quellen:
\begin{itemize}
	\item Frontend-API
\end{itemize}
Es findet bei dieser Funktion kein Abruf von Daten aus {\glqq COSP\grqq} statt.
\subsubsection{editCinemas}
\paragraph{Parameter} Die Funktion besitzt folgende Parameter:
\begin{table}[H]
	\begin{tabular}{|c|p{11cm}|}
		\hline
		\textbf{Parametername} & \textbf{Parameterbeschreibung} \\ \hline
		id & Identifikator einer Saalanzahl \\ \hline
	\end{tabular}
\end{table}
\paragraph{Beschreibung} Die Funktion schaltet das Änderungsfeld einer Saalanzahl frei. Es findet bei dieser Funktion kein Abruf von Daten aus {\glqq COSP\grqq} statt.
\subsubsection{showPopover}
\paragraph{Parameter} Die Funktion besitzt folgende Parameter:
\begin{table}[H]
	\begin{tabular}{|c|p{11cm}|}
		\hline
		\textbf{Parametername} & \textbf{Parameterbeschreibung} \\ \hline
		popoverId & Identifikator eines benutzerdefinierten Popovers \\ \hline
	\end{tabular}
\end{table}
\paragraph{Beschreibung} Die Funktion zeigt ein benutzerdefiniertes Popover an. Es findet bei dieser Funktion kein Abruf von Daten aus {\glqq COSP\grqq} statt.
\subsubsection{hidePopover}
\paragraph{Parameter} Die Funktion besitzt folgende Parameter:
\begin{table}[H]
	\begin{tabular}{|c|p{11cm}|}
		\hline
		\textbf{Parametername} & \textbf{Parameterbeschreibung} \\ \hline
		popoverId & Identifikator eines benutzerdefinierten Popovers \\ \hline
	\end{tabular}
\end{table}
\paragraph{Beschreibung} Die Funktion blendet ein benutzerdefiniertes Popover aus. Es findet bei dieser Funktion kein Abruf von Daten aus {\glqq COSP\grqq} statt.
\subsubsection{setCinemaType}
\paragraph{Parameter} Die Funktion besitzt folgende Parameter:
\begin{table}[H]
	\begin{tabular}{|c|p{11cm}|}
		\hline
		\textbf{Parametername} & \textbf{Parameterbeschreibung} \\ \hline
		typeId   & Identifikator eines Typs \\ \hline
		typeName & Name eines Typs \\ \hline
	\end{tabular}
\end{table}
\paragraph{Beschreibung} Die Funktion setzt den Identifikator eines Typs in ein verstecktes Input Feld und beschriftet den Dropdown-Button neu. Es findet bei dieser Funktion kein Abruf von Daten aus {\glqq COSP\grqq} statt.
\subsubsection{saveAddNewSourceShowMore}
\paragraph{Parameter} Die Funktion besitzt keine Parameter.
\paragraph{Beschreibung} Die Funktion fügt einem Interessenpunkt eine neue Quelle hinzu. Es findet bei dieser Funktion kein Abruf von Daten aus {\glqq COSP\grqq} statt.
\subsubsection{ShowMoreSources}
\paragraph{Parameter} Die Funktion besitzt folgende Parameter:
\begin{table}[H]
	\begin{tabular}{|c|p{11cm}|}
		\hline
		\textbf{Parametername} & \textbf{Parameterbeschreibung} \\ \hline
		poiid & Identifikator eines Interessenpunktes \\ \hline
	\end{tabular}
\end{table}
\paragraph{Beschreibung} Die Funktion lädt alle Quellen eines Interessenpunktes. Anschließend werden diese Angezeigt. Die Funktion nutzt folgende Quellen:
\begin{itemize}
	\item Frontend-API
\end{itemize}
Es findet bei dieser Funktion kein Abruf von Daten aus {\glqq COSP\grqq} statt.
\subsubsection{enableEditSourceShowMore}
\paragraph{Parameter} Die Funktion besitzt folgende Parameter:
\begin{table}[H]
	\begin{tabular}{|c|p{11cm}|}
		\hline
		\textbf{Parametername} & \textbf{Parameterbeschreibung} \\ \hline
		id & Identifikator einer Quelle \\ \hline
	\end{tabular}
\end{table}
\paragraph{Beschreibung} Die Funktion macht einen Quelleneintrag editierbar. Es findet bei dieser Funktion kein Abruf von Daten aus {\glqq COSP\grqq} statt.
\subsubsection{saveEditSourceShowMore}
\paragraph{Parameter} Die Funktion besitzt folgende Parameter:
\begin{table}[H]
	\begin{tabular}{|c|p{11cm}|}
		\hline
		\textbf{Parametername} & \textbf{Parameterbeschreibung} \\ \hline
		id & Identifikator einer Quelle \\ \hline
	\end{tabular}
\end{table}
\paragraph{Beschreibung} Die Funktion ändert einen Quelleneintrag. Es findet bei dieser Funktion kein Abruf von Daten aus {\glqq COSP\grqq} statt.
\subsubsection{finalDeleteSourceShowMore}
\paragraph{Parameter} Die Funktion besitzt folgende Parameter:
\begin{table}[H]
	\begin{tabular}{|c|p{11cm}|}
		\hline
		\textbf{Parametername} & \textbf{Parameterbeschreibung} \\ \hline
		id    & Identifikator einer Quelle \\ \hline
		poiid & Identifikator des aktuellen Interessenpunktes \\ \hline
	\end{tabular}
\end{table}
\paragraph{Beschreibung} Die Funktion löscht einen Quelleneintrag endgültig. Es findet bei dieser Funktion kein Abruf von Daten aus {\glqq COSP\grqq} statt.
\subsubsection{restoreSourceShowMore}
\paragraph{Parameter} Die Funktion besitzt folgende Parameter:
\begin{table}[H]
	\begin{tabular}{|c|p{11cm}|}
		\hline
		\textbf{Parametername} & \textbf{Parameterbeschreibung} \\ \hline
		id    & Identifikator einer Quelle \\ \hline
		poiid & Identifikator des aktuellen Interessenpunktes \\ \hline
	\end{tabular}
\end{table}
\paragraph{Beschreibung} Die Funktion stellt einen Quelleneintrag wieder her. Es findet bei dieser Funktion kein Abruf von Daten aus {\glqq COSP\grqq} statt.
\subsubsection{validatePoi}
\paragraph{Parameter} Die Funktion besitzt folgende Parameter:
\begin{table}[H]
	\begin{tabular}{|c|p{11cm}|}
		\hline
		\textbf{Parametername} & \textbf{Parameterbeschreibung} \\ \hline
		id    & Identifikator des aktuellen Interessenpunktes \\ \hline
	\end{tabular}
\end{table}
\paragraph{Beschreibung} Die Funktion validiert einen Interessenpunkt. Es findet bei dieser Funktion kein Abruf von Daten aus {\glqq COSP\grqq} statt.
\subsubsection{deletePoiMap}
\paragraph{Parameter} Die Funktion besitzt folgende Parameter:
\begin{table}[H]
	\begin{tabular}{|c|p{11cm}|}
		\hline
		\textbf{Parametername} & \textbf{Parameterbeschreibung} \\ \hline
		id    & Identifikator des aktuellen Interessenpunktes \\ \hline
	\end{tabular}
\end{table}
\paragraph{Beschreibung} Die Funktion löscht einen Interessenpunkt oder markiert diesen als gelöscht. Es findet bei dieser Funktion kein Abruf von Daten aus {\glqq COSP\grqq} statt.
\subsubsection{restorePoiMap}
\paragraph{Parameter} Die Funktion besitzt folgende Parameter:
\begin{table}[H]
	\begin{tabular}{|c|p{11cm}|}
		\hline
		\textbf{Parametername} & \textbf{Parameterbeschreibung} \\ \hline
		id    & Identifikator des aktuellen Interessenpunktes \\ \hline
	\end{tabular}
\end{table}
\paragraph{Beschreibung} Die Funktion stellt einen Interessenpunkt wieder her. Es findet bei dieser Funktion kein Abruf von Daten aus {\glqq COSP\grqq} statt.
\subsubsection{finalDeletePoiMap}
\paragraph{Parameter} Die Funktion besitzt folgende Parameter:
\begin{table}[H]
	\begin{tabular}{|c|p{11cm}|}
		\hline
		\textbf{Parametername} & \textbf{Parameterbeschreibung} \\ \hline
		id    & Identifikator des aktuellen Interessenpunktes \\ \hline
	\end{tabular}
\end{table}
\paragraph{Beschreibung} Die Funktion stellt einen Interessenpunkt wieder her. Es findet bei dieser Funktion kein Abruf von Daten aus {\glqq COSP\grqq} statt.
\subsubsection{changeIconSelect}
\paragraph{Parameter} Die Funktion besitzt keine Parameter.
\paragraph{Beschreibung} Die Funktion ändert die Icons in der Auswahlleiste des Typs der Interessenpunkte. Es findet bei dieser Funktion kein Abruf von Daten aus {\glqq COSP\grqq} statt.
\subsubsection{setfocus}
\paragraph{Parameter} Die Funktion besitzt folgende Parameter:
\begin{table}[H]
	\begin{tabular}{|c|p{11cm}|}
		\hline
		\textbf{Parametername} & \textbf{Parameterbeschreibung} \\ \hline
		lng & Längengrad \\ \hline
		lat & Breitengrad \\ \hline
	\end{tabular}
\end{table}
\paragraph{Beschreibung} Die Funktion setzt den Focus der Karte auf die angegebenen Koordinaten bei einem Zoom von 17,5. Es findet bei dieser Funktion kein Abruf von Daten aus {\glqq COSP\grqq} statt.
\newpage
\section{Marker}
\subsection{Allgemeines} Diese Datei enthält alle grundlegenden JavaScript-Funktionen.
Die Ausführung des Codes findet im Browser statt. Auch hier die Marker Icons, welche in verschiedenen anderen Javascripts verwendeten werden, festgelegt. Siehe hierzu:
\begin{lstlisting}[language=JavaScript]
	var redIcon = new L.Icon({
		iconUrl: 'images/map-marker-red.svg',
		iconSize: [25, 41],
		iconAnchor: [12, 41],
		popupAnchor: [1, -34],
		shadowSize: [41, 41]
	});
	
	var greenIcon = new L.Icon({
		iconUrl: 'images/map-marker-green.svg',
		iconSize: [25, 41],
		iconAnchor: [12, 41],
		popupAnchor: [1, -34],
		shadowSize: [41, 41]
	});
	
	var blackIcon = new L.Icon({
		iconUrl: 'images/map-marker-black.svg',
		iconSize: [25, 41],
		iconAnchor: [12, 41],
		popupAnchor: [1, -34],
		shadowSize: [41, 41]
	});
	
	var blueIcon = new L.Icon({
		iconUrl: 'images/map-marker-blue.svg',
		iconSize: [25, 41],
		iconAnchor: [12, 41],
		popupAnchor: [1, -34],
		shadowSize: [41, 41]
	});
	
	var deletedGreenIcon = new L.Icon({
		iconUrl: 'images/map-marker-green-del.svg',
		iconSize: [25, 41],
		iconAnchor: [12, 41],
		popupAnchor: [1, -34],
		shadowSize: [41, 41]
	});
	
	var deletedRedIcon = new L.Icon({
		iconUrl: 'images/map-marker-red-del.svg',
		iconSize: [25, 41],
		iconAnchor: [12, 41],
		popupAnchor: [1, -34],
		shadowSize: [41, 41]
	});
	
	var deletedBlueIcon = new L.Icon({
		iconUrl: 'images/map-marker-blue-del.svg',
		iconSize: [25, 41],
		iconAnchor: [12, 41],
		popupAnchor: [1, -34],
		shadowSize: [41, 41]
	});
\end{lstlisting}
\newpage
\section{MarkerFunctions}
\subsection{Allgemeines} Diese Datei enthält alle grundlegenden JavaScript-Funktionen für Kartenmarker.
Die Ausführung des Codes findet im Browser statt. Auch werden hier einige in verschiedenen anderen Javascripts verwendeten Variablen festgelegt.
\begin{lstlisting}[language=JavaScript]
var anz = 0;
var anzEditMap = 0;
\end{lstlisting}
\newpage
\subsection{Funktionen}
\subsubsection{delMark}
\paragraph{Parameter} Die Funktion besitzt keine Parameter.
\paragraph{Beschreibung} Die Funktion löscht den gesetzten Marker auf der Karte. Es findet bei dieser Funktion kein Abruf von Daten aus {\glqq COSP\grqq} statt.
\subsubsection{onMove}
\paragraph{Parameter} Die Funktion besitzt folgende Parameter:
\begin{table}[H]
	\begin{tabular}{|c|p{11cm}|}
		\hline
		\textbf{Parametername} & \textbf{Parameterbeschreibung} \\ \hline
		e       & Marker \\ \hline
		editMap & Gibt an, ob das JavaScript auf der Interessenpunkt-Ändern-Seite ausgeführt wird \\ \hline
		Minimap & Gibt an, ob der Marker auf der Minimap ist \\ \hline
	\end{tabular}
\end{table}
\paragraph{Beschreibung} Die Funktion aktualisiert die Koordinaten in den verdeckten Formularfeldern, wenn der Marker bewegt wird. Es findet bei dieser Funktion kein Abruf von Daten aus {\glqq COSP\grqq} statt.
\subsubsection{onMoveMM}
\paragraph{Parameter} Die Funktion besitzt folgende Parameter:
\begin{table}[H]
	\begin{tabular}{|c|p{11cm}|}
		\hline
		\textbf{Parametername} & \textbf{Parameterbeschreibung} \\ \hline
		latLng & Array mit Längen- und Breitengrad \\ \hline
	\end{tabular}
\end{table}
\paragraph{Beschreibung} Die Funktion aktualisiert die Koordinaten in den verdeckten Formularfeldern, wenn die Minimap beim einfügen eines neuen Interessenpunktes aktiv ist. Es findet bei dieser Funktion kein Abruf von Daten aus {\glqq COSP\grqq} statt.
\subsubsection{placeMarker}
\paragraph{Parameter} Die Funktion besitzt folgende Parameter:
\begin{table}[H]
	\begin{tabular}{|c|p{11cm}|}
		\hline
		\textbf{Parametername} & \textbf{Parameterbeschreibung} \\ \hline
		coordinates & Array mit Längen- und Breitengrad \\ \hline
		editMap     & Gibt an, ob das JavaScript auf der Interessenpunkt-Ändern-Seite ausgeführt wird \\ \hline
	\end{tabular}
\end{table}
\paragraph{Beschreibung} Die Funktion setzt den Interessenpunkt-Hinzufügen-Marker an der gegebenen Stelle. Es findet bei dieser Funktion kein Abruf von Daten aus {\glqq COSP\grqq} statt.
\subsubsection{insertCoordinates}
\paragraph{Parameter} Die Funktion besitzt folgende Parameter:
\begin{table}[H]
	\begin{tabular}{|c|p{11cm}|}
		\hline
		\textbf{Parametername} & \textbf{Parameterbeschreibung} \\ \hline
		editMap     & Gibt an, ob das JavaScript auf der Interessenpunkt-Ändern-Seite ausgeführt wird \\ \hline
	\end{tabular}
\end{table}
\paragraph{Beschreibung} Die Funktion aktualisiert die verdeckten Formularfelder mit aktualisierten Koordinaten. Es findet bei dieser Funktion kein Abruf von Daten aus {\glqq COSP\grqq} statt.
\subsubsection{resizeMap}
\paragraph{Parameter} Die Funktion besitzt folgende Parameter:
\begin{table}[H]
	\begin{tabular}{|c|p{11cm}|}
		\hline
		\textbf{Parametername} & \textbf{Parameterbeschreibung} \\ \hline
		map         & Variable, welche Karte darstellt \\ \hline
		coordinates & Array mit Längen- und Breitengrad \\ \hline
	\end{tabular}
\end{table}
\paragraph{Beschreibung} Die Funktion aktualisiert den gezeigten Ausschnitt der angegeben Karte mit 300 Milisekunden verzögerung. Es findet bei dieser Funktion kein Abruf von Daten aus {\glqq COSP\grqq} statt.
\newpage
\section{pictureUploadNew}
\subsection{Allgemeines} Diese Datei enthält alle grundlegenden JavaScript-Funktionen für Bilder Upload und Vorschau.
Die Ausführung des Codes findet im Browser statt. Hier werden die auszuführenden Funktionen beim Bedienen der entsprechenden Elemente gesetzt.
\subsection{Funktionen}
\subsubsection{readURL}
\paragraph{Parameter} Die Funktion besitzt folgende Parameter:
\begin{table}[H]
	\begin{tabular}{|c|p{11cm}|}
		\hline
		\textbf{Parametername} & \textbf{Parameterbeschreibung} \\ \hline
		input & Dateiauswahlfeld \\ \hline
	\end{tabular}
\end{table}
\paragraph{Beschreibung} Die Funktion lädt ein Bild zur Vorschau. Es findet bei dieser Funktion kein Abruf von Daten aus {\glqq COSP\grqq} statt.

\newpage
\section{search}
\subsection{Allgemeines} Diese Datei enthält diverse Hilfsfunktionen und ermöglicht das Suchen.
Die Ausführung des Codes findet im Browser statt. Hier wird der Trigger zum Ausführen der Suche bei Enter gesetzt, sowie zur Auswertung gesetzter Cookies:
\begin{lstlisting}[language=JavaScript]
$(document).ready(function () {
	// Trigger search button on enter at search input
	$("#search").keyup(function (e) {
		if(e.which === 13) {
			$("#poiSearchButton").click();
		}
	});
	
	//On pressing a key on "Search box" in "search.php" file. This function will be called.
	$("#poiSearchButton").click(function () {
		var name = $('#search').val();
		if (name === "") {
			$("#display").html("");
			pois = null;
		} else {
			if (pois == null) {
				$.ajax({
					type: "POST",
					contentType: "application/json",
					url: "Formular/api.php",
					data: JSON.stringify({
						type: "gpu"
					}),
					success: create
				});
			} else {
				create();
			}
		}
	});
	var OpenPersonalArea = getCookie("personalArea");
	if (OpenPersonalArea === "1") {
		var delayInMilliseconds = 300;
		setTimeout(function () {
			deleteCookie("personalArea");
			loadPersonalArea();
		}, delayInMilliseconds);
	}
});
\end{lstlisting} 
Auch werden hier einige in verschiedenen anderen Javascripts verwendeten Variablen festgelegt.
\begin{lstlisting}[language=JavaScript]
var pois = null;
var focusComment = -1;
\end{lstlisting} 
\newpage
\subsection{Funktionen}
\subsubsection{openSearchModal}
\paragraph{Parameter} Die Funktion besitzt folgende Parameter:
\begin{table}[H]
	\begin{tabular}{|c|p{11cm}|}
		\hline
		\textbf{Parametername} & \textbf{Parameterbeschreibung} \\ \hline
		json & Array mit Daten \\ \hline
	\end{tabular}
\end{table}
\subparagraph{\$json}Das Array enthält Einträge mit folgenden Elementen:
\begin{table}[H]
	\begin{tabular}{|c|p{11cm}|}
		\hline
		\textbf{Parametername} & \textbf{Parameterbeschreibung} \\ \hline
		name             & Namen eines Interessenpunktes  \\ \hline
		current\_address & Aktuelle Adresse eines Interessenpunktes  \\ \hline
		hist\_address    & historische Adresse eines Interessenpunktes (veraltet) \\ \hline
		operator         & Betreiber eines Interessenpunktes (veraltet) \\ \hline
		history          & Geschichte eines Interessenpunktes \\ \hline
		lng              & Längengrad eines Interessenpunktes \\ \hline
		lat              & Breitengrad eines Interessenpunktes \\ \hline
	\end{tabular}
\end{table}
\paragraph{Beschreibung} Die Funktion öffnet das Such-Modal und sucht in den gegebenen Daten. Die FUnktion nutzt folgende Quellen:
\begin{itemize}
	\item Frontend-API
\end{itemize}
Es findet bei dieser Funktion kein Abruf von Daten aus {\glqq COSP\grqq} statt.
\subsubsection{focusPOI}
\paragraph{Parameter} Die Funktion besitzt folgende Parameter:
\begin{table}[H]
	\begin{tabular}{|c|p{11cm}|}
		\hline
		\textbf{Parametername} & \textbf{Parameterbeschreibung} \\ \hline
		lng & Längengrad \\ \hline
		lat & Breitengrad \\ \hline
	\end{tabular}
\end{table}
\paragraph{Beschreibung} Die Funktion setzt den Focus auf die gegebenen Koordinaten und schließt das Such-Modal. Es findet bei dieser Funktion kein Abruf von Daten aus {\glqq COSP\grqq} statt.

\newpage
\section{slider}
\subsection{Allgemeines} Diese Datei dient zum Initialisieren des Sliders.
Die Ausführung des Codes findet im Browser statt. Hier wird der Slider mit entsprechenden Daten initialisiert und Aktionen, welche beim verschieben des Sliders ausgeführt werden sollen gesetzt.
\newpage
\section{statistics}
\subsection{Allgemeines} Diese Datei zeigt statistische Grafen an.
Die Datei ist direkt durch den Nutzer aufrufbar. Sie setzt auch die entsprechende Konstante und bindet alle notwendigen Dateien ein:
\begin{lstlisting}[language=php]
define('NICE_PROJECT', true);
require_once "bin/inc.php";
\end{lstlisting}
\subsection{Allgemeines}
Auf dieser Seite sind mittels {\glqq Chart.js\grqq} erzeugte Diagramme für statistische Daten wie neue oder geänderte Interessenpunkte beziehungsweise Kommentare zu sehen.
\subsection{Besonderheiten}
Diese Seite kann nur durch Mitarbeiter und Administratoren eingesehen werden.
\newpage
\section{StoryUpload}
\subsection{Allgemeines} Diese Datei dient enthält alle Funktionen, welche für die Geschichten-Seite zusätzlich benötigt werden.
Die Ausführung des Codes findet im Browser statt. Hier wird eine Variable für die Sortierung der Geschichten gesetzt:
\begin{lstlisting}[language=JavaScript]
var sortdown = true;
\end{lstlisting} 
Des Weiteren wird auch das initiale Laden der Daten hier ausgelöst:
\begin{lstlisting}[language=JavaScript]
window.onload = function () {
	getAllStories();
};
\end{lstlisting} 
\newpage
\subsection{Funktionen}
\subsubsection{updateSortType}
\paragraph{Parameter} Die Funktion besitzt folgende Parameter:
\begin{table}[H]
	\begin{tabular}{|c|p{11cm}|}
		\hline
		\textbf{Parametername} & \textbf{Parameterbeschreibung} \\ \hline
		sortDownState & Wahr, wenn Absteigend sortiert werden soll \\ \hline
	\end{tabular}
\end{table}
\paragraph{Beschreibung} Die Funktion updated die Variable, welche die Sortierung der Geschichten angibt und löst eine erneute Sortierung aus, bei welcher anschließend auch die Anzeige aktualisiert wird. Es findet bei dieser Funktion kein Abruf von Daten aus {\glqq COSP\grqq} statt.
\subsubsection{FilterStorys}
\paragraph{Parameter} Die Funktion besitzt keine Parameter.
\paragraph{Beschreibung} Die Funktion löst eine erneute Sortierung und Filterung der Geschichten aus, bei welcher anschließend auch die Anzeige aktualisiert wird. Es findet bei dieser Funktion kein Abruf von Daten aus {\glqq COSP\grqq} statt.
\subsubsection{saveLinkedPoi}
\paragraph{Parameter} Die Funktion besitzt keine Parameter.
\paragraph{Beschreibung} Die Funktion speichert einen Link zwischen einer Geschichte und einem Interessenpunkt. Die Funktion hat Auswirkungen auf folgende Quellen:
\begin{itemize}
	\item Frontend-API
\end{itemize}
Es findet bei dieser Funktion kein Abruf von Daten aus {\glqq COSP\grqq} statt.
\newpage
\section{personalArea}
\subsection{Allgemeines} Diese Datei enthält alle JavaScript-Funktionen, welche für den persönlichen Bereich benötigt werden.
Die Ausführung des Codes findet im Browser statt.
\subsection{Funktionen}
\subsubsection{loadPersonalArea}
\paragraph{Parameter} Die Funktion besitzt keine Parameter.
\paragraph{Beschreibung} Die Funktion lädt alle benötigten Daten für den persönlichen Bereich und zeigt diesen an. Die Funktion nutzt folgende Quellen:
\begin{itemize}
	\item Frontend-API
\end{itemize}
Es findet bei dieser Funktion kein Abruf von Daten aus {\glqq COSP\grqq} statt.
\subsubsection{deletePoiFinalPersonalArea}
\paragraph{Parameter} Die Funktion besitzt folgende Parameter:
\begin{table}[H]
	\begin{tabular}{|c|p{11cm}|}
		\hline
		\textbf{Parametername} & \textbf{Parameterbeschreibung} \\ \hline
		id & Identifikator eines Interessenpunktes \\ \hline
	\end{tabular}
\end{table}
\paragraph{Beschreibung} Die Funktion löscht einen Interessenpunkt endgültig und lädt den persönlichen Bereich neu. Die Funktion hat Auswirkungen auf folgende Quellen:
\begin{itemize}
	\item Frontend-API
\end{itemize}
Es findet bei dieser Funktion kein Abruf von Daten aus {\glqq COSP\grqq} statt.
\subsubsection{restorePoiPersonalArea}
\paragraph{Parameter} Die Funktion besitzt folgende Parameter:
\begin{table}[H]
	\begin{tabular}{|c|p{11cm}|}
		\hline
		\textbf{Parametername} & \textbf{Parameterbeschreibung} \\ \hline
		id & Identifikator eines Interessenpunktes \\ \hline
	\end{tabular}
\end{table}
\paragraph{Beschreibung} Die Funktion stellt einen Interessenpunkt wieder her und lädt den persönlichen Bereich neu. Die Funktion hat Auswirkungen auf folgende Quellen:
\begin{itemize}
	\item Frontend-API
\end{itemize}
Es findet bei dieser Funktion kein Abruf von Daten aus {\glqq COSP\grqq} statt.
\subsubsection{deleteCommentFinalPersonalArea}
\paragraph{Parameter} Die Funktion besitzt folgende Parameter:
\begin{table}[H]
	\begin{tabular}{|c|p{11cm}|}
		\hline
		\textbf{Parametername} & \textbf{Parameterbeschreibung} \\ \hline
		id & Identifikator eines Kommentars \\ \hline
	\end{tabular}
\end{table}
\paragraph{Beschreibung} Die Funktion löscht einen Kommentar endgültig und lädt den persönlichen Bereich neu. Die Funktion hat Auswirkungen auf folgende Quellen:
\begin{itemize}
	\item Frontend-API
\end{itemize}
Es findet bei dieser Funktion kein Abruf von Daten aus {\glqq COSP\grqq} statt.
\subsubsection{restoreCommentPersonalArea}
\paragraph{Parameter} Die Funktion besitzt folgende Parameter:
\begin{table}[H]
	\begin{tabular}{|c|p{11cm}|}
		\hline
		\textbf{Parametername} & \textbf{Parameterbeschreibung} \\ \hline
		id & Identifikator eines Kommentars \\ \hline
	\end{tabular}
\end{table}
\paragraph{Beschreibung} Die Funktion stellt einen Kommentar wieder her und lädt den persönlichen Bereich neu. Die Funktion hat Auswirkungen auf folgende Quellen:
\begin{itemize}
	\item Frontend-API
\end{itemize}
Es findet bei dieser Funktion kein Abruf von Daten aus {\glqq COSP\grqq} statt.
\subsubsection{editComment}
\paragraph{Parameter} Die Funktion besitzt folgende Parameter:
\begin{table}[H]
	\begin{tabular}{|c|p{11cm}|}
		\hline
		\textbf{Parametername} & \textbf{Parameterbeschreibung} \\ \hline
		commentID & Identifikator eines Kommentars \\ \hline
	\end{tabular}
\end{table}
\paragraph{Beschreibung} Die Funktion bereitet das Kommentar-Ändern-Modal vor und öffnet dieses. Es findet bei dieser Funktion kein Abruf von Daten aus {\glqq COSP\grqq} statt.
\subsubsection{saveEditedComment}
\paragraph{Parameter} Die Funktion besitzt keine Parameter.
\paragraph{Beschreibung} Die Funktion speichert einen geänderten Kommentar und schließt das entsprechende Modal. Die Funktion hat Auswirkungen auf folgende Quellen:
\begin{itemize}
	\item Frontend-API
\end{itemize}
Es findet bei dieser Funktion kein Abruf von Daten aus {\glqq COSP\grqq} statt.
\newpage
\section{archive}
\input{Kapitel/Files/js/archive}
\newpage
\section{mapFnc}
\subsection{Allgemeines} Diese Datei dient enthält alle Funktionen, welche für die Karten-Seite zusätzlich benötigt werden.
\newpage
\subsection{Funktionen}
\subsubsection{checkInputDataAddPOI}
\paragraph{Parameter} Die Funktion besitzt keine Parameter.
\paragraph{Beschreibung} Die Funktion prüft die Eingaben beim Erstellen eines neuen Interessenpunktes. Es findet bei dieser Funktion kein Abruf von Daten aus {\glqq COSP\grqq} statt.
\subsubsection{checkYearInputAddPoi}
\paragraph{Parameter} Die Funktion besitzt keine Parameter.
\paragraph{Beschreibung} Die Funktion prüft die Jahreszahleingaben beim Erstellen eines neuen Interessenpunktes. Es findet bei dieser Funktion kein Abruf von Daten aus {\glqq COSP\grqq} statt.
\subsubsection{toggleHistoricalAdressAdd}
\paragraph{Parameter} Die Funktion besitzt keine Parameter.
\paragraph{Beschreibung} Zeigt oder verbirgt Eingabefelder für das Anlegen einer neuen historischen Adresse. Es findet bei dieser Funktion kein Abruf von Daten aus {\glqq COSP\grqq} statt.
\newpage
\section{materialList}
\subsection{Allgemeines} Diese Datei dient enthält alle Funktionen, welche für die Archiv-Seite zusätzlich benötigt werden.
\newpage
\subsection{Funktionen}
\subsubsection{focusPoiOfPicture}
\paragraph{Parameter} Die Funktion besitzt folgende Parameter:
\begin{table}[H]
	\begin{tabular}{|c|p{11cm}|}
		\hline
		\textbf{Parametername} & \textbf{Parameterbeschreibung} \\ \hline
		lat & Breitengrad \\ \hline
		lng & Längengrad \\ \hline
	\end{tabular}
\end{table}
\paragraph{Beschreibung} Die Funktion leitet auf die Kartenseite weiter und fokussiert dabei den entsprechenden Interessenpunkt. Es findet bei dieser Funktion kein Abruf von Daten aus {\glqq COSP\grqq} statt.
\newpage
\section{poiEdit}
\subsection{Allgemeines} Diese Datei dient enthält alle Funktionen, welche für die Änderungsseite von Interessenpunkten zusätzlich benötigt werden.
\newpage
\subsection{Funktionen}
\subsubsection{preparePicSelectModal}
\paragraph{Parameter} Die Funktion besitzt keine Parameter.
\paragraph{Beschreibung} Die Funktion bereitet das Öffnen des Bildauswahl-Modals vor und leitet das Öffnen ein. Es findet bei dieser Funktion kein Abruf von Daten aus {\glqq COSP\grqq} statt.
\subsubsection{abortSelectMorePicturesEditPoi}
\paragraph{Parameter} Die Funktion besitzt keine Parameter.
\paragraph{Beschreibung} Die Funktion schließt das Bildauswahl-Modal. Es findet bei dieser Funktion kein Abruf von Daten aus {\glqq COSP\grqq} statt.
\newpage
\section{registration}
\subsection{Allgemeines} Diese Datei zeigt ein Formular zur Selbstregistrierung an.
Die Datei ist direkt durch den Nutzer aufrufbar. Sie setzt auch die entsprechende Konstante und bindet alle notwendigen Dateien ein:
\begin{lstlisting}[language=php]
	define('NICE_PROJECT', true);
	require_once "bin/inc.php";
\end{lstlisting}
\subsection{Allgemeines}
Auf dieser Seite kann sich Nutzer selbst registrieren.
\subsection{Besonderheiten}
Die Seite sendet die für eine Registrierung benötigten Informationen an {\glqq COSP\grqq}.

\end{document}