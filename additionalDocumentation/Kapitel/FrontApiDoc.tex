\newpage
\section{Frontend API}\label{api}
\subsection{Befehlsübersicht}
\begin{longtable}[H]{|c|p{12cm}|}
		\hline
		\textbf{Api-Befehl} & \textbf{Kurzbeschreibung}                                                                                                                          \\ \hline
		pac                 & Abfrage der Daten für den persönlichen Bereich                                                                                                     \\ \hline
		duc                 & Löschen eines Kommentares                                                                                                                          \\ \hline
		auc                 & Kommentar hinzufügen                                                                                                                               \\ \hline
		smd                 & Alle Daten für das "Mehr Anzeigen"-Modal abfragen                                                                                                  \\ \hline
		gpu                 & Liste aller Interessenpunkte für aktuellen Nutzer abfragen                                                                                         \\ \hline
		mmy                 & \begin{tabular}[c]{@{}l@{}}Minimum und Maximum für der Jahreszahlen \\ der Betriebszeit der Interessenpunkte abfragen\end{tabular}                 \\ \hline
		ccp                 & URI der Zentralplattform "COSP" abfragen                                                                                                           \\ \hline
		aus                 & Neue Geschichte anlegen                                                                                                                            \\ \hline
		gas                 & Daten zu allen für den Nutzer verfügbaren Geschichten laden                                                                                        \\ \hline
		dpi                 & Interessenpunkt löschen                                                                                                                            \\ \hline
		gcs                 & Einzelnen Kommentar laden                                                                                                                          \\ \hline
		sec                 & Speichern eines bereits vorhandenen Kommentares                                                                                                    \\ \hline
		dsm                 & Daten für ein Bildabruf abfragen                                                                                                                   \\ \hline
		ssm                 & Daten eines bereits vorhanden Bildes speichern                                                                                                     \\ \hline
		eus                 & Daten einer bereits vorhandenen Geschichte speichern                                                                                               \\ \hline
		aha                 & Historische Adresse zu einem Interessenpunkt hinzufügen                                                                                            \\ \hline
		ado                 & Betreiber zu einem Interessenpunkt hinzufügen                                                                                                      \\ \hline
		adn                 & Name zu einem Interessenpunkt hinzufügen                                                                                                           \\ \hline
		dha                 & Löschen einer historischen Adresse eines Interessenpunktes                                                                                         \\ \hline
		dop                 & Löschen eines Betreibers eines Interessenpunktes                                                                                                   \\ \hline
		dna                 & Löschen eines Namen eines Interessenpunktes                                                                                                        \\ \hline
		vha                 & Validieren einer historischen Adresse eines Interessenpunktes                                                                                      \\ \hline
		vop                 & Validieren eines Betreibers eines Interessenpunktes                                                                                                \\ \hline
		vna                 & Validieren eines Namen eines Interessenpunktes                                                                                                     \\ \hline
		vts                 & Geschichte eines Interessenpunktes validieren                                                                                                      \\ \hline
		vca                 & Validieren der aktuellen Geschichte eines Interessenpunktes                                                                                        \\ \hline
		vhi                 & Validieren des Betriebszeitraum eines Interessenpunktes                                                                                            \\ \hline
		uha                 & Speichern einer bereits vorhandenen historischen Adresse eines Interessenpunktes                                                                   \\ \hline
		uop                 & Speichern einer bereits vorhandenen Betreibers eines Interessenpunktes                                                                             \\ \hline
		una                 & Speichern einer bereits vorhandenen Namen eines Interessenpunktes                                                                                  \\ \hline
		dsp                 & Löschen eines einzelnen Bildes                                                                                                                     \\ \hline
		gpt                 & Fragt alle Titel der Interessenpunkte ab                                                                                                           \\ \hline
		aps                 & Stellt einen Link zwischen Interessenpunkt und Geschichte her                                                                                      \\ \hline
		gps                 & Fragt alle Links zwischen Interessenpunkt und Geschichte auf Basis des Identifikators der Geschichte ab                                            \\ \hline
		vps                 & Validiert Link zwischen Geschichte und Interessenpunkt                                                                                             \\ \hline
		dps                 & Löscht Link zwischen Geschichte und Interessenpunkt                                                                                                \\ \hline
		dus                 & Löscht eine Geschichte                                                                                                                             \\ \hline
		cha                 & Prüft eine eingegebene Adresse, ob diese bereits bekannt                                                                                           \\ \hline
		gpf                 & Gibt Daten zum abrufen eines Originalbildes und einer Vorschau zurück                                                                              \\ \hline
		app                 & Stellt Link zwischen Bild und Interessenpunkt her                                                                                                  \\ \hline
		vpp                 & Validiert Link zwischen Bild und Interessenpunkt                                                                                                   \\ \hline
		dpp                 & Löscht Link zwischen Bild und Interessenpunkt                                                                                                      \\ \hline
		lpp                 & Lädt Daten für Bildauswahl                                                                                                                         \\ \hline
		asc                 & Fügt Sitzplatzanzahl zu Interessenpunkt hinzu                                                                                                      \\ \hline
		vsc                 & Validiert eine Sitzplatzanzahl zu Interessenpunktes                                                                                                \\ \hline
		dsc                 & Löscht eine Sitzplatzanzahl zu Interessenpunktes                                                                                                   \\ \hline
		usc                 & Speichert eine geänderte Sitzplatzanzahl eines Interessenpunktes                                                                                   \\ \hline
		acc                 & Fügt Saalanzahl zu Interessenpunkt hinzu                                                                                                           \\ \hline
		vcc                 & Validiert eine Saalanzahl zu Interessenpunktes                                                                                                     \\ \hline
		dcc                 & Löscht eine Saalanzahl zu Interessenpunktes                                                                                                        \\ \hline
		ucc                 & Speichert eine geänderte Saalanzahl eines Interessenpunktes                                                                                        \\ \hline
		vty                 & Validiert Kinotyp eines Interessenpunktes                                                                                                          \\ \hline
		asg                 & Abfrage ob Nutzer Gast ist                                                                                                                         \\ \hline
		gsd                 & Abfrage von Statistikdaten                                                                                                                         \\ \hline
		asa                 & Freigeben einer Geschichte                                                                                                                         \\ \hline
		das                 & Sperren einer Geschichte                                                                                                                           \\ \hline
		snp                 & Abfrage aller Namen eines Interessenpunktes                                                                                                        \\ \hline
		sop                 & Abfrage aller Betreiber eines Interessenpunktes                                                                                                    \\ \hline
		shp                 & Abfrage aller historischen Adressen eines Interessenpunktes                                                                                        \\ \hline
		gue                 & Abfrage ob Nutzer Gast ist, basierend auf Nutzername                                                                                               \\ \hline
		scp                 & Abfrage aller Saalanzahlen eines Interessenpunktes                                                                                                 \\ \hline
		ssp                 & Abfrage aller Sitzplatzanzahlen eines Interessenpunktes                                                                                            \\ \hline
		slp                 & Abfrage aller Geschichten-Interessenpunkt-Links                                                                                                    \\ \hline
		gsp                 & Abfrage aller Title der Geschichten die noch keine Link zu dem gewählten Interessenpunkt haben                                                     \\ \hline
		plp                 & Abfrage der Daten zum Laden des Hauptbildes eines Interessenpunktes                                                                                \\ \hline
		apl                 & Abfragen der Daten zu den zusätzlichen Bildern eines Interessenpunktes                                                                             \\ \hline
		lcp                 & laden der Kommentare eines Interessenpunktes                                                                                                       \\ \hline
		cse                 & Abfrage des Aktivierungsstatus des Geschichtenmoduls                                                                                               \\ \hline
		cpa                 & Anfordern eines Captcha-Codes                                                                                                                      \\ \hline
		cmg                 & Absenden einer Kontaktnachricht                                                                                                                    \\ \hline
		fdp                 & Finales Löschen einer Verknüpfung zwischen Interessenpunkt und Bild                                                                                \\ \hline
		rdp                 & Wiederherstellen einer Verknüpfung zwischen Interessenpunkt und Bild                                                                               \\ \hline
		fna                 & Finales Löschen eines Namen eines Interessenpunktes                                                                                                \\ \hline
		rna                 & Wiederherstellen eines Namen eines Interessenpunktes                                                                                               \\ \hline
		fop                 & Finales Löschen eines Betreibers eines Interessenpunktes                                                                                           \\ \hline
		rop                 & Wiederherstellen eines Betreibers eines Interessenpunktes                                                                                          \\ \hline
		fsc                 & Finales Löschen einer Sitzplatzanzahl eines Interessenpunktes                                                                                      \\ \hline
		rsc                 & Wiederherstellen einer Sitzplatzanzahl eines Interessenpunktes                                                                                     \\ \hline
		fcc                 & Finales Löschen einer Saalanzahl eines Interessenpunktes                                                                                           \\ \hline
		rcc                 & Wiederherstellen einer Saalanzahl eines Interessenpunktes                                                                                          \\ \hline
		fha                 & Finales Löschen einer historischen Adresse eines Interessenpunktes                                                                                 \\ \hline
		rha                 & Wiederherstellen einer historischen Adresse eines Interessenpunktes                                                                                \\ \hline
		fsp                 & Finales Löschen einer Verknüpfung zwischen Interessenpunkt und Geschichte                                                                          \\ \hline
		rsp                 & Wiederherstellen einer Verknüpfung zwischen Interessenpunkt und Geschichte                                                                         \\ \hline
		fcp                 & Finales Löschen eines Kommentars eines Interessenpunktes                                                                                           \\ \hline
		rcp                 & Wiederherstellen eines Kommentars eines Interessenpunktes                                                                                          \\ \hline
		fpi                 & Finales Löschen eines Interessenpunktes                                                                                                            \\ \hline
		rpi                 & Wiederherstellen eines Interessenpunktes                                                                                                           \\ \hline
		fst                 & Finales Löschen einer Geschichte                                                                                                                   \\ \hline
		rst                 & Wiederherstellen einer Geschichte                                                                                                                  \\ \hline
		aan                 & Ankündigung hinzufügen                                                                                                                             \\ \hline
		gan                 & Ankündigung Abfragen                                                                                                                               \\ \hline
		uan                 & Ankündigung Ändern                                                                                                                                 \\ \hline
		dan                 & Ankündigung Löschen                                                                                                                                \\ \hline
		gca                 & Aktuelle Ankündigungen abfragen                                                                                                                    \\ \hline
		saa                 & Ankündigungen aktivieren/deaktivieren                                                                                                              \\ \hline
		asp                 & Quelle für Interessenpunkt hinzufügen                                                                                                              \\ \hline
		grp                 & Quellen von Interessenpunkt abrufen                                                                                                                \\ \hline
		grs                 & Alle Bezüge von Quellen abfragen                                                                                                                   \\ \hline
		gts                 & Alle Typen von Quellen abfragen                                                                                                                    \\ \hline
		usp                 & Quelleneintrag ändern                                                                                                                              \\ \hline
		des                 & Quelleneintrag löschen                                                                                                                             \\ \hline
		fds                 & Quelleneintrag endgültig löschen                                                                                                                   \\ \hline
		rso                 & Quelleneintrag wiederherstellen                                                                                                                    \\ \hline
		vpi                 & Interessenpunkt validieren                                                                                                                         \\ \hline
		ddl                 & Fragt ab, ob Daten Direkt gelöscht werden                                                                                                          \\ \hline
		emp					& Hauptbild eines Interessenpunktes ändern					                                                                                         \\ \hline
		cma					& Prüft, ob Mailadresse bereits verwendet wird				                                                                                         \\ \hline
\end{longtable}
\newpage
\subsection{Befehle}
\subsubsection{Abfrage der Daten für den Persönlichen Bereich}
\paragraph{Kurzbeschreibung}Dieser API-Request wird dazu genutzt um alle Daten zur Darstellung des persönlichen Bereiches zu laden.
\paragraph{Anfrage}Folgende Daten werden zu Anfrage benötigt:
\begin{table}[H]
	\begin{tabular}{|c|c|c|p{6.5cm}|}
		\hline
		\textbf{Paramtername} & \textbf{Datentyp} & \textbf{Konstante} & \textbf{Kurzbeschreibung}                                                                                               \\ \hline
		type                & string            & pac                & Anfragen Aller Daten für persönlichen Bereich \\ \hline
	\end{tabular}
\end{table}
\paragraph{Antwort}Die Antwort ist wie folgt aufgebaut:
\begin{table}[H]
	\begin{tabular}{|c|c|c|p{6.5cm}|}
		\hline
		\textbf{Paramtername} & \textbf{Datentyp} & \textbf{Konstante} & \textbf{Kurzbeschreibung}                                                                                               \\ \hline
		pois                & array            &                 & Alle Daten zu Interessenpunkten des Nutzers \\ \hline
		User                & array            &                 & Alle nutzerbezogenen Daten \\ \hline
		comments            & array            &                 & Alle Kommentare des Nutzers bei Interessenpunkten \\ \hline
		result              & string           &                 & Erfolgreich wenn Wert {\glqq ack\grqq} ist \\ \hline
		Code                & int              &                 & Erfolgreich wenn Wert {\glqq 0\grqq} ist \\ \hline
	\end{tabular}
\end{table}
\subparagraph{pois}Dieses Array enthält Elemente mit Einträgen, welche die nachstehend dargestellten Form haben:
\begin{table}[H]
	\begin{tabular}{|c|c|c|p{6.5cm}|}
		\hline
		\textbf{Paramtername} & \textbf{Datentyp} & \textbf{Konstante} & \textbf{Kurzbeschreibung}                                                                                               \\ \hline
		lat                & double            &                 & Geographischer Breitengrad \\ \hline
		lng                & double            &                 & Geographischer Längengrad \\ \hline
		name               & string            &                 & Name des Interessenpunktes \\ \hline
		poi\_id            & int               &                 & Identifikator des Interessenpunktes \\ \hline
		edit\_enable       & bool              &                 & Gibt an, ob Bearbeitungsfunktionen verfügbar sind\\ \hline
	\end{tabular}
\end{table}
\subparagraph{User}Dieses Array enthält Einträge in der nachstehend dargestellten Form haben:
\begin{table}[H]
	\begin{tabular}{|c|c|c|p{6.5cm}|}
		\hline
		\textbf{Paramtername} & \textbf{Datentyp} & \textbf{Konstante} & \textbf{Kurzbeschreibung}    \\ \hline
		username           & string            &                 & Nutzername \\ \hline
		firstname          & string            &                 & Angegebener Vorname \\ \hline
		lastname           & string            &                 & Angegebener Nachname \\ \hline
		email              & string            &                 & Angegebene E-Mailadresse \\ \hline
	\end{tabular}
\end{table}
\subparagraph{comments}Dieses Array enthält Elemente mit Einträgen, welche die nachstehend dargestellten Form haben:
\begin{table}[H]
	\begin{tabular}{|c|c|c|p{6.5cm}|}
		\hline
		\textbf{Paramtername} & \textbf{Datentyp} & \textbf{Konstante} & \textbf{Kurzbeschreibung}                                                                                               \\ \hline
		date               & timestamp         &                 & Datum und Uhrzeit der Erstellung \\ \hline
		content            & string            &                 & Inhalt des Kommentares \\ \hline
		poiname            & string            &                 & Name des Interessenpunktes des Kommentares \\ \hline
		poiid              & int               &                 & Identifikator des Interessenpunktes des Kommentares \\ \hline
		lat                & double            &                 & Geographischer Breitengrad des Interessenpunktes \\ \hline
		lng                & double            &                 & Geographischer Längengrad  des Interessenpunktes \\ \hline
	\end{tabular}
\end{table}
\subsubsection{Löschen eines Kommentares}
\paragraph{Kurzbeschreibung}Dieser API-Request wird dazu genutzt um einen bestimmten Kommentar zu löschen.
\paragraph{Anfrage}Folgende Daten werden zu Anfrage benötigt:
\begin{table}[H]
	\begin{tabular}{|c|c|c|p{6.5cm}|}
		\hline
		\textbf{Paramtername} & \textbf{Datentyp} & \textbf{Konstante} & \textbf{Kurzbeschreibung}                                                                                               \\ \hline
		type                & string            & duc                & Löschen eines bestimmten Kommentares \\ \hline
		commentid           & int               &                    & Identifikator des Kommentares \\ \hline
	\end{tabular}
\end{table}
\paragraph{Antwort}Die Antwort ist wie folgt aufgebaut:
\begin{table}[H]
	\begin{tabular}{|c|c|c|p{6.5cm}|}
		\hline
		\textbf{Paramtername} & \textbf{Datentyp} & \textbf{Konstante} & \textbf{Kurzbeschreibung}                                                                                               \\ \hline
		result              & string           &                 & Erfolgreich wenn Wert {\glqq ack\grqq} ist \\ \hline
		Code                & int              &                 & Erfolgreich wenn Wert {\glqq 0\grqq} ist \\ \hline
	\end{tabular}
\end{table}
\subsubsection{Nutzerkommentar zu Interessenpunkt hinzufügen}
\paragraph{Kurzbeschreibung}Dieser API-Request wird dazu genutzt um einen Kommentar zu einem bestimmten Interessenpunkt hinzuzufügen.
\paragraph{Anfrage}Folgende Daten werden zu Anfrage benötigt:
\begin{table}[H]
	\begin{tabular}{|c|c|c|p{6.5cm}|}
		\hline
		\textbf{Paramtername} & \textbf{Datentyp} & \textbf{Konstante} & \textbf{Kurzbeschreibung}                                                                                               \\ \hline
		type                & string            & auc                & Neuanlegen eines Kommentares \\ \hline
		comment             & string            &                    & Inhalt des Kommentares \\ \hline
		poi\_id             & int               &                    & Identifikator des Interessenpunktes \\ \hline
	\end{tabular}
\end{table}
\paragraph{Antwort}Die Antwort ist wie folgt aufgebaut:
\begin{table}[H]
	\begin{tabular}{|c|c|c|p{6.5cm}|}
		\hline
		\textbf{Paramtername} & \textbf{Datentyp} & \textbf{Konstante} & \textbf{Kurzbeschreibung}                                                                                               \\ \hline
		result              & string           &                 & Erfolgreich wenn Wert {\glqq ack\grqq} ist \\ \hline
		Code                & int              &                 & Erfolgreich wenn Wert {\glqq 0\grqq} ist \\ \hline
		comment             & string           &                 & Inhalt des Kommentares \\ \hline
		poi\_id             & int              &                 & Identifikator des Interessenpunktes \\ \hline
	\end{tabular}
\end{table}
\subsubsection{Datenabfrage für bestimmten Interessenpunkt}
\paragraph{Kurzbeschreibung}Dieser API-Request wird dazu genutzt um alle Daten zu einem bestimmten Interessenpunkt im Backend abzufragen.
\paragraph{Anfrage}Folgende Daten werden zu Anfrage benötigt:
\begin{table}[H]
	\begin{tabular}{|c|c|c|p{6.5cm}|}
		\hline
		\textbf{Paramtername} & \textbf{Datentyp} & \textbf{Konstante} & \textbf{Kurzbeschreibung}                                                                                               \\ \hline
		type                & string            & smd                & Daten eines Interessenpunktes laden \\ \hline
		poi\_id             & int               &                    & Identifikator des Interessenpunktes \\ \hline
	\end{tabular}
\end{table}
\paragraph{Antwort}Die Antwort ist wie folgt aufgebaut:
\begin{table}[H]
	\begin{tabular}{|c|c|c|p{6.5cm}|}
		\hline
		\textbf{Paramtername} & \textbf{Datentyp} & \textbf{Konstante} & \textbf{Kurzbeschreibung}                                                                                               \\ \hline
		result              & string           &                 & Erfolgreich wenn Wert {\glqq ack\grqq} ist \\ \hline
		Code                & int              &                 & Erfolgreich wenn Wert {\glqq 0\grqq} ist \\ \hline
		data                & array            &                 & Daten des Interessenpunktes \\ \hline
	\end{tabular}
\end{table}
\subparagraph{data}Dieses Array enthält Einträge in der nachstehend dargestellten Form haben:
\begin{table}[H]
	\begin{tabular}{|c|c|c|p{6.5cm}|}
		\hline
		\textbf{Paramtername} & \textbf{Datentyp} & \textbf{Konstante} & \textbf{Kurzbeschreibung}    \\ \hline
		City               & string            &                 & Stadt der aktuellen Adresse \\ \hline
		Housenumber        & string            &                 & Hausnummer der aktuellen Adresse \\ \hline
		Postalcode         & int               &                 & Postleitzahl der aktuellen Adresse \\ \hline
		Streetname         & string            &                 & Straßenname der aktuellen Adresse \\ \hline
		category           & int               &                 & Kategorie des Interessenpunktes \\ \hline
		currAddr\_validate & bool              &                 & Validierungsstatus der aktuellen Adresse \\ \hline
		end                & int               &                 & Ende der Betriebszeit \\ \hline
		history            & string            &                 & Geschichte des Interessenpunktes \\ \hline
		history\_validate  & bool              &                 & Validierungsstatus der Geschichte \\ \hline
		poi\_id            & int               &                 & Identifikator des Interessenpunktes \\ \hline
		poi\_name          & string            &                 & Titel des Interessenpunktes \\ \hline
		start              & int               &                 & Anfang der Betriebszeit \\ \hline
		timespan\_validate & bool              &                 & Validierungsstatus des Betriebszeitraumes \\ \hline
		user\_name         & string            &                 & Nutzername des Erstellers \\ \hline
		deleted            & bool              &                 & Gibt an, ob Interessenpunkt als gelöscht gilt \\ \hline
		duty               & bool              &                 & Gibt an, ob der Interessenpunkt aktuell noch in Betrieb ist \\ \hline
		validated          & bool              &                 & Gibt an, ob der Interessenpunkt bereits Validiert ist\\ \hline
		editable           & bool              &                 & Gibt an, ob der Interessenpunkt änderbar ist \\ \hline
		deletable          & bool              &                 & Gibt an, ob der Interessenpunkt löschbar ist \\ \hline
		validatable        & bool              &                 & Gibt an, ob der Interessenpunkt validierbar ist \\ \hline
		finalDelete        & bool              &                 & Gibt an, ob der Interessenpunkt final löschbar ist \\ \hline
		editLink           & string            &                 & Link zum ändern der Daten des Interessenpunktes \\ \hline
	\end{tabular}
\end{table}
\subsubsection{Datenabfrage für alle Interessenpunkte für Nutzer zur Kartendarstellung}
\paragraph{Kurzbeschreibung}Dieser API-Request wird dazu genutzt um Kartendarstellungsdaten spezifisch für diesen Nutzer zu laden.
\paragraph{Anfrage}Folgende Daten werden zu Anfrage benötigt:
\begin{table}[H]
	\begin{tabular}{|c|c|c|p{6.5cm}|}
		\hline
		\textbf{Paramtername} & \textbf{Datentyp} & \textbf{Konstante} & \textbf{Kurzbeschreibung}                                                                                               \\ \hline
		type                & string            & gpu                & Laden der Kartendarstellungsdaten \\ \hline
	\end{tabular}
\end{table}
\paragraph{Antwort}Die Antwort ist wie folgt aufgebaut:
\begin{table}[H]
	\begin{tabular}{|c|c|c|p{6.5cm}|}
		\hline
		\textbf{Paramtername} & \textbf{Datentyp} & \textbf{Konstante} & \textbf{Kurzbeschreibung}                                                                                               \\ \hline
		result              & string           &                 & Erfolgreich wenn Wert {\glqq ack\grqq} ist \\ \hline
		Code                & int              &                 & Erfolgreich wenn Wert {\glqq 0\grqq} ist \\ \hline
		data                & array            &                 & Daten des Interessenpunktes \\ \hline
	\end{tabular}
\end{table}
\subparagraph{data}Dieses Array enthält Elemente mit Einträgen, welche die nachstehend dargestellten Form haben:
\begin{table}[H]
	\begin{tabular}{|c|c|c|p{6.5cm}|}
		\hline
		\textbf{Paramtername} & \textbf{Datentyp} & \textbf{Konstante} & \textbf{Kurzbeschreibung}    \\ \hline
		City               & string            &                 & Stadt der aktuellen Adresse \\ \hline
		Housenumber        & string            &                 & Hausnummer der aktuellen Adresse \\ \hline
		Postalcode         & int               &                 & Postleitzahl der aktuellen Adresse \\ \hline
		Streetname         & string            &                 & Straßenname der aktuellen Adresse \\ \hline
		category           & int               &                 & Kategorie des Interessenpunktes \\ \hline
		end                & int               &                 & Ende der Betriebszeit \\ \hline
		history            & string            &                 & Geschichte des Interessenpunktes \\ \hline
		poi\_id            & int               &                 & Identifikator des Interessenpunktes \\ \hline
		name               & string            &                 & Titel des Interessenpunktes \\ \hline
		start              & int               &                 & Anfang der Betriebszeit \\ \hline
		username           & string            &                 & Nutzername des Erstellers \\ \hline
		validated          & bool              &                 & Validierungsstatus des Interessenpunktes \\ \hline
		validatedByUser    & bool              &                 & Wahr wenn Nutzer diesen Interessenpunkt validiert hat \\ \hline
		lat                & double            &                 & Geographischer Breitengrad des Interessenpunktes \\ \hline
		lng                & double            &                 & Geographischer Längengrad  des Interessenpunktes \\ \hline
		duty               & bool              &                 & Legt fest ob Interessenpunkt aktuell in Betrieb \\ \hline
	\end{tabular}
\end{table}
\subsubsection{Abfrage des kleinsten und größten Jahres der Betriebszeiten aller Kinos}
\paragraph{Kurzbeschreibung}Dieser API-Request wird dazu genutzt um das minimale und maximale Jahr des Sliders abzufragen.
\paragraph{Anfrage}Folgende Daten werden zu Anfrage benötigt:
\begin{table}[H]
	\begin{tabular}{|c|c|c|p{6.5cm}|}
		\hline
		\textbf{Paramtername} & \textbf{Datentyp} & \textbf{Konstante} & \textbf{Kurzbeschreibung}                                                                                               \\ \hline
		type                & string            & mmy                & Abfrage des minimalen und maximalen Jahres \\ \hline
	\end{tabular}
\end{table}
\paragraph{Antwort}Die Antwort ist wie folgt aufgebaut:
\begin{table}[H]
	\begin{tabular}{|c|c|c|p{6.5cm}|}
		\hline
		\textbf{Paramtername} & \textbf{Datentyp} & \textbf{Konstante} & \textbf{Kurzbeschreibung}                                                                                               \\ \hline
		result              & string           &                 & Erfolgreich wenn Wert {\glqq ack\grqq} ist \\ \hline
		Code                & int              &                 & Erfolgreich wenn Wert {\glqq 0\grqq} ist \\ \hline
		MinYear             & int              &                 & Minimales Jahr \\ \hline
		MaxYear             & int              &                 & Maximales Jahr \\ \hline
	\end{tabular}
\end{table}
\subsubsection{Abfrage der {\glqq COSP\grqq}-URI}
\paragraph{Kurzbeschreibung}Dieser API-Request wird dazu genutzt um die {\glqq COSP\grqq}-URI abzufragen.
\paragraph{Anfrage}Folgende Daten werden zu Anfrage benötigt:
\begin{table}[H]
	\begin{tabular}{|c|c|c|p{6.5cm}|}
		\hline
		\textbf{Paramtername} & \textbf{Datentyp} & \textbf{Konstante} & \textbf{Kurzbeschreibung}                                                                                               \\ \hline
		type                & string            & ccp                & Abfrage der {\glqq COSP\grqq}-URI \\ \hline
	\end{tabular}
\end{table}
\paragraph{Antwort}Die Antwort ist wie folgt aufgebaut:
\begin{table}[H]
	\begin{tabular}{|c|c|c|p{6.5cm}|}
		\hline
		\textbf{Paramtername} & \textbf{Datentyp} & \textbf{Konstante} & \textbf{Kurzbeschreibung}                                                                                               \\ \hline
		result              & string           &                 & Erfolgreich wenn Wert {\glqq ack\grqq} ist \\ \hline
		Code                & int              &                 & Erfolgreich wenn Wert {\glqq 0\grqq} ist \\ \hline
		data                & array            &                 & Daten des Interessenpunktes \\ \hline
	\end{tabular}
\end{table}
\subparagraph{data}Dieses Array enthält Einträge in der nachstehend dargestellten Form haben:
\begin{table}[H]
	\begin{tabular}{|c|c|c|p{6.5cm}|}
		\hline
		\textbf{Paramtername} & \textbf{Datentyp} & \textbf{Konstante} & \textbf{Kurzbeschreibung}    \\ \hline
		uapi               & string            &                 & {\glqq COSP\grqq}-URI \\ \hline
	\end{tabular}
\end{table}
\subsubsection{Anlegen einer neuen Geschichte}\label{api:NewStoryAdd}
\paragraph{Kurzbeschreibung}Dieser API-Request wird dazu genutzt um eine neue Geschichte im Biographien-Modul anzulegen.
\paragraph{Anfrage}Folgende Daten werden zu Anfrage benötigt:
\begin{table}[H]
	\begin{tabular}{|c|c|c|p{6.5cm}|}
		\hline
		\textbf{Paramtername} & \textbf{Datentyp} & \textbf{Konstante} & \textbf{Kurzbeschreibung}                                                                                               \\ \hline
		type                & string            & aus                & Hinzufügen einer neuen Geschichte \\ \hline
		story               & string            &                    & Inhalt der Geschichte \\ \hline
		title               & string            &                    & Titel der Geschichte \\ \hline
		rights              & bool              &                    & Wahrheitswert des Rechte-Checkbox \\ \hline
	\end{tabular}
\end{table}
\paragraph{Antwort}Die Antwort ist wie folgt aufgebaut:
\begin{table}[H]
	\begin{tabular}{|c|c|c|p{6.5cm}|}
		\hline
		\textbf{Paramtername} & \textbf{Datentyp} & \textbf{Konstante} & \textbf{Kurzbeschreibung}                                                                                               \\ \hline
		result              & string           &                 & Erfolgreich wenn Wert {\glqq ack\grqq} ist \\ \hline
		Code                & int              &                 & Erfolgreich wenn Wert {\glqq 0\grqq} ist \\ \hline
		story               & string           &                 & Inhalt der Geschichte \\ \hline
		title               & string           &                 & Titel der Geschichte \\ \hline
		rights              & bool             &                 & Wahrheitswert des Rechte-Checkbox \\ \hline
		token               & string           &                 & Identifikator der Geschichte \\ \hline
		username            & string           &                 & Nutzername des Erstellers \\ \hline
	\end{tabular}
\end{table}
\subsubsection{Abfrage von Geschichten}
\paragraph{Kurzbeschreibung}Dieser API-Request wird dazu genutzt um alle für den Nutzer verfügbaren Geschichten abzufragen.
\paragraph{Anfrage}Folgende Daten werden zu Anfrage benötigt:
\begin{table}[H]
	\begin{tabular}{|c|c|c|p{6.5cm}|}
		\hline
		\textbf{Paramtername} & \textbf{Datentyp} & \textbf{Konstante} & \textbf{Kurzbeschreibung}                                                                                               \\ \hline
		type                & string            & gas                & Abfragen aller Geschichten für Nutzer\\ \hline
	\end{tabular}
\end{table}
\paragraph{Antwort}Die Antwort ist wie folgt aufgebaut:
\begin{table}[H]
	\begin{tabular}{|c|c|c|p{6.5cm}|}
		\hline
		\textbf{Paramtername} & \textbf{Datentyp} & \textbf{Konstante} & \textbf{Kurzbeschreibung}                                                                                               \\ \hline
		result              & string           &                 & Erfolgreich wenn Wert {\glqq ack\grqq} ist \\ \hline
		Code                & int              &                 & Erfolgreich wenn Wert {\glqq 0\grqq} ist \\ \hline
		data                & array            &                 & Daten zur Darstellung\\ \hline
	\end{tabular}
\end{table}
\subparagraph{data}Dieses Array enthält Einträge in der nachstehend dargestellten Form haben:
\begin{table}[H]
	\begin{tabular}{|c|c|c|p{6.5cm}|}
		\hline
		\textbf{Paramtername} & \textbf{Datentyp} & \textbf{Konstante} & \textbf{Kurzbeschreibung}    \\ \hline
		approver               & bool            &                 & Wahr, wenn Nutzer Geschichten freischalten darf \\ \hline
		guest                  & bool            &                 & Wahr, wenn Nutzer Gast ist \\ \hline
		result                 & array           &                 & Daten zum abrufen der Geschichten \\ \hline
	\end{tabular}
\end{table}
\subparagraph{result}Dieses Array enthält Einträge in der nachstehend dargestellten Form haben:
\begin{table}[H]
	\begin{tabular}{|c|c|c|p{6.5cm}|}
		\hline
		\textbf{Paramtername} & \textbf{Datentyp} & \textbf{Konstante} & \textbf{Kurzbeschreibung}    \\ \hline
		data                   & string          &                 & Gesicherte und signierte Daten \\ \hline
		seccode                & string          &                 & Sicherheitscode \\ \hline
		time                   & int             &                 & Zeitstempel des Requests \\ \hline
		type                   & string          & gas             & Abfragen aller Geschichten für Nutzer\\ \hline
		url                    & string          &                 & URI der {\glqq COSP\grqq}-Nutzer-API \\ \hline
	\end{tabular}
\end{table}
\subsubsection{Interessenpunkt löschen}
\paragraph{Kurzbeschreibung}Dieser API-Request wird dazu genutzt um einen Interessenpunkt zu löschen.
\paragraph{Anfrage}Folgende Daten werden zu Anfrage benötigt:
\begin{table}[H]
	\begin{tabular}{|c|c|c|p{6.5cm}|}
		\hline
		\textbf{Paramtername} & \textbf{Datentyp} & \textbf{Konstante} & \textbf{Kurzbeschreibung}                                                                                               \\ \hline
		type                & string            & dpi                & Interessenpunkt löschen \\ \hline
		poiid               & int               &                    & Wahrheitswert des Rechte-Checkbox \\ \hline
	\end{tabular}
\end{table}
\paragraph{Antwort}Die Antwort ist wie folgt aufgebaut:
\begin{table}[H]
	\begin{tabular}{|c|c|c|p{6.5cm}|}
		\hline
		\textbf{Paramtername} & \textbf{Datentyp} & \textbf{Konstante} & \textbf{Kurzbeschreibung}                                                                                               \\ \hline
		result              & string           &                 & Erfolgreich wenn Wert {\glqq ack\grqq} ist \\ \hline
		Code                & int              &                 & Erfolgreich wenn Wert {\glqq 0\grqq} ist \\ \hline
	\end{tabular}
\end{table}
\subsubsection{Ändern eines Kommentares/Speichern eines bereits vorhandenen Kommentares}
\paragraph{Kurzbeschreibung}Dieser API-Request wird dazu genutzt um einen bereits vorhandenen Kommentar in eventuell geänderter Form ab zu speichern.
\paragraph{Anfrage}Folgende Daten werden zu Anfrage benötigt:
\begin{table}[H]
	\begin{tabular}{|c|c|c|p{6.5cm}|}
		\hline
		\textbf{Paramtername} & \textbf{Datentyp} & \textbf{Konstante} & \textbf{Kurzbeschreibung}                                                                                               \\ \hline
		type                & string            & sec                & Kommentar speichern \\ \hline
		commentid           & int               &                    & Identifikator des Kommentares \\ \hline
		commentContent      & string            &                    & Inhalt des Kommentares \\ \hline
	\end{tabular}
\end{table}
\paragraph{Antwort}Die Antwort ist wie folgt aufgebaut:
\begin{table}[H]
	\begin{tabular}{|c|c|c|p{6.5cm}|}
		\hline
		\textbf{Paramtername} & \textbf{Datentyp} & \textbf{Konstante} & \textbf{Kurzbeschreibung}                                                                                               \\ \hline
		result              & string           &                 & Erfolgreich wenn Wert {\glqq ack\grqq} ist \\ \hline
		Code                & int              &                 & Erfolgreich wenn Wert {\glqq 0\grqq} ist \\ \hline
	\end{tabular}
\end{table}
\subsubsection{Abfrage der Abrufdaten eines Bildes}
\paragraph{Kurzbeschreibung}Dieser API-Request wird dazu genutzt um Daten zum Abrufen eines bestimmten Bildes zu erhalten.
\paragraph{Anfrage}Folgende Daten werden zu Anfrage benötigt:
\begin{table}[H]
	\begin{tabular}{|c|c|c|p{6.5cm}|}
		\hline
		\textbf{Paramtername} & \textbf{Datentyp} & \textbf{Konstante} & \textbf{Kurzbeschreibung}                                                                                               \\ \hline
		type                & string            & dsm                & Bild-URI abfragen \\ \hline
		token               & string            &                    & Identifikator des Bildes \\ \hline
	\end{tabular}
\end{table}
\paragraph{Antwort}Die Antwort ist wie folgt aufgebaut:
\begin{table}[H]
	\begin{tabular}{|c|c|c|p{6.5cm}|}
		\hline
		\textbf{Paramtername} & \textbf{Datentyp} & \textbf{Konstante} & \textbf{Kurzbeschreibung}                                                                                               \\ \hline
		result              & string           &                 & Erfolgreich wenn Wert {\glqq ack\grqq} ist \\ \hline
		Code                & int              &                 & Erfolgreich wenn Wert {\glqq 0\grqq} ist \\ \hline
		token               & string           &                 & Identifikator des Bildes \\ \hline
		title               & string           &                 & Titel des Bildes \\ \hline
		description         & string           &                 & Beschreibung des Bildes \\ \hline
		picture             & string           &                 & Bild-URI \\ \hline
		sourcetype          & string           &                 & Name des Typs der Quelle \\ \hline
		source              & string           &                 & Quellenangabe \\ \hline
		sourcetypeid        & int              &                 & Identifikator des Typs der Quelle \\ \hline
	\end{tabular}
\end{table}
\subsubsection{Ändern der Metadaten eines Bildes}
\paragraph{Kurzbeschreibung}Dieser API-Request wird dazu genutzt um Daten eines Bildes zu ändern.
\paragraph{Anfrage}Folgende Daten werden zu Anfrage benötigt:
\begin{table}[H]
	\begin{tabular}{|c|c|c|p{6.5cm}|}
		\hline
		\textbf{Paramtername} & \textbf{Datentyp} & \textbf{Konstante} & \textbf{Kurzbeschreibung}                                                                                               \\ \hline
		type                & string            & ssm                & Ändern von Bildmetadaten \\ \hline
		token               & string            &                    & Identifikator des Bildes \\ \hline
		title               & string            &                    & Titel des Bildes \\ \hline
		description         & string            &                    & Beschreibung des Bildes \\ \hline
		source              & string            &                    & Quellenangabe (optional) \\ \hline
		sourcetype          & int               &                    & Identifikator des Typs der Quelle (optional) \\ \hline
	\end{tabular}
\end{table}
\paragraph{Antwort}Die Antwort ist wie folgt aufgebaut:
\begin{table}[H]
	\begin{tabular}{|c|c|c|p{6.5cm}|}
		\hline
		\textbf{Paramtername} & \textbf{Datentyp} & \textbf{Konstante} & \textbf{Kurzbeschreibung}                                                                                               \\ \hline
		result              & string           &                 & Erfolgreich wenn Wert {\glqq ack\grqq} ist \\ \hline
		Code                & int              &                 & Erfolgreich wenn Wert {\glqq 0\grqq} ist \\ \hline
	\end{tabular}
\end{table}
\subsubsection{Historische Adresse hinzufügen}
\paragraph{Kurzbeschreibung}Dieser API-Request wird dazu genutzt um eine historische Adresse zu einem Interessenpunkt hinzuzufügen.
\paragraph{Anfrage}Folgende Daten werden zu Anfrage benötigt:
\begin{table}[H]
	\begin{tabular}{|c|c|c|p{6.5cm}|}
		\hline
		\textbf{Paramtername} & \textbf{Datentyp} & \textbf{Konstante} & \textbf{Kurzbeschreibung}                                                                                               \\ \hline
		type                & string            & aha                & Historische Adresse hinzufügen \\ \hline
		poi\_id             & int               &                    & Identifikator des Interessenpunktes \\ \hline
		from                & int               &                    & Adresse benutzt ab \\ \hline
		till                & int               &                    & Adresse benutzt bis \\ \hline
		streetname          & string            &                    & Straßenname \\ \hline
		housenumber         & string            &                    & Hausnummer \\ \hline
		city                & string            &                    & Stadt \\ \hline
		postalcode          & int               &                    & Postleitzahl \\ \hline
	\end{tabular}
\end{table}
\paragraph{Antwort}Die Antwort ist wie folgt aufgebaut:
\begin{table}[H]
	\begin{tabular}{|c|c|c|p{6.5cm}|}
		\hline
		\textbf{Paramtername} & \textbf{Datentyp} & \textbf{Konstante} & \textbf{Kurzbeschreibung}                                                                                               \\ \hline
		result              & string           &                 & Erfolgreich wenn Wert {\glqq ack\grqq} ist \\ \hline
		Code                & int              &                 & Erfolgreich wenn Wert {\glqq 0\grqq} ist \\ \hline
		type                & string           & aha             & Historische Adresse hinzufügen \\ \hline
		poi\_id             & int              &                 & Identifikator des Interessenpunktes \\ \hline
		from                & int              &                 & Adresse benutzt ab \\ \hline
		till                & int              &                 & Adresse benutzt bis \\ \hline
		streetname          & string           &                 & Straßenname \\ \hline
		housenumber         & string           &                 & Hausnummer \\ \hline
		city                & string           &                 & Stadt \\ \hline
		postalcode          & int              &                 & Postleitzahl \\ \hline
	\end{tabular}
\end{table}
\subsubsection{Betreiber hinzufügen}
\paragraph{Kurzbeschreibung}Dieser API-Request wird dazu genutzt um einen Betreiber zu einem Interessenpunkt hinzuzufügen.
\paragraph{Anfrage}Folgende Daten werden zu Anfrage benötigt:
\begin{table}[H]
	\begin{tabular}{|c|c|c|p{6.5cm}|}
		\hline
		\textbf{Paramtername} & \textbf{Datentyp} & \textbf{Konstante} & \textbf{Kurzbeschreibung}                                                                                               \\ \hline
		type                & string            & ado                & Historische Adresse hinzufügen \\ \hline
		poi\_id             & int               &                    & Identifikator des Interessenpunktes \\ \hline
		from                & int               &                    & Betreiber ab \\ \hline
		till                & int               &                    & Betreiber bis \\ \hline
		operator            & string            &                    & Betreibername \\ \hline
	\end{tabular}
\end{table}
\paragraph{Antwort}Die Antwort ist wie folgt aufgebaut:
\begin{table}[H]
	\begin{tabular}{|c|c|c|p{6.5cm}|}
		\hline
		\textbf{Paramtername} & \textbf{Datentyp} & \textbf{Konstante} & \textbf{Kurzbeschreibung}                                                                                               \\ \hline
		result              & string           &                 & Erfolgreich wenn Wert {\glqq ack\grqq} ist \\ \hline
		Code                & int              &                 & Erfolgreich wenn Wert {\glqq 0\grqq} ist \\ \hline
		type                & string           & ado             & Historische Adresse hinzufügen \\ \hline
		poi\_id             & int              &                 & Identifikator des Interessenpunktes \\ \hline
		from                & int              &                 & Betreiber ab \\ \hline
		till                & int              &                 & Betreiber bis \\ \hline
		operator            & string           &                 & Betreibername \\ \hline
	\end{tabular}
\end{table}
\subsubsection{Name hinzufügen}
\paragraph{Kurzbeschreibung}Dieser API-Request wird dazu genutzt um einen Namen zu einem Interessenpunkt hinzuzufügen.
\paragraph{Anfrage}Folgende Daten werden zu Anfrage benötigt:
\begin{table}[H]
	\begin{tabular}{|c|c|c|p{6.5cm}|}
		\hline
		\textbf{Paramtername} & \textbf{Datentyp} & \textbf{Konstante} & \textbf{Kurzbeschreibung}                                                                                               \\ \hline
		type                & string            & adn                & Namen hinzufügen \\ \hline
		poi\_id             & int               &                    & Identifikator des Interessenpunktes \\ \hline
		from                & int               &                    & Name benutzt ab \\ \hline
		till                & int               &                    & Name benutzt bis \\ \hline
		name                & string            &                    & Name des Interessenpunkts \\ \hline
	\end{tabular}
\end{table}
\paragraph{Antwort}Die Antwort ist wie folgt aufgebaut:
\begin{table}[H]
	\begin{tabular}{|c|c|c|p{6.5cm}|}
		\hline
		\textbf{Paramtername} & \textbf{Datentyp} & \textbf{Konstante} & \textbf{Kurzbeschreibung}                                                                                               \\ \hline
		result              & string           &                 & Erfolgreich wenn Wert {\glqq ack\grqq} ist \\ \hline
		Code                & int              &                 & Erfolgreich wenn Wert {\glqq 0\grqq} ist \\ \hline
		type                & string           & adn             & Namen hinzufügen \\ \hline
		poi\_id             & int              &                 & Identifikator des Interessenpunktes \\ \hline
		from                & int              &                 & Name benutzt ab \\ \hline
		till                & int              &                 & Name benutzt bis \\ \hline
 		name                & string           &                 & Name des Interessenpunkts \\ \hline
	\end{tabular}
\end{table}
\subsubsection{Historische Adresse löschen}
\paragraph{Kurzbeschreibung}Dieser API-Request wird dazu genutzt um eine historische Adresse eines Interessenpunkts zu löschen.
\paragraph{Anfrage}Folgende Daten werden zu Anfrage benötigt:
\begin{table}[H]
	\begin{tabular}{|c|c|c|p{6.5cm}|}
		\hline
		\textbf{Paramtername} & \textbf{Datentyp} & \textbf{Konstante} & \textbf{Kurzbeschreibung}                                                                                               \\ \hline
		type                & string            & dha                & historische Adresse löschen \\ \hline
		IDent               & int               &                    & Identifikator der historischen Adresse \\ \hline
	\end{tabular}
\end{table}
\paragraph{Antwort}Die Antwort ist wie folgt aufgebaut:
\begin{table}[H]
	\begin{tabular}{|c|c|c|p{6.5cm}|}
		\hline
		\textbf{Paramtername} & \textbf{Datentyp} & \textbf{Konstante} & \textbf{Kurzbeschreibung}                                                                                               \\ \hline
		result              & string           &                 & Erfolgreich wenn Wert {\glqq ack\grqq} ist \\ \hline
		Code                & int              &                 & Erfolgreich wenn Wert {\glqq 0\grqq} ist \\ \hline
		type                & string           & dha             & historische Adresse löschen \\ \hline
		IDent               & int              &                 & Identifikator der historischen Adresse \\ \hline
	\end{tabular}
\end{table}
\subsubsection{Betreiber löschen}
\paragraph{Kurzbeschreibung}Dieser API-Request wird dazu genutzt um einen Betreiber eines Interessenpunkts zu löschen.
\paragraph{Anfrage}Folgende Daten werden zu Anfrage benötigt:
\begin{table}[H]
	\begin{tabular}{|c|c|c|p{6.5cm}|}
		\hline
		\textbf{Paramtername} & \textbf{Datentyp} & \textbf{Konstante} & \textbf{Kurzbeschreibung}                                                                                               \\ \hline
		type                & string            & dop                & Betreiber löschen \\ \hline
		IDent               & int               &                    & Identifikator des Betreibers \\ \hline
	\end{tabular}
\end{table}
\paragraph{Antwort}Die Antwort ist wie folgt aufgebaut:
\begin{table}[H]
	\begin{tabular}{|c|c|c|p{6.5cm}|}
		\hline
		\textbf{Paramtername} & \textbf{Datentyp} & \textbf{Konstante} & \textbf{Kurzbeschreibung}                                                                                               \\ \hline
		result              & string           &                 & Erfolgreich wenn Wert {\glqq ack\grqq} ist \\ \hline
		Code                & int              &                 & Erfolgreich wenn Wert {\glqq 0\grqq} ist \\ \hline
		type                & string           & dop             & Betreiber löschen \\ \hline
		IDent               & int              &                 & Identifikator des Betreibers \\ \hline
	\end{tabular}
\end{table}
\subsubsection{Name löschen}
\paragraph{Kurzbeschreibung}Dieser API-Request wird dazu genutzt um einen Namen eines Interessenpunkts zu löschen.
\paragraph{Anfrage}Folgende Daten werden zu Anfrage benötigt:
\begin{table}[H]
	\begin{tabular}{|c|c|c|p{6.5cm}|}
		\hline
		\textbf{Paramtername} & \textbf{Datentyp} & \textbf{Konstante} & \textbf{Kurzbeschreibung}                                                                                               \\ \hline
		type                & string            & dna                & Name löschen \\ \hline
		IDent               & int               &                    & Identifikator des Namens \\ \hline
	\end{tabular}
\end{table}
\paragraph{Antwort}Die Antwort ist wie folgt aufgebaut:
\begin{table}[H]
	\begin{tabular}{|c|c|c|p{6.5cm}|}
		\hline
		\textbf{Paramtername} & \textbf{Datentyp} & \textbf{Konstante} & \textbf{Kurzbeschreibung}                                                                                               \\ \hline
		result              & string           &                 & Erfolgreich wenn Wert {\glqq ack\grqq} ist \\ \hline
		Code                & int              &                 & Erfolgreich wenn Wert {\glqq 0\grqq} ist \\ \hline
		type                & string           & dna             & Name löschen \\ \hline
		IDent               & int              &                 & Identifikator des Namens \\ \hline
	\end{tabular}
\end{table}
\subsubsection{Historische Adresse validieren}
\paragraph{Kurzbeschreibung}Dieser API-Request wird dazu genutzt um eine historische Adresse eines Interessenpunkts zu validieren.
\paragraph{Anfrage}Folgende Daten werden zu Anfrage benötigt:
\begin{table}[H]
	\begin{tabular}{|c|c|c|p{6.5cm}|}
		\hline
		\textbf{Paramtername} & \textbf{Datentyp} & \textbf{Konstante} & \textbf{Kurzbeschreibung}                                                                                               \\ \hline
		type                & string            & vha                & historische Adresse validieren \\ \hline
		ADDRESSID           & int               &                    & Identifikator der historischen Adresse \\ \hline
	\end{tabular}
\end{table}
\paragraph{Antwort}Die Antwort ist wie folgt aufgebaut:
\begin{table}[H]
	\begin{tabular}{|c|c|c|p{6.5cm}|}
		\hline
		\textbf{Paramtername} & \textbf{Datentyp} & \textbf{Konstante} & \textbf{Kurzbeschreibung}                                                                                               \\ \hline
		result              & string           &                 & Erfolgreich wenn Wert {\glqq ack\grqq} ist \\ \hline
		Code                & int              &                 & Erfolgreich wenn Wert {\glqq 0\grqq} ist \\ \hline
		type                & string           & vha             & historische Adresse validieren \\ \hline
		ADDRESSID           & int              &                 & Identifikator der historischen Adresse \\ \hline
	\end{tabular}
\end{table}
\subsubsection{Betreiber validieren}
\paragraph{Kurzbeschreibung}Dieser API-Request wird dazu genutzt um einen Betreiber eines Interessenpunkts zu validieren.
\paragraph{Anfrage}Folgende Daten werden zu Anfrage benötigt:
\begin{table}[H]
	\begin{tabular}{|c|c|c|p{6.5cm}|}
		\hline
		\textbf{Paramtername} & \textbf{Datentyp} & \textbf{Konstante} & \textbf{Kurzbeschreibung}                                                                                               \\ \hline
		type                & string            & vop                & Betreiber validieren \\ \hline
		OPERATORID          & int               &                    & Identifikator des Betreibers \\ \hline
	\end{tabular}
\end{table}
\paragraph{Antwort}Die Antwort ist wie folgt aufgebaut:
\begin{table}[H]
	\begin{tabular}{|c|c|c|p{6.5cm}|}
		\hline
		\textbf{Paramtername} & \textbf{Datentyp} & \textbf{Konstante} & \textbf{Kurzbeschreibung}                                                                                               \\ \hline
		result              & string           &                 & Erfolgreich wenn Wert {\glqq ack\grqq} ist \\ \hline
		Code                & int              &                 & Erfolgreich wenn Wert {\glqq 0\grqq} ist \\ \hline
		type                & string           & vop             & Betreiber validieren \\ \hline
		OPERATORID          & int              &                 & Identifikator des Betreibers \\ \hline
	\end{tabular}
\end{table}
\subsubsection{Name validieren}
\paragraph{Kurzbeschreibung}Dieser API-Request wird dazu genutzt um einen Namen eines Interessenpunkts zu validieren.
\paragraph{Anfrage}Folgende Daten werden zu Anfrage benötigt:
\begin{table}[H]
	\begin{tabular}{|c|c|c|p{6.5cm}|}
		\hline
		\textbf{Paramtername} & \textbf{Datentyp} & \textbf{Konstante} & \textbf{Kurzbeschreibung}                                                                                               \\ \hline
		type                & string            & vna                & Name validieren \\ \hline
		NAMEID              & int               &                    & Identifikator des Namens \\ \hline
	\end{tabular}
\end{table}
\paragraph{Antwort}Die Antwort ist wie folgt aufgebaut:
\begin{table}[H]
	\begin{tabular}{|c|c|c|p{6.5cm}|}
		\hline
		\textbf{Paramtername} & \textbf{Datentyp} & \textbf{Konstante} & \textbf{Kurzbeschreibung}                                                                                               \\ \hline
		result              & string           &                 & Erfolgreich wenn Wert {\glqq ack\grqq} ist \\ \hline
		Code                & int              &                 & Erfolgreich wenn Wert {\glqq 0\grqq} ist \\ \hline
		type                & string           & vna             & Name validieren \\ \hline
		NAMEID              & int              &                 & Identifikator des Namens \\ \hline
	\end{tabular}
\end{table}
\subsubsection{Geschichte eines Interessenpunktes validieren}
\paragraph{Kurzbeschreibung}Dieser API-Request wird dazu genutzt um die Geschichte eines Interessenpunktes zu validieren.
\paragraph{Anfrage}Folgende Daten werden zu Anfrage benötigt:
\begin{table}[H]
	\begin{tabular}{|c|c|c|p{6.5cm}|}
		\hline
		\textbf{Paramtername} & \textbf{Datentyp} & \textbf{Konstante} & \textbf{Kurzbeschreibung}                                                                                               \\ \hline
		type                & string            & vts                & Geschichte eines Interessenpunktes validieren \\ \hline
		POIID               & int               &                    & Identifikator des Interessenpunktes \\ \hline
	\end{tabular}
\end{table}
\paragraph{Antwort}Die Antwort ist wie folgt aufgebaut:
\begin{table}[H]
	\begin{tabular}{|c|c|c|p{6.5cm}|}
		\hline
		\textbf{Paramtername} & \textbf{Datentyp} & \textbf{Konstante} & \textbf{Kurzbeschreibung}                                                                                               \\ \hline
		result              & string           &                 & Erfolgreich wenn Wert {\glqq ack\grqq} ist \\ \hline
		Code                & int              &                 & Erfolgreich wenn Wert {\glqq 0\grqq} ist \\ \hline
		type                & string           & vts             & Geschichte eines Interessenpunktes validieren \\ \hline
		POIID               & int              &                 & Identifikator des Interessenpunktes \\ \hline
	\end{tabular}
\end{table}
\subsubsection{Aktuelle Adresse eines Interessenpunktes validieren}
\paragraph{Kurzbeschreibung}Dieser API-Request wird dazu genutzt um die aktuelle Adresse eines Interessenpunktes zu validieren.
\paragraph{Anfrage}Folgende Daten werden zu Anfrage benötigt:
\begin{table}[H]
	\begin{tabular}{|c|c|c|p{6.5cm}|}
		\hline
		\textbf{Paramtername} & \textbf{Datentyp} & \textbf{Konstante} & \textbf{Kurzbeschreibung}                                                                                               \\ \hline
		type                & string            & vca                & Aktuelle Adresse eines Interessenpunktes validieren \\ \hline
		POIID               & int               &                    & Identifikator des Interessenpunktes \\ \hline
	\end{tabular}
\end{table}
\paragraph{Antwort}Die Antwort ist wie folgt aufgebaut:
\begin{table}[H]
	\begin{tabular}{|c|c|c|p{6.5cm}|}
		\hline
		\textbf{Paramtername} & \textbf{Datentyp} & \textbf{Konstante} & \textbf{Kurzbeschreibung}                                                                                               \\ \hline
		result              & string           &                 & Erfolgreich wenn Wert {\glqq ack\grqq} ist \\ \hline
		Code                & int              &                 & Erfolgreich wenn Wert {\glqq 0\grqq} ist \\ \hline
		type                & string           & vca             & Aktuelle Adresse eines Interessenpunktes validieren \\ \hline
		POIID               & int              &                 & Identifikator des Interessenpunktes \\ \hline
	\end{tabular}
\end{table}
\subsubsection{Betriebszeitraum eines Interessenpunktes validieren}
\paragraph{Kurzbeschreibung}Dieser API-Request wird dazu genutzt um den Betriebszeitraum eines Interessenpunktes zu validieren.
\paragraph{Anfrage}Folgende Daten werden zu Anfrage benötigt:
\begin{table}[H]
	\begin{tabular}{|c|c|c|p{6.5cm}|}
		\hline
		\textbf{Paramtername} & \textbf{Datentyp} & \textbf{Konstante} & \textbf{Kurzbeschreibung}                                                                                               \\ \hline
		type                & string            & vhi                & Betriebszeitraum eines Interessenpunktes validieren \\ \hline
		POIID               & int               &                    & Identifikator des Interessenpunktes \\ \hline
	\end{tabular}
\end{table}
\paragraph{Antwort}Die Antwort ist wie folgt aufgebaut:
\begin{table}[H]
	\begin{tabular}{|c|c|c|p{6.5cm}|}
		\hline
		\textbf{Paramtername} & \textbf{Datentyp} & \textbf{Konstante} & \textbf{Kurzbeschreibung}                                                                                               \\ \hline
		result              & string           &                 & Erfolgreich wenn Wert {\glqq ack\grqq} ist \\ \hline
		Code                & int              &                 & Erfolgreich wenn Wert {\glqq 0\grqq} ist \\ \hline
		type                & string           & vhi             & Betriebszeitraum eines Interessenpunktes validieren \\ \hline
		POIID               & int              &                 & Identifikator des Interessenpunktes \\ \hline
	\end{tabular}
\end{table}

\subsubsection{Historische Adresse ändern}
\paragraph{Kurzbeschreibung}Dieser API-Request wird dazu genutzt um eine historische Adresse zu einem Interessenpunkt ändern.
\paragraph{Anfrage}Folgende Daten werden zu Anfrage benötigt:
\begin{table}[H]
	\begin{tabular}{|c|c|c|p{6.5cm}|}
		\hline
		\textbf{Paramtername} & \textbf{Datentyp} & \textbf{Konstante} & \textbf{Kurzbeschreibung}                                                                                               \\ \hline
		type                & string            & uha                & Historische Adresse ändern \\ \hline
		id                  & int               &                    & Identifikator der historischen Adresse \\ \hline
		from                & int               &                    & Adresse benutzt ab \\ \hline
		till                & int               &                    & Adresse benutzt bis \\ \hline
		streetname          & string            &                    & Straßenname \\ \hline
		housenumber         & string            &                    & Hausnummer \\ \hline
		city                & string            &                    & Stadt \\ \hline
		postalcode          & int               &                    & Postleitzahl \\ \hline
	\end{tabular}
\end{table}
\paragraph{Antwort}Die Antwort ist wie folgt aufgebaut:
\begin{table}[H]
	\begin{tabular}{|c|c|c|p{6.5cm}|}
		\hline
		\textbf{Paramtername} & \textbf{Datentyp} & \textbf{Konstante} & \textbf{Kurzbeschreibung}                                                                                               \\ \hline
		result              & string           &                 & Erfolgreich wenn Wert {\glqq ack\grqq} ist \\ \hline
		Code                & int              &                 & Erfolgreich wenn Wert {\glqq 0\grqq} ist \\ \hline
		state               & bool             &                 & Wahr, wenn erfolgreich \\ \hline
	\end{tabular}
\end{table}
\subsubsection{Betreiber ändern}
\paragraph{Kurzbeschreibung}Dieser API-Request wird dazu genutzt um einen Betreiber zu einem Interessenpunkt ändern.
\paragraph{Anfrage}Folgende Daten werden zu Anfrage benötigt:
\begin{table}[H]
	\begin{tabular}{|c|c|c|p{6.5cm}|}
		\hline
		\textbf{Paramtername} & \textbf{Datentyp} & \textbf{Konstante} & \textbf{Kurzbeschreibung}                                                                                               \\ \hline
		type                & string            & uop                & Historische Adresse ändern \\ \hline
		id                  & int               &                    & Identifikator des Betreibers \\ \hline
		from                & int               &                    & Betreiber ab \\ \hline
		till                & int               &                    & Betreiber bis \\ \hline
		operator            & string            &                    & Betreibername \\ \hline
	\end{tabular}
\end{table}
\paragraph{Antwort}Die Antwort ist wie folgt aufgebaut:
\begin{table}[H]
	\begin{tabular}{|c|c|c|p{6.5cm}|}
		\hline
		\textbf{Paramtername} & \textbf{Datentyp} & \textbf{Konstante} & \textbf{Kurzbeschreibung}                                                                                               \\ \hline
		result              & string           &                 & Erfolgreich wenn Wert {\glqq ack\grqq} ist \\ \hline
		Code                & int              &                 & Erfolgreich wenn Wert {\glqq 0\grqq} ist \\ \hline
		state               & bool             &                 & Wahr, wenn erfolgreich \\ \hline
	\end{tabular}
\end{table}
\subsubsection{Name ändern}
\paragraph{Kurzbeschreibung}Dieser API-Request wird dazu genutzt um einen Namen zu einem Interessenpunkt ändern.
\paragraph{Anfrage}Folgende Daten werden zu Anfrage benötigt:
\begin{table}[H]
	\begin{tabular}{|c|c|c|p{6.5cm}|}
		\hline
		\textbf{Paramtername} & \textbf{Datentyp} & \textbf{Konstante} & \textbf{Kurzbeschreibung}                                                                                               \\ \hline
		type                & string            & una                & Namen ändern \\ \hline
		id                  & int               &                    & Identifikator des Namens \\ \hline
		from                & int               &                    & Name benutzt ab \\ \hline
		till                & int               &                    & Name benutzt bis \\ \hline
		name                & string            &                    & Name des Interessenpunkts \\ \hline
	\end{tabular}
\end{table}
\paragraph{Antwort}Die Antwort ist wie folgt aufgebaut:
\begin{table}[H]
	\begin{tabular}{|c|c|c|p{6.5cm}|}
		\hline
		\textbf{Paramtername} & \textbf{Datentyp} & \textbf{Konstante} & \textbf{Kurzbeschreibung}                                                                                               \\ \hline
		result              & string           &                 & Erfolgreich wenn Wert {\glqq ack\grqq} ist \\ \hline
		Code                & int              &                 & Erfolgreich wenn Wert {\glqq 0\grqq} ist \\ \hline
		state               & bool             &                 & Wahr, wenn erfolgreich \\ \hline
	\end{tabular}
\end{table}
\subsubsection{Bild löschen}
\paragraph{Kurzbeschreibung}Dieser API-Request wird dazu genutzt um ein Bild zu löschen.
\paragraph{Anfrage}Folgende Daten werden zu Anfrage benötigt:
\begin{table}[H]
	\begin{tabular}{|c|c|c|p{6.5cm}|}
		\hline
		\textbf{Paramtername} & \textbf{Datentyp} & \textbf{Konstante} & \textbf{Kurzbeschreibung}                                                                                               \\ \hline
		type                & string            & dsp                & Bild löschen\\ \hline
		token               & string            &                    & Identifikator des Bildes \\ \hline
	\end{tabular}
\end{table}
\paragraph{Antwort}Die Antwort ist wie folgt aufgebaut:
\begin{table}[H]
	\begin{tabular}{|c|c|c|p{6.5cm}|}
		\hline
		\textbf{Paramtername} & \textbf{Datentyp} & \textbf{Konstante} & \textbf{Kurzbeschreibung}                                                                                               \\ \hline
		result              & string           &                 & Erfolgreich wenn Wert {\glqq ack\grqq} ist \\ \hline
		Code                & int              &                 & Erfolgreich wenn Wert {\glqq 0\grqq} ist \\ \hline
	\end{tabular}
\end{table}
\subsubsection{Abfrage der Titel aller Interessenpunkte}
\paragraph{Kurzbeschreibung}Dieser API-Request wird dazu genutzt um die Titel aller Interessenpunkt abzufragen, welche noch nicht mit einer spezifischen Geschichte verlinkt sind.
\paragraph{Anfrage}Folgende Daten werden zu Anfrage benötigt:
\begin{table}[H]
	\begin{tabular}{|c|c|c|p{6.5cm}|}
		\hline
		\textbf{Paramtername} & \textbf{Datentyp} & \textbf{Konstante} & \textbf{Kurzbeschreibung}                                                                                               \\ \hline
		type                & string            & gpt                & Titel aller Interessenpunkte abfragen\\ \hline
		storytoken          & string            &                    & Identifikator der Geschichte \\ \hline
	\end{tabular}
\end{table}
\paragraph{Antwort}Die Antwort ist wie folgt aufgebaut:
\begin{table}[H]
	\begin{tabular}{|c|c|c|p{6.5cm}|}
		\hline
		\textbf{Paramtername} & \textbf{Datentyp} & \textbf{Konstante} & \textbf{Kurzbeschreibung}                                                                                               \\ \hline
		result              & string           &                 & Erfolgreich wenn Wert {\glqq ack\grqq} ist \\ \hline
		Code                & int              &                 & Erfolgreich wenn Wert {\glqq 0\grqq} ist \\ \hline
		data                & array            &                 & Liste der Titel \\ \hline
	\end{tabular}
\end{table}
\subsubsection{Verlinkung von Interessenpunkt und Geschichte}
\paragraph{Kurzbeschreibung}Dieser API-Request wird dazu genutzt um eine Geschichte und einen Interessenpunkt zu verlinken.
\paragraph{Anfrage}Folgende Daten werden zu Anfrage benötigt:
\begin{table}[H]
	\begin{tabular}{|c|c|c|p{6.5cm}|}
		\hline
		\textbf{Paramtername} & \textbf{Datentyp} & \textbf{Konstante} & \textbf{Kurzbeschreibung}                                                                                               \\ \hline
		type                & string            & aps                & Geschichte und Interessenpunkt verlinken\\ \hline
		storytoken          & string            &                    & Identifikator der Geschichte \\ \hline
		poiid               & int               &                    & Identifikator des Interessenpunktes \\ \hline
	\end{tabular}
\end{table}
\paragraph{Antwort}Die Antwort ist wie folgt aufgebaut:
\begin{table}[H]
	\begin{tabular}{|c|c|c|p{6.5cm}|}
		\hline
		\textbf{Paramtername} & \textbf{Datentyp} & \textbf{Konstante} & \textbf{Kurzbeschreibung}                                                                                               \\ \hline
		result              & string           &                 & Erfolgreich wenn Wert {\glqq ack\grqq} ist \\ \hline
		Code                & int              &                 & Erfolgreich wenn Wert {\glqq 0\grqq} ist \\ \hline
		data                & bool             &                 & Erfolgreich wenn Wert {\glqq true\grqq} ist \\ \hline
	\end{tabular}
\end{table}
\subsubsection{Abfrage von Verlinkungen zwischen Interessenpunkt und Geschichte}
\paragraph{Kurzbeschreibung}Dieser API-Request wird dazu genutzt um die Verlinkungen einer bestimmten Geschichte mit Interessenpunkten ab zu fragen.
\paragraph{Anfrage}Folgende Daten werden zu Anfrage benötigt:
\begin{table}[H]
	\begin{tabular}{|c|c|c|p{6.5cm}|}
		\hline
		\textbf{Paramtername} & \textbf{Datentyp} & \textbf{Konstante} & \textbf{Kurzbeschreibung}                                                                                               \\ \hline
		type                & string            & gps                & Abfrage Verknüpfung Geschichte und Interessenpunkt \\ \hline
		storytoken          & string            &                    & Identifikator der Geschichte \\ \hline
	\end{tabular}
\end{table}
\paragraph{Antwort}Die Antwort ist wie folgt aufgebaut:
\begin{table}[H]
	\begin{tabular}{|c|c|c|p{6.5cm}|}
		\hline
		\textbf{Paramtername} & \textbf{Datentyp} & \textbf{Konstante} & \textbf{Kurzbeschreibung}                                                                                               \\ \hline
		result              & string           &                 & Erfolgreich wenn Wert {\glqq ack\grqq} ist \\ \hline
		Code                & int              &                 & Erfolgreich wenn Wert {\glqq 0\grqq} ist \\ \hline
		data                & array            &                 & Strukturiertes Ergebnis \\ \hline
	\end{tabular}
\end{table}
\subparagraph{data}Dieses Array enthält Einträge in der nachstehend dargestellten Form haben:
\begin{table}[H]
	\begin{tabular}{|c|c|c|p{6.5cm}|}
		\hline
		\textbf{Paramtername} & \textbf{Datentyp} & \textbf{Konstante} & \textbf{Kurzbeschreibung}    \\ \hline
		guest                  & bool            &                 & Wahr, wenn Nutzer Gast \\ \hline
		valpos                 & int             &                 & Wahr, wenn Nutzer Validierungswert größer 0 hat \\ \hline
		pois                   & array           &                 & Liste der verknüpften Interessenpunkte \\ \hline
	\end{tabular}
\end{table}
\subparagraph{pois}Dieses Array enthält Elemente mit Einträge in der nachstehend dargestellten Form haben:
\begin{table}[H]
	\begin{tabular}{|c|c|c|p{6.5cm}|}
		\hline
		\textbf{Paramtername} & \textbf{Datentyp} & \textbf{Konstante} & \textbf{Kurzbeschreibung}    \\ \hline
		deletable              & bool            &                 & Wahr, wenn Verlinkung Löschen darf \\ \hline
		deleted                & bool            &                 & Wahr, wenn Link als gelöscht gilt \\ \hline
		validate               & int             &                 & Wahr, wenn Link validiert \\ \hline
		id                     & int             &                 & Identifikator des Links \\ \hline
		poi\_id                & int             &                 & Identifikator des Interessenpunktes \\ \hline
		name                   & string          &                 & Name des Interessenpunktes \\ \hline
		restrictions           & bool            &                 & Wahr, wenn Wiederherstellung gelöschte Abhängigkeiten hat \\ \hline
	\end{tabular}
\end{table}
\subsubsection{Validierung einer Verlinkungen zwischen Interessenpunkt und Geschichte}
\paragraph{Kurzbeschreibung}Dieser API-Request wird dazu genutzt um die Verlinkungen einer bestimmten Geschichte mit einem Interessenpunkt zu validieren.
\paragraph{Anfrage}Folgende Daten werden zu Anfrage benötigt:
\begin{table}[H]
	\begin{tabular}{|c|c|c|p{6.5cm}|}
		\hline
		\textbf{Paramtername} & \textbf{Datentyp} & \textbf{Konstante} & \textbf{Kurzbeschreibung}                                                                                               \\ \hline
		type                & string            & vps                & Validierung Verknüpfung Geschichte und Interessenpunkt \\ \hline
		poiStoryId          & int               &                    & Identifikator der Verlinkung \\ \hline
	\end{tabular}
\end{table}
\paragraph{Antwort}Die Antwort ist wie folgt aufgebaut:
\begin{table}[H]
	\begin{tabular}{|c|c|c|p{6.5cm}|}
		\hline
		\textbf{Paramtername} & \textbf{Datentyp} & \textbf{Konstante} & \textbf{Kurzbeschreibung}                                                                                               \\ \hline
		result              & string           &                 & Erfolgreich wenn Wert {\glqq ack\grqq} ist \\ \hline
		Code                & int              &                 & Erfolgreich wenn Wert {\glqq 0\grqq} ist \\ \hline
		type                & string           & vps             & Validierung Verknüpfung Geschichte und Interessenpunkt \\ \hline
		poiStoryId          & int              &                 & Identifikator der Verlinkung \\ \hline
	\end{tabular}
\end{table}
\subsubsection{Löschen einer Verlinkungen zwischen Interessenpunkt und Geschichte}
\paragraph{Kurzbeschreibung}Dieser API-Request wird dazu genutzt um die Verlinkungen einer bestimmten Geschichte mit einem Interessenpunkt zu löschen.
\paragraph{Anfrage}Folgende Daten werden zu Anfrage benötigt:
\begin{table}[H]
	\begin{tabular}{|c|c|c|p{6.5cm}|}
		\hline
		\textbf{Paramtername} & \textbf{Datentyp} & \textbf{Konstante} & \textbf{Kurzbeschreibung}                                                                                               \\ \hline
		type                & string            & dps                & Löschung Verknüpfung Geschichte und Interessenpunkt \\ \hline
		poiStoryId          & int               &                    & Identifikator der Verlinkung \\ \hline
	\end{tabular}
\end{table}
\paragraph{Antwort}Die Antwort ist wie folgt aufgebaut:
\begin{table}[H]
	\begin{tabular}{|c|c|c|p{6.5cm}|}
		\hline
		\textbf{Paramtername} & \textbf{Datentyp} & \textbf{Konstante} & \textbf{Kurzbeschreibung}                                                                                               \\ \hline
		result              & string           &                 & Erfolgreich wenn Wert {\glqq ack\grqq} ist \\ \hline
		Code                & int              &                 & Erfolgreich wenn Wert {\glqq 0\grqq} ist \\ \hline
		type                & string           & dps             & Löschung Verknüpfung Geschichte und Interessenpunkt \\ \hline
		poiStoryId          & int              &                 & Identifikator der Verlinkung \\ \hline
	\end{tabular}
\end{table}
\subsubsection{Löschen einer Geschichte}
\paragraph{Kurzbeschreibung}Dieser API-Request wird dazu genutzt um die Verlinkungen eine bestimmten Geschichte zu löschen.
\paragraph{Anfrage}Folgende Daten werden zu Anfrage benötigt:
\begin{table}[H]
	\begin{tabular}{|c|c|c|p{6.5cm}|}
		\hline
		\textbf{Paramtername} & \textbf{Datentyp} & \textbf{Konstante} & \textbf{Kurzbeschreibung}                                                                                               \\ \hline
		type                & string            & dus                & Löschung Verknüpfung Geschichte und Interessenpunkt \\ \hline
		story\_token        & string            &                    & Identifikator der Geschichte \\ \hline
	\end{tabular}
\end{table}
\paragraph{Antwort}Die Antwort ist wie folgt aufgebaut:
\begin{table}[H]
	\begin{tabular}{|c|c|c|p{6.5cm}|}
		\hline
		\textbf{Paramtername} & \textbf{Datentyp} & \textbf{Konstante} & \textbf{Kurzbeschreibung}                                                                                               \\ \hline
		result              & string           &                 & Erfolgreich wenn Wert {\glqq ack\grqq} ist \\ \hline
		Code                & int              &                 & Erfolgreich wenn Wert {\glqq 0\grqq} ist \\ \hline
		data                & array            &                 & Strukturiert wie die Anfrage \\ \hline
	\end{tabular}
\end{table}
\subparagraph{data}Dieses Array enthält Einträge in der nachstehend dargestellten Form haben:
\begin{table}[H]
	\begin{tabular}{|c|c|c|p{6.5cm}|}
		\hline
		\textbf{Paramtername} & \textbf{Datentyp} & \textbf{Konstante} & \textbf{Kurzbeschreibung}    \\ \hline
		type                   & string          & dus             & Löschung Verknüpfung Geschichte und Interessenpunkt \\ \hline
		story\_token           & string          &                 & Identifikator der Geschichte \\ \hline
	\end{tabular}
\end{table}
\subsubsection{Existenzprüfung einer Adresse}
\paragraph{Kurzbeschreibung}Dieser API-Request wird dazu genutzt um zu überprüfen, ob eine bestimmte Adresse bereits in der Datenbank vorhanden ist.
\paragraph{Anfrage}Folgende Daten werden zu Anfrage benötigt:
\begin{table}[H]
	\begin{tabular}{|c|c|c|p{6.5cm}|}
		\hline
		\textbf{Paramtername} & \textbf{Datentyp} & \textbf{Konstante} & \textbf{Kurzbeschreibung}                                                                                               \\ \hline
		type                & string            & cha                & Existenzprüfung einer Adresse \\ \hline
		st                  & string            &                    & Straßenname \\ \hline
		hn                  & string            &                    & Hausnummer \\ \hline
		ct                  & string            &                    & Stadt oder Ort \\ \hline
		pc                  & int               &                    & Postleitzahl \\ \hline
	\end{tabular}
\end{table}
\paragraph{Antwort}Die Antwort ist wie folgt aufgebaut:
\begin{table}[H]
	\begin{tabular}{|c|c|c|p{6.5cm}|}
		\hline
		\textbf{Paramtername} & \textbf{Datentyp} & \textbf{Konstante} & \textbf{Kurzbeschreibung}                                                                                               \\ \hline
		result              & string           &                 & Erfolgreich wenn Wert {\glqq ack\grqq} ist \\ \hline
		Code                & int              &                 & Erfolgreich wenn Wert {\glqq 0\grqq} ist \\ \hline
		data                & bool             &                 & Wahr, wenn Adresse bereits in Datenbank \\ \hline
		request             & array            &                 & Strukturiert wie Anfrage \\ \hline
	\end{tabular}
\end{table}
\subparagraph{request}Dieses Array enthält Einträge in der nachstehend dargestellten Form haben:
\begin{table}[H]
	\begin{tabular}{|c|c|c|p{6.5cm}|}
		\hline
		\textbf{Paramtername} & \textbf{Datentyp} & \textbf{Konstante} & \textbf{Kurzbeschreibung}    \\ \hline
		type                   & string          & cha             & Existenzprüfung einer Adresse \\ \hline
		st                     & string          &                 & Straßenname \\ \hline
		hn                     & string          &                 & Hausnummer \\ \hline
		ct                     & string          &                 & Stadt oder Ort \\ \hline
		pc                     & int             &                 & Postleitzahl \\ \hline
	\end{tabular}
\end{table}
\subsubsection{Abfrage der Ladedaten von Bildern}
\paragraph{Kurzbeschreibung}Dieser API-Request wird dazu genutzt um die Daten zum laden mehrerer Bilder zu bekommen.
\paragraph{Anfrage}Folgende Daten werden zu Anfrage benötigt:
\begin{table}[H]
	\begin{tabular}{|c|c|c|p{6.5cm}|}
		\hline
		\textbf{Paramtername} & \textbf{Datentyp} & \textbf{Konstante} & \textbf{Kurzbeschreibung}                                                                                               \\ \hline
		type                & string            & gpf                & Abfrage der Ladedaten von Bildern \\ \hline
	\end{tabular}
\end{table}
\paragraph{Antwort}Die Antwort ist wie folgt aufgebaut:
\begin{table}[H]
	\begin{tabular}{|c|c|c|p{6.5cm}|}
		\hline
		\textbf{Paramtername} & \textbf{Datentyp} & \textbf{Konstante} & \textbf{Kurzbeschreibung}                                                                                               \\ \hline
		result              & string           &                 & Erfolgreich wenn Wert {\glqq ack\grqq} ist \\ \hline
		Code                & int              &                 & Erfolgreich wenn Wert {\glqq 0\grqq} ist \\ \hline
		data                & array            &                 & Array mit allen benötigten Daten \\ \hline
	\end{tabular}
\end{table}
\subparagraph{data}Dieses Array enthält Einträge in der nachstehend dargestellten Form haben:
\begin{table}[H]
	\begin{tabular}{|c|c|c|p{6.5cm}|}
		\hline
		\textbf{Paramtername} & \textbf{Datentyp} & \textbf{Konstante} & \textbf{Kurzbeschreibung}    \\ \hline
		description            & string          &                 & Beschreibung des Bildes \\ \hline
		fullsize               & string          &                 & URI zum Laden des Bildes \\ \hline
		id                     & int             &                 & Identifikator des Bildes \\ \hline
		identifier             & string          &                 & zweiter Identifikator des Bildes \\ \hline
		preview                & string          &                 & URI zum Laden einer Vorschau \\ \hline
		title                  & string          &                 & Titel des Bildes \\ \hline
		token                  & array           &                 & Informationen zum Laden des Bildes \\ \hline
		username               & string          &                 & Nutzername des Erstellers \\ \hline
		valUsers               & array           &                 & Array mit allen Validatoren \\ \hline
		validationValue        & int             &                 & Validierungsstatus \\ \hline
	\end{tabular}
\end{table}
\subparagraph{token}Dieses Array enthält Einträge in der nachstehend dargestellten Form haben:
\begin{table}[H]
	\begin{tabular}{|c|c|c|p{6.5cm}|}
		\hline
		\textbf{Paramtername} & \textbf{Datentyp} & \textbf{Konstante} & \textbf{Kurzbeschreibung}    \\ \hline
		seccode                & string          &                 & Security Code \\ \hline
		token                  & string          &                 & Identifikator des Bildes \\ \hline
		time                   & int             &                 & Timestamp \\ \hline
	\end{tabular}
\end{table}
\subsubsection{Verknüpfung zwischen Interessenpunkt und Bild hinzufügen}
\paragraph{Kurzbeschreibung}Dieser API-Request wird dazu genutzt um eine Verlinkung zwischen einem Bild und einem Interessenpunkt herzustellen.
\paragraph{Anfrage}Folgende Daten werden zu Anfrage benötigt:
\begin{table}[H]
	\begin{tabular}{|c|c|c|p{6.5cm}|}
		\hline
		\textbf{Paramtername} & \textbf{Datentyp} & \textbf{Konstante} & \textbf{Kurzbeschreibung}                                                                                               \\ \hline
		type                & string            & app                & Anlegen eines neuen Links \\ \hline
		poi                 & int               &                    & Identifikator des Interessenpunktes \\ \hline
		data                & array             &                    & Array mit Bilderidentifikatoren \\ \hline
	\end{tabular}
\end{table}
\paragraph{Antwort}Die Antwort ist wie folgt aufgebaut:
\begin{table}[H]
	\begin{tabular}{|c|c|c|p{6.5cm}|}
		\hline
		\textbf{Paramtername} & \textbf{Datentyp} & \textbf{Konstante} & \textbf{Kurzbeschreibung}                                                                                               \\ \hline
		result              & string           &                 & Erfolgreich wenn Wert {\glqq ack\grqq} ist \\ \hline
		Code                & int              &                 & Erfolgreich wenn Wert {\glqq 0\grqq} ist \\ \hline
	\end{tabular}
\end{table}
\subsubsection{Validierung einer Verknüpfung zwischen Interessenpunkt und Bild hinzufügen}
\paragraph{Kurzbeschreibung}Dieser API-Request wird dazu genutzt um eine Verlinkung zwischen einem Bild und einem Interessenpunkt zu validieren.
\paragraph{Anfrage}Folgende Daten werden zu Anfrage benötigt:
\begin{table}[H]
	\begin{tabular}{|c|c|c|p{6.5cm}|}
		\hline
		\textbf{Paramtername} & \textbf{Datentyp} & \textbf{Konstante} & \textbf{Kurzbeschreibung}                                                                                               \\ \hline
		type                & string            & vpp                & Anlegen eines neuen Links \\ \hline
		id                  & int               &                    & Identifikator der Verknüpfung \\ \hline
	\end{tabular}
\end{table}
\paragraph{Antwort}Die Antwort ist wie folgt aufgebaut:
\begin{table}[H]
	\begin{tabular}{|c|c|c|p{6.5cm}|}
		\hline
		\textbf{Paramtername} & \textbf{Datentyp} & \textbf{Konstante} & \textbf{Kurzbeschreibung}                                                                                               \\ \hline
		result              & string           &                 & Erfolgreich wenn Wert {\glqq ack\grqq} ist \\ \hline
		Code                & int              &                 & Erfolgreich wenn Wert {\glqq 0\grqq} ist \\ \hline
	\end{tabular}
\end{table}
\subsubsection{Löschung einer Verknüpfung zwischen Interessenpunkt und Bild hinzufügen}
\paragraph{Kurzbeschreibung}Dieser API-Request wird dazu genutzt um eine Verlinkung zwischen einem Bild und einem Interessenpunkt zu löschen.
\paragraph{Anfrage}Folgende Daten werden zu Anfrage benötigt:
\begin{table}[H]
	\begin{tabular}{|c|c|c|p{6.5cm}|}
		\hline
		\textbf{Paramtername} & \textbf{Datentyp} & \textbf{Konstante} & \textbf{Kurzbeschreibung}                                                                                               \\ \hline
		type                & string            & vpp                & Löschen eines Links \\ \hline
		id                  & int               &                    & Identifikator der Verknüpfung \\ \hline
	\end{tabular}
\end{table}
\paragraph{Antwort}Die Antwort ist wie folgt aufgebaut:
\begin{table}[H]
	\begin{tabular}{|c|c|c|p{6.5cm}|}
		\hline
		\textbf{Paramtername} & \textbf{Datentyp} & \textbf{Konstante} & \textbf{Kurzbeschreibung}                                                                                               \\ \hline
		result              & string           &                 & Erfolgreich wenn Wert {\glqq ack\grqq} ist \\ \hline
		Code                & int              &                 & Erfolgreich wenn Wert {\glqq 0\grqq} ist \\ \hline
	\end{tabular}
\end{table}
\subsubsection{Abfrage alle Informationen zum Laden des Interessenpunkt-Bild-Link-Modals}
\paragraph{Kurzbeschreibung}Dieser API-Request wird dazu genutzt um alle Informationen zum Anzeigen des Bild-Interessenpunkt-Verknüpfers im Biographien Modul zu laden.
\paragraph{Anfrage}Folgende Daten werden zu Anfrage benötigt:
\begin{table}[H]
	\begin{tabular}{|c|c|c|p{6.5cm}|}
		\hline
		\textbf{Paramtername} & \textbf{Datentyp} & \textbf{Konstante} & \textbf{Kurzbeschreibung}                                                                                               \\ \hline
		type                & string            & lpp                & Abfrage aller verfügbaren Interessenpunktitel \\ \hline
		pictoken            & string            &                    & Identifikator eines Bildes \\ \hline
	\end{tabular}
\end{table}
\paragraph{Antwort}Die Antwort ist wie folgt aufgebaut:
\begin{table}[H]
	\begin{tabular}{|c|c|c|p{6.5cm}|}
		\hline
		\textbf{Paramtername} & \textbf{Datentyp} & \textbf{Konstante} & \textbf{Kurzbeschreibung}                                                                                               \\ \hline
		result              & string           &                 & Erfolgreich wenn Wert {\glqq ack\grqq} ist \\ \hline
		Code                & int              &                 & Erfolgreich wenn Wert {\glqq 0\grqq} ist \\ \hline
		data                & array            &                 & Array mit Verknüpfungsinformationen \\ \hline
	\end{tabular}
\end{table}
\subparagraph{token}Dieses Array enthält Einträge in der nachstehend dargestellten Form haben:
\begin{table}[H]
	\begin{tabular}{|c|c|c|p{6.5cm}|}
		\hline
		\textbf{Paramtername} & \textbf{Datentyp} & \textbf{Konstante} & \textbf{Kurzbeschreibung}    \\ \hline
		guest                  & bool            &                 & Wahr wenn Nutzer Gast ist \\ \hline
		linked                 & Array           &                 & Bestehende Verlinkungen \\ \hline
		options                & int             &                 & Nicht bestehende Verlinkungen \\ \hline
		valpos                 & bool            &                 & Wahr wenn Nutzer validieren darf \\ \hline
	\end{tabular}
\end{table}
\subparagraph{linked}Dieses Array enthält Elemente mit Einträgen in der nachstehend dargestellten Form haben:
\begin{table}[H]
	\begin{tabular}{|c|c|c|p{6.5cm}|}
		\hline
		\textbf{Paramtername} & \textbf{Datentyp} & \textbf{Konstante} & \textbf{Kurzbeschreibung}    \\ \hline
		deletable              & bool            &                 & Gibt an, ob Nutzer Link löschen darf \\ \hline
		lid                    & int             &                 & Identifikator des Links\\ \hline
		name                   & string          &                 & Name des Interessenpunktes \\ \hline
		poi\_id                & int             &                 & Identifikator des Interessenpunktes \\ \hline
		validated              & bool            &                 & Validierungsstatus des Links \\ \hline
		deleted                & bool            &                 & Wahr, wenn Link als gelöscht markiert \\ \hline
		restrictions           & bool            &                 & Wahr, wenn Abhängigkeiten als gelöscht gelten \\ \hline
	\end{tabular}
\end{table}
\subparagraph{options}Dieses Array enthält Elemente mit Einträgen in der nachstehend dargestellten Form haben:
\begin{table}[H]
	\begin{tabular}{|c|c|c|p{6.5cm}|}
		\hline
		\textbf{Paramtername} & \textbf{Datentyp} & \textbf{Konstante} & \textbf{Kurzbeschreibung}    \\ \hline
		name                   & string          &                 & Name des Interessenpunktes \\ \hline
		poi\_id                & int             &                 & Identifikator des Interessenpunktes \\ \hline
	\end{tabular}
\end{table}
\subsubsection{Sitzplatzanzahl zu Interessenpunkt hinzufügen}
\paragraph{Kurzbeschreibung}Dieser API-Request wird dazu genutzt um eine neue Sitzplatzanzahl zu einem Interessenpunkt hinzuzufügen.
\paragraph{Anfrage}Folgende Daten werden zu Anfrage benötigt:
\begin{table}[H]
	\begin{tabular}{|c|c|c|p{6.5cm}|}
		\hline
		\textbf{Paramtername} & \textbf{Datentyp} & \textbf{Konstante} & \textbf{Kurzbeschreibung}                                                                                               \\ \hline
		type                & string            & asc                & Sitzplatzanzahl hinzufügen \\ \hline
		poi\_id             & int               &                    & Identifikator eines Interessenpunktes \\ \hline
		from                & int               &                    & Zeitpunkt des Startes der Sitzplatzanzahl \\ \hline
		till                & int               &                    & Zeitpunkt des Endes der Sitzplatzanzahl \\ \hline
		seats               & int               &                    & Sitzplatzanzahl \\ \hline
	\end{tabular}
\end{table}
\paragraph{Antwort}Die Antwort ist wie folgt aufgebaut:
\begin{table}[H]
	\begin{tabular}{|c|c|c|p{6.5cm}|}
		\hline
		\textbf{Paramtername} & \textbf{Datentyp} & \textbf{Konstante} & \textbf{Kurzbeschreibung}                                                                                               \\ \hline
		result              & string           &                 & Erfolgreich wenn Wert {\glqq ack\grqq} ist \\ \hline
		Code                & int              &                 & Erfolgreich wenn Wert {\glqq 0\grqq} ist \\ \hline
		type                & string           & asc             & Sitzplatzanzahl hinzufügen \\ \hline
		poi\_id             & int              &                 & Identifikator eines Interessenpunktes \\ \hline
		from                & int              &                 & Zeitpunkt des Startes der Sitzplatzanzahl \\ \hline
		till                & int              &                 & Zeitpunkt des Endes der Sitzplatzanzahl \\ \hline
		seats               & int              &                 & Sitzplatzanzahl \\ \hline
	\end{tabular}
\end{table}
\subsubsection{Validierung einer Sitzplatzanzahl}
\paragraph{Kurzbeschreibung}Dieser API-Request wird dazu genutzt um eine Sitzplatzanzahl zu validieren.
\paragraph{Anfrage}Folgende Daten werden zu Anfrage benötigt:
\begin{table}[H]
	\begin{tabular}{|c|c|c|p{6.5cm}|}
		\hline
		\textbf{Paramtername} & \textbf{Datentyp} & \textbf{Konstante} & \textbf{Kurzbeschreibung}                                                                                               \\ \hline
		type                & string            & vsc                & Sitzplatzanzahl validieren \\ \hline
		SEATID              & int               &                    & Identifikator der Sitzplatzanzahl \\ \hline
	\end{tabular}
\end{table}
\paragraph{Antwort}Die Antwort ist wie folgt aufgebaut:
\begin{table}[H]
	\begin{tabular}{|c|c|c|p{6.5cm}|}
		\hline
		\textbf{Paramtername} & \textbf{Datentyp} & \textbf{Konstante} & \textbf{Kurzbeschreibung}                                                                                               \\ \hline
		result              & string           &                 & Erfolgreich wenn Wert {\glqq ack\grqq} ist \\ \hline
		Code                & int              &                 & Erfolgreich wenn Wert {\glqq 0\grqq} ist \\ \hline
		type                & string           & vsc             & Sitzplatzanzahl validieren \\ \hline
		SEATID              & int              &                 & Identifikator der Sitzplatzanzahl \\ \hline
	\end{tabular}
\end{table}
\subsubsection{Löschung einer Sitzplatzanzahl}
\paragraph{Kurzbeschreibung}Dieser API-Request wird dazu genutzt um eine Sitzplatzanzahl zu löschen.
\paragraph{Anfrage}Folgende Daten werden zu Anfrage benötigt:
\begin{table}[H]
	\begin{tabular}{|c|c|c|p{6.5cm}|}
		\hline
		\textbf{Paramtername} & \textbf{Datentyp} & \textbf{Konstante} & \textbf{Kurzbeschreibung}                                                                                               \\ \hline
		type                & string            & dsc                & Sitzplatzanzahl löschen \\ \hline
		SEATID              & int               &                    & Identifikator der Sitzplatzanzahl \\ \hline
	\end{tabular}
\end{table}
\paragraph{Antwort}Die Antwort ist wie folgt aufgebaut:
\begin{table}[H]
	\begin{tabular}{|c|c|c|p{6.5cm}|}
		\hline
		\textbf{Paramtername} & \textbf{Datentyp} & \textbf{Konstante} & \textbf{Kurzbeschreibung}                                                                                               \\ \hline
		result              & string           &                 & Erfolgreich wenn Wert {\glqq ack\grqq} ist \\ \hline
		Code                & int              &                 & Erfolgreich wenn Wert {\glqq 0\grqq} ist \\ \hline
		type                & string           & dsc             & Sitzplatzanzahl löschen \\ \hline
		SEATID              & int              &                 & Identifikator der Sitzplatzanzahl \\ \hline
	\end{tabular}
\end{table}
\subsubsection{Speichern einer Sitzplatzanzahl}
\paragraph{Kurzbeschreibung}Dieser API-Request wird dazu genutzt um eine bestehende Sitzplatzanzahl zu Speichern.
\paragraph{Anfrage}Folgende Daten werden zu Anfrage benötigt:
\begin{table}[H]
	\begin{tabular}{|c|c|c|p{6.5cm}|}
		\hline
		\textbf{Paramtername} & \textbf{Datentyp} & \textbf{Konstante} & \textbf{Kurzbeschreibung}                                                                                               \\ \hline
		type                & string            & usc                & Sitzplatzanzahl löschen \\ \hline
		id                  & int               &                    & Identifikator der Sitzplatzanzahl \\ \hline
		start               & int               &                    & Zeitpunkt des Startes der Sitzplatzanzahl \\ \hline
		end                 & int               &                    & Zeitpunkt des Endes der Sitzplatzanzahl \\ \hline
		seats               & int               &                    & Sitzplatzanzahl \\ \hline
	\end{tabular}
\end{table}
\paragraph{Antwort}Die Antwort ist wie folgt aufgebaut:
\begin{table}[H]
	\begin{tabular}{|c|c|c|p{6.5cm}|}
		\hline
		\textbf{Paramtername} & \textbf{Datentyp} & \textbf{Konstante} & \textbf{Kurzbeschreibung}                                                                                               \\ \hline
		result              & string           &                 & Erfolgreich wenn Wert {\glqq ack\grqq} ist \\ \hline
		Code                & int              &                 & Erfolgreich wenn Wert {\glqq 0\grqq} ist \\ \hline
		type                & string           & usc             & Sitzplatzanzahl löschen \\ \hline
		SEATID              & int              &                 & Identifikator der Sitzplatzanzahl \\ \hline
	\end{tabular}
\end{table}

\subsubsection{Saalanzahl zu Interessenpunkt hinzufügen}
\paragraph{Kurzbeschreibung}Dieser API-Request wird dazu genutzt um eine neue Saalanzahl zu einem Interessenpunkt hinzuzufügen.
\paragraph{Anfrage}Folgende Daten werden zu Anfrage benötigt:
\begin{table}[H]
	\begin{tabular}{|c|c|c|p{6.5cm}|}
		\hline
		\textbf{Paramtername} & \textbf{Datentyp} & \textbf{Konstante} & \textbf{Kurzbeschreibung}                                                                                               \\ \hline
		type                & string            & acc                & Saalanzahl hinzufügen \\ \hline
		poi\_id             & int               &                    & Identifikator eines Interessenpunktes \\ \hline
		from                & int               &                    & Zeitpunkt des Startes der Saalanzahl \\ \hline
		till                & int               &                    & Zeitpunkt des Endes der Saalanzahl \\ \hline
		cinemas             & int               &                    & Saalanzahl \\ \hline
	\end{tabular}
\end{table}
\paragraph{Antwort}Die Antwort ist wie folgt aufgebaut:
\begin{table}[H]
	\begin{tabular}{|c|c|c|p{6.5cm}|}
		\hline
		\textbf{Paramtername} & \textbf{Datentyp} & \textbf{Konstante} & \textbf{Kurzbeschreibung}                                                                                               \\ \hline
		result              & string           &                 & Erfolgreich wenn Wert {\glqq ack\grqq} ist \\ \hline
		Code                & int              &                 & Erfolgreich wenn Wert {\glqq 0\grqq} ist \\ \hline
		type                & string           & acc             & Saalanzahl hinzufügen \\ \hline
		poi\_id             & int              &                 & Identifikator eines Interessenpunktes \\ \hline
		from                & int              &                 & Zeitpunkt des Startes der Saalanzahl \\ \hline
		till                & int              &                 & Zeitpunkt des Endes der Saalanzahl \\ \hline
		cinemas             & int              &                 & Saalanzahl \\ \hline
	\end{tabular}
\end{table}
\subsubsection{Validierung einer Saalanzahl}
\paragraph{Kurzbeschreibung}Dieser API-Request wird dazu genutzt um eine Saalanzahl zu validieren.
\paragraph{Anfrage}Folgende Daten werden zu Anfrage benötigt:
\begin{table}[H]
	\begin{tabular}{|c|c|c|p{6.5cm}|}
		\hline
		\textbf{Paramtername} & \textbf{Datentyp} & \textbf{Konstante} & \textbf{Kurzbeschreibung}                                                                                               \\ \hline
		type                & string            & vcc                & Saalanzahl validieren \\ \hline
		CINEMAID            & int               &                    & Identifikator der Saalanzahl \\ \hline
	\end{tabular}
\end{table}
\paragraph{Antwort}Die Antwort ist wie folgt aufgebaut:
\begin{table}[H]
	\begin{tabular}{|c|c|c|p{6.5cm}|}
		\hline
		\textbf{Paramtername} & \textbf{Datentyp} & \textbf{Konstante} & \textbf{Kurzbeschreibung}                                                                                               \\ \hline
		result              & string           &                 & Erfolgreich wenn Wert {\glqq ack\grqq} ist \\ \hline
		Code                & int              &                 & Erfolgreich wenn Wert {\glqq 0\grqq} ist \\ \hline
		type                & string           & vcc             & Saalanzahl validieren \\ \hline
		CINEMAID            & int              &                 & Identifikator der Saalanzahl \\ \hline
	\end{tabular}
\end{table}
\subsubsection{Löschung einer Saalanzahl}
\paragraph{Kurzbeschreibung}Dieser API-Request wird dazu genutzt um eine Saalanzahl zu löschen.
\paragraph{Anfrage}Folgende Daten werden zu Anfrage benötigt:
\begin{table}[H]
	\begin{tabular}{|c|c|c|p{6.5cm}|}
		\hline
		\textbf{Paramtername} & \textbf{Datentyp} & \textbf{Konstante} & \textbf{Kurzbeschreibung}                                                                                               \\ \hline
		type                & string            & dcc                & Saalanzahl löschen \\ \hline
		IDent               & int               &                    & Identifikator der Saalanzahl \\ \hline
	\end{tabular}
\end{table}
\paragraph{Antwort}Die Antwort ist wie folgt aufgebaut:
\begin{table}[H]
	\begin{tabular}{|c|c|c|p{6.5cm}|}
		\hline
		\textbf{Paramtername} & \textbf{Datentyp} & \textbf{Konstante} & \textbf{Kurzbeschreibung}                                                                                               \\ \hline
		result              & string           &                 & Erfolgreich wenn Wert {\glqq ack\grqq} ist \\ \hline
		Code                & int              &                 & Erfolgreich wenn Wert {\glqq 0\grqq} ist \\ \hline
		type                & string           & dcc             & Saalanzahl löschen \\ \hline
		IDent               & int              &                 & Identifikator der Saalanzahl \\ \hline
	\end{tabular}
\end{table}
\subsubsection{Speichern einer Saalanzahl}
\paragraph{Kurzbeschreibung}Dieser API-Request wird dazu genutzt um eine bestehende Saalanzahl zu Speichern.
\paragraph{Anfrage}Folgende Daten werden zu Anfrage benötigt:
\begin{table}[H]
	\begin{tabular}{|c|c|c|p{6.5cm}|}
		\hline
		\textbf{Paramtername} & \textbf{Datentyp} & \textbf{Konstante} & \textbf{Kurzbeschreibung}                                                                                               \\ \hline
		type                & string            & ucc                & Saalanzahl löschen \\ \hline
		id                  & int               &                    & Identifikator der Saalanzahl \\ \hline
		start               & int               &                    & Zeitpunkt des Startes der Saalanzahl \\ \hline
		end                 & int               &                    & Zeitpunkt des Endes der Saalanzahl \\ \hline
		cinemas             & int               &                    & Saalanzahl \\ \hline
	\end{tabular}
\end{table}
\paragraph{Antwort}Die Antwort ist wie folgt aufgebaut:
\begin{table}[H]
	\begin{tabular}{|c|c|c|p{6.5cm}|}
		\hline
		\textbf{Paramtername} & \textbf{Datentyp} & \textbf{Konstante} & \textbf{Kurzbeschreibung}                                                                                               \\ \hline
		result              & string           &                 & Erfolgreich wenn Wert {\glqq ack\grqq} ist \\ \hline
		Code                & int              &                 & Erfolgreich wenn Wert {\glqq 0\grqq} ist \\ \hline
		type                & string           & ucc             & Saalanzahl löschen \\ \hline
		cinemas             & int              &                 & Identifikator der Saalanzahl \\ \hline
	\end{tabular}
\end{table}

\subsubsection{Validierung des Spielstättentyp}
\paragraph{Kurzbeschreibung}Dieser API-Request wird dazu genutzt um den Spielstättentyp eines Interessenpunktes zu validieren.
\paragraph{Anfrage}Folgende Daten werden zu Anfrage benötigt:
\begin{table}[H]
	\begin{tabular}{|c|c|c|p{6.5cm}|}
		\hline
		\textbf{Paramtername} & \textbf{Datentyp} & \textbf{Konstante} & \textbf{Kurzbeschreibung}                                                                                               \\ \hline
		type                & string            & vty                & Validierung Spielstättentyp \\ \hline
		POIID               & int               &                    & Identifikator des Interessenpunktes \\ \hline
	\end{tabular}
\end{table}
\paragraph{Antwort}Die Antwort ist wie folgt aufgebaut:
\begin{table}[H]
	\begin{tabular}{|c|c|c|p{6.5cm}|}
		\hline
		\textbf{Paramtername} & \textbf{Datentyp} & \textbf{Konstante} & \textbf{Kurzbeschreibung}                                                                                               \\ \hline
		result              & string           &                 & Erfolgreich wenn Wert {\glqq ack\grqq} ist \\ \hline
		Code                & int              &                 & Erfolgreich wenn Wert {\glqq 0\grqq} ist \\ \hline
		type                & string           & vty             & Validierung Spielstättentyp \\ \hline
		POIID               & int              &                 & Identifikator des Interessenpunktes \\ \hline
	\end{tabular}
\end{table}
\subsubsection{Abfrage Nutzer als Gast}
\paragraph{Kurzbeschreibung}Dieser API-Request wird dazu genutzt um abzufragen, ob aktueller Nutzer Gast ist.
\paragraph{Anfrage}Folgende Daten werden zu Anfrage benötigt:
\begin{table}[H]
	\begin{tabular}{|c|c|c|p{6.5cm}|}
		\hline
		\textbf{Paramtername} & \textbf{Datentyp} & \textbf{Konstante} & \textbf{Kurzbeschreibung}                                                                                               \\ \hline
		type                & string            & asg                & Gastabfrage \\ \hline
		POIID               & int               &                    & Identifikator des Interessenpunktes \\ \hline
	\end{tabular}
\end{table}
\paragraph{Antwort}Die Antwort ist wie folgt aufgebaut:
\begin{table}[H]
	\begin{tabular}{|c|c|c|p{6.5cm}|}
		\hline
		\textbf{Paramtername} & \textbf{Datentyp} & \textbf{Konstante} & \textbf{Kurzbeschreibung}                                                                                               \\ \hline
		result              & string           &                 & Erfolgreich wenn Wert {\glqq ack\grqq} ist \\ \hline
		Code                & int              &                 & Erfolgreich wenn Wert {\glqq 0\grqq} ist \\ \hline
		data                & bool             &                 & Wahr, wenn Nutzer Gast ist \\ \hline
	\end{tabular}
\end{table}
\subsubsection{Abfrage von Statistikdaten}
\paragraph{Kurzbeschreibung}Dieser API-Request wird dazu genutzt um Statistikdaten abzufragen.
\paragraph{Anfrage}Folgende Daten werden zu Anfrage benötigt:
\begin{table}[H]
	\begin{tabular}{|c|c|c|p{6.5cm}|}
		\hline
		\textbf{Paramtername} & \textbf{Datentyp} & \textbf{Konstante} & \textbf{Kurzbeschreibung}                                                                                               \\ \hline
		type                & string            & gsd                & Statistikdaten Abfragen\\ \hline
		data                & array             &                    & Informationen zur Abfrage \\ \hline
	\end{tabular}
\end{table}
\subparagraph{data}Dieses Array enthält Einträge in der nachstehend dargestellten Form haben:
\begin{table}[H]
	\begin{tabular}{|c|c|c|p{6.5cm}|}
		\hline
		\textbf{Paramtername} & \textbf{Datentyp} & \textbf{Konstante} & \textbf{Kurzbeschreibung}    \\ \hline
		data                   & array           &                 & Abfrage von Zeiträumen \\ \hline
		src                    & string          &                 & Quelle der statistischen Informationen \\ \hline
	\end{tabular}
\end{table}
\subparagraph{data}Dieses Array enthält Elemente mit Einträgen in der nachstehend dargestellten Form haben:
\begin{table}[H]
	\begin{tabular}{|c|c|c|p{6.5cm}|}
		\hline
		\textbf{Paramtername} & \textbf{Datentyp} & \textbf{Konstante} & \textbf{Kurzbeschreibung}    \\ \hline
		type                   & char            &                 & Zeiteinheit (D: Tag, W: Week, M: Month, Y:Year) \\ \hline
		Amount                 & int             &                 & Anzahl an Zeiteinheiten \\ \hline
		ID                     & int             &                 & fortlaufender Identifikator \\ \hline
	\end{tabular}
\end{table}
\paragraph{Antwort}Die Antwort ist wie folgt aufgebaut:
\begin{table}[H]
	\begin{tabular}{|c|c|c|p{6.5cm}|}
		\hline
		\textbf{Paramtername} & \textbf{Datentyp} & \textbf{Konstante} & \textbf{Kurzbeschreibung}                                                                                               \\ \hline
		result              & string           &                 & Erfolgreich wenn Wert {\glqq ack\grqq} ist \\ \hline
		Code                & int              &                 & Erfolgreich wenn Wert {\glqq 0\grqq} ist \\ \hline
		data                & array            &                 & Ergebnisse \\ \hline
	\end{tabular}
\end{table}
\subparagraph{data}Dieses Array enthält Elemente mit Einträgen in der nachstehend dargestellten Form haben:
\begin{table}[H]
	\begin{tabular}{|c|c|c|p{6.5cm}|}
		\hline
		\textbf{Paramtername} & \textbf{Datentyp} & \textbf{Konstante} & \textbf{Kurzbeschreibung}    \\ \hline
		type                   & char            &                 & Zeiteinheit (D: Tag, W: Week, M: Month, Y:Year) \\ \hline
		Amount                 & int             &                 & Anzahl an Zeiteinheiten \\ \hline
		ID                     & int             &                 & fortlaufender Identifikator \\ \hline
		data                   & array           &                 & Darstellungsinformationen für Chart.js \\ \hline
	\end{tabular}
\end{table}
\subparagraph{Anmerkungen}Für den Aufbau der Darstellungsinformationen ist die Dokumentation von Chart.js zu konsultieren.

\subsubsection{Geschichte freigeben}
\paragraph{Kurzbeschreibung}Dieser API-Request wird dazu genutzt um eine durch einen Nutzer veröffentlichte Geschichte freizugeben.
\paragraph{Anfrage}Folgende Daten werden zu Anfrage benötigt:
\begin{table}[H]
	\begin{tabular}{|c|c|c|p{6.5cm}|}
		\hline
		\textbf{Paramtername} & \textbf{Datentyp} & \textbf{Konstante} & \textbf{Kurzbeschreibung}                                                                                               \\ \hline
		type                & string            & asa                & Geschichte freigeben \\ \hline
		story\_token        & string            &                    & Identifikator einer Geschichte \\ \hline
	\end{tabular}
\end{table}
\paragraph{Antwort}Die Antwort ist wie folgt aufgebaut:
\begin{table}[H]
	\begin{tabular}{|c|c|c|p{6.5cm}|}
		\hline
		\textbf{Paramtername} & \textbf{Datentyp} & \textbf{Konstante} & \textbf{Kurzbeschreibung}                                                                                               \\ \hline
		result              & string           &                 & Erfolgreich wenn Wert {\glqq ack\grqq} ist \\ \hline
		Code                & int              &                 & Erfolgreich wenn Wert {\glqq 0\grqq} ist \\ \hline
	\end{tabular}
\end{table}
\subsubsection{Geschichte sperren}
\paragraph{Kurzbeschreibung}Dieser API-Request wird dazu genutzt um eine durch einen Nutzer veröffentlichte Geschichte zu sperren.
\paragraph{Anfrage}Folgende Daten werden zu Anfrage benötigt:
\begin{table}[H]
	\begin{tabular}{|c|c|c|p{6.5cm}|}
		\hline
		\textbf{Paramtername} & \textbf{Datentyp} & \textbf{Konstante} & \textbf{Kurzbeschreibung}                                                                                               \\ \hline
		type                & string            & das                & Geschichte sperren \\ \hline
		story\_token        & string            &                    & Identifikator einer Geschichte \\ \hline
	\end{tabular}
\end{table}
\paragraph{Antwort}Die Antwort ist wie folgt aufgebaut:
\begin{table}[H]
	\begin{tabular}{|c|c|c|p{6.5cm}|}
		\hline
		\textbf{Paramtername} & \textbf{Datentyp} & \textbf{Konstante} & \textbf{Kurzbeschreibung}                                                                                               \\ \hline
		result              & string           &                 & Erfolgreich wenn Wert {\glqq ack\grqq} ist \\ \hline
		Code                & int              &                 & Erfolgreich wenn Wert {\glqq 0\grqq} ist \\ \hline
	\end{tabular}
\end{table}

\subsubsection{Namen eines Interessenpunktes abfragen}
\paragraph{Kurzbeschreibung}Dieser API-Request wird dazu genutzt um alle Namen eines Interessenpunktes ab zu fragen.
\paragraph{Anfrage}Folgende Daten werden zu Anfrage benötigt:
\begin{table}[H]
	\begin{tabular}{|c|c|c|p{6.5cm}|}
		\hline
		\textbf{Paramtername} & \textbf{Datentyp} & \textbf{Konstante} & \textbf{Kurzbeschreibung}                                                                                               \\ \hline
		type                & string            & snp                & Namen abfragen \\ \hline
		poi\_id             & int               &                    & Identifikator eines Interessenpunktes \\ \hline
	\end{tabular}
\end{table}
\paragraph{Antwort}Die Antwort ist wie folgt aufgebaut:
\begin{table}[H]
	\begin{tabular}{|c|c|c|p{6.5cm}|}
		\hline
		\textbf{Paramtername} & \textbf{Datentyp} & \textbf{Konstante} & \textbf{Kurzbeschreibung}                                                                                               \\ \hline
		result              & string           &                 & Erfolgreich wenn Wert {\glqq ack\grqq} ist \\ \hline
		Code                & int              &                 & Erfolgreich wenn Wert {\glqq 0\grqq} ist \\ \hline
		data                & array            &                 & Angefragte Informationen \\ \hline
	\end{tabular}
\end{table}
\subparagraph{data}Dieses Array enthält Elemente mit Einträgen in der nachstehend dargestellten Form haben:
\begin{table}[H]
	\begin{tabular}{|c|c|c|p{6.5cm}|}
		\hline
		\textbf{Paramtername} & \textbf{Datentyp} & \textbf{Konstante} & \textbf{Kurzbeschreibung}    \\ \hline
		ID                     & int             &                 & Identifikator des Namens (ab Position >0) \\ \hline
		editable               & bool            &                 & Wahr, wenn Eintrag bearbeitbar (ab Position >0) \\ \hline
		end                    & int             &                 & Ende der Nutzungszeit des Namens \\ \hline
		name                   & string          &                 & Name \\ \hline
		start                  & string          &                 & Anfang der Nutzungszeit des Namens \\ \hline
		validatable            & bool            &                 & Wahr, wenn Eintrag validierbar (ab Position >0) \\ \hline
		deleted                & bool            &                 & Wahr, wenn Eintrag als gelöscht gilt \\ \hline
	\end{tabular}
\end{table}

\subsubsection{Betreiber eines Interessenpunktes abfragen}
\paragraph{Kurzbeschreibung}Dieser API-Request wird dazu genutzt um alle Betreiber eines Interessenpunktes ab zu fragen.
\paragraph{Anfrage}Folgende Daten werden zu Anfrage benötigt:
\begin{table}[H]
	\begin{tabular}{|c|c|c|p{6.5cm}|}
		\hline
		\textbf{Paramtername} & \textbf{Datentyp} & \textbf{Konstante} & \textbf{Kurzbeschreibung}                                                                                               \\ \hline
		type                & string            & sop                & Betreiber abfragen \\ \hline
		poi\_id             & int               &                    & Identifikator eines Interessenpunktes \\ \hline
	\end{tabular}
\end{table}
\paragraph{Antwort}Die Antwort ist wie folgt aufgebaut:
\begin{table}[H]
	\begin{tabular}{|c|c|c|p{6.5cm}|}
		\hline
		\textbf{Paramtername} & \textbf{Datentyp} & \textbf{Konstante} & \textbf{Kurzbeschreibung}                                                                                               \\ \hline
		result              & string           &                 & Erfolgreich wenn Wert {\glqq ack\grqq} ist \\ \hline
		Code                & int              &                 & Erfolgreich wenn Wert {\glqq 0\grqq} ist \\ \hline
		data                & array            &                 & Angefragte Informationen \\ \hline
	\end{tabular}
\end{table}
\subparagraph{data}Dieses Array enthält Elemente mit Einträgen in der nachstehend dargestellten Form haben:
\begin{table}[H]
	\begin{tabular}{|c|c|c|p{6.5cm}|}
		\hline
		\textbf{Paramtername} & \textbf{Datentyp} & \textbf{Konstante} & \textbf{Kurzbeschreibung}    \\ \hline
		ID                     & int             &                 & Identifikator des Betreibers \\ \hline
		editable               & bool            &                 & Wahr, wenn Eintrag bearbeitbar \\ \hline
		end                    & int             &                 & Ende der Nutzungszeit Betreiber \\ \hline
		Operator               & string          &                 & Betreiber \\ \hline
		start                  & string          &                 & Anfang der Nutzungszeit durch Betreiber \\ \hline
		validatable            & bool            &                 & Wahr, wenn Eintrag validierbar \\ \hline
		deleted                & bool            &                 & Wahr, wenn Eintrag als gelöscht gilt \\ \hline
	\end{tabular}
\end{table}

\subsubsection{Historisch Adressen eines Interessenpunktes abfragen}
\paragraph{Kurzbeschreibung}Dieser API-Request wird dazu genutzt um alle historischen Adressen eines Interessenpunktes ab zu fragen.
\paragraph{Anfrage}Folgende Daten werden zu Anfrage benötigt:
\begin{table}[H]
	\begin{tabular}{|c|c|c|p{6.5cm}|}
		\hline
		\textbf{Paramtername} & \textbf{Datentyp} & \textbf{Konstante} & \textbf{Kurzbeschreibung}                                                                                               \\ \hline
		type                & string            & shp                & Historisch Adressen abfragen \\ \hline
		poi\_id             & int               &                    & Identifikator eines Interessenpunktes \\ \hline
	\end{tabular}
\end{table}
\paragraph{Antwort}Die Antwort ist wie folgt aufgebaut:
\begin{table}[H]
	\begin{tabular}{|c|c|c|p{6.5cm}|}
		\hline
		\textbf{Paramtername} & \textbf{Datentyp} & \textbf{Konstante} & \textbf{Kurzbeschreibung}                                                                                               \\ \hline
		result              & string           &                 & Erfolgreich wenn Wert {\glqq ack\grqq} ist \\ \hline
		Code                & int              &                 & Erfolgreich wenn Wert {\glqq 0\grqq} ist \\ \hline
		data                & array            &                 & Angefragte Informationen \\ \hline
	\end{tabular}
\end{table}
\subparagraph{data}Dieses Array enthält Elemente mit Einträgen in der nachstehend dargestellten Form haben:
\begin{table}[H]
	\begin{tabular}{|c|c|c|p{6.5cm}|}
		\hline
		\textbf{Paramtername} & \textbf{Datentyp} & \textbf{Konstante} & \textbf{Kurzbeschreibung}    \\ \hline
		ID                     & int             &                 & Identifikator der historischen Adresse \\ \hline
		editable               & bool            &                 & Wahr, wenn Eintrag bearbeitbar \\ \hline
		end                    & int             &                 & Ende der Nutzungszeit der Adresse \\ \hline
		start                  & string          &                 & Anfang der Nutzungszeit der Adresse \\ \hline
		validatable            & bool            &                 & Wahr, wenn Eintrag validierbar \\ \hline
		City                   & string          &                 & Stadt \\ \hline
		Housenumber            & string          &                 & Hausnummer \\ \hline
		Postalcode             & int             &                 & Postleitzahl \\ \hline
		Streetname             & string          &                 & Straßenname \\ \hline
		deleted                & bool            &                 & Wahr, wenn Eintrag als gelöscht gilt \\ \hline
	\end{tabular}
\end{table}


\subsubsection{Gastabfrage}
\paragraph{Kurzbeschreibung}Dieser API-Request wird dazu genutzt um abzufragen, ob Nutzer Gast ist.
\paragraph{Anfrage}Folgende Daten werden zu Anfrage benötigt:
\begin{table}[H]
	\begin{tabular}{|c|c|c|p{6.5cm}|}
		\hline
		\textbf{Paramtername} & \textbf{Datentyp} & \textbf{Konstante} & \textbf{Kurzbeschreibung}                                                                                               \\ \hline
		type                & string            & gue                & Gastabfrage \\ \hline
	\end{tabular}
\end{table}
\paragraph{Antwort}Die Antwort ist wie folgt aufgebaut:
\begin{table}[H]
	\begin{tabular}{|c|c|c|p{6.5cm}|}
		\hline
		\textbf{Paramtername} & \textbf{Datentyp} & \textbf{Konstante} & \textbf{Kurzbeschreibung}                                                                                               \\ \hline
		result              & string           &                 & Erfolgreich wenn Wert {\glqq ack\grqq} ist \\ \hline
		Code                & int              &                 & Erfolgreich wenn Wert {\glqq 0\grqq} ist \\ \hline
		data                & bool             &                 & Wahr, wenn Nutzer Gast ist \\ \hline
	\end{tabular}
\end{table}

\subsubsection{Saalanzahlen eines Interessenpunktes abfragen}
\paragraph{Kurzbeschreibung}Dieser API-Request wird dazu genutzt um alle Saalanzahlen eines Interessenpunktes ab zu fragen.
\paragraph{Anfrage}Folgende Daten werden zu Anfrage benötigt:
\begin{table}[H]
	\begin{tabular}{|c|c|c|p{6.5cm}|}
		\hline
		\textbf{Paramtername} & \textbf{Datentyp} & \textbf{Konstante} & \textbf{Kurzbeschreibung}                                                                                               \\ \hline
		type                & string            & scp                & Saalanzahlen abfragen \\ \hline
		poi\_id             & int               &                    & Identifikator eines Interessenpunktes \\ \hline
	\end{tabular}
\end{table}
\paragraph{Antwort}Die Antwort ist wie folgt aufgebaut:
\begin{table}[H]
	\begin{tabular}{|c|c|c|p{6.5cm}|}
		\hline
		\textbf{Paramtername} & \textbf{Datentyp} & \textbf{Konstante} & \textbf{Kurzbeschreibung}                                                                                               \\ \hline
		result              & string           &                 & Erfolgreich wenn Wert {\glqq ack\grqq} ist \\ \hline
		Code                & int              &                 & Erfolgreich wenn Wert {\glqq 0\grqq} ist \\ \hline
		data                & array            &                 & Angefragte Informationen \\ \hline
	\end{tabular}
\end{table}
\subparagraph{data}Dieses Array enthält Elemente mit Einträgen in der nachstehend dargestellten Form haben:
\begin{table}[H]
	\begin{tabular}{|c|c|c|p{6.5cm}|}
		\hline
		\textbf{Paramtername} & \textbf{Datentyp} & \textbf{Konstante} & \textbf{Kurzbeschreibung}    \\ \hline
		ID                     & int             &                 & Identifikator der Saalanzahl \\ \hline
		editable               & bool            &                 & Wahr, wenn Eintrag bearbeitbar \\ \hline
		end                    & int             &                 & Ende \\ \hline
		cinemas                & string          &                 & Saalanzahl \\ \hline
		start                  & string          &                 & Anfang \\ \hline
		validatable            & bool            &                 & Wahr, wenn Eintrag validierbar \\ \hline
		deleted                & bool            &                 & Wahr, wenn Eintrag als gelöscht gilt \\ \hline
	\end{tabular}
\end{table}

\subsubsection{Sitzplatzanzahl eines Interessenpunktes abfragen}
\paragraph{Kurzbeschreibung}Dieser API-Request wird dazu genutzt um alle Sitzplatzanzahlen eines Interessenpunktes ab zu fragen.
\paragraph{Anfrage}Folgende Daten werden zu Anfrage benötigt:
\begin{table}[H]
	\begin{tabular}{|c|c|c|p{6.5cm}|}
		\hline
		\textbf{Paramtername} & \textbf{Datentyp} & \textbf{Konstante} & \textbf{Kurzbeschreibung}                                                                                               \\ \hline
		type                & string            & ssp                & Sitzplatzanzahlen abfragen \\ \hline
		poi\_id             & int               &                    & Identifikator eines Interessenpunktes \\ \hline
	\end{tabular}
\end{table}
\paragraph{Antwort}Die Antwort ist wie folgt aufgebaut:
\begin{table}[H]
	\begin{tabular}{|c|c|c|p{6.5cm}|}
		\hline
		\textbf{Paramtername} & \textbf{Datentyp} & \textbf{Konstante} & \textbf{Kurzbeschreibung}                                                                                               \\ \hline
		result              & string           &                 & Erfolgreich wenn Wert {\glqq ack\grqq} ist \\ \hline
		Code                & int              &                 & Erfolgreich wenn Wert {\glqq 0\grqq} ist \\ \hline
		data                & array            &                 & Angefragte Informationen \\ \hline
	\end{tabular}
\end{table}
\subparagraph{data}Dieses Array enthält Elemente mit Einträgen in der nachstehend dargestellten Form haben:
\begin{table}[H]
	\begin{tabular}{|c|c|c|p{6.5cm}|}
		\hline
		\textbf{Paramtername} & \textbf{Datentyp} & \textbf{Konstante} & \textbf{Kurzbeschreibung}    \\ \hline
		ID                     & int             &                 & Identifikator der Sitzplatzanzahl \\ \hline
		editable               & bool            &                 & Wahr, wenn Eintrag bearbeitbar \\ \hline
		end                    & int             &                 & Ende \\ \hline
		seats                  & string          &                 & Sitzplatzanzahl \\ \hline
		start                  & string          &                 & Anfang \\ \hline
		validatable            & bool            &                 & Wahr, wenn Eintrag validierbar \\ \hline
		deleted                & bool            &                 & Wahr, wenn Eintrag als gelöscht gilt \\ \hline
	\end{tabular}
\end{table}

\subsubsection{Geschichtsverknüpfungen eines Interessenpunktes abfragen}
\paragraph{Kurzbeschreibung}Dieser API-Request wird dazu genutzt um alle Geschichtsverknüpfungen eines Interessenpunktes ab zu fragen.
\paragraph{Anfrage}Folgende Daten werden zu Anfrage benötigt:
\begin{table}[H]
	\begin{tabular}{|c|c|c|p{6.5cm}|}
		\hline
		\textbf{Paramtername} & \textbf{Datentyp} & \textbf{Konstante} & \textbf{Kurzbeschreibung}                                                                                               \\ \hline
		type                & string            & slp                & Geschichtsverknüpfungen abfragen \\ \hline
		poi\_id             & int               &                    & Identifikator eines Interessenpunktes \\ \hline
	\end{tabular}
\end{table}
\paragraph{Antwort}Die Antwort ist wie folgt aufgebaut:
\begin{table}[H]
	\begin{tabular}{|c|c|c|p{6.5cm}|}
		\hline
		\textbf{Paramtername} & \textbf{Datentyp} & \textbf{Konstante} & \textbf{Kurzbeschreibung}                                                                                               \\ \hline
		result              & string           &                 & Erfolgreich wenn Wert {\glqq ack\grqq} ist \\ \hline
		Code                & int              &                 & Erfolgreich wenn Wert {\glqq 0\grqq} ist \\ \hline
		data                & array            &                 & Angefragte Informationen \\ \hline
	\end{tabular}
\end{table}
\subparagraph{data}Dieses Array enthält Elemente mit Einträgen in der nachstehend dargestellten Form haben:
\begin{table}[H]
	\begin{tabular}{|c|c|c|p{6.5cm}|}
		\hline
		\textbf{Paramtername} & \textbf{Datentyp} & \textbf{Konstante} & \textbf{Kurzbeschreibung}    \\ \hline
		id                     & int             &                 & Identifikator des Links \\ \hline
		LinkDeletable          & bool            &                 & Wahr, wenn Link löschbar \\ \hline
		LinkValidated          & bool            &                 & Wahr, wenn Link validiert \\ \hline
		name                   & string          &                 & Name des Erstellers \\ \hline
		date                   & timestamp       &                 & Erstellungsdatum \\ \hline
		title                  & string          &                 & Titel der Geschichte \\ \hline
		story                  & string          &                 & Inhalt der Geschichte \\ \hline
		token                  & string          &                 & Identifikator der Geschichte \\ \hline
		validate               & bool            &                 & Validierungsstatus der Geschichte \\ \hline
		deleted                & bool            &                 & Gibt an, ob Link als gelöscht gilt \\ \hline
		restrictions           & bool            &                 & Wahr, wenn Abhängigkeiten als gelöscht gelten \\ \hline
	\end{tabular}
\end{table}

\subsubsection{Liste der unverknüpften Geschichten eines Interessenpunktes }
\paragraph{Kurzbeschreibung}Dieser API-Request wird dazu genutzt um eine Liste der unverknüpften Geschichten eines Interessenpunktes aburagen.
\paragraph{Anfrage}Folgende Daten werden zu Anfrage benötigt:
\begin{table}[H]
	\begin{tabular}{|c|c|c|p{6.5cm}|}
		\hline
		\textbf{Paramtername} & \textbf{Datentyp} & \textbf{Konstante} & \textbf{Kurzbeschreibung}                                                                                               \\ \hline
		type                & string            & gsp                & Liste abfragen \\ \hline
		poi\_id             & int               &                    & Identifikator eines Interessenpunktes \\ \hline
	\end{tabular}
\end{table}
\paragraph{Antwort}Die Antwort ist wie folgt aufgebaut:
\begin{table}[H]
	\begin{tabular}{|c|c|c|p{6.5cm}|}
		\hline
		\textbf{Paramtername} & \textbf{Datentyp} & \textbf{Konstante} & \textbf{Kurzbeschreibung}                                                                                               \\ \hline
		result              & string           &                 & Erfolgreich wenn Wert {\glqq ack\grqq} ist \\ \hline
		Code                & int              &                 & Erfolgreich wenn Wert {\glqq 0\grqq} ist \\ \hline
		data                & array            &                 & Angefragte Informationen \\ \hline
	\end{tabular}
\end{table}
\subparagraph{data}Dieses Array enthält Elemente mit Einträgen in der nachstehend dargestellten Form haben:
\begin{table}[H]
	\begin{tabular}{|c|c|c|p{6.5cm}|}
		\hline
		\textbf{Paramtername} & \textbf{Datentyp} & \textbf{Konstante} & \textbf{Kurzbeschreibung}    \\ \hline
		title                  & string          &                 & Titel der Geschichte \\ \hline
		token                  & string          &                 & Identifikator der Geschichte \\ \hline
	\end{tabular}
\end{table}

\subsubsection{Hauptbild eines Interessenpunktes }
\paragraph{Kurzbeschreibung}Dieser API-Request wird dazu genutzt um das Hauptbild eines Interessenpunktes abzufragen.
\paragraph{Anfrage}Folgende Daten werden zu Anfrage benötigt:
\begin{table}[H]
	\begin{tabular}{|c|c|c|p{6.5cm}|}
		\hline
		\textbf{Paramtername} & \textbf{Datentyp} & \textbf{Konstante} & \textbf{Kurzbeschreibung}                                                                                               \\ \hline
		type                & string            & plp                & Hauptbild abfragen \\ \hline
		poi\_id             & int               &                    & Identifikator eines Interessenpunktes \\ \hline
	\end{tabular}
\end{table}
\paragraph{Antwort}Die Antwort ist wie folgt aufgebaut:
\begin{table}[H]
	\begin{tabular}{|c|c|c|p{6.5cm}|}
		\hline
		\textbf{Paramtername} & \textbf{Datentyp} & \textbf{Konstante} & \textbf{Kurzbeschreibung}                                                                                               \\ \hline
		result              & string           &                 & Erfolgreich wenn Wert {\glqq ack\grqq} ist \\ \hline
		Code                & int              &                 & Erfolgreich wenn Wert {\glqq 0\grqq} ist \\ \hline
		data                & string           &                 & URI des Hauptbildes \\ \hline
		deleted             & bool             &                 & Gibt an, ob das Bild als gelöscht gilt \\ \hline
		source              & string           &                 & Quellenangabe \\ \hline
		sourceType          & string           &                 & Typ der Quelle \\ \hline
	\end{tabular}
\end{table}

\subsubsection{Zusätzliche Bilder eines Interessenpunktes}
\paragraph{Kurzbeschreibung}Dieser API-Request wird dazu genutzt um zusätzliche Bilder eines Interessenpunktes zu laden.
\paragraph{Anfrage}Folgende Daten werden zu Anfrage benötigt:
\begin{table}[H]
	\begin{tabular}{|c|c|c|p{6.5cm}|}
		\hline
		\textbf{Paramtername} & \textbf{Datentyp} & \textbf{Konstante} & \textbf{Kurzbeschreibung}                                                                                               \\ \hline
		type                & string            & apl                & Liste abfragen \\ \hline
		poi\_id             & int               &                    & Identifikator eines Interessenpunktes \\ \hline
	\end{tabular}
\end{table}
\paragraph{Antwort}Die Antwort ist wie folgt aufgebaut:
\begin{table}[H]
	\begin{tabular}{|c|c|c|p{6.5cm}|}
		\hline
		\textbf{Paramtername} & \textbf{Datentyp} & \textbf{Konstante} & \textbf{Kurzbeschreibung}                                                                                               \\ \hline
		result              & string           &                 & Erfolgreich wenn Wert {\glqq ack\grqq} ist \\ \hline
		Code                & int              &                 & Erfolgreich wenn Wert {\glqq 0\grqq} ist \\ \hline
		data                & array            &                 & Angefragte Informationen \\ \hline
	\end{tabular}
\end{table}
\subparagraph{data}Dieses Array enthält Elemente mit Einträgen in der nachstehend dargestellten Form haben:
\begin{table}[H]
	\begin{tabular}{|c|c|c|p{6.5cm}|}
		\hline
		\textbf{Paramtername} & \textbf{Datentyp} & \textbf{Konstante} & \textbf{Kurzbeschreibung}    \\ \hline
		deletable              & bool            &                 & Wahr, wenn Link löschbar \\ \hline
		deleted                & bool            &                 & Wahr, wenn Link als gelöscht gilt \\ \hline
		description            & string          &                 & Bildbeschreibung \\ \hline
		fullsize               & string          &                 & URI des Vollbildes \\ \hline
		id                     & int             &                 & Identifikator des Bildes \\ \hline
		identifier             & string          &                 & zweiter Identifikator des Bildes \\ \hline
		ppid                   & int             &                 & Identifikator des Links \\ \hline
		preview                & string          &                 & URI des Vorschaubildes \\ \hline
		title                  & string          &                 & Titel des Bildes \\ \hline
		token                  & array           &                 & Informationen zum Laden des Bildes \\ \hline
		username               & string          &                 & Nutzername des Erstellers des Bildes \\ \hline
		valUsers               & array           &                 & Array mit Nutzernamen der Validatoren \\ \hline
		validated              & bool            &                 & Validierungsstatus des Links \\ \hline
		validationValue        & bool            &                 & Validierungsstatus des Bildes \\ \hline
		restrictions           & bool            &                 & Wahr, wenn Abhängigkeiten als gelöscht gelten \\ \hline
	\end{tabular}
\end{table}

\subsubsection{Kommentare eines Interessenpunktes}
\paragraph{Kurzbeschreibung}Dieser API-Request wird dazu genutzt um Kommentare eines Interessenpunktes ab zu fragen.
\paragraph{Anfrage}Folgende Daten werden zu Anfrage benötigt:
\begin{table}[H]
	\begin{tabular}{|c|c|c|p{6.5cm}|}
		\hline
		\textbf{Paramtername} & \textbf{Datentyp} & \textbf{Konstante} & \textbf{Kurzbeschreibung}                                                                                               \\ \hline
		type                & string            & lcp                & Kommentare abfragen \\ \hline
		poi\_id             & int               &                    & Identifikator eines Interessenpunktes \\ \hline
	\end{tabular}
\end{table}
\paragraph{Antwort}Die Antwort ist wie folgt aufgebaut:
\begin{table}[H]
	\begin{tabular}{|c|c|c|p{6.5cm}|}
		\hline
		\textbf{Paramtername} & \textbf{Datentyp} & \textbf{Konstante} & \textbf{Kurzbeschreibung}                                                                                               \\ \hline
		result              & string           &                 & Erfolgreich wenn Wert {\glqq ack\grqq} ist \\ \hline
		Code                & int              &                 & Erfolgreich wenn Wert {\glqq 0\grqq} ist \\ \hline
		data                & array            &                 & Angefragte Informationen \\ \hline
	\end{tabular}
\end{table}
\subparagraph{data}Dieses Array Einträge in der nachstehend dargestellten Form haben:
\begin{table}[H]
	\begin{tabular}{|c|c|c|p{6.5cm}|}
		\hline
		\textbf{Paramtername} & \textbf{Datentyp} & \textbf{Konstante} & \textbf{Kurzbeschreibung}    \\ \hline
		deleteComments         & bool            &                 & Wahr, wenn Nutzer Kommentare löschen darf \\ \hline
		poi\_name              & string          &                 & Name des Interessenpunktes \\ \hline
		comments               & array           &                 & Kommentare \\ \hline
	\end{tabular}
\end{table}
\subparagraph{comments}Dieses Array enthält Elemente mit Einträgen in der nachstehend dargestellten Form haben:
\begin{table}[H]
	\begin{tabular}{|c|c|c|p{6.5cm}|}
		\hline
		\textbf{Paramtername} & \textbf{Datentyp} & \textbf{Konstante} & \textbf{Kurzbeschreibung}    \\ \hline
		comment\_id            & int             &                 & Identifikator des Kommentars \\ \hline
		content                & string          &                 & Inhalt des Kommentars \\ \hline
		name                   & string          &                 & Ersteller des Kommentars \\ \hline
		timestamp              & timestamp       &                 & Erstellungszeitpunkt des Kommentars \\ \hline
		deleted                & bool            &                 & Gibt an, ob Kommentar als gelöscht gilt \\ \hline
		deletable              & bool            &                 & Gibt an, ob Kommentar für Nutzer löschbar ist \\ \hline
	\end{tabular}
\end{table}

\subsubsection{Aktivierungsstatus der Biographienfunktion}
\paragraph{Kurzbeschreibung}Dieser API-Request wird dazu genutzt den Status der Biographienfunktion abzufragen.
\paragraph{Anfrage}Folgende Daten werden zu Anfrage benötigt:
\begin{table}[H]
	\begin{tabular}{|c|c|c|p{6.5cm}|}
		\hline
		\textbf{Paramtername} & \textbf{Datentyp} & \textbf{Konstante} & \textbf{Kurzbeschreibung}                                                                                               \\ \hline
		type                & string            & cse                & Kommentare abfragen \\ \hline
	\end{tabular}
\end{table}
\paragraph{Antwort}Die Antwort ist wie folgt aufgebaut:
\begin{table}[H]
	\begin{tabular}{|c|c|c|p{6.5cm}|}
		\hline
		\textbf{Paramtername} & \textbf{Datentyp} & \textbf{Konstante} & \textbf{Kurzbeschreibung}                                                                                               \\ \hline
		result              & string           &                 & Erfolgreich wenn Wert {\glqq ack\grqq} ist \\ \hline
		Code                & int              &                 & Erfolgreich wenn Wert {\glqq 0\grqq} ist \\ \hline
		data                & bool             &                 & Wahr, wenn Biographienfunktion aktiv \\ \hline
	\end{tabular}
\end{table}

\subsubsection{Captcha-Code anfordern}
\paragraph{Kurzbeschreibung}Dieser API-Request wird dazu genutzt einen Captcha-Code anzufordern.
\paragraph{Anfrage}Folgende Daten werden zu Anfrage benötigt:
\begin{table}[H]
	\begin{tabular}{|c|c|c|p{6.5cm}|}
		\hline
		\textbf{Paramtername} & \textbf{Datentyp} & \textbf{Konstante} & \textbf{Kurzbeschreibung}                                                                                               \\ \hline
		type                & string            & cpa                & Captcha-Code anfordern \\ \hline
	\end{tabular}
\end{table}
\paragraph{Antwort}Die Antwort ist wie folgt aufgebaut:
\begin{table}[H]
	\begin{tabular}{|c|c|c|p{6.5cm}|}
		\hline
		\textbf{Paramtername} & \textbf{Datentyp} & \textbf{Konstante} & \textbf{Kurzbeschreibung}                                                                                               \\ \hline
		result              & string           &                 & Erfolgreich wenn Wert {\glqq ack\grqq} ist \\ \hline
		Code                & int              &                 & Erfolgreich wenn Wert {\glqq 0\grqq} ist \\ \hline
		data                & string           &                 & Base64-Codiertes Captcha Bild im JPEG-Format \\ \hline
	\end{tabular}
\end{table}

\subsubsection{Kontaktnachricht senden}
\paragraph{Kurzbeschreibung}Dieser API-Request wird dazu genutzt einen Kontaktnachricht zu versenden.
\paragraph{Anfrage}Folgende Daten werden zu Anfrage benötigt:
\begin{table}[H]
	\begin{tabular}{|c|c|c|p{6.5cm}|}
		\hline
		\textbf{Paramtername} & \textbf{Datentyp} & \textbf{Konstante} & \textbf{Kurzbeschreibung}                                                                                               \\ \hline
		type                & string            & cmg                & Captcha-Code anfordern \\ \hline
		cap                 & string            &                    & Eingabe des gelesenen Captcha-Codes \\ \hline
		email               & string            &                    & Nutzermailadresse (Wahlweise, wenn Nutzer Gast ist) \\ \hline
		msg                 & string            &                    & Mailinhalt \\ \hline
		title               & string            &                    & Betreff der Mail \\ \hline
	\end{tabular}
\end{table}
\paragraph{Antwort}Die Antwort ist wie folgt aufgebaut:
\begin{table}[H]
	\begin{tabular}{|c|c|c|p{6.5cm}|}
		\hline
		\textbf{Paramtername} & \textbf{Datentyp} & \textbf{Konstante} & \textbf{Kurzbeschreibung}                                                                                               \\ \hline
		result              & string           &                 & Erfolgreich wenn Wert {\glqq ack\grqq} ist \\ \hline
		Code                & int              &                 & Erfolgreich wenn Wert {\glqq 0\grqq} ist \\ \hline
	\end{tabular}
\end{table}
\subsubsection{Verknüpfung löschen zwischen Interessenpunkt und Bild}
\paragraph{Kurzbeschreibung}Dieser API-Request wird dazu genutzt um eine Verknüpfung zwischen einem Bild und einem Interessenpunkt zu löschen.
\paragraph{Anfrage}Folgende Daten werden zu Anfrage benötigt:
\begin{table}[H]
	\begin{tabular}{|c|c|c|p{6.5cm}|}
		\hline
		\textbf{Paramtername} & \textbf{Datentyp} & \textbf{Konstante} & \textbf{Kurzbeschreibung}                                                                                               \\ \hline
		type                & string            & fdp                & Link löschen \\ \hline
		IDent               & int               &                    & Identifikator des Links \\ \hline
	\end{tabular}
\end{table}
\paragraph{Antwort}Die Antwort ist wie folgt aufgebaut:
\begin{table}[H]
	\begin{tabular}{|c|c|c|p{6.5cm}|}
		\hline
		\textbf{Paramtername} & \textbf{Datentyp} & \textbf{Konstante} & \textbf{Kurzbeschreibung}                                                                                               \\ \hline
		result              & string           &                 & Erfolgreich wenn Wert {\glqq ack\grqq} ist \\ \hline
		Code                & int              &                 & Erfolgreich wenn Wert {\glqq 0\grqq} ist \\ \hline
	\end{tabular}
\end{table}
\subsubsection{Verknüpfung zwischen Interessenpunkt und Bild wiederherstellen}
\paragraph{Kurzbeschreibung}Dieser API-Request wird dazu genutzt um eine Verknüpfung zwischen einem Bild und einem Interessenpunkt wiederherzustellen.
\paragraph{Anfrage}Folgende Daten werden zu Anfrage benötigt:
\begin{table}[H]
	\begin{tabular}{|c|c|c|p{6.5cm}|}
		\hline
		\textbf{Paramtername} & \textbf{Datentyp} & \textbf{Konstante} & \textbf{Kurzbeschreibung}                                                                                               \\ \hline
		type                & string            & rdp                & Link wiederherstellen \\ \hline
		IDent               & int               &                    & Identifikator des Links \\ \hline
	\end{tabular}
\end{table}
\paragraph{Antwort}Die Antwort ist wie folgt aufgebaut:
\begin{table}[H]
	\begin{tabular}{|c|c|c|p{6.5cm}|}
		\hline
		\textbf{Paramtername} & \textbf{Datentyp} & \textbf{Konstante} & \textbf{Kurzbeschreibung}                                                                                               \\ \hline
		result              & string           &                 & Erfolgreich wenn Wert {\glqq ack\grqq} ist \\ \hline
		Code                & int              &                 & Erfolgreich wenn Wert {\glqq 0\grqq} ist \\ \hline
	\end{tabular}
\end{table}
\subsubsection{Namen eines Interessenpunktes löschen}
\paragraph{Kurzbeschreibung}Dieser API-Request wird dazu genutzt um einen Namen eines Interessenpunktes final zu löschen.
\paragraph{Anfrage}Folgende Daten werden zu Anfrage benötigt:
\begin{table}[H]
	\begin{tabular}{|c|c|c|p{6.5cm}|}
		\hline
		\textbf{Paramtername} & \textbf{Datentyp} & \textbf{Konstante} & \textbf{Kurzbeschreibung}                                                                                               \\ \hline
		type                & string            & fna                & Name löschen \\ \hline
		IDent               & int               &                    & Identifikator des Namen \\ \hline
	\end{tabular}
\end{table}
\paragraph{Antwort}Die Antwort ist wie folgt aufgebaut:
\begin{table}[H]
	\begin{tabular}{|c|c|c|p{6.5cm}|}
		\hline
		\textbf{Paramtername} & \textbf{Datentyp} & \textbf{Konstante} & \textbf{Kurzbeschreibung}                                                                                               \\ \hline
		result              & string           &                 & Erfolgreich wenn Wert {\glqq ack\grqq} ist \\ \hline
		Code                & int              &                 & Erfolgreich wenn Wert {\glqq 0\grqq} ist \\ \hline
	\end{tabular}
\end{table}
\subsubsection{Namen eines Interessenpunktes wiederherstellen}
\paragraph{Kurzbeschreibung}Dieser API-Request wird dazu genutzt um einen Namen eines Interessenpunktes wiederherzustellen.
\paragraph{Anfrage}Folgende Daten werden zu Anfrage benötigt:
\begin{table}[H]
	\begin{tabular}{|c|c|c|p{6.5cm}|}
		\hline
		\textbf{Paramtername} & \textbf{Datentyp} & \textbf{Konstante} & \textbf{Kurzbeschreibung}                                                                                               \\ \hline
		type                & string            & rna                & Name wiederherstellen \\ \hline
		IDent               & int               &                    & Identifikator des Namen \\ \hline
	\end{tabular}
\end{table}
\paragraph{Antwort}Die Antwort ist wie folgt aufgebaut:
\begin{table}[H]
	\begin{tabular}{|c|c|c|p{6.5cm}|}
		\hline
		\textbf{Paramtername} & \textbf{Datentyp} & \textbf{Konstante} & \textbf{Kurzbeschreibung}                                                                                               \\ \hline
		result              & string           &                 & Erfolgreich wenn Wert {\glqq ack\grqq} ist \\ \hline
		Code                & int              &                 & Erfolgreich wenn Wert {\glqq 0\grqq} ist \\ \hline
	\end{tabular}
\end{table}
\subsubsection{Betreiber eines Interessenpunktes löschen}
\paragraph{Kurzbeschreibung}Dieser API-Request wird dazu genutzt um einen Betreiber eines Interessenpunktes final zu löschen.
\paragraph{Anfrage}Folgende Daten werden zu Anfrage benötigt:
\begin{table}[H]
	\begin{tabular}{|c|c|c|p{6.5cm}|}
		\hline
		\textbf{Paramtername} & \textbf{Datentyp} & \textbf{Konstante} & \textbf{Kurzbeschreibung}                                                                                               \\ \hline
		type                & string            & fop                & Betreiber löschen \\ \hline
		IDent               & int               &                    & Identifikator des Betreibers \\ \hline
	\end{tabular}
\end{table}
\paragraph{Antwort}Die Antwort ist wie folgt aufgebaut:
\begin{table}[H]
	\begin{tabular}{|c|c|c|p{6.5cm}|}
		\hline
		\textbf{Paramtername} & \textbf{Datentyp} & \textbf{Konstante} & \textbf{Kurzbeschreibung}                                                                                               \\ \hline
		result              & string           &                 & Erfolgreich wenn Wert {\glqq ack\grqq} ist \\ \hline
		Code                & int              &                 & Erfolgreich wenn Wert {\glqq 0\grqq} ist \\ \hline
	\end{tabular}
\end{table}
\subsubsection{Betreiber eines Interessenpunktes wiederherstellen}
\paragraph{Kurzbeschreibung}Dieser API-Request wird dazu genutzt um einen Betreiber eines Interessenpunktes wiederherzustellen.
\paragraph{Anfrage}Folgende Daten werden zu Anfrage benötigt:
\begin{table}[H]
	\begin{tabular}{|c|c|c|p{6.5cm}|}
		\hline
		\textbf{Paramtername} & \textbf{Datentyp} & \textbf{Konstante} & \textbf{Kurzbeschreibung}                                                                                               \\ \hline
		type                & string            & rop                & Betreiber wiederherstellen \\ \hline
		IDent               & int               &                    & Identifikator des Betreibers \\ \hline
	\end{tabular}
\end{table}
\paragraph{Antwort}Die Antwort ist wie folgt aufgebaut:
\begin{table}[H]
	\begin{tabular}{|c|c|c|p{6.5cm}|}
		\hline
		\textbf{Paramtername} & \textbf{Datentyp} & \textbf{Konstante} & \textbf{Kurzbeschreibung}                                                                                               \\ \hline
		result              & string           &                 & Erfolgreich wenn Wert {\glqq ack\grqq} ist \\ \hline
		Code                & int              &                 & Erfolgreich wenn Wert {\glqq 0\grqq} ist \\ \hline
	\end{tabular}
\end{table}
\subsubsection{Sitzplatzanzahl eines Interessenpunktes löschen}
\paragraph{Kurzbeschreibung}Dieser API-Request wird dazu genutzt um eine Sitzplatzanzahl eines Interessenpunktes final zu löschen.
\paragraph{Anfrage}Folgende Daten werden zu Anfrage benötigt:
\begin{table}[H]
	\begin{tabular}{|c|c|c|p{6.5cm}|}
		\hline
		\textbf{Paramtername} & \textbf{Datentyp} & \textbf{Konstante} & \textbf{Kurzbeschreibung}                                                                                               \\ \hline
		type                & string            & fsc                & Sitzplatzanzahl löschen \\ \hline
		IDent               & int               &                    & Identifikator der Sitzplatzanzahl \\ \hline
	\end{tabular}
\end{table}
\paragraph{Antwort}Die Antwort ist wie folgt aufgebaut:
\begin{table}[H]
	\begin{tabular}{|c|c|c|p{6.5cm}|}
		\hline
		\textbf{Paramtername} & \textbf{Datentyp} & \textbf{Konstante} & \textbf{Kurzbeschreibung}                                                                                               \\ \hline
		result              & string           &                 & Erfolgreich wenn Wert {\glqq ack\grqq} ist \\ \hline
		Code                & int              &                 & Erfolgreich wenn Wert {\glqq 0\grqq} ist \\ \hline
	\end{tabular}
\end{table}
\subsubsection{Sitzplatzanzahl eines Interessenpunktes wiederherstellen}
\paragraph{Kurzbeschreibung}Dieser API-Request wird dazu genutzt um eine Sitzplatzanzahl eines Interessenpunktes wiederherzustellen.
\paragraph{Anfrage}Folgende Daten werden zu Anfrage benötigt:
\begin{table}[H]
	\begin{tabular}{|c|c|c|p{6.5cm}|}
		\hline
		\textbf{Paramtername} & \textbf{Datentyp} & \textbf{Konstante} & \textbf{Kurzbeschreibung}                                                                                               \\ \hline
		type                & string            & rsc                & Sitzplatzanzahl wiederherstellen \\ \hline
		IDent               & int               &                    & Identifikator der Sitzplatzanzahl \\ \hline
	\end{tabular}
\end{table}
\paragraph{Antwort}Die Antwort ist wie folgt aufgebaut:
\begin{table}[H]
	\begin{tabular}{|c|c|c|p{6.5cm}|}
		\hline
		\textbf{Paramtername} & \textbf{Datentyp} & \textbf{Konstante} & \textbf{Kurzbeschreibung}                                                                                               \\ \hline
		result              & string           &                 & Erfolgreich wenn Wert {\glqq ack\grqq} ist \\ \hline
		Code                & int              &                 & Erfolgreich wenn Wert {\glqq 0\grqq} ist \\ \hline
	\end{tabular}
\end{table}
\subsubsection{Saalanzahl eines Interessenpunktes löschen}
\paragraph{Kurzbeschreibung}Dieser API-Request wird dazu genutzt um eine Saalanzahl eines Interessenpunktes final zu löschen.
\paragraph{Anfrage}Folgende Daten werden zu Anfrage benötigt:
\begin{table}[H]
	\begin{tabular}{|c|c|c|p{6.5cm}|}
		\hline
		\textbf{Paramtername} & \textbf{Datentyp} & \textbf{Konstante} & \textbf{Kurzbeschreibung}                                                                                               \\ \hline
		type                & string            & fcc                & Saalanzahl löschen \\ \hline
		IDent               & int               &                    & Identifikator der Saalanzahl \\ \hline
	\end{tabular}
\end{table}
\paragraph{Antwort}Die Antwort ist wie folgt aufgebaut:
\begin{table}[H]
	\begin{tabular}{|c|c|c|p{6.5cm}|}
		\hline
		\textbf{Paramtername} & \textbf{Datentyp} & \textbf{Konstante} & \textbf{Kurzbeschreibung}                                                                                               \\ \hline
		result              & string           &                 & Erfolgreich wenn Wert {\glqq ack\grqq} ist \\ \hline
		Code                & int              &                 & Erfolgreich wenn Wert {\glqq 0\grqq} ist \\ \hline
	\end{tabular}
\end{table}
\subsubsection{Saalanzahl eines Interessenpunktes wiederherstellen}
\paragraph{Kurzbeschreibung}Dieser API-Request wird dazu genutzt um eine Saalanzahl eines Interessenpunktes wiederherzustellen.
\paragraph{Anfrage}Folgende Daten werden zu Anfrage benötigt:
\begin{table}[H]
	\begin{tabular}{|c|c|c|p{6.5cm}|}
		\hline
		\textbf{Paramtername} & \textbf{Datentyp} & \textbf{Konstante} & \textbf{Kurzbeschreibung}                                                                                               \\ \hline
		type                & string            & rcc                & Saalanzahl wiederherstellen \\ \hline
		IDent               & int               &                    & Identifikator der Saalanzahl \\ \hline
	\end{tabular}
\end{table}
\paragraph{Antwort}Die Antwort ist wie folgt aufgebaut:
\begin{table}[H]
	\begin{tabular}{|c|c|c|p{6.5cm}|}
		\hline
		\textbf{Paramtername} & \textbf{Datentyp} & \textbf{Konstante} & \textbf{Kurzbeschreibung}                                                                                               \\ \hline
		result              & string           &                 & Erfolgreich wenn Wert {\glqq ack\grqq} ist \\ \hline
		Code                & int              &                 & Erfolgreich wenn Wert {\glqq 0\grqq} ist \\ \hline
	\end{tabular}
\end{table}
\subsubsection{Historischen Adresse eines Interessenpunktes löschen}
\paragraph{Kurzbeschreibung}Dieser API-Request wird dazu genutzt um eine Historischen Adresse eines Interessenpunktes final zu löschen.
\paragraph{Anfrage}Folgende Daten werden zu Anfrage benötigt:
\begin{table}[H]
	\begin{tabular}{|c|c|c|p{6.5cm}|}
		\hline
		\textbf{Paramtername} & \textbf{Datentyp} & \textbf{Konstante} & \textbf{Kurzbeschreibung}                                                                                               \\ \hline
		type                & string            & fha                & Historischen Adresse löschen \\ \hline
		IDent               & int               &                    & Identifikator der historischen Adresse \\ \hline
	\end{tabular}
\end{table}
\paragraph{Antwort}Die Antwort ist wie folgt aufgebaut:
\begin{table}[H]
	\begin{tabular}{|c|c|c|p{6.5cm}|}
		\hline
		\textbf{Paramtername} & \textbf{Datentyp} & \textbf{Konstante} & \textbf{Kurzbeschreibung}                                                                                               \\ \hline
		result              & string           &                 & Erfolgreich wenn Wert {\glqq ack\grqq} ist \\ \hline
		Code                & int              &                 & Erfolgreich wenn Wert {\glqq 0\grqq} ist \\ \hline
	\end{tabular}
\end{table}
\subsubsection{Historischen Adresse eines Interessenpunktes wiederherstellen}
\paragraph{Kurzbeschreibung}Dieser API-Request wird dazu genutzt um eine Historischen Adresse eines Interessenpunktes wiederherzustellen.
\paragraph{Anfrage}Folgende Daten werden zu Anfrage benötigt:
\begin{table}[H]
	\begin{tabular}{|c|c|c|p{6.5cm}|}
		\hline
		\textbf{Paramtername} & \textbf{Datentyp} & \textbf{Konstante} & \textbf{Kurzbeschreibung}                                                                                               \\ \hline
		type                & string            & rha                & Historischen Adresse wiederherstellen \\ \hline
		IDent               & int               &                    & Identifikator der historischen Adresse \\ \hline
	\end{tabular}
\end{table}
\paragraph{Antwort}Die Antwort ist wie folgt aufgebaut:
\begin{table}[H]
	\begin{tabular}{|c|c|c|p{6.5cm}|}
		\hline
		\textbf{Paramtername} & \textbf{Datentyp} & \textbf{Konstante} & \textbf{Kurzbeschreibung}                                                                                               \\ \hline
		result              & string           &                 & Erfolgreich wenn Wert {\glqq ack\grqq} ist \\ \hline
		Code                & int              &                 & Erfolgreich wenn Wert {\glqq 0\grqq} ist \\ \hline
	\end{tabular}
\end{table}
\subsubsection{Verknüpfung löschen zwischen Interessenpunkt und Geschichte}
\paragraph{Kurzbeschreibung}Dieser API-Request wird dazu genutzt um eine Verknüpfung zwischen einer Geschichte und einem Interessenpunkt zu löschen.
\paragraph{Anfrage}Folgende Daten werden zu Anfrage benötigt:
\begin{table}[H]
	\begin{tabular}{|c|c|c|p{6.5cm}|}
		\hline
		\textbf{Paramtername} & \textbf{Datentyp} & \textbf{Konstante} & \textbf{Kurzbeschreibung}                                                                                               \\ \hline
		type                & string            & fsp                & Link löschen \\ \hline
		IDent               & int               &                    & Identifikator des Links \\ \hline
	\end{tabular}
\end{table}
\paragraph{Antwort}Die Antwort ist wie folgt aufgebaut:
\begin{table}[H]
	\begin{tabular}{|c|c|c|p{6.5cm}|}
		\hline
		\textbf{Paramtername} & \textbf{Datentyp} & \textbf{Konstante} & \textbf{Kurzbeschreibung}                                                                                               \\ \hline
		result              & string           &                 & Erfolgreich wenn Wert {\glqq ack\grqq} ist \\ \hline
		Code                & int              &                 & Erfolgreich wenn Wert {\glqq 0\grqq} ist \\ \hline
	\end{tabular}
\end{table}
\subsubsection{Verknüpfung zwischen Interessenpunkt und Geschichte wiederherstellen}
\paragraph{Kurzbeschreibung}Dieser API-Request wird dazu genutzt um eine Verknüpfung zwischen einer Geschichte und einem Interessenpunkt wiederherzustellen.
\paragraph{Anfrage}Folgende Daten werden zu Anfrage benötigt:
\begin{table}[H]
	\begin{tabular}{|c|c|c|p{6.5cm}|}
		\hline
		\textbf{Paramtername} & \textbf{Datentyp} & \textbf{Konstante} & \textbf{Kurzbeschreibung}                                                                                               \\ \hline
		type                & string            & rsp                & Link wiederherstellen \\ \hline
		IDent               & int               &                    & Identifikator des Links \\ \hline
	\end{tabular}
\end{table}
\paragraph{Antwort}Die Antwort ist wie folgt aufgebaut:
\begin{table}[H]
	\begin{tabular}{|c|c|c|p{6.5cm}|}
		\hline
		\textbf{Paramtername} & \textbf{Datentyp} & \textbf{Konstante} & \textbf{Kurzbeschreibung}                                                                                               \\ \hline
		result              & string           &                 & Erfolgreich wenn Wert {\glqq ack\grqq} ist \\ \hline
		Code                & int              &                 & Erfolgreich wenn Wert {\glqq 0\grqq} ist \\ \hline
	\end{tabular}
\end{table}
\subsubsection{Kommentar eines Interessenpunktes löschen}
\paragraph{Kurzbeschreibung}Dieser API-Request wird dazu genutzt um einen Kommentar eines Interessenpunktes final zu löschen.
\paragraph{Anfrage}Folgende Daten werden zu Anfrage benötigt:
\begin{table}[H]
	\begin{tabular}{|c|c|c|p{6.5cm}|}
		\hline
		\textbf{Paramtername} & \textbf{Datentyp} & \textbf{Konstante} & \textbf{Kurzbeschreibung}                                                                                               \\ \hline
		type                & string            & fcp                & Kommentar löschen \\ \hline
		IDent               & int               &                    & Identifikator des Kommentars \\ \hline
	\end{tabular}
\end{table}
\paragraph{Antwort}Die Antwort ist wie folgt aufgebaut:
\begin{table}[H]
	\begin{tabular}{|c|c|c|p{6.5cm}|}
		\hline
		\textbf{Paramtername} & \textbf{Datentyp} & \textbf{Konstante} & \textbf{Kurzbeschreibung}                                                                                               \\ \hline
		result              & string           &                 & Erfolgreich wenn Wert {\glqq ack\grqq} ist \\ \hline
		Code                & int              &                 & Erfolgreich wenn Wert {\glqq 0\grqq} ist \\ \hline
	\end{tabular}
\end{table}
\subsubsection{Kommentar eines Interessenpunktes wiederherstellen}
\paragraph{Kurzbeschreibung}Dieser API-Request wird dazu genutzt um einen Kommentar eines Interessenpunktes wiederherzustellen.
\paragraph{Anfrage}Folgende Daten werden zu Anfrage benötigt:
\begin{table}[H]
	\begin{tabular}{|c|c|c|p{6.5cm}|}
		\hline
		\textbf{Paramtername} & \textbf{Datentyp} & \textbf{Konstante} & \textbf{Kurzbeschreibung}                                                                                               \\ \hline
		type                & string            & rcp                & Kommentar wiederherstellen \\ \hline
		IDent               & int               &                    & Identifikator des Kommentars \\ \hline
	\end{tabular}
\end{table}
\paragraph{Antwort}Die Antwort ist wie folgt aufgebaut:
\begin{table}[H]
	\begin{tabular}{|c|c|c|p{6.5cm}|}
		\hline
		\textbf{Paramtername} & \textbf{Datentyp} & \textbf{Konstante} & \textbf{Kurzbeschreibung}                                                                                               \\ \hline
		result              & string           &                 & Erfolgreich wenn Wert {\glqq ack\grqq} ist \\ \hline
		Code                & int              &                 & Erfolgreich wenn Wert {\glqq 0\grqq} ist \\ \hline
	\end{tabular}
\end{table}
\subsubsection{Interessenpunkt löschen}
\paragraph{Kurzbeschreibung}Dieser API-Request wird dazu genutzt um einen Interessenpunkt final zu löschen.
\paragraph{Anfrage}Folgende Daten werden zu Anfrage benötigt:
\begin{table}[H]
	\begin{tabular}{|c|c|c|p{6.5cm}|}
		\hline
		\textbf{Paramtername} & \textbf{Datentyp} & \textbf{Konstante} & \textbf{Kurzbeschreibung}                                                                                               \\ \hline
		type                & string            & fpi                & Interessenpunkt löschen \\ \hline
		IDent               & int               &                    & Identifikator des Interessenpunktes \\ \hline
	\end{tabular}
\end{table}
\paragraph{Antwort}Die Antwort ist wie folgt aufgebaut:
\begin{table}[H]
	\begin{tabular}{|c|c|c|p{6.5cm}|}
		\hline
		\textbf{Paramtername} & \textbf{Datentyp} & \textbf{Konstante} & \textbf{Kurzbeschreibung}                                                                                               \\ \hline
		result              & string           &                 & Erfolgreich wenn Wert {\glqq ack\grqq} ist \\ \hline
		Code                & int              &                 & Erfolgreich wenn Wert {\glqq 0\grqq} ist \\ \hline
	\end{tabular}
\end{table}
\subsubsection{Interessenpunkt wiederherstellen}
\paragraph{Kurzbeschreibung}Dieser API-Request wird dazu genutzt um einen Interessenpunkt wiederherzustellen.
\paragraph{Anfrage}Folgende Daten werden zu Anfrage benötigt:
\begin{table}[H]
	\begin{tabular}{|c|c|c|p{6.5cm}|}
		\hline
		\textbf{Paramtername} & \textbf{Datentyp} & \textbf{Konstante} & \textbf{Kurzbeschreibung}                                                                                               \\ \hline
		type                & string            & rpi                & Interessenpunkt wiederherstellen \\ \hline
		IDent               & int               &                    & Identifikator des Interessenpunktes \\ \hline
	\end{tabular}
\end{table}
\paragraph{Antwort}Die Antwort ist wie folgt aufgebaut:
\begin{table}[H]
	\begin{tabular}{|c|c|c|p{6.5cm}|}
		\hline
		\textbf{Paramtername} & \textbf{Datentyp} & \textbf{Konstante} & \textbf{Kurzbeschreibung}                                                                                               \\ \hline
		result              & string           &                 & Erfolgreich wenn Wert {\glqq ack\grqq} ist \\ \hline
		Code                & int              &                 & Erfolgreich wenn Wert {\glqq 0\grqq} ist \\ \hline
	\end{tabular}
\end{table}

\subsubsection{Geschichte löschen}
\paragraph{Kurzbeschreibung}Dieser API-Request wird dazu genutzt um eine Geschichte final zu löschen.
\paragraph{Anfrage}Folgende Daten werden zu Anfrage benötigt:
\begin{table}[H]
	\begin{tabular}{|c|c|c|p{6.5cm}|}
		\hline
		\textbf{Paramtername} & \textbf{Datentyp} & \textbf{Konstante} & \textbf{Kurzbeschreibung}                                                                                               \\ \hline
		type                & string            & fst               & Geschichte löschen \\ \hline
		IDent               & string            &                   & Identifikator der Geschichte \\ \hline
	\end{tabular}
\end{table}
\paragraph{Antwort}Die Antwort ist wie folgt aufgebaut:
\begin{table}[H]
	\begin{tabular}{|c|c|c|p{6.5cm}|}
		\hline
		\textbf{Paramtername} & \textbf{Datentyp} & \textbf{Konstante} & \textbf{Kurzbeschreibung}                                                                                               \\ \hline
		result              & string           &                 & Erfolgreich wenn Wert {\glqq ack\grqq} ist \\ \hline
		Code                & int              &                 & Erfolgreich wenn Wert {\glqq 0\grqq} ist \\ \hline
	\end{tabular}
\end{table}
\subsubsection{Geschichte wiederherstellen}
\paragraph{Kurzbeschreibung}Dieser API-Request wird dazu genutzt um eine Geschichte wiederherzustellen.
\paragraph{Anfrage}Folgende Daten werden zu Anfrage benötigt:
\begin{table}[H]
	\begin{tabular}{|c|c|c|p{6.5cm}|}
		\hline
		\textbf{Paramtername} & \textbf{Datentyp} & \textbf{Konstante} & \textbf{Kurzbeschreibung}                                                                                               \\ \hline
		type                & string            & rst                & Geschichte wiederherstellen \\ \hline
		IDent               & string            &                    & Identifikator der Geschichte \\ \hline
	\end{tabular}
\end{table}
\paragraph{Antwort}Die Antwort ist wie folgt aufgebaut:
\begin{table}[H]
	\begin{tabular}{|c|c|c|p{6.5cm}|}
		\hline
		\textbf{Paramtername} & \textbf{Datentyp} & \textbf{Konstante} & \textbf{Kurzbeschreibung}                                                                                               \\ \hline
		result              & string           &                 & Erfolgreich wenn Wert {\glqq ack\grqq} ist \\ \hline
		Code                & int              &                 & Erfolgreich wenn Wert {\glqq 0\grqq} ist \\ \hline
	\end{tabular}
\end{table}
\subsubsection{Ankündigung Hinzufügen}
\paragraph{Kurzbeschreibung}Dieser API-Request wird dazu genutzt um eine Ankündigung an zu legen.
\paragraph{Anfrage}Folgende Daten werden zu Anfrage benötigt:
\begin{table}[H]
	\begin{tabular}{|c|c|c|p{6.5cm}|}
		\hline
		\textbf{Paramtername} & \textbf{Datentyp} & \textbf{Konstante} & \textbf{Kurzbeschreibung}                                                                                               \\ \hline
		type                & string            & aan                & Ankündigung Hinzufügen \\ \hline
		title               & string            &                    & Titel der Ankündigung \\ \hline
		content             & string            &                    & Inhalt der Ankündigung \\ \hline
		start               & string            &                    & Starttag der Ankündigung \\ \hline
		end                 & string            &                    & Endtag  der Ankündigung \\ \hline
	\end{tabular}
\end{table}
\paragraph{Antwort}Die Antwort ist wie folgt aufgebaut:
\begin{table}[H]
	\begin{tabular}{|c|c|c|p{6.5cm}|}
		\hline
		\textbf{Paramtername} & \textbf{Datentyp} & \textbf{Konstante} & \textbf{Kurzbeschreibung}                                                                                               \\ \hline
		result              & string           &                 & Erfolgreich wenn Wert {\glqq ack\grqq} ist \\ \hline
		Code                & int              &                 & Erfolgreich wenn Wert {\glqq 0\grqq} ist \\ \hline
	\end{tabular}
\end{table}
\subsubsection{Ankündigung Abfragen}
\paragraph{Kurzbeschreibung}Dieser API-Request wird dazu genutzt um eine Ankündigung abzufragen.
\paragraph{Anfrage}Folgende Daten werden zu Anfrage benötigt:
\begin{table}[H]
	\begin{tabular}{|c|c|c|p{6.5cm}|}
		\hline
		\textbf{Paramtername} & \textbf{Datentyp} & \textbf{Konstante} & \textbf{Kurzbeschreibung}                                                                                               \\ \hline
		type                & string            & gan                & Ankündigung Abfragen \\ \hline
		id                  & id                &                    & Identifikator der Ankündigung \\ \hline
	\end{tabular}
\end{table}
\paragraph{Antwort}Die Antwort ist wie folgt aufgebaut:
\begin{table}[H]
	\begin{tabular}{|c|c|c|p{6.5cm}|}
		\hline
		\textbf{Paramtername} & \textbf{Datentyp} & \textbf{Konstante} & \textbf{Kurzbeschreibung}                                                                                               \\ \hline
		result              & string           &                 & Erfolgreich wenn Wert {\glqq ack\grqq} ist \\ \hline
		Code                & int              &                 & Erfolgreich wenn Wert {\glqq 0\grqq} ist \\ \hline
		data                & array            &                 & Strukturierte Daten \\ \hline
	\end{tabular}
\end{table}
\subparagraph{data}Dieses Array enthält Einträge in der nachstehend dargestellten Form haben:
\begin{table}[H]
	\begin{tabular}{|c|c|c|p{6.5cm}|}
		\hline
		\textbf{Paramtername} & \textbf{Datentyp} & \textbf{Konstante} & \textbf{Kurzbeschreibung}    \\ \hline
		id      & int               &                 & Identifikator der Ankündigung\\ \hline
		title   & string            &                 & Titel der Ankündigung \\ \hline
		content & string            &                 & Inhalt der Ankündigung \\ \hline
		start   & string            &                 & Startzeitpunkt der Ankündigung \\ \hline
		end     & string            &                 & Endzeit der Ankündigung \\ \hline
	\end{tabular}
\end{table}
\subsubsection{Ankündigung Ändern}
\paragraph{Kurzbeschreibung}Dieser API-Request wird dazu genutzt um eine Ankündigung zu ändern.
\paragraph{Anfrage}Folgende Daten werden zu Anfrage benötigt:
\begin{table}[H]
	\begin{tabular}{|c|c|c|p{6.5cm}|}
		\hline
		\textbf{Paramtername} & \textbf{Datentyp} & \textbf{Konstante} & \textbf{Kurzbeschreibung}                                                                                               \\ \hline
		type                & string            & uan                & Ankündigung ändern \\ \hline
		title               & string            &                    & Titel der Ankündigung \\ \hline
		content             & string            &                    & Inhalt der Ankündigung \\ \hline
		start               & string            &                    & Starttag der Ankündigung \\ \hline
		end                 & string            &                    & Endtag  der Ankündigung \\ \hline
		id                  & int               &                    & Identifikator der Ankündigung \\ \hline
	\end{tabular}
\end{table}
\paragraph{Antwort}Die Antwort ist wie folgt aufgebaut:
\begin{table}[H]
	\begin{tabular}{|c|c|c|p{6.5cm}|}
		\hline
		\textbf{Paramtername} & \textbf{Datentyp} & \textbf{Konstante} & \textbf{Kurzbeschreibung}                                                                                               \\ \hline
		result              & string           &                 & Erfolgreich wenn Wert {\glqq ack\grqq} ist \\ \hline
		Code                & int              &                 & Erfolgreich wenn Wert {\glqq 0\grqq} ist \\ \hline
	\end{tabular}
\end{table}
\subsubsection{Ankündigung Löschen}
\paragraph{Kurzbeschreibung}Dieser API-Request wird dazu genutzt um eine Ankündigung abzufragen.
\paragraph{Anfrage}Folgende Daten werden zu Anfrage benötigt:
\begin{table}[H]
	\begin{tabular}{|c|c|c|p{6.5cm}|}
		\hline
		\textbf{Paramtername} & \textbf{Datentyp} & \textbf{Konstante} & \textbf{Kurzbeschreibung}                                                                                               \\ \hline
		type                & string            & dan                & Ankündigung Löschen \\ \hline
		id                  & id                &                    & Identifikator der Ankündigung \\ \hline
	\end{tabular}
\end{table}
\paragraph{Antwort}Die Antwort ist wie folgt aufgebaut:
\begin{table}[H]
	\begin{tabular}{|c|c|c|p{6.5cm}|}
		\hline
		\textbf{Paramtername} & \textbf{Datentyp} & \textbf{Konstante} & \textbf{Kurzbeschreibung}                                                                                               \\ \hline
		result              & string           &                 & Erfolgreich wenn Wert {\glqq ack\grqq} ist \\ \hline
		Code                & int              &                 & Erfolgreich wenn Wert {\glqq 0\grqq} ist \\ \hline
	\end{tabular}
\end{table}
\subsubsection{Aktuelle Ankündigungen abfragen}
\paragraph{Kurzbeschreibung}Dieser API-Request wird dazu genutzt um die aktuellen Ankündigungen abzufragen.
\paragraph{Anfrage}Folgende Daten werden zu Anfrage benötigt:
\begin{table}[H]
	\begin{tabular}{|c|c|c|p{6.5cm}|}
		\hline
		\textbf{Paramtername} & \textbf{Datentyp} & \textbf{Konstante} & \textbf{Kurzbeschreibung}                                                                                               \\ \hline
		type                & string            & gca                & Ankündigung Löschen \\ \hline
	\end{tabular}
\end{table}
\paragraph{Antwort}Die Antwort ist wie folgt aufgebaut:
\begin{table}[H]
	\begin{tabular}{|c|c|c|p{6.5cm}|}
		\hline
		\textbf{Paramtername} & \textbf{Datentyp} & \textbf{Konstante} & \textbf{Kurzbeschreibung}                                                                                               \\ \hline
		result              & string           &                 & Erfolgreich wenn Wert {\glqq ack\grqq} ist \\ \hline
		Code                & int              &                 & Erfolgreich wenn Wert {\glqq 0\grqq} ist \\ \hline
		data                & array            &                 & Strukturierte Daten \\ \hline
	\end{tabular}
\end{table}
\subparagraph{data}Dieses Array enthält Einträge in der nachstehend dargestellten Form haben:
\begin{table}[H]
	\begin{tabular}{|c|c|c|p{6.5cm}|}
		\hline
		\textbf{Paramtername} & \textbf{Datentyp} & \textbf{Konstante} & \textbf{Kurzbeschreibung}    \\ \hline
		id      & int               &                 & Identifikator der Ankündigung\\ \hline
		title   & string            &                 & Titel der Ankündigung \\ \hline
		content & string            &                 & Inhalt der Ankündigung \\ \hline
	\end{tabular}
\end{table}
\subsubsection{Aktuelle Ankündigungen aktivieren}
\paragraph{Kurzbeschreibung}Dieser API-Request wird dazu genutzt um die aktuellen Ankündigungen abzufragen.
\paragraph{Anfrage}Folgende Daten werden zu Anfrage benötigt:
\begin{table}[H]
	\begin{tabular}{|c|c|c|p{6.5cm}|}
		\hline
		\textbf{Paramtername} & \textbf{Datentyp} & \textbf{Konstante} & \textbf{Kurzbeschreibung}                                                                                               \\ \hline
		type                & string            & saa                & Ankündigung Löschen \\ \hline
		id                  & int               &                    & Identifikator der Ankündigung \\ \hline
	\end{tabular}
\end{table}
\paragraph{Antwort}Die Antwort ist wie folgt aufgebaut:
\begin{table}[H]
	\begin{tabular}{|c|c|c|p{6.5cm}|}
		\hline
		\textbf{Paramtername} & \textbf{Datentyp} & \textbf{Konstante} & \textbf{Kurzbeschreibung}                                                                                               \\ \hline
		result              & string           &                 & Erfolgreich wenn Wert {\glqq ack\grqq} ist \\ \hline
		Code                & int              &                 & Erfolgreich wenn Wert {\glqq 0\grqq} ist \\ \hline
	\end{tabular}
\end{table}
\subsubsection{Quelle zu Interessenpunkt hinzufügen}
\paragraph{Kurzbeschreibung}Dieser API-Request wird dazu genutzt um einem Interessenpunkt eine neue Quelle hinzuzufügen.
\paragraph{Anfrage}Folgende Daten werden zu Anfrage benötigt:
\begin{table}[H]
	\begin{tabular}{|c|c|c|p{6.5cm}|}
		\hline
		\textbf{Paramtername} & \textbf{Datentyp} & \textbf{Konstante} & \textbf{Kurzbeschreibung}                                                                                               \\ \hline
		type                & string            & asp                & Quelle hinzufügen \\ \hline
		typeSource          & int               &                    & Indentifikator des Typs der Quelle \\ \hline
		source              & string            &                    & Quellenangabe \\ \hline
		relation            & int               &                    & Identifikator des Informationsbezugs der Quelle \\ \hline
		poiid               & int               &                    & Identifikator des Interessenpunktes \\ \hline
	\end{tabular}
\end{table}
\paragraph{Antwort}Die Antwort ist wie folgt aufgebaut:
\begin{table}[H]
	\begin{tabular}{|c|c|c|p{6.5cm}|}
		\hline
		\textbf{Paramtername} & \textbf{Datentyp} & \textbf{Konstante} & \textbf{Kurzbeschreibung}                                                                                               \\ \hline
		result              & string           &                 & Erfolgreich wenn Wert {\glqq ack\grqq} ist \\ \hline
		Code                & int              &                 & Erfolgreich wenn Wert {\glqq 0\grqq} ist \\ \hline
	\end{tabular}
\end{table}
\subsubsection{Quelle zu Interessenpunkt abrufen}
\paragraph{Kurzbeschreibung}Dieser API-Request wird dazu genutzt um einem Interessenpunkt eine neue Quelle hinzuzufügen.
\paragraph{Anfrage}Folgende Daten werden zu Anfrage benötigt:
\begin{table}[H]
	\begin{tabular}{|c|c|c|p{6.5cm}|}
		\hline
		\textbf{Paramtername} & \textbf{Datentyp} & \textbf{Konstante} & \textbf{Kurzbeschreibung}                                                                                               \\ \hline
		type                & string            & grp                & Ankündigung Löschen \\ \hline
		poiid               & int               &                    & Identifikator des Interessenpunktes \\ \hline
	\end{tabular}
\end{table}
\paragraph{Antwort}Die Antwort ist wie folgt aufgebaut:
\begin{table}[H]
	\begin{tabular}{|c|c|c|p{6.5cm}|}
		\hline
		\textbf{Paramtername} & \textbf{Datentyp} & \textbf{Konstante} & \textbf{Kurzbeschreibung}                                                                                               \\ \hline
		result              & string           &                 & Erfolgreich wenn Wert {\glqq ack\grqq} ist \\ \hline
		code                & int              &                 & Erfolgreich wenn Wert {\glqq 0\grqq} ist \\ \hline
		data                & array            &                 & Strukturierte angeforderte Daten \\ \hline
	\end{tabular}
\end{table}
\subparagraph{data}Dieses Array enthält Einträge in der nachstehend dargestellten Form haben:
\begin{table}[H]
	\begin{tabular}{|c|c|c|p{6.5cm}|}
		\hline
		\textbf{Paramtername} & \textbf{Datentyp} & \textbf{Konstante} & \textbf{Kurzbeschreibung}    \\ \hline
		id         & int               &                 & Identifikator der Quelle \\ \hline
		type       & string            &                 & Typ der Quelle \\ \hline
		typeid     & int               &                 & Identifikator des Typs der Quelle \\ \hline
		source     & string            &                 & Inhalt der Ankündigung \\ \hline
		relation   & string            &                 & Bezug der Quelle \\ \hline
		relationid & int               &                 & Identifikator des Bezugs der Quelle \\ \hline
		editable   & boolean           &                 & Nutzer kann Quellenangabe ändern \\ \hline
		deleted    & boolean           &                 & gibt an, ob Quelle als gelöscht zählt \\ \hline
	\end{tabular}
\end{table}
\subsubsection{Alle Quellenbezüge abfragen}
\paragraph{Kurzbeschreibung}Dieser API-Request wird dazu genutzt um alle Bezüge von Quellen abzufragen.
\paragraph{Anfrage}Folgende Daten werden zu Anfrage benötigt:
\begin{table}[H]
	\begin{tabular}{|c|c|c|p{6.5cm}|}
		\hline
		\textbf{Paramtername} & \textbf{Datentyp} & \textbf{Konstante} & \textbf{Kurzbeschreibung}                                                                                               \\ \hline
		type                & string            & grs                & Quellenbeziehungen abfragen \\ \hline
	\end{tabular}
\end{table}
\paragraph{Antwort}Die Antwort ist wie folgt aufgebaut:
\begin{table}[H]
	\begin{tabular}{|c|c|c|p{6.5cm}|}
		\hline
		\textbf{Paramtername} & \textbf{Datentyp} & \textbf{Konstante} & \textbf{Kurzbeschreibung}                                                                                               \\ \hline
		result              & string           &                 & Erfolgreich wenn Wert {\glqq ack\grqq} ist \\ \hline
		code                & int              &                 & Erfolgreich wenn Wert {\glqq 0\grqq} ist \\ \hline
		data                & array            &                 & Strukturierte angeforderte Daten \\ \hline
	\end{tabular}
\end{table}
\subparagraph{data}Dieses Array enthält Einträge in der nachstehend dargestellten Form haben:
\begin{table}[H]
	\begin{tabular}{|c|c|c|p{6.5cm}|}
		\hline
		\textbf{Paramtername} & \textbf{Datentyp} & \textbf{Konstante} & \textbf{Kurzbeschreibung}    \\ \hline
		id         & int               &                 & Identifikator des Bezugs der Quelle \\ \hline
		name       & string            &                 & Name des Bezugs der Quelle \\ \hline
	\end{tabular}
\end{table}
\subsubsection{Alle Quellentypen abfragen}
\paragraph{Kurzbeschreibung}Dieser API-Request wird dazu genutzt um alle Typen von Quellen abzufragen.
\paragraph{Anfrage}Folgende Daten werden zu Anfrage benötigt:
\begin{table}[H]
	\begin{tabular}{|c|c|c|p{6.5cm}|}
		\hline
		\textbf{Paramtername} & \textbf{Datentyp} & \textbf{Konstante} & \textbf{Kurzbeschreibung}                                                                                               \\ \hline
		type                & string            & gts                & Quellenbeziehungen abfragen \\ \hline
	\end{tabular}
\end{table}
\paragraph{Antwort}Die Antwort ist wie folgt aufgebaut:
\begin{table}[H]
	\begin{tabular}{|c|c|c|p{6.5cm}|}
		\hline
		\textbf{Paramtername} & \textbf{Datentyp} & \textbf{Konstante} & \textbf{Kurzbeschreibung}                                                                                               \\ \hline
		result              & string           &                 & Erfolgreich wenn Wert {\glqq ack\grqq} ist \\ \hline
		code                & int              &                 & Erfolgreich wenn Wert {\glqq 0\grqq} ist \\ \hline
		data                & array            &                 & Strukturierte angeforderte Daten \\ \hline
	\end{tabular}
\end{table}
\subparagraph{data}Dieses Array enthält Einträge in der nachstehend dargestellten Form haben:
\begin{table}[H]
	\begin{tabular}{|c|c|c|p{6.5cm}|}
		\hline
		\textbf{Paramtername} & \textbf{Datentyp} & \textbf{Konstante} & \textbf{Kurzbeschreibung}    \\ \hline
		id         & int               &                 & Identifikator des Bezugs der Quelle \\ \hline
		name       & string            &                 & Name des Bezugs der Quelle \\ \hline
	\end{tabular}
\end{table}
\subsubsection{Quelleneintrag ändern}
\paragraph{Kurzbeschreibung}Dieser API-Request wird dazu genutzt um eine Quelle zu ändern.
\paragraph{Anfrage}Folgende Daten werden zu Anfrage benötigt:
\begin{table}[H]
	\begin{tabular}{|c|c|c|p{6.5cm}|}
		\hline
		\textbf{Paramtername} & \textbf{Datentyp} & \textbf{Konstante} & \textbf{Kurzbeschreibung}                                                                                               \\ \hline
		type                & string            & usp                & Quelleneintrag ändern \\ \hline
		id                  & int               &                    & Identifikator der Quelle \\ \hline
		typeSource          & int               &                    & Identifikator des Typs der Quelle \\ \hline
		source              & string            &                    & Quellenangabe \\ \hline
		relation            & int               &                    & Identifikator des Informationsbezugs der Quelle \\ \hline
	\end{tabular}
\end{table}
\paragraph{Antwort}Die Antwort ist wie folgt aufgebaut:
\begin{table}[H]
	\begin{tabular}{|c|c|c|p{6.5cm}|}
		\hline
		\textbf{Paramtername} & \textbf{Datentyp} & \textbf{Konstante} & \textbf{Kurzbeschreibung}                                                                                               \\ \hline
		result              & string           &                 & Erfolgreich wenn Wert {\glqq ack\grqq} ist \\ \hline
		Code                & int              &                 & Erfolgreich wenn Wert {\glqq 0\grqq} ist \\ \hline
	\end{tabular}
\end{table}
\subsubsection{Quelleneintrag löschen}
\paragraph{Kurzbeschreibung}Dieser API-Request wird dazu genutzt um eine Quelle zu löschen oder als gelöscht zu markieren.
\paragraph{Anfrage}Folgende Daten werden zu Anfrage benötigt:
\begin{table}[H]
	\begin{tabular}{|c|c|c|p{6.5cm}|}
		\hline
		\textbf{Paramtername} & \textbf{Datentyp} & \textbf{Konstante} & \textbf{Kurzbeschreibung}                                                                                               \\ \hline
		type                & string            & des                & Quelleneintrag löschen \\ \hline
		id                  & int               &                    & Identifikator der Quelle \\ \hline
	\end{tabular}
\end{table}
\paragraph{Antwort}Die Antwort ist wie folgt aufgebaut:
\begin{table}[H]
	\begin{tabular}{|c|c|c|p{6.5cm}|}
		\hline
		\textbf{Paramtername} & \textbf{Datentyp} & \textbf{Konstante} & \textbf{Kurzbeschreibung}                                                                                               \\ \hline
		result              & string           &                 & Erfolgreich wenn Wert {\glqq ack\grqq} ist \\ \hline
		Code                & int              &                 & Erfolgreich wenn Wert {\glqq 0\grqq} ist \\ \hline
	\end{tabular}
\end{table}
\subsubsection{Quelleneintrag endgültig löschen}
\paragraph{Kurzbeschreibung}Dieser API-Request wird dazu genutzt um eine Quelle zu endgültig zu löschen.
\paragraph{Anfrage}Folgende Daten werden zu Anfrage benötigt:
\begin{table}[H]
	\begin{tabular}{|c|c|c|p{6.5cm}|}
		\hline
		\textbf{Paramtername} & \textbf{Datentyp} & \textbf{Konstante} & \textbf{Kurzbeschreibung}                                                                                               \\ \hline
		type                & string            & fds                & Quelleneintrag endgültig löschen \\ \hline
		id                  & int               &                    & Identifikator der Quelle \\ \hline
	\end{tabular}
\end{table}
\paragraph{Antwort}Die Antwort ist wie folgt aufgebaut:
\begin{table}[H]
	\begin{tabular}{|c|c|c|p{6.5cm}|}
		\hline
		\textbf{Paramtername} & \textbf{Datentyp} & \textbf{Konstante} & \textbf{Kurzbeschreibung}                                                                                               \\ \hline
		result              & string           &                 & Erfolgreich wenn Wert {\glqq ack\grqq} ist \\ \hline
		Code                & int              &                 & Erfolgreich wenn Wert {\glqq 0\grqq} ist \\ \hline
	\end{tabular}
\end{table}
\subsubsection{Quelleneintrag Wiederherstellen}
\paragraph{Kurzbeschreibung}Dieser API-Request wird dazu genutzt um eine Quelle wiederherzustellen.
\paragraph{Anfrage}Folgende Daten werden zu Anfrage benötigt:
\begin{table}[H]
	\begin{tabular}{|c|c|c|p{6.5cm}|}
		\hline
		\textbf{Paramtername} & \textbf{Datentyp} & \textbf{Konstante} & \textbf{Kurzbeschreibung}                                                                                               \\ \hline
		type                & string            & rso                & Quelleneintrag wiederherstellen \\ \hline
		id                  & int               &                    & Identifikator der Quelle \\ \hline
	\end{tabular}
\end{table}
\paragraph{Antwort}Die Antwort ist wie folgt aufgebaut:
\begin{table}[H]
	\begin{tabular}{|c|c|c|p{6.5cm}|}
		\hline
		\textbf{Paramtername} & \textbf{Datentyp} & \textbf{Konstante} & \textbf{Kurzbeschreibung}                                                                                               \\ \hline
		result              & string           &                 & Erfolgreich wenn Wert {\glqq ack\grqq} ist \\ \hline
		Code                & int              &                 & Erfolgreich wenn Wert {\glqq 0\grqq} ist \\ \hline
	\end{tabular}
\end{table}
\subsubsection{Interessenpunkt validieren}
\paragraph{Kurzbeschreibung}Dieser API-Request wird dazu genutzt um einen Interessenpunkt zu validieren.
\paragraph{Anfrage}Folgende Daten werden zu Anfrage benötigt:
\begin{table}[H]
	\begin{tabular}{|c|c|c|p{6.5cm}|}
		\hline
		\textbf{Paramtername} & \textbf{Datentyp} & \textbf{Konstante} & \textbf{Kurzbeschreibung}                                                                                               \\ \hline
		type                & string            & vpi                & Interessenpunkt validieren \\ \hline
		id                  & int               &                    & Identifikator des Interessenpunktes \\ \hline
	\end{tabular}
\end{table}
\paragraph{Antwort}Die Antwort ist wie folgt aufgebaut:
\begin{table}[H]
	\begin{tabular}{|c|c|c|p{6.5cm}|}
		\hline
		\textbf{Paramtername} & \textbf{Datentyp} & \textbf{Konstante} & \textbf{Kurzbeschreibung}                                                                                               \\ \hline
		result              & string           &                 & Erfolgreich wenn Wert {\glqq ack\grqq} ist \\ \hline
		Code                & int              &                 & Erfolgreich wenn Wert {\glqq 0\grqq} ist \\ \hline
	\end{tabular}
\end{table}
\subsubsection{Direkt löschen abfragen}
\paragraph{Kurzbeschreibung}Dieser API-Request wird dazu genutzt um zu prüfen, ob Daten direkt gelöscht werden.
\paragraph{Anfrage}Folgende Daten werden zu Anfrage benötigt:
\begin{table}[H]
	\begin{tabular}{|c|c|c|p{6.5cm}|}
		\hline
		\textbf{Paramtername} & \textbf{Datentyp} & \textbf{Konstante} & \textbf{Kurzbeschreibung}                                                                                               \\ \hline
		type                & string            & vpi                & Interessenpunkt validieren \\ \hline
	\end{tabular}
\end{table}
\paragraph{Antwort}Die Antwort ist wie folgt aufgebaut:
\begin{table}[H]
	\begin{tabular}{|c|c|c|p{6.5cm}|}
		\hline
		\textbf{Paramtername} & \textbf{Datentyp} & \textbf{Konstante} & \textbf{Kurzbeschreibung}                                                                                               \\ \hline
		result              & string           &                 & Erfolgreich wenn Wert {\glqq ack\grqq} ist \\ \hline
		Code                & int              &                 & Erfolgreich wenn Wert {\glqq 0\grqq} ist \\ \hline
		data                & bool             &                 & Wenn Wahr, werden Daten direkt gelöscht \\ \hline
	\end{tabular}
\end{table}
\subsubsection{Hauptbild eines Interessenpunktes ändern}
\paragraph{Kurzbeschreibung}Dieser API-Request wird dazu genutzt um zu prüfen, ob Daten direkt gelöscht werden.
\paragraph{Anfrage}Folgende Daten werden zu Anfrage benötigt:
\begin{table}[H]
	\begin{tabular}{|c|c|c|p{6.5cm}|}
		\hline
		\textbf{Paramtername} & \textbf{Datentyp} & \textbf{Konstante} & \textbf{Kurzbeschreibung}                                                                                               \\ \hline
		type                & string            & emp                & Hauptbild ändern \\ \hline
		poiid				& int				& 					 & Identifikator eines Interessenpunktes \\ \hline
		token				& string			&					 & Identifikator eines Bildes \\ \hline
	\end{tabular}
\end{table}
\paragraph{Antwort}Die Antwort ist wie folgt aufgebaut:
\begin{table}[H]
	\begin{tabular}{|c|c|c|p{6.5cm}|}
		\hline
		\textbf{Paramtername} & \textbf{Datentyp} & \textbf{Konstante} & \textbf{Kurzbeschreibung}                                                                                               \\ \hline
		result              & string           &                 & Erfolgreich wenn Wert {\glqq ack\grqq} ist \\ \hline
		Code                & int              &                 & Erfolgreich wenn Wert {\glqq 0\grqq} ist \\ \hline
		data                & array            &                 & Leeres Array \\ \hline
	\end{tabular}
\end{table}
\subsubsection{Mailadresse bereits existent}
\paragraph{Kurzbeschreibung}Dieser API-Request wird dazu genutzt um zu prüfen, ob eine Mailadresse bereits verwendet wird.
\paragraph{Anfrage}Folgende Daten werden zu Anfrage benötigt:
\begin{table}[H]
	\begin{tabular}{|c|c|c|p{6.5cm}|}
		\hline
		\textbf{Paramtername} & \textbf{Datentyp} & \textbf{Konstante} & \textbf{Kurzbeschreibung}                                                                                               \\ \hline
		type                & string            & cma                & Mailadresse prüfen \\ \hline
		mail				& string			& 					 & Mailadresse\\ \hline
	\end{tabular}
\end{table}
\paragraph{Antwort}Die Antwort ist wie folgt aufgebaut:
\begin{table}[H]
	\begin{tabular}{|c|c|c|p{6.5cm}|}
		\hline
		\textbf{Paramtername} & \textbf{Datentyp} & \textbf{Konstante} & \textbf{Kurzbeschreibung}                                                                                               \\ \hline
		result              & string           &                 & Erfolgreich wenn Wert {\glqq ack\grqq} ist \\ \hline
		Code                & int              &                 & Erfolgreich wenn Wert {\glqq 0\grqq} ist \\ \hline
		data                & bool             &                 & True, wenn Mailadresse verwendet wird \\ \hline
	\end{tabular}
\end{table}