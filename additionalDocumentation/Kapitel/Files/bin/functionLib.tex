\subsection{Allgemeines} Diese Datei enthält alle Funktionen, welche an multiplen Stellen verwendet werden und daher nicht klar zugeordnet werden können.
\begin{table}[H]
	\begin{tabular}{|c|p{11cm}|}
		\hline
		\textbf{Einbindungspunkt} & inc.php \\ \hline
		\textbf{Einbindungspunkt} & inc-sub.php \\ \hline
	\end{tabular}
\end{table}
Die Datei ist nicht direkt durch den Nutzer aufrufbar, dies wird durch folgenden Code-Ausschnitt sichergestellt:
\begin{lstlisting}[language=php]
if (!defined('NICE_PROJECT')) {
	die('Permission denied.');
}
\end{lstlisting}
Der Globale Wert {\glqq NICE\_PROJECT\grqq} wird durch für den Nutzer valide Aufrufpunkte festgelegt, z.B. {\glqq api.php\grqq}.
\newpage
\subsection{Funktionen}
\subsubsection{generateHeader}
\paragraph{Parameter} Die Funktion besitzt folgende Parameter:
\begin{table}[H]
	\begin{tabular}{|c|p{11cm}|}
		\hline
		\textbf{Parametername} & \textbf{Parameterbeschreibung} \\ \hline
		\$Login     & Gibt an, ob Nutzer eingeloggt ist. \\ \hline
		\$lg        & Array für Mehrsprachigkeit (siehe \autoref{lang:de} oder \autoref{lang:en}) \\ \hline
		\$map       & Gibt an, ob einbindende Seite Karte ist \\ \hline
		\$loginpage & Gibt an, ob einbindende Seite Login- beziehungsweise Logout-Seite ist \\ \hline
	\end{tabular}
\end{table}
\paragraph{Beschreibung} Die Funktion erzeugt die Navbar sowie mehrfach verwendete Modals. Die Funktion nutzt folgende Quellen:
\begin{itemize}
	\item COSP
\end{itemize}
Es findet bei dieser Funktion ein Abruf von Daten aus {\glqq COSP\grqq} statt. Die Antwort wird direkt Ausgegeben.
\subsubsection{dump}
\paragraph{Parameter} Die Funktion besitzt folgende Parameter:
\begin{table}[H]
	\begin{tabular}{|c|p{11cm}|}
		\hline
		\textbf{Parametername} & \textbf{Parameterbeschreibung} \\ \hline
		\$data  & Daten für var\_dump \\ \hline
		\$level & Optionale Angabe des Debuglevels, wird mit Angabe aus Konfiguration verglichen , siehe \autoref{config:debug-level} \\ \hline
	\end{tabular}
\end{table}
\paragraph{Beschreibung} Die Funktion dient der Ausgabe von Ergebnissen von Funktionen zu Entwicklungszwecken. Die Funktion nutzt folgende Quellen:
\begin{itemize}
	\item Konfigurationsdatei
\end{itemize}
Es findet bei dieser Funktion kein Abruf von Daten aus {\glqq COSP\grqq} statt. Die Antwort wird direkt Ausgegeben.
\subsubsection{Redirect}
\paragraph{Parameter} Die Funktion besitzt folgende Parameter:
\begin{table}[H]
	\begin{tabular}{|c|p{11cm}|}
		\hline
		\textbf{Parametername} & \textbf{Parameterbeschreibung} \\ \hline
		\$url       & Leitet die Anfrage auf eine andere Seite weiter \\ \hline
		\$permanent & Legt fest, ob Weiterleitung permanent ist. \\ \hline
	\end{tabular}
\end{table}
\paragraph{Beschreibung} Die Funktion leitet den Aufrufer auf eine andere Seite weiter und beendet die Ausführung des aktuellen PHP-Scriptes. Es findet bei dieser Funktion kein Abruf von Daten aus {\glqq COSP\grqq} statt. Die Antwort wird direkt Ausgegeben.
\subsubsection{permissionDenied}
\paragraph{Parameter} Die Funktion besitzt folgende Parameter:
\begin{table}[H]
	\begin{tabular}{|c|p{11cm}|}
		\hline
		\textbf{Parametername} & \textbf{Parameterbeschreibung} \\ \hline
		\$string & Optionale Angab des Grundes \\ \hline
	\end{tabular}
\end{table}
\paragraph{Beschreibung} Die Funktion verhindert ein weiteres Ausführen des PHP-Scriptes und gibt einen entsprechenden HTTP-Statuscode zurück. Es findet bei dieser Funktion kein Abruf von Daten aus {\glqq COSP\grqq} statt. Die Antwort wird direkt Ausgegeben.
\subsubsection{generateHeaderTags}
\paragraph{Parameter} Die Funktion besitzt folgende Parameter:
\begin{table}[H]
	\begin{tabular}{|c|p{11cm}|}
		\hline
		\textbf{Parametername} & \textbf{Parameterbeschreibung} \\ \hline
		\$additional & Optional ein zu bindende Daten \\ \hline
	\end{tabular}
\end{table}
\subparagraph{\$additional}Das Array enthält Einträge mit folgenden Elementen:
\begin{table}[H]
	\begin{tabular}{|c|p{11cm}|}
		\hline
		\textbf{Parametername} & \textbf{Parameterbeschreibung} \\ \hline
		type    & Typ der Datei ({\glqq link\grqq} für zum Beispiel CSS-Files oder {\glqq script\grqq} für zum Beispiel javaScript-Files ) \\ \hline
		rel     & Gibt den Rel-Tag eines HTML-Link Elements an \\ \hline
		href    & Gibt die Position der Datei an \\ \hline
		hrefmin & Minimierte Version der Datei \\ \hline
		typeval & Gibt den Typ der Datei an (zum Beispiel: {\glqq text/javascript\grqq}) \\ \hline
	\end{tabular}
\end{table}
\paragraph{Beschreibung} Die Funktion dient der Generierung von HTML-Head Elementen und der zentralen Pflege der eingebundenen Dateien. Es findet bei dieser Funktion kein Abruf von Daten aus {\glqq COSP\grqq} statt. Die Antwort wird direkt Ausgegeben.
\paragraph{Besonderheiten} Die Einbindung zusätzlicher Dateien erfolgt in der im Array angegebenen Reihenfolge mittels:
\begin{lstlisting}[language=php]
foreach ($additional as $line) {
	switch ($line['type']) {
		case 'link':
			if (isset($line['typeval']) === false || $line['typeval'] === "") {
				echo '<link rel="' . $line['rel'] . '" href="' . $line['href'] . '" >';
			} else {
				echo '<link rel="' . $line['rel'] . '" type="' . $line['typeval'] . '" href="' . $line['href'] . '" >';
			}
			break;
		case 'script':
			echo '<script type="' . $line['typeval'] . '" src="' . $line['href'] . '" ></script>';
			break;
	}
}
\end{lstlisting}
\subsubsection{decode\_json}
\paragraph{Parameter} Die Funktion besitzt folgende Parameter:
\begin{table}[H]
	\begin{tabular}{|c|p{11cm}|}
		\hline
		\textbf{Parametername} & \textbf{Parameterbeschreibung} \\ \hline
		\$string & Eingabe des JSON als Zeichenkette \\ \hline
	\end{tabular}
\end{table}
\paragraph{Beschreibung} Die Funktion decodiert Daten im JSON-Format in PHP-Arrays. Es findet bei dieser Funktion kein Abruf von Daten aus {\glqq COSP\grqq} statt. Die Antwort wird als strukturiertes Array an den Aufrufer zurückgegeben.
\subsubsection{checkPoiExists}
\paragraph{Parameter} Die Funktion besitzt folgende Parameter:
\begin{table}[H]
	\begin{tabular}{|c|p{11cm}|}
		\hline
		\textbf{Parametername} & \textbf{Parameterbeschreibung} \\ \hline
		\$POIID & Identifikator eines Interessenpunktes \\ \hline
	\end{tabular}
\end{table}
\paragraph{Beschreibung} Die Funktion prüft die Existenz eines Interessenpunktes. Die Funktion nutzt folgende Quellen:
\begin{itemize}
	\item Interessenpunkt-Tabelle
\end{itemize}
Es findet bei dieser Funktion kein Abruf von Daten aus {\glqq COSP\grqq} statt. Die Antwort wird als Boolean an den Aufrufer zurückgegeben.
\subsubsection{remoteLogin}
\paragraph{Parameter} Die Funktion besitzt folgende Parameter:
\begin{table}[H]
	\begin{tabular}{|c|p{11cm}|}
		\hline
		\textbf{Parametername} & \textbf{Parameterbeschreibung} \\ \hline
		\$name & Nutzername, dessen Daten abgefragt werden sollen \\ \hline
	\end{tabular}
\end{table}
\paragraph{Beschreibung} Die Funktion fragt die vollständigen Nutzerdaten aus {\glqq COSP\grqq} ab. Die Funktion nutzt folgende Quellen:
\begin{itemize}
	\item COSP
\end{itemize}
Es findet bei dieser Funktion ein Abruf von Daten aus {\glqq COSP\grqq} statt. Die Antwort wird als strukturiertes Array an den Aufrufer zurückgegeben.
\subsubsection{remoteRole}
\paragraph{Parameter} Die Funktion besitzt folgende Parameter:
\begin{table}[H]
	\begin{tabular}{|c|p{11cm}|}
		\hline
		\textbf{Parametername} & \textbf{Parameterbeschreibung} \\ \hline
		\$name              & Nutzername, dessen Rolle abgefragt werden sollen \\ \hline
		\$ignoreDeaktivatet & Ignoriert Freischaltungsstatus des Nutzers \\ \hline
	\end{tabular}
\end{table}
\paragraph{Beschreibung} Die Funktion fragt die Rangdaten eines Nutzers aus {\glqq COSP\grqq} ab. Die Funktion nutzt folgende Quellen:
\begin{itemize}
	\item COSP
\end{itemize}
Es findet bei dieser Funktion ein Abruf von Daten aus {\glqq COSP\grqq} statt. Die Antwort wird als strukturiertes Array an den Aufrufer zurückgegeben.
\subsubsection{addRemoteUser}
\paragraph{Parameter} Die Funktion besitzt folgende Parameter:
\begin{table}[H]
	\begin{tabular}{|c|p{11cm}|}
		\hline
		\textbf{Parametername} & \textbf{Parameterbeschreibung} \\ \hline
		\$Username  & Nutzername des neuen Nutzers\\ \hline
		\$EMail     & E-Mailadresse des neuen Nutzers \\ \hline
		\$pwd       & Passwort als Hash \\ \hline
		\$firstname & Vorname des neuen Nutzers \\ \hline
		\$lastname  & Nachname des neuen Nutzers \\ \hline
	\end{tabular}
\end{table}
\paragraph{Beschreibung} Die Funktion fügt einen neuen Nutzer zu {\glqq COSP\grqq} hinzu. Die Funktion hat Auswirkungen auf folgende Quellen:
\begin{itemize}
	\item COSP
\end{itemize}
Es findet bei dieser Funktion kein Abruf von Daten aus {\glqq COSP\grqq} statt. Es werden jedoch Daten an {\glqq COSP\grqq} gesendet. Die Antwort wird als strukturiertes Array an den Aufrufer zurückgegeben.
\subsubsection{getRemoteAllUsernames}
\paragraph{Parameter} Die Funktion besitzt keine Parameter.
\paragraph{Beschreibung} Die Funktion fragt alle Nutzernamen aus {\glqq COSP\grqq} ab. Die Funktion nutzt folgende Quellen:
\begin{itemize}
	\item 
\end{itemize}
Es findet bei dieser Funktion ein Abruf von Daten aus {\glqq COSP\grqq} statt. Die Antwort wird als strukturiertes Array an den Aufrufer zurückgegeben.
\subsubsection{UploadPicture}
\paragraph{Parameter} Die Funktion besitzt folgende Parameter:
\begin{table}[H]
	\begin{tabular}{|c|p{11cm}|}
		\hline
		\textbf{Parametername} & \textbf{Parameterbeschreibung} \\ \hline
		\$title      & Titel des Bildes \\ \hline
		\$desc       & Beschreibung des Bildes \\ \hline
		\$filepath   & Lokaler Pfad zum Bild \\ \hline
		\$ftype      & Gibt den Dateityp des Bildes an \\ \hline
		\$username   & Nutzername des hochladenden Nutzers \\ \hline
		\$source     & Quellenangabe (optional) \\ \hline
		\$sourceType & Identifikator des Typs der Quelle (optional) \\ \hline
	\end{tabular}
\end{table}
\paragraph{Beschreibung} Die Funktion lädt ein Bild auf {\glqq COSP\grqq} hoch. Die Funktion hat Auswirkungen auf folgende Quellen:
\begin{itemize}
	\item COSP
\end{itemize}
Es findet bei dieser Funktion kein Abruf von Daten aus {\glqq COSP\grqq} statt. Es werden jedoch Daten an {\glqq COSP\grqq} gesendet. Die Antwort wird als strukturiertes Array an den Aufrufer zurückgegeben.
\subsubsection{getRemoteSeccode}
\paragraph{Parameter} Die Funktion besitzt folgende Parameter:
\begin{table}[H]
	\begin{tabular}{|c|p{11cm}|}
		\hline
		\textbf{Parametername} & \textbf{Parameterbeschreibung} \\ \hline
		\$pictureToken & alphanumerischer Identifikator eines Bildes \\ \hline
	\end{tabular}
\end{table}
\paragraph{Beschreibung} Die Funktion fragt Daten zum Laden eines bestimmten Bildes bei {\glqq COSP\grqq} an. Die Funktion nutzt folgende Quellen:
\begin{itemize}
	\item COSP
\end{itemize}
Es findet bei dieser Funktion ein Abruf von Daten aus {\glqq COSP\grqq} statt. Die Antwort wird als strukturiertes Array an den Aufrufer zurückgegeben.
\subsubsection{getRemotePictureList}
\paragraph{Parameter} Die Funktion besitzt keine Parameter.
\paragraph{Beschreibung} Die Funktion fragt Daten zum Laden aller Bilder bei {\glqq COSP\grqq} an. Die Funktion nutzt folgende Quellen:
\begin{itemize}
	\item COSP
\end{itemize}
Es findet bei dieser Funktion ein Abruf von Daten aus {\glqq COSP\grqq} statt. Die Antwort wird als strukturiertes Array an den Aufrufer zurückgegeben.
\subsubsection{ApiCall}
\paragraph{Parameter} Die Funktion besitzt folgende Parameter:
\begin{table}[H]
	\begin{tabular}{|c|p{11cm}|}
		\hline
		\textbf{Parametername} & \textbf{Parameterbeschreibung} \\ \hline
		\$params       & Array mit Parametern des API-Aufrufs \\ \hline
		\$type         & Typ des API-Aufrufs \\ \hline
		\$file\_upload & Gibt an, ob eine Datei hochgeladen werden soll \\ \hline
	\end{tabular}
\end{table}
\subparagraph{\$params}Das Array besteht aus Key-Value-Paaren.
\paragraph{Beschreibung} Die Funktion führt einen API-Aufruf an {\glqq COSP\grqq} aus. Die Funktion nutzt folgende Quellen:
\begin{itemize}
	\item Konfigurationsdatei
\end{itemize}
Es findet bei dieser Funktion ein Abruf von Daten aus {\glqq COSP\grqq} statt. Es können auch Daten an {\glqq COSP\grqq} gesendet werden. Die Antwort wird als strukturiertes Array an den Aufrufer zurückgegeben.
\subsubsection{checkMailAddress}
\paragraph{Parameter} Die Funktion besitzt folgende Parameter:
\begin{table}[H]
	\begin{tabular}{|c|p{11cm}|}
		\hline
		\textbf{Parametername} & \textbf{Parameterbeschreibung} \\ \hline
		\$email & zu prüfende E-Mailadresse \\ \hline
	\end{tabular}
\end{table}
\paragraph{Beschreibung} Die Funktion prüft, ob die angegebene Zeichenkette eine E-Mailadresse darstellt. Es findet bei dieser Funktion kein Abruf von Daten aus {\glqq COSP\grqq} statt. Die Antwort wird als strukturiertes Array an den Aufrufer zurückgegeben.
\subsubsection{inspectPassword}
\paragraph{Parameter} Die Funktion besitzt folgende Parameter:
\begin{table}[H]
	\begin{tabular}{|c|p{11cm}|}
		\hline
		\textbf{Parametername} & \textbf{Parameterbeschreibung} \\ \hline
		\$PasswordField1Val & Inhalt des ersten Passwortfeldes \\ \hline
		\$PasswordField2Val & Inhalt des zweiten Passwortfeldes \\ \hline
	\end{tabular}
\end{table}
\paragraph{Beschreibung} Die Funktion prüft, ob die angegebenen Inhalt der Passworteingabe Felder übereinstimmen und den geforderten Kriterien entsprechen. Es findet bei dieser Funktion kein Abruf von Daten aus {\glqq COSP\grqq} statt. Die Antwort wird als strukturiertes Array an den Aufrufer zurückgegeben.\\
\subsubsection{checkPermission}
\paragraph{Parameter} Die Funktion besitzt folgende Parameter:
\begin{table}[H]
	\begin{tabular}{|c|p{11cm}|}
		\hline
		\textbf{Parametername} & \textbf{Parameterbeschreibung} \\ \hline
		\$requiredPermission & erforderlicher Freigabewert \\ \hline
	\end{tabular}
\end{table}
\paragraph{Beschreibung} Die Funktion prüft, ob ein Nutzer eine angegebene Berechtigungsstufe besitzt. Sollte der Nutzer nicht die geforderte Berechtigungsstufe besitzen so wird die Ausführung des PHP-Scriptes beendet. Es findet bei dieser Funktion kein Abruf von Daten aus {\glqq COSP\grqq} statt. Die Antwort wird als strukturiertes Array an den Aufrufer zurückgegeben.
\subsubsection{getRemoteRank}
\paragraph{Parameter} Die Funktion besitzt folgende Parameter:
\begin{table}[H]
	\begin{tabular}{|c|p{11cm}|}
		\hline
		\textbf{Parametername} & \textbf{Parameterbeschreibung} \\ \hline
		\$username & Nutzername, dessen Rang abgefragt werden soll \\ \hline
	\end{tabular}
\end{table}
\paragraph{Beschreibung} Die Funktion fragt den Rang eines Nutzers in {\glqq COSP\grqq} ab. Die Funktion nutzt folgende Quellen:
\begin{itemize}
	\item COSP
\end{itemize}
Es findet bei dieser Funktion ein Abruf von Daten aus {\glqq COSP\grqq} statt. Die Antwort wird als strukturiertes Array an den Aufrufer zurückgegeben.
\subsubsection{getRanktypes}
\paragraph{Parameter} Die Funktion besitzt keine Parameter.
\paragraph{Beschreibung} Die Funktion fragt alle verfügbaren Ränge in {\glqq COSP\grqq} ab. Die Funktion nutzt folgende Quellen:
\begin{itemize}
	\item COSP
\end{itemize}
Es findet bei dieser Funktion ein Abruf von Daten aus {\glqq COSP\grqq} statt. Die Antwort wird als strukturiertes Array an den Aufrufer zurückgegeben.
\subsubsection{getRanklist}
\paragraph{Parameter} Die Funktion besitzt keine Parameter.
\paragraph{Beschreibung} Die Funktion fragt eine Rangliste für dieses Modul in {\glqq COSP\grqq} ab. Die Funktion nutzt folgende Quellen:
\begin{itemize}
	\item COSP
\end{itemize}
Es findet bei dieser Funktion ein Abruf von Daten aus {\glqq COSP\grqq} statt. Die Antwort wird als strukturiertes Array an den Aufrufer zurückgegeben.
\subsubsection{addRankPoints}
\paragraph{Parameter} Die Funktion besitzt folgende Parameter:
\begin{table}[H]
	\begin{tabular}{|c|p{11cm}|}
		\hline
		\textbf{Parametername} & \textbf{Parameterbeschreibung} \\ \hline
		\$username & Nutzername zu welchem Punkte hinzugefügt werden sollen \\ \hline
		\$points   & hinzuzufügende Punkte \\ \hline
		\$reason   & Begründung der Punkte \\ \hline
	\end{tabular}
\end{table}
\paragraph{Beschreibung} Die Funktion fügt einem Nutzer Rangpunkte hinzu. Die Funktion hat Auswirkungen auf folgende Quellen:
\begin{itemize}
	\item COSP
\end{itemize}
Es findet bei dieser Funktion kein Abruf von Daten aus {\glqq COSP\grqq} statt. Es werden jedoch Daten an {\glqq COSP\grqq} gesendet. Die Antwort wird als strukturiertes Array an den Aufrufer zurückgegeben.
\subsubsection{getValidationValue}
\paragraph{Parameter} Die Funktion besitzt folgende Parameter:
\begin{table}[H]
	\begin{tabular}{|c|p{11cm}|}
		\hline
		\textbf{Parametername} & \textbf{Parameterbeschreibung} \\ \hline
		\$name & Nutzername zu welchem Rolle abgefragt werden soll \\ \hline
	\end{tabular}
\end{table}
\paragraph{Beschreibung} Die Funktion bestimmt den Validierungswert des aktuellen Nutzers. Die Funktion nutzt folgende Quellen:
\begin{itemize}
	\item COSP
	\item Konfigurationsdatei
\end{itemize}
Es findet bei dieser Funktion ein Abruf von Daten aus {\glqq COSP\grqq} statt. Die Antwort wird als strukturiertes Array an den Aufrufer zurückgegeben.
\subsubsection{validatePoi}
\paragraph{Parameter} Die Funktion besitzt folgende Parameter:
\begin{table}[H]
	\begin{tabular}{|c|p{11cm}|}
		\hline
		\textbf{Parametername} & \textbf{Parameterbeschreibung} \\ \hline
		\$poiid & Identifikator eines Interessenpunktes \\ \hline
	\end{tabular}
\end{table}
\paragraph{Beschreibung} Die Funktion fügt eine Validierung einem Interessenpunkt hinzu und vergibt Rangpunkte. Die Funktion hat Auswirkungen auf folgende Quellen:
\begin{itemize}
	\item Tabelle mit Validierungsdaten zu Interessenpunkten
	\item COSP
\end{itemize}
Es findet bei dieser Funktion kein Abruf von Daten aus {\glqq COSP\grqq} statt. Es werden jedoch Daten an {\glqq COSP\grqq} gesendet. Die Antwort wird als strukturiertes Array an den Aufrufer zurückgegeben.
\subsubsection{validateTimeSpanPoi}
\paragraph{Parameter} Die Funktion besitzt folgende Parameter:
\begin{table}[H]
	\begin{tabular}{|c|p{11cm}|}
		\hline
		\textbf{Parametername} & \textbf{Parameterbeschreibung} \\ \hline
		\$poiid & Identifikator eines Interessenpunktes \\ \hline
	\end{tabular}
\end{table}
\paragraph{Beschreibung} Die Funktion fügt eine Validierung zur Zeitspanne eines Interessenpunkt hinzu und vergibt Rangpunkte. Die Funktion hat Auswirkungen auf folgende Quellen:
\begin{itemize}
	\item Tabelle mit Validierungsdaten zur Zeitspanne von Interessenpunkten
	\item COSP
\end{itemize}
Es findet bei dieser Funktion kein Abruf von Daten aus {\glqq COSP\grqq} statt. Es werden jedoch Daten an {\glqq COSP\grqq} gesendet. Die Antwort wird als strukturiertes Array an den Aufrufer zurückgegeben.
\subsubsection{validateCurrentAddressPoi}
\paragraph{Parameter} Die Funktion besitzt folgende Parameter:
\begin{table}[H]
	\begin{tabular}{|c|p{11cm}|}
		\hline
		\textbf{Parametername} & \textbf{Parameterbeschreibung} \\ \hline
		\$poiid & Identifikator eines Interessenpunktes \\ \hline
	\end{tabular}
\end{table}
\paragraph{Beschreibung} Die Funktion fügt eine Validierung zur aktuellen Adresse eines Interessenpunkt hinzu und vergibt Rangpunkte. Die Funktion hat Auswirkungen auf folgende Quellen:
\begin{itemize}
	\item Tabelle mit Validierungsdaten zur aktuellen Adresse von Interessenpunkten
	\item COSP
\end{itemize}
Es findet bei dieser Funktion kein Abruf von Daten aus {\glqq COSP\grqq} statt. Es werden jedoch Daten an {\glqq COSP\grqq} gesendet. Die Antwort wird als strukturiertes Array an den Aufrufer zurückgegeben.
\subsubsection{validateHistoryPoi}
\paragraph{Parameter} Die Funktion besitzt folgende Parameter:
\begin{table}[H]
	\begin{tabular}{|c|p{11cm}|}
		\hline
		\textbf{Parametername} & \textbf{Parameterbeschreibung} \\ \hline
		\$poiid & Identifikator eines Interessenpunktes \\ \hline
	\end{tabular}
\end{table}
\paragraph{Beschreibung} Die Funktion fügt eine Validierung zur Geschichte eines Interessenpunkt hinzu und vergibt Rangpunkte. Die Funktion hat Auswirkungen auf folgende Quellen:
\begin{itemize}
	\item Tabelle mit Validierungsdaten zur Geschichte von Interessenpunkten
	\item COSP
\end{itemize}
Es findet bei dieser Funktion kein Abruf von Daten aus {\glqq COSP\grqq} statt. Es werden jedoch Daten an {\glqq COSP\grqq} gesendet. Die Antwort wird als strukturiertes Array an den Aufrufer zurückgegeben.
\subsubsection{validatePoiName}
\paragraph{Parameter} Die Funktion besitzt folgende Parameter:
\begin{table}[H]
	\begin{tabular}{|c|p{11cm}|}
		\hline
		\textbf{Parametername} & \textbf{Parameterbeschreibung} \\ \hline
		\$nameid & Identifikator eines Namens \\ \hline
	\end{tabular}
\end{table}
\paragraph{Beschreibung} Die Funktion fügt eine Validierung zu einem Namen hinzu und vergibt Rangpunkte. Die Funktion hat Auswirkungen auf folgende Quellen:
\begin{itemize}
	\item Tabelle mit Validierungsdaten zu Namen
	\item COSP
\end{itemize}
Es findet bei dieser Funktion kein Abruf von Daten aus {\glqq COSP\grqq} statt. Es werden jedoch Daten an {\glqq COSP\grqq} gesendet. Die Antwort wird als strukturiertes Array an den Aufrufer zurückgegeben.
\subsubsection{validatePoiOperator}
\paragraph{Parameter} Die Funktion besitzt folgende Parameter:
\begin{table}[H]
	\begin{tabular}{|c|p{11cm}|}
		\hline
		\textbf{Parametername} & \textbf{Parameterbeschreibung} \\ \hline
		\$opertorid & Identifikator eines Betreibers \\ \hline
	\end{tabular}
\end{table}
\paragraph{Beschreibung} Die Funktion fügt eine Validierung zu einem Betreiber hinzu und vergibt Rangpunkte. Die Funktion hat Auswirkungen auf folgende Quellen:
\begin{itemize}
	\item Tabelle mit Validierungsdaten zu Betreibern
	\item COSP
\end{itemize}
Es findet bei dieser Funktion kein Abruf von Daten aus {\glqq COSP\grqq} statt. Es werden jedoch Daten an {\glqq COSP\grqq} gesendet. Die Antwort wird als strukturiertes Array an den Aufrufer zurückgegeben.
\subsubsection{validatePoiHistAddress}
\paragraph{Parameter} Die Funktion besitzt folgende Parameter:
\begin{table}[H]
	\begin{tabular}{|c|p{11cm}|}
		\hline
		\textbf{Parametername} & \textbf{Parameterbeschreibung} \\ \hline
		\$histAddrId & Identifikator einer historischen Adresse \\ \hline
	\end{tabular}
\end{table}
\paragraph{Beschreibung} Die Funktion fügt eine Validierung zu einer historischen Adresse hinzu und vergibt Rangpunkte. Die Funktion hat Auswirkungen auf folgende Quellen:
\begin{itemize}
	\item Tabelle mit Validierungsdaten zu historischen Adressen
	\item COSP
\end{itemize}
Es findet bei dieser Funktion kein Abruf von Daten aus {\glqq COSP\grqq} statt. Es werden jedoch Daten an {\glqq COSP\grqq} gesendet. Die Antwort wird als strukturiertes Array an den Aufrufer zurückgegeben.
\subsubsection{validatePoiStory}
\paragraph{Parameter} Die Funktion besitzt folgende Parameter:
\begin{table}[H]
	\begin{tabular}{|c|p{11cm}|}
		\hline
		\textbf{Parametername} & \textbf{Parameterbeschreibung} \\ \hline
		\$story\_poi\_id & Identifikator eines Links zwischen einer Geschichte und einem Interessenpunkt \\ \hline
	\end{tabular}
\end{table}
\paragraph{Beschreibung} Die Funktion fügt eine Validierung zu einem Link zwischen einem Interessenpunkt und einer Geschichte hinzu und vergibt Rangpunkte. Die Funktion hat Auswirkungen auf folgende Quellen:
\begin{itemize}
	\item Tabelle mit Validierungsdaten zu Links zwischen Geschichten und Interessenpunkten
	\item COSP
\end{itemize}
Es findet bei dieser Funktion kein Abruf von Daten aus {\glqq COSP\grqq} statt. Es werden jedoch Daten an {\glqq COSP\grqq} gesendet. Die Antwort wird als strukturiertes Array an den Aufrufer zurückgegeben.
\subsubsection{getValidatedByPOI}
\paragraph{Parameter} Die Funktion besitzt keine Parameter.
\paragraph{Beschreibung} Die Funktion fragt alle Validierungswerte für alle Interessenpunkte ab. Die Funktion nutzt folgende Quellen:
\begin{itemize}
	\item Interessenpunkt-Tabelle
\end{itemize}
Es findet bei dieser Funktion kein Abruf von Daten aus {\glqq COSP\grqq} statt. Die Antwort wird als strukturiertes Array an den Aufrufer zurückgegeben.
\subsubsection{getPoisForUser}
\paragraph{Parameter} Die Funktion besitzt keine Parameter.
\paragraph{Beschreibung} Die Funktion fragt alle Interessenpunkte, welche für den Nutzer freigegeben sind, ab. Die Funktion nutzt folgende Quellen:
\begin{itemize}
	\item Interessenpunkt-Tabelle
	\item Tabelle mit Validierungsdaten zu Intressenpunkten
\end{itemize}
Es findet bei dieser Funktion kein Abruf von Daten aus {\glqq COSP\grqq} statt. Die Antwort wird als strukturiertes Array an den Aufrufer zurückgegeben.
\subsubsection{getPoisForUser}
\paragraph{Parameter} Die Funktion besitzt keine Parameter.
\paragraph{Beschreibung} Die Funktion fragt die Titel aller Interessenpunkte, welche für den Nutzer freigegeben sind, ab. Die Funktion nutzt folgende Quellen:
\begin{itemize}
	\item Interessenpunkt-Tabelle
	\item Tabelle mit Validierungsdaten zu Intressenpunkten
\end{itemize}
Es findet bei dieser Funktion kein Abruf von Daten aus {\glqq COSP\grqq} statt. Die Antwort wird als strukturiertes Array an den Aufrufer zurückgegeben.
\subsubsection{getValidationsByUserForPOI}
\paragraph{Parameter} Die Funktion besitzt folgende Parameter:
\begin{table}[H]
	\begin{tabular}{|c|p{11cm}|}
		\hline
		\textbf{Parametername} & \textbf{Parameterbeschreibung} \\ \hline
		\$uid & Angabe eines Nutzeridentifikators \\ \hline
	\end{tabular}
\end{table}
\paragraph{Beschreibung} Die Funktion liefert alle Identifikatoren von Interessenpunkten, welche bereits durch den Nutzer validiert wurden. Die Funktion nutzt folgende Quellen:
\begin{itemize}
	\item Tabelle mit Validierungsdaten zu Intressenpunkten
\end{itemize}
Es findet bei dieser Funktion kein Abruf von Daten aus {\glqq COSP\grqq} statt. Die Antwort wird als strukturiertes Array an den Aufrufer zurückgegeben.
\subsubsection{getValidationsByUserForTimeSpans}
\paragraph{Parameter} Die Funktion besitzt folgende Parameter:
\begin{table}[H]
	\begin{tabular}{|c|p{11cm}|}
		\hline
		\textbf{Parametername} & \textbf{Parameterbeschreibung} \\ \hline
		\$uid & Angabe eines Nutzeridentifikators \\ \hline
	\end{tabular}
\end{table}
\paragraph{Beschreibung} Die Funktion liefert alle Identifikatoren von Interessenpunkten, bei welchen bereits durch den Nutzer die Zeitspanne validiert wurde. Die Funktion nutzt folgende Quellen:
\begin{itemize}
	\item Tabelle mit Validierungsdaten zu Zeitspannen von Interessenpunkten
\end{itemize}
Es findet bei dieser Funktion kein Abruf von Daten aus {\glqq COSP\grqq} statt. Die Antwort wird als strukturiertes Array an den Aufrufer zurückgegeben.
\subsubsection{getValidationsByUserForCurrentAddress}
\paragraph{Parameter} Die Funktion besitzt folgende Parameter:
\begin{table}[H]
	\begin{tabular}{|c|p{11cm}|}
		\hline
		\textbf{Parametername} & \textbf{Parameterbeschreibung} \\ \hline
		\$uid & Angabe eines Nutzeridentifikators \\ \hline
	\end{tabular}
\end{table}
\paragraph{Beschreibung} Die Funktion liefert alle Identifikatoren von Interessenpunkten, bei welchen bereits durch den Nutzer die aktuelle Adresse validiert wurde. Die Funktion nutzt folgende Quellen:
\begin{itemize}
	\item Tabelle mit Validierungsdaten zu aktuellen Adressen von Interessenpunkten
\end{itemize}
Es findet bei dieser Funktion kein Abruf von Daten aus {\glqq COSP\grqq} statt. Die Antwort wird als strukturiertes Array an den Aufrufer zurückgegeben.
\subsubsection{getValidationsByUserForHistory}
\paragraph{Parameter} Die Funktion besitzt folgende Parameter:
\begin{table}[H]
	\begin{tabular}{|c|p{11cm}|}
		\hline
		\textbf{Parametername} & \textbf{Parameterbeschreibung} \\ \hline
		\$uid & Angabe eines Nutzeridentifikators \\ \hline
	\end{tabular}
\end{table}
\paragraph{Beschreibung} Die Funktion liefert alle Identifikatoren von Interessenpunkten, bei welchen bereits durch den Nutzer die Geschichte validiert wurde. Die Funktion nutzt folgende Quellen:
\begin{itemize}
	\item Tabelle mit Validierungsdaten zur Geschichte von Interessenpunkten
\end{itemize}
Es findet bei dieser Funktion kein Abruf von Daten aus {\glqq COSP\grqq} statt. Die Antwort wird als strukturiertes Array an den Aufrufer zurückgegeben.
\subsubsection{getValidationsByUserForPoiNames}
\paragraph{Parameter} Die Funktion besitzt folgende Parameter:
\begin{table}[H]
	\begin{tabular}{|c|p{11cm}|}
		\hline
		\textbf{Parametername} & \textbf{Parameterbeschreibung} \\ \hline
		\$uid & Angabe eines Nutzeridentifikators \\ \hline
	\end{tabular}
\end{table}
\paragraph{Beschreibung} Die Funktion liefert alle Identifikatoren von Namen, welche bereits durch den Nutzer die validiert wurden. Die Funktion nutzt folgende Quellen:
\begin{itemize}
	\item Tabelle mit Validierungsdaten zu Namen
\end{itemize}
Es findet bei dieser Funktion kein Abruf von Daten aus {\glqq COSP\grqq} statt. Die Antwort wird als strukturiertes Array an den Aufrufer zurückgegeben.
\subsubsection{getValidationsByUserForPoiOperators}
\paragraph{Parameter} Die Funktion besitzt folgende Parameter:
\begin{table}[H]
	\begin{tabular}{|c|p{11cm}|}
		\hline
		\textbf{Parametername} & \textbf{Parameterbeschreibung} \\ \hline
		\$uid & Angabe eines Nutzeridentifikators \\ \hline
	\end{tabular}
\end{table}
\paragraph{Beschreibung} Die Funktion liefert alle Identifikatoren von Betreibern, welche bereits durch den Nutzer die validiert wurden. Die Funktion nutzt folgende Quellen:
\begin{itemize}
	\item Tabelle mit Validierungsdaten zu Betreibern
\end{itemize}
Es findet bei dieser Funktion kein Abruf von Daten aus {\glqq COSP\grqq} statt. Die Antwort wird als strukturiertes Array an den Aufrufer zurückgegeben.
\subsubsection{getValidationsByUserForPoiHistAddresses}
\paragraph{Parameter} Die Funktion besitzt folgende Parameter:
\begin{table}[H]
	\begin{tabular}{|c|p{11cm}|}
		\hline
		\textbf{Parametername} & \textbf{Parameterbeschreibung} \\ \hline
		\$uid & Angabe eines Nutzeridentifikators \\ \hline
	\end{tabular}
\end{table}
\paragraph{Beschreibung} Die Funktion liefert alle Identifikatoren von historischen Adressen, welche bereits durch den Nutzer die validiert wurden. Die Funktion nutzt folgende Quellen:
\begin{itemize}
	\item Tabelle mit Validierungsdaten zu historischen Adressen
\end{itemize}
Es findet bei dieser Funktion kein Abruf von Daten aus {\glqq COSP\grqq} statt. Die Antwort wird als strukturiertes Array an den Aufrufer zurückgegeben.
\subsubsection{getValidationsByUserForLinkPoiStory}
\paragraph{Parameter} Die Funktion besitzt folgende Parameter:
\begin{table}[H]
	\begin{tabular}{|c|p{11cm}|}
		\hline
		\textbf{Parametername} & \textbf{Parameterbeschreibung} \\ \hline
		\$uid & Angabe eines Nutzeridentifikators \\ \hline
	\end{tabular}
\end{table}
\paragraph{Beschreibung} Die Funktion liefert alle Identifikatoren von Links zwischen Interessenpunkten und Geschichten, welche bereits durch den Nutzer die validiert wurden. Die Funktion nutzt folgende Quellen:
\begin{itemize}
	\item Tabelle mit Validierungsdaten zu Links zwischen Interessenpunkten und Geschichten
\end{itemize}
Es findet bei dieser Funktion kein Abruf von Daten aus {\glqq COSP\grqq} statt. Die Antwort wird als strukturiertes Array an den Aufrufer zurückgegeben.
\subsubsection{getAllStoriesData}
\paragraph{Parameter} Die Funktion besitzt keine Parameter.
\paragraph{Beschreibung} Die Funktion fragt Daten für das abrufen aller für das Modul verfügbaren Geschichten ab. Die Funktion nutzt folgende Quellen:
\begin{itemize}
	\item COSP
\end{itemize}
Es findet bei dieser Funktion ein Abruf von Daten aus {\glqq COSP\grqq} statt. Die Antwort wird als strukturiertes Array an den Aufrufer zurückgegeben.
\subsubsection{addUserStoryRemote}
\paragraph{Parameter} Die Funktion besitzt folgende Parameter:
\begin{table}[H]
	\begin{tabular}{|c|p{11cm}|}
		\hline
		\textbf{Parametername} & \textbf{Parameterbeschreibung} \\ \hline
		\$json & Daten der Geschichte \\ \hline
	\end{tabular}
\end{table}
\subparagraph{\$json}Das Array enthält Elemente nach \autoref{api:NewStoryAdd} und \autoref{api-functions:addUserStory}.
\paragraph{Beschreibung} Die Funktion fügt eine neue Geschichte zu {\glqq COSP\grqq} hinzu. Die Funktion hat Auswirkungen auf folgende Quellen:
\begin{itemize}
	\item COSP
\end{itemize}
Es findet bei dieser Funktion kein Abruf von Daten aus {\glqq COSP\grqq} statt. Es werden jedoch Daten an {\glqq COSP\grqq} gesendet. Die Antwort wird als strukturiertes Array an den Aufrufer zurückgegeben.
\subsubsection{deletePOIComplete}
\paragraph{Parameter} Die Funktion besitzt folgende Parameter:
\begin{table}[H]
	\begin{tabular}{|c|p{11cm}|}
		\hline
		\textbf{Parametername} & \textbf{Parameterbeschreibung} \\ \hline
		\$poiid     & Identifikator eines Interessenpunktes \\ \hline
		\$overwrite & Legt fest, ob Daten direkt gelöscht werden sollen \\ \hline
	\end{tabular}
\end{table}
\paragraph{Beschreibung} Die Funktion löscht einen Interessenpunkt mitsamt aller zugehöriger Daten. Die Funktion hat Auswirkungen auf:
\begin{itemize}
	\item Interessenpunkt-Tabelle
	\item Kommentar-Tabelle
	\item Saalanzahl-Tabelle
	\item Sitzplatzanzahl-Tabelle
	\item Namen-Tabelle
	\item Betreiber-Tabelle
	\item Tabelle mit historischen Adressen
	\item Tabelle mit Links zwischen Interessenpunkten und Bildern
	\item Tabelle mit Links zwischen Interessenpunkten und Geschichten
	\item Tabelle mit Validierungsinformationen zu Interessenpunkten
	\item Tabelle mit Validierungsinformationen zu Saalanzahlen
	\item Tabelle mit Validierungsinformationen zu Sitzplatzanzahlen
	\item Tabelle mit Validierungsinformationen zu Namen
	\item Tabelle mit Validierungsinformationen zu historischen Adressen
	\item Tabelle mit Validierungsinformationen zu Links zwischen Interessenpunkten und Bildern
	\item Tabelle mit Validierungsinformationen zu Links zwischen Interessenpunkten und Geschichten
\end{itemize}
Es findet bei dieser Funktion kein Abruf von Daten aus {\glqq COSP\grqq} statt. Die Antwort wird als strukturiertes Array an den Aufrufer zurückgegeben.
\subsubsection{restorePOI}
\paragraph{Parameter} Die Funktion besitzt folgende Parameter:
\begin{table}[H]
	\begin{tabular}{|c|p{11cm}|}
		\hline
		\textbf{Parametername} & \textbf{Parameterbeschreibung} \\ \hline
		\$poiid     & Identifikator eines Interessenpunktes \\ \hline
	\end{tabular}
\end{table}
\paragraph{Beschreibung} Die Funktion stellt einen als gelöscht markierten Interessenpunkt mitsamt aller zugehöriger Daten wieder her. Die Funktion hat Auswirkungen auf:
\begin{itemize}
	\item Interessenpunkt-Tabelle
	\item Kommentar-Tabelle
	\item Saalanzahl-Tabelle
	\item Sitzplatzanzahl-Tabelle
	\item Namen-Tabelle
	\item Betreiber-Tabelle
	\item Tabelle mit historischen Adressen
	\item Tabelle mit Links zwischen Interessenpunkten und Bildern
	\item Tabelle mit Links zwischen Interessenpunkten und Geschichten
\end{itemize}
Es findet bei dieser Funktion kein Abruf von Daten aus {\glqq COSP\grqq} statt. Die Antwort wird als strukturiertes Array an den Aufrufer zurückgegeben.
\subsubsection{deleteSourceByPoi}
\paragraph{Parameter} Die Funktion besitzt folgende Parameter:
\begin{table}[H]
	\begin{tabular}{|c|p{11cm}|}
		\hline
		\textbf{Parametername} & \textbf{Parameterbeschreibung} \\ \hline
		\$poiid     & Identifikator eines Interessenpunktes \\ \hline
		\$overwrite & Legt fest, ob Daten direkt gelöscht werden sollen \\ \hline
	\end{tabular}
\end{table}
\paragraph{Beschreibung} Die Funktion löscht alle Quellen eines Interessenpunktes oder markiert diese als gelöscht. Die Funktion hat Auswirkung auf folgende Quellen:
\begin{itemize}
	\item Tabelle mit Quellenangaben
\end{itemize}
Es findet bei dieser Funktion kein Abruf von Daten aus {\glqq COSP\grqq} statt. Die Antwort wird als strukturiertes Array an den Aufrufer zurückgegeben.
\subsubsection{restoreSourceByPoi}
\paragraph{Parameter} Die Funktion besitzt folgende Parameter:
\begin{table}[H]
	\begin{tabular}{|c|p{11cm}|}
		\hline
		\textbf{Parametername} & \textbf{Parameterbeschreibung} \\ \hline
		\$poiid     & Identifikator eines Interessenpunktes \\ \hline
	\end{tabular}
\end{table}
\paragraph{Beschreibung} Die Funktion stellt alle Quellen eines Interessenpunktes, welche als gelöscht markiert wurden, wieder her. Die Funktion hat Auswirkung auf folgende Quellen:
\begin{itemize}
	\item Tabelle mit Quellenangaben
\end{itemize}
Es findet bei dieser Funktion kein Abruf von Daten aus {\glqq COSP\grqq} statt. Die Antwort wird als strukturiertes Array an den Aufrufer zurückgegeben.
\subsubsection{deleteCinemasByPoi}
\paragraph{Parameter} Die Funktion besitzt folgende Parameter:
\begin{table}[H]
	\begin{tabular}{|c|p{11cm}|}
		\hline
		\textbf{Parametername} & \textbf{Parameterbeschreibung} \\ \hline
		\$poiid     & Identifikator eines Interessenpunktes \\ \hline
		\$overwrite & Legt fest, ob Daten direkt gelöscht werden sollen \\ \hline
	\end{tabular}
\end{table}
\paragraph{Beschreibung} Die Funktion löscht alle Saalanzahlen eines Interessenpunktes oder markiert diese als gelöscht. Die Funktion hat Auswirkung auf folgende Quellen:
\begin{itemize}
	\item Saalanzahl-Tabelle
\end{itemize}
Es findet bei dieser Funktion kein Abruf von Daten aus {\glqq COSP\grqq} statt. Die Antwort wird als strukturiertes Array an den Aufrufer zurückgegeben.
\subsubsection{restoreCinemasByPoi}
\paragraph{Parameter} Die Funktion besitzt folgende Parameter:
\begin{table}[H]
	\begin{tabular}{|c|p{11cm}|}
		\hline
		\textbf{Parametername} & \textbf{Parameterbeschreibung} \\ \hline
		\$poiid     & Identifikator eines Interessenpunktes \\ \hline
	\end{tabular}
\end{table}
\paragraph{Beschreibung} Die Funktion stellt alle Saalanzahlen eines Interessenpunktes, welche als gelöscht markiert wurden, wieder her. Die Funktion hat Auswirkung auf folgende Quellen:
\begin{itemize}
	\item Saalanzahl-Tabelle
\end{itemize}
Es findet bei dieser Funktion kein Abruf von Daten aus {\glqq COSP\grqq} statt. Die Antwort wird als strukturiertes Array an den Aufrufer zurückgegeben.
\subsubsection{deleteNamesByPoi}
\paragraph{Parameter} Die Funktion besitzt folgende Parameter:
\begin{table}[H]
	\begin{tabular}{|c|p{11cm}|}
		\hline
		\textbf{Parametername} & \textbf{Parameterbeschreibung} \\ \hline
		\$poiid     & Identifikator eines Interessenpunktes \\ \hline
		\$overwrite & Legt fest, ob Daten direkt gelöscht werden sollen \\ \hline
	\end{tabular}
\end{table}
\paragraph{Beschreibung} Die Funktion löscht alle Namen eines Interessenpunktes oder markiert diese als gelöscht. Die Funktion hat Auswirkung auf folgende Quellen:
\begin{itemize}
	\item Namen-Tabelle
\end{itemize}
Es findet bei dieser Funktion kein Abruf von Daten aus {\glqq COSP\grqq} statt. Die Antwort wird als strukturiertes Array an den Aufrufer zurückgegeben.
\subsubsection{restoreNamesByPoi}
\paragraph{Parameter} Die Funktion besitzt folgende Parameter:
\begin{table}[H]
	\begin{tabular}{|c|p{11cm}|}
		\hline
		\textbf{Parametername} & \textbf{Parameterbeschreibung} \\ \hline
		\$poiid     & Identifikator eines Interessenpunktes \\ \hline
	\end{tabular}
\end{table}
\paragraph{Beschreibung} Die Funktion stellt alle Namen eines Interessenpunktes, welche als gelöscht markiert wurden, wieder her. Die Funktion hat Auswirkung auf folgende Quellen:
\begin{itemize}
	\item Namen-Tabelle
\end{itemize}
Es findet bei dieser Funktion kein Abruf von Daten aus {\glqq COSP\grqq} statt. Die Antwort wird als strukturiertes Array an den Aufrufer zurückgegeben.
\subsubsection{deletePoiStoryByPoi}
\paragraph{Parameter} Die Funktion besitzt folgende Parameter:
\begin{table}[H]
	\begin{tabular}{|c|p{11cm}|}
		\hline
		\textbf{Parametername} & \textbf{Parameterbeschreibung} \\ \hline
		\$poiid     & Identifikator eines Interessenpunktes \\ \hline
		\$overwrite & Legt fest, ob Daten direkt gelöscht werden sollen \\ \hline
	\end{tabular}
\end{table}
\paragraph{Beschreibung} Die Funktion löscht alle Links eines Interessenpunktes mit Geschichten oder markiert diese als gelöscht. Die Funktion hat Auswirkung auf folgende Quellen:
\begin{itemize}
	\item Tabelle mit Links von Interessenpunkten und Geschichten
\end{itemize}
Es findet bei dieser Funktion kein Abruf von Daten aus {\glqq COSP\grqq} statt. Die Antwort wird als strukturiertes Array an den Aufrufer zurückgegeben.
\subsubsection{restorePoiStoryByPoi}
\paragraph{Parameter} Die Funktion besitzt folgende Parameter:
\begin{table}[H]
	\begin{tabular}{|c|p{11cm}|}
		\hline
		\textbf{Parametername} & \textbf{Parameterbeschreibung} \\ \hline
		\$poiid     & Identifikator eines Interessenpunktes \\ \hline
	\end{tabular}
\end{table}
\paragraph{Beschreibung} Die Funktion stellt alle Links eines Interessenpunktes mit Geschichten, welche als gelöscht markiert wurden, wieder her. Die Funktion hat Auswirkung auf folgende Quellen:
\begin{itemize}
	\item Tabelle mit Links von Interessenpunkten und Geschichten
\end{itemize}
Es findet bei dieser Funktion kein Abruf von Daten aus {\glqq COSP\grqq} statt. Die Antwort wird als strukturiertes Array an den Aufrufer zurückgegeben.
\subsubsection{deleteSeatsByPoi}
\paragraph{Parameter} Die Funktion besitzt folgende Parameter:
\begin{table}[H]
	\begin{tabular}{|c|p{11cm}|}
		\hline
		\textbf{Parametername} & \textbf{Parameterbeschreibung} \\ \hline
		\$poiid     & Identifikator eines Interessenpunktes \\ \hline
		\$overwrite & Legt fest, ob Daten direkt gelöscht werden sollen \\ \hline
	\end{tabular}
\end{table}
\paragraph{Beschreibung} Die Funktion löscht alle Sitzplatzanzahlen eines Interessenpunktes oder markiert diese als gelöscht. Die Funktion hat Auswirkung auf folgende Quellen:
\begin{itemize}
	\item Sitzplatzanzahl-Tabelle
\end{itemize}
Es findet bei dieser Funktion kein Abruf von Daten aus {\glqq COSP\grqq} statt. Die Antwort wird als strukturiertes Array an den Aufrufer zurückgegeben.
\subsubsection{restoreSeatsByPoi}
\paragraph{Parameter} Die Funktion besitzt folgende Parameter:
\begin{table}[H]
	\begin{tabular}{|c|p{11cm}|}
		\hline
		\textbf{Parametername} & \textbf{Parameterbeschreibung} \\ \hline
		\$poiid     & Identifikator eines Interessenpunktes \\ \hline
	\end{tabular}
\end{table}
\paragraph{Beschreibung} Die Funktion stellt alle Sitzplatzanzahlen eines Interessenpunktes, welche als gelöscht markiert wurden, wieder her. Die Funktion hat Auswirkung auf folgende Quellen:
\begin{itemize}
	\item Sitzplatzanzahl-Tabelle
\end{itemize}
Es findet bei dieser Funktion kein Abruf von Daten aus {\glqq COSP\grqq} statt. Die Antwort wird als strukturiertes Array an den Aufrufer zurückgegeben.
\subsubsection{deleteHistAddressByPoi}
\paragraph{Parameter} Die Funktion besitzt folgende Parameter:
\begin{table}[H]
	\begin{tabular}{|c|p{11cm}|}
		\hline
		\textbf{Parametername} & \textbf{Parameterbeschreibung} \\ \hline
		\$poiid     & Identifikator eines Interessenpunktes \\ \hline
		\$overwrite & Legt fest, ob Daten direkt gelöscht werden sollen \\ \hline
	\end{tabular}
\end{table}
\paragraph{Beschreibung} Die Funktion löscht alle historischen Adressen eines Interessenpunktes oder markiert diese als gelöscht. Die Funktion hat Auswirkung auf folgende Quellen:
\begin{itemize}
	\item Tabelle mit historischen Adressen
\end{itemize}
Es findet bei dieser Funktion kein Abruf von Daten aus {\glqq COSP\grqq} statt. Die Antwort wird als strukturiertes Array an den Aufrufer zurückgegeben.
\subsubsection{restoreHistAddressByPoi}
\paragraph{Parameter} Die Funktion besitzt folgende Parameter:
\begin{table}[H]
	\begin{tabular}{|c|p{11cm}|}
		\hline
		\textbf{Parametername} & \textbf{Parameterbeschreibung} \\ \hline
		\$poiid     & Identifikator eines Interessenpunktes \\ \hline
	\end{tabular}
\end{table}
\paragraph{Beschreibung} Die Funktion stellt alle historischen Adressen eines Interessenpunktes, welche als gelöscht markiert wurden, wieder her. Die Funktion hat Auswirkung auf folgende Quellen:
\begin{itemize}
	\item Tabelle mit historischen Adressen
\end{itemize}
Es findet bei dieser Funktion kein Abruf von Daten aus {\glqq COSP\grqq} statt. Die Antwort wird als strukturiertes Array an den Aufrufer zurückgegeben.
\subsubsection{deleteOperatorsByPoi}
\paragraph{Parameter} Die Funktion besitzt folgende Parameter:
\begin{table}[H]
	\begin{tabular}{|c|p{11cm}|}
		\hline
		\textbf{Parametername} & \textbf{Parameterbeschreibung} \\ \hline
		\$poiid     & Identifikator eines Interessenpunktes \\ \hline
		\$overwrite & Legt fest, ob Daten direkt gelöscht werden sollen \\ \hline
	\end{tabular}
\end{table}
\paragraph{Beschreibung} Die Funktion löscht alle Betreiber eines Interessenpunktes oder markiert diese als gelöscht. Die Funktion hat Auswirkung auf folgende Quellen:
\begin{itemize}
	\item Betreiber-Tabelle
\end{itemize}
Es findet bei dieser Funktion kein Abruf von Daten aus {\glqq COSP\grqq} statt. Die Antwort wird als strukturiertes Array an den Aufrufer zurückgegeben.
\subsubsection{restoreOperatorsByPoi}
\paragraph{Parameter} Die Funktion besitzt folgende Parameter:
\begin{table}[H]
	\begin{tabular}{|c|p{11cm}|}
		\hline
		\textbf{Parametername} & \textbf{Parameterbeschreibung} \\ \hline
		\$poiid     & Identifikator eines Interessenpunktes \\ \hline
	\end{tabular}
\end{table}
\paragraph{Beschreibung} Die Funktion stellt alle Betreiber eines Interessenpunktes, welche als gelöscht markiert wurden, wieder her. Die Funktion hat Auswirkung auf folgende Quellen:
\begin{itemize}
	\item Betreiber-Tabelle
\end{itemize}
Es findet bei dieser Funktion kein Abruf von Daten aus {\glqq COSP\grqq} statt. Die Antwort wird als strukturiertes Array an den Aufrufer zurückgegeben.
\subsubsection{deletePoiPicLinkByPoi}
\paragraph{Parameter} Die Funktion besitzt folgende Parameter:
\begin{table}[H]
	\begin{tabular}{|c|p{11cm}|}
		\hline
		\textbf{Parametername} & \textbf{Parameterbeschreibung} \\ \hline
		\$poiid     & Identifikator eines Interessenpunktes \\ \hline
		\$overwrite & Legt fest, ob Daten direkt gelöscht werden sollen \\ \hline
	\end{tabular}
\end{table}
\paragraph{Beschreibung} Die Funktion löscht alle Links eines Interessenpunktes mit Bildern oder markiert diese als gelöscht. Die Funktion hat Auswirkung auf folgende Quellen:
\begin{itemize}
	\item Tabelle mit Links von Interessenpunkten und Bildern
\end{itemize}
Es findet bei dieser Funktion kein Abruf von Daten aus {\glqq COSP\grqq} statt. Die Antwort wird als strukturiertes Array an den Aufrufer zurückgegeben.
\subsubsection{restorePoiPicLinkByPoi}
\paragraph{Parameter} Die Funktion besitzt folgende Parameter:
\begin{table}[H]
	\begin{tabular}{|c|p{11cm}|}
		\hline
		\textbf{Parametername} & \textbf{Parameterbeschreibung} \\ \hline
		\$poiid     & Identifikator eines Interessenpunktes \\ \hline
	\end{tabular}
\end{table}
\paragraph{Beschreibung} Die Funktion stellt alle Links eines Interessenpunktes mit Bildern, welche als gelöscht markiert wurden, wieder her. Die Funktion hat Auswirkung auf folgende Quellen:
\begin{itemize}
	\item Tabelle mit Links von Interessenpunkten und Bildern
\end{itemize}
Es findet bei dieser Funktion kein Abruf von Daten aus {\glqq COSP\grqq} statt. Die Antwort wird als strukturiertes Array an den Aufrufer zurückgegeben.
\subsubsection{GetDataForSingleMaterial}
\paragraph{Parameter} Die Funktion besitzt folgende Parameter:
\begin{table}[H]
	\begin{tabular}{|c|p{11cm}|}
		\hline
		\textbf{Parametername} & \textbf{Parameterbeschreibung} \\ \hline
		\$token & alphanumerischer Identifikator eines Bildes \\ \hline
	\end{tabular}
\end{table}
\paragraph{Beschreibung} Die Funktion fragt alle zum Daten zum Abrufen eines einzelnen Bildes ab. Die Funktion nutzt folgende Quellen:
\begin{itemize}
	\item COSP
\end{itemize}
Es findet bei dieser Funktion ein Abruf von Daten aus {\glqq COSP\grqq} statt. Die Antwort wird als strukturiertes Array an den Aufrufer zurückgegeben.
\subsubsection{SaveDataForSingleMaterial}
\paragraph{Parameter} Die Funktion besitzt folgende Parameter:
\begin{table}[H]
	\begin{tabular}{|c|p{11cm}|}
		\hline
		\textbf{Parametername} & \textbf{Parameterbeschreibung} \\ \hline
		\$token       & alphanumerischer Identifikator eines Bildes \\ \hline
		\$title       & Titel des Bildes \\ \hline
		\$description & Beschreibung des Bildes \\ \hline
		\$source      & Quellenangabe (optional) \\ \hline
		\$sourceType  & Identifikator des Typs der Quelle \\ \hline
	\end{tabular}
\end{table}
\paragraph{Beschreibung} Die Funktion aktualisiert den Titel und/oder die Beschreibung eines Bildes in {\glqq COSP\grqq}. Die Funktion hat Auswirkungen auf folgende Quellen:
\begin{itemize}
	\item COSP
\end{itemize}
Es findet bei dieser Funktion kein Abruf von Daten aus {\glqq COSP\grqq} statt. Es werden jedoch Daten an {\glqq COSP\grqq} gesendet. Die Antwort wird als strukturiertes Array an den Aufrufer zurückgegeben.
\subsubsection{saveStoryEditedDataToCOSP}
\paragraph{Parameter} Die Funktion besitzt folgende Parameter:
\begin{table}[H]
	\begin{tabular}{|c|p{11cm}|}
		\hline
		\textbf{Parametername} & \textbf{Parameterbeschreibung} \\ \hline
		\$token & alphanumerischer Identifikator einer Geschichte \\ \hline
		\$title & Titel der Geschichte \\ \hline
		\$story & Inhalt der Geschichte \\ \hline
	\end{tabular}
\end{table}
\paragraph{Beschreibung} Die Funktion aktualisiert den Titel und/oder den Inhalt einer Geschichte in {\glqq COSP\grqq}. Die Funktion hat Auswirkungen auf folgende Quellen:
\begin{itemize}
	\item COSP
\end{itemize}
Es findet bei dieser Funktion kein Abruf von Daten aus {\glqq COSP\grqq} statt. Es werden jedoch Daten an {\glqq COSP\grqq} gesendet. Die Antwort wird als strukturiertes Array an den Aufrufer zurückgegeben.
\subsubsection{resetUserPassword}
\paragraph{Parameter} Die Funktion besitzt folgende Parameter:
\begin{table}[H]
	\begin{tabular}{|c|p{11cm}|}
		\hline
		\textbf{Parametername} & \textbf{Parameterbeschreibung} \\ \hline
		\$username & Nutzername \\ \hline
	\end{tabular}
\end{table}
\paragraph{Beschreibung} Die Funktion fordert ein zurücksetzen des Passwortes in {\glqq COSP\grqq} an. Die Funktion hat Auswirkungen auf folgende Quellen:
\begin{itemize}
	\item COSP
\end{itemize}
Es findet bei dieser Funktion kein Abruf von Daten aus {\glqq COSP\grqq} statt. Es werden jedoch Daten an {\glqq COSP\grqq} gesendet. Die Antwort wird als strukturiertes Array an den Aufrufer zurückgegeben.
\subsubsection{getCompleteInformationOfPoiNames}
\paragraph{Parameter} Die Funktion besitzt folgende Parameter:
\begin{table}[H]
	\begin{tabular}{|c|p{11cm}|}
		\hline
		\textbf{Parametername} & \textbf{Parameterbeschreibung} \\ \hline
		\$poiid      & Identifikator eines Interessenpunktes \\ \hline
		\$main\_name & Name des Interessenpunktes aus der Interessenpunkt-Tabelle \\ \hline
	\end{tabular}
\end{table}
\paragraph{Beschreibung} Die Funktion fragt alle Namen eines Interessenpunktes mit Validierungswerten ab. Die Funktion nutzt folgende Quellen:
\begin{itemize}
	\item Namen-Tabelle
	\item Tabelle mit Validierungsinformationen zu Namen
\end{itemize}
Es findet bei dieser Funktion kein Abruf von Daten aus {\glqq COSP\grqq} statt. Die Antwort wird als strukturiertes Array an den Aufrufer zurückgegeben.
\subsubsection{getCompleteInformationOfPoiOperators}
\paragraph{Parameter} Die Funktion besitzt folgende Parameter:
\begin{table}[H]
	\begin{tabular}{|c|p{11cm}|}
		\hline
		\textbf{Parametername} & \textbf{Parameterbeschreibung} \\ \hline
		\$poiid      & Identifikator eines Interessenpunktes \\ \hline
	\end{tabular}
\end{table}
\paragraph{Beschreibung} Die Funktion fragt alle Betreiber eines Interessenpunktes mit Validierungswerten ab. Die Funktion nutzt folgende Quellen:
\begin{itemize}
	\item Betreiber-Tabelle
	\item Tabelle mit Validierungsinformationen zu Betreibern
\end{itemize}
Es findet bei dieser Funktion kein Abruf von Daten aus {\glqq COSP\grqq} statt. Die Antwort wird als strukturiertes Array an den Aufrufer zurückgegeben.
\subsubsection{getCompleteInformationOfPoiHistAddress}
\paragraph{Parameter} Die Funktion besitzt folgende Parameter:
\begin{table}[H]
	\begin{tabular}{|c|p{11cm}|}
		\hline
		\textbf{Parametername} & \textbf{Parameterbeschreibung} \\ \hline
		\$poiid      & Identifikator eines Interessenpunktes \\ \hline
	\end{tabular}
\end{table}
\paragraph{Beschreibung} Die Funktion fragt alle historischen Adressen eines Interessenpunktes mit Validierungswerten ab. Die Funktion nutzt folgende Quellen:
\begin{itemize}
	\item Tabelle mit historischen Adressen
	\item Tabelle mit Validierungsinformationen zu historischen Adressen
\end{itemize}
Es findet bei dieser Funktion kein Abruf von Daten aus {\glqq COSP\grqq} statt. Die Antwort wird als strukturiertes Array an den Aufrufer zurückgegeben.
\subsubsection{deleteMaterial}
\paragraph{Parameter} Die Funktion besitzt folgende Parameter:
\begin{table}[H]
	\begin{tabular}{|c|p{11cm}|}
		\hline
		\textbf{Parametername} & \textbf{Parameterbeschreibung} \\ \hline
		\$picToken      & alphanumerischer Identifikator eines Bildes \\ \hline
	\end{tabular}
\end{table}
\paragraph{Beschreibung} Die Funktion fragt die Löschung eines Bildes in {\glqq COSP\grqq} an. Die Funktion hat Auswirkungen auf folgende Quellen:
\begin{itemize}
	\item COSP
\end{itemize}
Es findet bei dieser Funktion kein Abruf von Daten aus {\glqq COSP\grqq} statt. Es werden jedoch Daten an {\glqq COSP\grqq} gesendet. Die Antwort wird als strukturiertes Array an den Aufrufer zurückgegeben.
\subsubsection{getStoriesAsListFromCOSP}
\paragraph{Parameter} Die Funktion besitzt folgende Parameter:
\begin{table}[H]
	\begin{tabular}{|c|p{11cm}|}
		\hline
		\textbf{Parametername} & \textbf{Parameterbeschreibung} \\ \hline
		\$tokenList      & Array mit alphanumerischen Identifikatoren von Geschichten \\ \hline
	\end{tabular}
\end{table}
\paragraph{Beschreibung} Die Funktion fragt alle Daten zum Abrufen mehrerer bestimmter Geschichten aus {\glqq COSP\grqq} an. Die Funktion nutzt folgende Quellen:
\begin{itemize}
	\item COSP
\end{itemize}
Es findet bei dieser Funktion ein Abruf von Daten aus {\glqq COSP\grqq} statt. Die Antwort wird als strukturiertes Array an den Aufrufer zurückgegeben.
\subsubsection{deleteMaterial}
\paragraph{Parameter} Die Funktion besitzt folgende Parameter:
\begin{table}[H]
	\begin{tabular}{|c|p{11cm}|}
		\hline
		\textbf{Parametername} & \textbf{Parameterbeschreibung} \\ \hline
		\$data      & alphanumerischer Identifikator einer Geschichte \\ \hline
	\end{tabular}
\end{table}
\paragraph{Beschreibung} Die Funktion fragt die Löschung einer Geschichte in {\glqq COSP\grqq} an. Die Funktion hat Auswirkungen auf folgende Quellen:
\begin{itemize}
	\item COSP
\end{itemize}
Es findet bei dieser Funktion kein Abruf von Daten aus {\glqq COSP\grqq} statt. Es werden jedoch Daten an {\glqq COSP\grqq} gesendet. Die Antwort wird als strukturiertes Array an den Aufrufer zurückgegeben.
\subsubsection{GetPoiPicLinkValidators}
\paragraph{Parameter} Die Funktion besitzt folgende Parameter:
\begin{table}[H]
	\begin{tabular}{|c|p{11cm}|}
		\hline
		\textbf{Parametername} & \textbf{Parameterbeschreibung} \\ \hline
		\$lid      & Identifikatoren eines Links zwischen einem Bild und einer Geschichte \\ \hline
	\end{tabular}
\end{table}
\paragraph{Beschreibung} Die Funktion fragt alle Validatoren eines Links zwischen einer Geschichte und einem Bild in {\glqq COSP\grqq} ab. Die Funktion nutzt folgende Quellen:
\begin{itemize}
	\item COSP
\end{itemize}
Es findet bei dieser Funktion ein Abruf von Daten aus {\glqq COSP\grqq} statt. Die Antwort wird als strukturiertes Array an den Aufrufer zurückgegeben.
\subsubsection{getCompleteInformationOfPoiSeats}
\paragraph{Parameter} Die Funktion besitzt folgende Parameter:
\begin{table}[H]
	\begin{tabular}{|c|p{11cm}|}
		\hline
		\textbf{Parametername} & \textbf{Parameterbeschreibung} \\ \hline
		\$poiid      & Identifikator eines Interessenpunktes \\ \hline
	\end{tabular}
\end{table}
\paragraph{Beschreibung} Die Funktion fragt alle Sitzplatzanzahlen eines Interessenpunktes mit Validierungswerten ab. Die Funktion nutzt folgende Quellen:
\begin{itemize}
	\item Sitzplatzanzahl-Tabelle
	\item Tabelle mit Validierungswerten zu Sitzplatzanzahlen
\end{itemize}
Es findet bei dieser Funktion kein Abruf von Daten aus {\glqq COSP\grqq} statt. Die Antwort wird als strukturiertes Array an den Aufrufer zurückgegeben.
\subsubsection{getValidationsByUserForPoiSeats}
\paragraph{Parameter} Die Funktion besitzt folgende Parameter:
\begin{table}[H]
	\begin{tabular}{|c|p{11cm}|}
		\hline
		\textbf{Parametername} & \textbf{Parameterbeschreibung} \\ \hline
		\$uid & Angabe eines Nutzeridentifikators \\ \hline
	\end{tabular}
\end{table}
\paragraph{Beschreibung} Die Funktion liefert alle Identifikatoren von Sitzplatzanzahlen, welche bereits durch den Nutzer validiert wurden. Die Funktion nutzt folgende Quellen:
\begin{itemize}
	\item Tabelle mit Validierungsdaten zu Sitzplatzanzahlen
\end{itemize}
Es findet bei dieser Funktion kein Abruf von Daten aus {\glqq COSP\grqq} statt. Die Antwort wird als strukturiertes Array an den Aufrufer zurückgegeben.
\subsubsection{validatePoiSeats}
\paragraph{Parameter} Die Funktion besitzt folgende Parameter:
\begin{table}[H]
	\begin{tabular}{|c|p{11cm}|}
		\hline
		\textbf{Parametername} & \textbf{Parameterbeschreibung} \\ \hline
		\$seatid & Identifikator der Sitzplatzanzahl \\ \hline
	\end{tabular}
\end{table}
\paragraph{Beschreibung} Die Funktion fügt einer Sitzplatzanzahl eine Validierung hinzu. Die Funktion hat Auswirkungen auf folgende Quellen:
\begin{itemize}
	\item Tabelle mit Validierungsdaten zu Sitzplatzanzahlen
\end{itemize}
Es findet bei dieser Funktion kein Abruf von Daten aus {\glqq COSP\grqq} statt. Die Antwort wird als strukturiertes Array an den Aufrufer zurückgegeben.
\subsubsection{getCompleteInformationOfPoiCinemas}
\paragraph{Parameter} Die Funktion besitzt folgende Parameter:
\begin{table}[H]
	\begin{tabular}{|c|p{11cm}|}
		\hline
		\textbf{Parametername} & \textbf{Parameterbeschreibung} \\ \hline
		\$poiid      & Identifikator eines Interessenpunktes \\ \hline
	\end{tabular}
\end{table}
\paragraph{Beschreibung} Die Funktion fragt alle Saalanzahlen eines Interessenpunktes mit Validierungswerten ab. Die Funktion nutzt folgende Quellen:
\begin{itemize}
	\item Saalanzahl-Tabelle
	\item Tabelle mit Validierungswerten zu Saalanzahlen
\end{itemize}
Es findet bei dieser Funktion kein Abruf von Daten aus {\glqq COSP\grqq} statt. Die Antwort wird als strukturiertes Array an den Aufrufer zurückgegeben.
\subsubsection{getValidationsByUserForPoiCinemas}
\paragraph{Parameter} Die Funktion besitzt folgende Parameter:
\begin{table}[H]
	\begin{tabular}{|c|p{11cm}|}
		\hline
		\textbf{Parametername} & \textbf{Parameterbeschreibung} \\ \hline
		\$uid & Angabe eines Nutzeridentifikators \\ \hline
	\end{tabular}
\end{table}
\paragraph{Beschreibung} Die Funktion liefert alle Identifikatoren von Sitzplatzanzahlen, welche bereits durch den Nutzer validiert wurden. Die Funktion nutzt folgende Quellen:
\begin{itemize}
	\item Tabelle mit Validierungsdaten zu Saalanzahlen
\end{itemize}
Es findet bei dieser Funktion kein Abruf von Daten aus {\glqq COSP\grqq} statt. Die Antwort wird als strukturiertes Array an den Aufrufer zurückgegeben.
\subsubsection{validatePoiCinemas}
\paragraph{Parameter} Die Funktion besitzt folgende Parameter:
\begin{table}[H]
	\begin{tabular}{|c|p{11cm}|}
		\hline
		\textbf{Parametername} & \textbf{Parameterbeschreibung} \\ \hline
		\$cinemasid & Identifikator einer Saalanzahl \\ \hline
	\end{tabular}
\end{table}
\paragraph{Beschreibung} Die Funktion fügt einer Sitzplatzanzahl eine Validierung hinzu. Die Funktion hat Auswirkungen auf folgende Quellen:
\begin{itemize}
	\item Tabelle mit Validierungsdaten zu Saalanzahlen
\end{itemize}
Es findet bei dieser Funktion kein Abruf von Daten aus {\glqq COSP\grqq} statt. Die Antwort wird als strukturiertes Array an den Aufrufer zurückgegeben.
\subsubsection{getValidationsByUserForCinemaType}
\paragraph{Parameter} Die Funktion besitzt folgende Parameter:
\begin{table}[H]
	\begin{tabular}{|c|p{11cm}|}
		\hline
		\textbf{Parametername} & \textbf{Parameterbeschreibung} \\ \hline
		\$uid & Angabe eines Nutzeridentifikators \\ \hline
	\end{tabular}
\end{table}
\paragraph{Beschreibung} Die Funktion liefert alle Identifikatoren von Interessenpunkten, bei welchen der Typ bereits durch den Nutzer validiert wurden. Die Funktion nutzt folgende Quellen:
\begin{itemize}
	\item Tabelle mit Validierungsdaten zum Typ eines Interessenpunktes
\end{itemize}
Es findet bei dieser Funktion kein Abruf von Daten aus {\glqq COSP\grqq} statt. Die Antwort wird als strukturiertes Array an den Aufrufer zurückgegeben.
\subsubsection{validateTypePoi}
\paragraph{Parameter} Die Funktion besitzt folgende Parameter:
\begin{table}[H]
	\begin{tabular}{|c|p{11cm}|}
		\hline
		\textbf{Parametername} & \textbf{Parameterbeschreibung} \\ \hline
		\$poiid & Identifikator eines Interessenpunktes \\ \hline
	\end{tabular}
\end{table}
\paragraph{Beschreibung} Die Funktion fügt einem Interessenpunkt eine Validierung für seinen Typ hinzu. Die Funktion hat Auswirkungen auf folgende Quellen:
\begin{itemize}
	\item Tabelle mit Validierungsdaten zum Typ eines Interessenpunktes
\end{itemize}
Es findet bei dieser Funktion kein Abruf von Daten aus {\glqq COSP\grqq} statt. Es gibt keinen Rückgabewert.
\subsubsection{RedirectMainBetaIndex}
\paragraph{Parameter} Die Funktion besitzt keine Parameter.
\paragraph{Beschreibung} Die Funktion leitet den bereits im Wartungs- beziehungsweise Beta-Modus angemeldeten Nutzern wider auf die Startseite ohne die URL nochmals eingeben zu müssen. Es findet bei dieser Funktion kein Abruf von Daten aus {\glqq COSP\grqq} statt. Die Antwort wird als strukturiertes Array an den Aufrufer zurückgegeben.
\subsubsection{SendStoryApprovalChange}
\paragraph{Parameter} Die Funktion besitzt folgende Parameter:
\begin{table}[H]
	\begin{tabular}{|c|p{11cm}|}
		\hline
		\textbf{Parametername} & \textbf{Parameterbeschreibung} \\ \hline
		\$story\_token & alphanumerischer Identifikator einer Geschichte \\ \hline
		\$state        & Status der Freischaltung der Geschichte \\ \hline
	\end{tabular}
\end{table}
\paragraph{Beschreibung} Die Funktion schaltet oder sperrt eine Geschichte in {\glqq COSP\grqq}. Die Funktion hat Auswirkungen auf folgende Quellen:
\begin{itemize}
	\item COSP
\end{itemize}
Es findet bei dieser Funktion kein Abruf von Daten aus {\glqq COSP\grqq} statt. Es werden jedoch Daten an {\glqq COSP\grqq} gesendet. Es gibt keine Antwort.
\subsubsection{GetCaptchaFromCOSP}
\paragraph{Parameter} Die Funktion besitzt keine Parameter.
\paragraph{Beschreibung} Die Funktion ruft ein Captcha von {\glqq COSP\grqq} ab. Die Funktion nutzt folgende Quellen:
\begin{itemize}
	\item Konfigurationsdatei
\end{itemize}
Es findet bei dieser Funktion ein Abruf von Daten aus {\glqq COSP\grqq} statt. Die Antwort wird als strukturiertes Array an den Aufrufer zurückgegeben.
\subsubsection{sendContactMail}
\paragraph{Parameter} Die Funktion besitzt folgende Parameter:
\begin{table}[H]
	\begin{tabular}{|c|p{11cm}|}
		\hline
		\textbf{Parametername} & \textbf{Parameterbeschreibung} \\ \hline
		\$mail    & Mailadresse des Senders \\ \hline
		\$subject & Betreff der Mail \\ \hline
		\$msg     & Inhalt der Mail \\ \hline
	\end{tabular}
\end{table}
\paragraph{Beschreibung} Die Funktion fordert das senden einer Kontakmail durch {\glqq COSP\grqq} an. Die Funktion nutzt folgende Quellen:
\begin{itemize}
	\item Konfigurationsdatei
\end{itemize}
Es findet bei dieser Funktion kein Abruf von Daten aus {\glqq COSP\grqq} statt. Es werden jedoch Daten an {\glqq COSP\grqq} gesendet. Die Antwort wird als strukturiertes Array an den Aufrufer zurückgegeben.
\subsubsection{getAllSourceTypesCOSP}
\paragraph{Parameter} Die Funktion besitzt keine Parameter.
\paragraph{Beschreibung} Die Funktion ruft alle Quelltypen von COSP ab. Es findet bei dieser Funktion ein Abruf von Daten aus {\glqq COSP\grqq} statt. Die Antwort wird als strukturiertes Array an den Aufrufer zurückgegeben.
\subsubsection{getUserIp}
\paragraph{Parameter} Die Funktion besitzt keine Parameter.
\paragraph{Beschreibung} Die Funktion bestimmt die IP-Adresse des Aufrufenden Nutzers. Es findet bei dieser Funktion ein Abruf von Daten aus {\glqq COSP\grqq} statt. Die Antwort wird als strukturiertes Array an den Aufrufer zurückgegeben.