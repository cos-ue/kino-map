\label{lang:de}
\subsection{Allgemeines} Diese Datei enthält ein Array mit Werten um eine Mehrsprachlichkeit in Teilen zu ermöglichen. Speziell beinhaltet diese Datei die Deutsche Version
\begin{table}[H]
	\begin{tabular}{|c|p{11cm}|}
		\hline
		\textbf{Einbindungspunkt} & inc.php \\ \hline
		\textbf{Einbindungspunkt} & inc-sub.php \\ \hline
	\end{tabular}
\end{table}
Die Datei ist nicht direkt durch den Nutzer aufrufbar, dies wird durch folgenden Code-Ausschnitt sichergestellt:
\begin{lstlisting}[language=php]
if (!defined('NICE_PROJECT')) {
	die('Permission denied.');
}
\end{lstlisting}
Der Globale Wert {\glqq NICE\_PROJECT\grqq} wird durch für den Nutzer valide Aufrufpunkte festgelegt, z.B. {\glqq api.php\grqq}.
\newpage
\subsection{Aufbau}
\subsubsection{Array}
Das Nachfolgend zu sehende Array beinhaltet alle verwendeten Schlüssel um eine Mehrsprachlichkeit zu ermöglichen. Viele der Texte sind jedoch nicht durch einen Schlüssel hinterlegt.
\begin{lstlisting}[language=php]
$lang = array(
	"Lehrer"=>"Lehrer",
	"NeuesBenutzerkonto"=>"Neues Benutzerkonto",
	"Erstellen"=>"Erstellen",
	"Klassenname"=>"Klassenname",
	"NeueKlasse"=>"Neue Klasse",
	"NeuesPasswort"=>"Neues Passwort",
	"Rolle"=>"Rolle",
	"Benutzername"=>"Benutzername",
	"Vorname"=>"Vorname",
	"Nachname"=>"Nachname",
	"Löschen"=>"Löschen",
	"Klassdel"=>"Klasse löschen",
	"CSV"=>"CSV-Datei hochladen",
	"Benutzerkontenverwalten"=>"Benutzerkonten verwalten",
	"Kommentare"=>"Persönlicher Bereich",
	"Speichern"=>"Speichern",
	"Spielstätte"=>"Spielstätte",
	"Koordinaten"=>"Koordinaten",
	"Auswahl"=>"Wähle ein Bild (*.png oder *.jpeg) aus.",
	"FotoVideo"=>"Foto oder Video",
	"Geschichte"=>"Hintergrundinformationen zum Kino",
	"Besonderheiten"=>"Besonderheiten",
	"Hist_Adresse"=>"Historische Adresse",
	"Akt_Adresse"=>"Aktuelle Adresse *",
	"Längengrad"=>"Längengrad",
	"Breitengrad"=>"Breitengrad",
	"Name"=>"Name (Pflichtfeld)",
	"Poi"=>"Interessenpunkt hinzufügen",
	"Abmelden"=>"Abmelden",
	"Suche"=>"Kino-Suche",
	"Klasse"=>"Klasse wählen",
	"Sprachen"=>"Sprachen",
	"Benutzerkonten"=>"Benutzerkonten",
	"infos" => "Persönlicher Bereich",
	"lang_en" => "Englisch",
	"lang_de" => "Deutsch",
	"Betreiber" => "Betreiber",
	"betrieb_von" => "Betrieb von",
	"betrieb_bis" => "Betrieb bis",
	"Verw" => "Kartenfunktionen",
	"eintragsverwaltung" => "Eintragsverwaltung"
);
\end{lstlisting}
\subsubsection{Besonderheiten}
Aktuell werden keine neuen Schlüssel hinzugefügt und die Ermöglichung einer Mehrsprachlichkeit nicht verfolgt.