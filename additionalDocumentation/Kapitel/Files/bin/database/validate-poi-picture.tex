\subsection{Allgemeines} Diese Datei enthält alle Funktionen, welche die Tabelle mit Validierungsinformationen zu Links zwischen Bildern und Interessenpunkten benutzen.
\begin{table}[H]
	\begin{tabular}{|c|p{11cm}|}
		\hline
		\textbf{Einbindungspunkt} & inc-db.php \\ \hline
		\textbf{Einbindungspunkt} & inc-db-sub.php \\ \hline
	\end{tabular}
\end{table}
Die Datei ist nicht direkt durch den Nutzer aufrufbar, dies wird durch folgenden Code-Ausschnitt sichergestellt:
\begin{lstlisting}[language=php]
	if (!defined('NICE_PROJECT')) {
		die('Permission denied.');
	}
\end{lstlisting}
Der Globale Wert {\glqq NICE\_PROJECT\grqq} wird durch für den Nutzer valide Aufrufpunkte festgelegt, z.B. {\glqq api.php\grqq}.
\newpage
\subsection{Funktionen}
\subsubsection{insertValidatePoiPicLink}
\paragraph{Parameter} Die Funktion besitzt folgende Parameter:
\begin{table}[H]
	\begin{tabular}{|c|p{11cm}|}
		\hline
		\textbf{Parametername} & \textbf{Parameterbeschreibung} \\ \hline
		\$pic\_poi\_id & Identifikator eines Links zwischen einem Bild und einem Interessenpunkt \\ \hline
		\$value        & Wert der Validierung \\ \hline
	\end{tabular}
\end{table}
\paragraph{Beschreibung} Die Funktion fügt einem Link zwischen einem Bild und einem Interessenpunkt eine neue Validierung hinzu. Die Funktion hat Auswirkungen auf folgende Quellen
\begin{itemize}
	\item Tabelle mit Validierungsinformationen zu Links zwischen Bildern und Interessenpunkten
\end{itemize}
Es findet bei dieser Funktion kein Abruf von Daten aus {\glqq COSP\grqq} statt. Es gibt einen Rückgabewert.
\subsubsection{getValidateSumPoiPic}
\paragraph{Parameter} Die Funktion besitzt folgende Parameter:
\begin{table}[H]
	\begin{tabular}{|c|p{11cm}|}
		\hline
		\textbf{Parametername} & \textbf{Parameterbeschreibung} \\ \hline
		\$pic\_poi\_id & Identifikator eines Links zwischen einem Bild und einem Interessenpunkt \\ \hline
	\end{tabular}
\end{table}
\paragraph{Beschreibung} Die Funktion ruft die Summe der Validierungen eines Links zwischen einem Bild und einem Interessenpunkt ab. Die Funktion nutzt folgende Quellen:
\begin{itemize}
	\item Tabelle mit Validierungsinformationen zu Links zwischen Bildern und Interessenpunkten
\end{itemize}
Es findet bei dieser Funktion kein Abruf von Daten aus {\glqq COSP\grqq} statt. Die Antwort wird als strukturiertes Array an den Aufrufer zurückgegeben.
\subsubsection{getAllValidatedForPoiPicLink}
\paragraph{Parameter} Die Funktion besitzt keine Parameter.
\paragraph{Beschreibung} Die Funktion ruft alle Validierungsinformationen zu Links zwischen Bildern und Interessenpunkten ab. Die Funktion nutzt folgende Quellen:
\begin{itemize}
	\item Tabelle mit Validierungsinformationen zu Links zwischen Bildern und Interessenpunkten
\end{itemize}
Es findet bei dieser Funktion kein Abruf von Daten aus {\glqq COSP\grqq} statt. Die Antwort wird als strukturiertes Array an den Aufrufer zurückgegeben.
\subsubsection{deleteValidationsForCertainPoiPicLink}
\paragraph{Parameter} Die Funktion besitzt folgende Parameter:
\begin{table}[H]
	\begin{tabular}{|c|p{11cm}|}
		\hline
		\textbf{Parametername} & \textbf{Parameterbeschreibung} \\ \hline
		\$lid & Identifikator eines Links zwischen einem Bild und einem Interessenpunkt \\ \hline
	\end{tabular}
\end{table}
\paragraph{Beschreibung} Die Funktion löscht alle Validierungen eines Links zwischen einem Bild und einem Interessenpunkt. Die Funktion hat Auswirkungen auf folgende Quellen
\begin{itemize}
	\item Tabelle mit Validierungsinformationen zu Links zwischen Bildern und Interessenpunkten
\end{itemize}
Es findet bei dieser Funktion kein Abruf von Daten aus {\glqq COSP\grqq} statt. Es gibt einen Rückgabewert.