\subsection{Allgemeines} Diese Datei enthält alle Funktionen für die Ankündigungstabelle-Tabelle.
\begin{table}[H]
	\begin{tabular}{|c|p{11cm}|}
		\hline
		\textbf{Einbindungspunkt} & inc-db.php \\ \hline
		\textbf{Einbindungspunkt} & inc-db-sub.php \\ \hline
	\end{tabular}
\end{table}
Die Datei ist nicht direkt durch den Nutzer aufrufbar, dies wird durch folgenden Code-Ausschnitt sichergestellt:
\begin{lstlisting}[language=php]
if (!defined('NICE_PROJECT')) {
	die('Permission denied.');
}
\end{lstlisting}
Der Globale Wert {\glqq NICE\_PROJECT\grqq} wird durch für den Nutzer valide Aufrufpunkte festgelegt, z.B. {\glqq api.php\grqq}.
\newpage
\subsection{Funktionen}
\subsubsection{addAnnouncement}
\paragraph{Parameter} Die Funktion besitzt folgende Parameter:
\begin{table}[H]
	\begin{tabular}{|c|p{11cm}|}
		\hline
		\textbf{Parametername} & \textbf{Parameterbeschreibung} \\ \hline
		\$title & Titel der Ankündigung \\ \hline
		\$content & Inhalt der Ankündigung \\ \hline
		\$start & Starttag der Ankündigung \\ \hline
		\$end & Endtag der Ankündigung \\ \hline
	\end{tabular}
\end{table}
\paragraph{Beschreibung} Die Funktion fügt eine neue Ankündigung hinzu. Die Funktion hat Auswirkungen auf folgende Quellen:
\begin{itemize}
	\item Ankündigungstabelle
\end{itemize}
Es findet bei dieser Funktion kein Abruf von Daten aus {\glqq COSP\grqq} statt. Die Antwort wird als strukturiertes Array an den Aufrufer zurückgegeben.
\subsubsection{getAllAnnouncements}
\paragraph{Parameter} Die Funktion besitzt keine Parameter.
\paragraph{Beschreibung} Die Funktion ruft alle bestehenden Ankündigungen ab. Die Funktion nutzt folgende Quellen:
\begin{itemize}
	\item Ankündigungstabelle
\end{itemize}
Es findet bei dieser Funktion kein Abruf von Daten aus {\glqq COSP\grqq} statt. Die Antwort wird als strukturiertes Array an den Aufrufer zurückgegeben.
\subsubsection{getAnnouncement}
\paragraph{Parameter} Die Funktion besitzt folgende Parameter:
\begin{table}[H]
	\begin{tabular}{|c|p{11cm}|}
		\hline
		\textbf{Parametername} & \textbf{Parameterbeschreibung} \\ \hline
		\$id & Identifikator einer Ankündigung\\ \hline
	\end{tabular}
\end{table}
\paragraph{Beschreibung} Die Funktion ruft die Daten einer bestimmten Ankündigung ab. Die Funktion nutzt folgende Quellen:
\begin{itemize}
	\item Ankündigungstabelle
\end{itemize}
Es findet bei dieser Funktion kein Abruf von Daten aus {\glqq COSP\grqq} statt. Die Antwort wird als strukturiertes Array an den Aufrufer zurückgegeben.
\subsubsection{updateAnnouncement}
\paragraph{Parameter} Die Funktion besitzt folgende Parameter:
\begin{table}[H]
	\begin{tabular}{|c|p{11cm}|}
		\hline
		\textbf{Parametername} & \textbf{Parameterbeschreibung} \\ \hline
		\$id      & Identifikator der Ankündigung \\ \hline
		\$title   & Titel der Ankündigung \\ \hline
		\$content & Inhalt der Ankündigung \\ \hline
		\$start   & Starttag der Ankündigung \\ \hline
		\$end     & Endtag der Ankündigung \\ \hline
	\end{tabular}
\end{table}
\paragraph{Beschreibung} Die Funktion ändert eine Ankündigung. Die Funktion hat Auswirkungen auf folgende Quellen:
\begin{itemize}
	\item Ankündigungstabelle
\end{itemize}
Es findet bei dieser Funktion kein Abruf von Daten aus {\glqq COSP\grqq} statt. Die Antwort wird als strukturiertes Array an den Aufrufer zurückgegeben.
\subsubsection{deleteAnnouncement}
\paragraph{Parameter} Die Funktion besitzt folgende Parameter:
\begin{table}[H]
	\begin{tabular}{|c|p{11cm}|}
		\hline
		\textbf{Parametername} & \textbf{Parameterbeschreibung} \\ \hline
		\$id & Identifikator einer Ankündigung\\ \hline
	\end{tabular}
\end{table}
\paragraph{Beschreibung} Die Funktion löscht die Daten einer bestimmten Ankündigung. Die Funktion nutzt folgende Quellen:
\begin{itemize}
	\item Ankündigungstabelle
\end{itemize}
Es findet bei dieser Funktion kein Abruf von Daten aus {\glqq COSP\grqq} statt. Die Antwort wird als strukturiertes Array an den Aufrufer zurückgegeben.
\subsubsection{getCurrentAnnouncement}
\paragraph{Parameter} Die Funktion besitzt keine Parameter.
\paragraph{Beschreibung} Die Funktion ruft alle aktuellen Ankündigungen ab. Die Funktion nutzt folgende Quellen:
\begin{itemize}
	\item Ankündigungstabelle
\end{itemize}
Es findet bei dieser Funktion kein Abruf von Daten aus {\glqq COSP\grqq} statt. Die Antwort wird als strukturiertes Array an den Aufrufer zurückgegeben.
\subsubsection{updateAktivationStateAnnouncement}
\paragraph{Parameter} Die Funktion besitzt folgende Parameter:
\begin{table}[H]
	\begin{tabular}{|c|p{11cm}|}
		\hline
		\textbf{Parametername} & \textbf{Parameterbeschreibung} \\ \hline
		\$id    & Identifikator einer Ankündigung \\ \hline
		\$state & Aktivierungsstatus einer Ankündigung \\ \hline
	\end{tabular}
\end{table}
\paragraph{Beschreibung} Die Funktion setzt den Aktivierungsstatus einer bestimmten Ankündigung. Die Funktion hat Auswirkungen auf folgende Quellen:
\begin{itemize}
	\item Ankündigungstabelle
\end{itemize}
Es findet bei dieser Funktion kein Abruf von Daten aus {\glqq COSP\grqq} statt. Die Antwort wird als strukturiertes Array an den Aufrufer zurückgegeben.