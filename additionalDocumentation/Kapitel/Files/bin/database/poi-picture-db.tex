\subsection{Allgemeines} Diese Datei enthält alle Funktionen, welche die Tabelle mit Links zwischen Bildern und Interessenpunkten benutzen.
\begin{table}[H]
	\begin{tabular}{|c|p{11cm}|}
		\hline
		\textbf{Einbindungspunkt} & inc-db.php \\ \hline
		\textbf{Einbindungspunkt} & inc-db-sub.php \\ \hline
	\end{tabular}
\end{table}
Die Datei ist nicht direkt durch den Nutzer aufrufbar, dies wird durch folgenden Code-Ausschnitt sichergestellt:
\begin{lstlisting}[language=php]
	if (!defined('NICE_PROJECT')) {
		die('Permission denied.');
	}
\end{lstlisting}
Der Globale Wert {\glqq NICE\_PROJECT\grqq} wird durch für den Nutzer valide Aufrufpunkte festgelegt, z.B. {\glqq api.php\grqq}.
\newpage
\subsection{Funktionen}
\subsubsection{insertPoiPicture}
\paragraph{Parameter} Die Funktion besitzt folgende Parameter:
\begin{table}[H]
	\begin{tabular}{|c|p{11cm}|}
		\hline
		\textbf{Parametername} & \textbf{Parameterbeschreibung} \\ \hline
		\$pictureId & alphanumerischer Identifikator eines Bildes \\ \hline
		\$PoiId     & Identifikator eines Interessenpunktes \\ \hline
	\end{tabular}
\end{table}
\paragraph{Beschreibung} Die Funktion fügt einen neuen Link zwischen einem Bild und einem Interessenpunkt hinzu. Die Funktion hat Auswirkungen auf folgende Quellen:
\begin{itemize}
	\item Tabelle mit Links zwischen Bildern und Interessenpunkten
\end{itemize}
Es findet bei dieser Funktion kein Abruf von Daten aus {\glqq COSP\grqq} statt. Es gibt einen Rückgabewert.
\subsubsection{getPicturesForPoi}
\paragraph{Parameter} Die Funktion besitzt folgende Parameter:
\begin{table}[H]
	\begin{tabular}{|c|p{11cm}|}
		\hline
		\textbf{Parametername} & \textbf{Parameterbeschreibung} \\ \hline
		\$poiid & Identifikator eines Interessenpunktes \\ \hline
	\end{tabular}
\end{table}
\paragraph{Beschreibung} Die Funktion fragt alle Links zwischen einem bestimmten Interessenpunkt und Bildern ab. Die Funktion nutzt folgende Quellen:
\begin{itemize}
	\item Tabelle mit Links zwischen Bildern und Interessenpunkten
\end{itemize}
Es findet bei dieser Funktion kein Abruf von Daten aus {\glqq COSP\grqq} statt. Die Antwort wird als strukturiertes Array an den Aufrufer zurückgegeben.
\subsubsection{getLinkIdPoiPic}
\paragraph{Parameter} Die Funktion besitzt folgende Parameter:
\begin{table}[H]
	\begin{tabular}{|c|p{11cm}|}
		\hline
		\textbf{Parametername} & \textbf{Parameterbeschreibung} \\ \hline
		\$poiid & Identifikator eines Interessenpunktes \\ \hline
		\$picid & alphanumerischer Identifikator eines Bildes \\ \hline
	\end{tabular}
\end{table}
\paragraph{Beschreibung} Die Funktion fragt den Identifikator eines Links zwischen einem bestimmten Interessenpunkt und einem bestimmten Bilder ab. Die Funktion nutzt folgende Quellen:
\begin{itemize}
	\item Tabelle mit Links zwischen Bildern und Interessenpunkten
\end{itemize}
Es findet bei dieser Funktion kein Abruf von Daten aus {\glqq COSP\grqq} statt. Die Antwort wird als strukturiertes Array an den Aufrufer zurückgegeben.
\subsubsection{getLinkIdsForPoi}
\paragraph{Parameter} Die Funktion besitzt folgende Parameter:
\begin{table}[H]
	\begin{tabular}{|c|p{11cm}|}
		\hline
		\textbf{Parametername} & \textbf{Parameterbeschreibung} \\ \hline
		\$poiid & Identifikator eines Interessenpunktes \\ \hline
	\end{tabular}
\end{table}
\paragraph{Beschreibung} Die Funktion fragt alle IDs von Links zwischen einem bestimmten Interessenpunkt und Bildern ab. Die Funktion nutzt folgende Quellen:
\begin{itemize}
	\item Tabelle mit Links zwischen Bildern und Interessenpunkten
\end{itemize}
Es findet bei dieser Funktion kein Abruf von Daten aus {\glqq COSP\grqq} statt. Die Antwort wird als strukturiertes Array an den Aufrufer zurückgegeben.
\subsubsection{getCreatorByPoiPicId}
\paragraph{Parameter} Die Funktion besitzt folgende Parameter:
\begin{table}[H]
	\begin{tabular}{|c|p{11cm}|}
		\hline
		\textbf{Parametername} & \textbf{Parameterbeschreibung} \\ \hline
		\$pic\_poi\_id & Identifikator eines Links zwischen einem Bild und einem Interessenpunkt \\ \hline
	\end{tabular}
\end{table}
\paragraph{Beschreibung} Die Funktion fragt den Ersteller eines Links zwischen einem Interessenpunkt und einem Bild ab. Die Funktion nutzt folgende Quellen:
\begin{itemize}
	\item Tabelle mit Links zwischen Bildern und Interessenpunkten
\end{itemize}
Es findet bei dieser Funktion kein Abruf von Daten aus {\glqq COSP\grqq} statt. Der Rückgabewert ist eine Zeichenkette.
\subsubsection{deleteCertainPoiPicLink}
\paragraph{Parameter} Die Funktion besitzt folgende Parameter:
\begin{table}[H]
	\begin{tabular}{|c|p{11cm}|}
		\hline
		\textbf{Parametername} & \textbf{Parameterbeschreibung} \\ \hline
		\$lid & Identifikator eines Links zwischen einem Bild und einem Interessenpunkt \\ \hline
	\end{tabular}
\end{table}
\paragraph{Beschreibung} Die Funktion löscht einen Link zwischen einem Interessenpunkt und einem Bild. Die Funktion hat Auswirkungen auf folgende Quellen:
\begin{itemize}
	\item Tabelle mit Links zwischen Bildern und Interessenpunkten
\end{itemize}
Es findet bei dieser Funktion kein Abruf von Daten aus {\glqq COSP\grqq} statt. Es gibt einen Rückgabewert.
\subsubsection{getPoisForPic}
\paragraph{Parameter} Die Funktion besitzt folgende Parameter:
\begin{table}[H]
	\begin{tabular}{|c|p{11cm}|}
		\hline
		\textbf{Parametername} & \textbf{Parameterbeschreibung} \\ \hline
		\$token & alphanumerischer Identifikator eines Bildes \\ \hline
	\end{tabular}
\end{table}
\paragraph{Beschreibung} Die Funktion fragt alle Links zwischen Interessenpunkten und einem bestimmten Bild ab. Die Funktion nutzt folgende Quellen:
\begin{itemize}
	\item Tabelle mit Links zwischen Bildern und Interessenpunkten
\end{itemize}
Es findet bei dieser Funktion kein Abruf von Daten aus {\glqq COSP\grqq} statt. Die Antwort wird als strukturiertes Array an den Aufrufer zurückgegeben.
\subsubsection{getLinkIdsPoiPic}
\paragraph{Parameter} Die Funktion besitzt folgende Parameter:
\begin{table}[H]
	\begin{tabular}{|c|p{11cm}|}
		\hline
		\textbf{Parametername} & \textbf{Parameterbeschreibung} \\ \hline
		\$picid & alphanumerischer Identifikator eines Bildes \\ \hline
	\end{tabular}
\end{table}
\paragraph{Beschreibung} Die Funktion fragt alle IDs von Links zwischen Interessenpunkten und einem bestimmten Bild ab. Die Funktion nutzt folgende Quellen:
\begin{itemize}
	\item Tabelle mit Links zwischen Bildern und Interessenpunkten
\end{itemize}
Es findet bei dieser Funktion kein Abruf von Daten aus {\glqq COSP\grqq} statt. Die Antwort wird als strukturiertes Array an den Aufrufer zurückgegeben.
\subsubsection{updateDeletionStateLinkPoiPicByID}
\paragraph{Parameter} Die Funktion besitzt folgende Parameter:
\begin{table}[H]
	\begin{tabular}{|c|p{11cm}|}
		\hline
		\textbf{Parametername} & \textbf{Parameterbeschreibung} \\ \hline
		\$id    & Identifikator eines Links zwischen einem Bild und einem Interessenpunkt \\ \hline
		\$state & Status der Löschung des Links \\ \hline
	\end{tabular}
\end{table}
\paragraph{Beschreibung} Die Funktion markiert einen Link zwischen einem Interessenpunkt und einem Bild als gelöscht. Die Funktion hat Auswirkungen auf folgende Quellen:
\begin{itemize}
	\item Tabelle mit Links zwischen Bildern und Interessenpunkten
\end{itemize}
Es findet bei dieser Funktion kein Abruf von Daten aus {\glqq COSP\grqq} statt. Es gibt einen Rückgabewert.
\subsubsection{updateDeletionPoiStateLinkPoiPicByID}
\paragraph{Parameter} Die Funktion besitzt folgende Parameter:
\begin{table}[H]
	\begin{tabular}{|c|p{11cm}|}
		\hline
		\textbf{Parametername} & \textbf{Parameterbeschreibung} \\ \hline
		\$id    & Identifikator eines Links zwischen einem Bild und einem Interessenpunkt \\ \hline
		\$state & Status der Löschung des Links \\ \hline
	\end{tabular}
\end{table}
\paragraph{Beschreibung} Die Funktion markiert einen Link zwischen einem Interessenpunkt und einem Bild als gelöscht, aufgrund eines gelöschten Interessenpunktes. Die Funktion hat Auswirkungen auf folgende Quellen:
\begin{itemize}
	\item Tabelle mit Links zwischen Bildern und Interessenpunkten
\end{itemize}
Es findet bei dieser Funktion kein Abruf von Daten aus {\glqq COSP\grqq} statt. Es gibt einen Rückgabewert.
\subsubsection{updateDeletionPicStateLinkPoiPicByID}
\paragraph{Parameter} Die Funktion besitzt folgende Parameter:
\begin{table}[H]
	\begin{tabular}{|c|p{11cm}|}
		\hline
		\textbf{Parametername} & \textbf{Parameterbeschreibung} \\ \hline
		\$id    & Identifikator eines Links zwischen einem Bild und einem Interessenpunkt \\ \hline
		\$state & Status der Löschung des Links \\ \hline
	\end{tabular}
\end{table}
\paragraph{Beschreibung} Die Funktion markiert einen Link zwischen einem Interessenpunkt und einem Bild als gelöscht, aufgrund eines gelöschten Bildes. Die Funktion hat Auswirkungen auf folgende Quellen:
\begin{itemize}
	\item Tabelle mit Links zwischen Bildern und Interessenpunkten
\end{itemize}
Es findet bei dieser Funktion kein Abruf von Daten aus {\glqq COSP\grqq} statt. Es gibt einen Rückgabewert.
\subsubsection{LinkPoiPicRestictedPOI}
\paragraph{Parameter} Die Funktion besitzt folgende Parameter:
\begin{table}[H]
	\begin{tabular}{|c|p{11cm}|}
		\hline
		\textbf{Parametername} & \textbf{Parameterbeschreibung} \\ \hline
		\$id    & Identifikator eines Links zwischen einem Bild und einem Interessenpunkt \\ \hline
	\end{tabular}
\end{table}
\paragraph{Beschreibung} Die Funktion fragt ab, ob eine Restriktion der Wiederherstellung aufgrund eines gelöschten Interessenpunktes besteht. Die Funktion nutzt folgende Quellen:
\begin{itemize}
	\item Tabelle mit Links zwischen Bildern und Interessenpunkten
\end{itemize}
Es findet bei dieser Funktion kein Abruf von Daten aus {\glqq COSP\grqq} statt. Der Rückgabewert ist ein Boolean.
\subsubsection{LinkPoiPicRestictedPic}
\paragraph{Parameter} Die Funktion besitzt folgende Parameter:
\begin{table}[H]
	\begin{tabular}{|c|p{11cm}|}
		\hline
		\textbf{Parametername} & \textbf{Parameterbeschreibung} \\ \hline
		\$id    & Identifikator eines Links zwischen einem Bild und einem Interessenpunkt \\ \hline
	\end{tabular}
\end{table}
\paragraph{Beschreibung} Die Funktion fragt ab, ob eine Restriktion der Wiederherstellung aufgrund eines gelöschten Bildes besteht. Die Funktion nutzt folgende Quellen:
\begin{itemize}
	\item Tabelle mit Links zwischen Bildern und Interessenpunkten
\end{itemize}
Es findet bei dieser Funktion kein Abruf von Daten aus {\glqq COSP\grqq} statt. Der Rückgabewert ist ein Boolean.