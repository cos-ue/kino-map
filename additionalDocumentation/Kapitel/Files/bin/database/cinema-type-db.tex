\subsection{Allgemeines} Diese Datei enthält alle Funktionen für die Typen-Tabelle.
\begin{table}[H]
	\begin{tabular}{|c|p{11cm}|}
		\hline
		\textbf{Einbindungspunkt} & inc-db.php \\ \hline
		\textbf{Einbindungspunkt} & inc-db-sub.php \\ \hline
	\end{tabular}
\end{table}
Die Datei ist nicht direkt durch den Nutzer aufrufbar, dies wird durch folgenden Code-Ausschnitt sichergestellt:
\begin{lstlisting}[language=php]
if (!defined('NICE_PROJECT')) {
	die('Permission denied.');
}
\end{lstlisting}
Der Globale Wert {\glqq NICE\_PROJECT\grqq} wird durch für den Nutzer valide Aufrufpunkte festgelegt, z.B. {\glqq api.php\grqq}.
\newpage
\subsection{Funktionen}
\subsubsection{getAllCinemaTypes}
\paragraph{Parameter} Die Funktion besitzt keine Parameter.
\paragraph{Beschreibung} Die Funktion fragt alle verfügbaren Kinotypen ab. Die Funktion nutzt folgenden Quellen:
\begin{itemize}
	\item Typen-Tabelle
\end{itemize}
Es findet bei dieser Funktion kein Abruf von Daten aus {\glqq COSP\grqq} statt. Die Antwort wird als strukturiertes Array an den Aufrufer zurückgegeben.
\subsubsection{getCinemaTypeNameByTypeId}
\paragraph{Parameter} Die Funktion besitzt folgende Parameter:
\begin{table}[H]
	\begin{tabular}{|c|p{11cm}|}
		\hline
		\textbf{Parametername} & \textbf{Parameterbeschreibung} \\ \hline
		\$id & numerischer Identifikator des Types \\ \hline
	\end{tabular}
\end{table}
\paragraph{Beschreibung} Die Funktion fragt den Namen eines Typs ab. Die Funktion nutzt folgenden Quellen:
\begin{itemize}
	\item Typen-Tabelle
\end{itemize}
Es findet bei dieser Funktion kein Abruf von Daten aus {\glqq COSP\grqq} statt. Die Antwort wird als Zeichenkette an den Aufrufer zurückgegeben.

