\subsection{Allgemeines} Diese Datei enthält Funktionen, welche auf die Datenbanktabelle mit historischen Adressen zugreifen.
\begin{table}[H]
	\begin{tabular}{|c|p{11cm}|}
		\hline
		\textbf{Einbindungspunkt} & inc-db.php \\ \hline
		\textbf{Einbindungspunkt} & inc-db-sub.php \\ \hline
	\end{tabular}
\end{table}
Die Datei ist nicht direkt durch den Nutzer aufrufbar, dies wird durch folgenden Code-Ausschnitt sichergestellt:
\begin{lstlisting}[language=php]
if (!defined('NICE_PROJECT')) {
	die('Permission denied.');
}
\end{lstlisting}
Der Globale Wert {\glqq NICE\_PROJECT\grqq} wird durch für den Nutzer valide Aufrufpunkte festgelegt, z.B. {\glqq api.php\grqq}.
\newpage
\subsection{Funktionen}
\subsubsection{insertCinemasOfPOI}
\paragraph{Parameter} Die Funktion besitzt folgende Parameter:
\begin{table}[H]
	\begin{tabular}{|c|p{11cm}|}
		\hline
		\textbf{Parametername} & \textbf{Parameterbeschreibung} \\ \hline
		\$poi\_id       & Identifikator des zugehörigen Interessenpunktes \\ \hline
		\$start         & Startjahr \\ \hline
		\$end           & Endjahr \\ \hline
		\$cinema\_count & Anzahl der Kinosäle \\ \hline
	\end{tabular}
\end{table}
\paragraph{Beschreibung} Die Funktion fügt einem Interessenpunkt eine neue Saalanzahl hinzu. Die Funktion hat Auswirkungen auf folgende Quellen:
\begin{itemize}
	\item Tabelle mit Kinosaalanzahlen
\end{itemize}
Es findet bei dieser Funktion kein Abruf von Daten aus {\glqq COSP\grqq} statt. Es wird eine Antwort zurück gegeben.
\subsubsection{getCinemasByPoiId}
\paragraph{Parameter} Die Funktion besitzt folgende Parameter:
\begin{table}[H]
	\begin{tabular}{|c|p{11cm}|}
		\hline
		\textbf{Parametername} & \textbf{Parameterbeschreibung} \\ \hline
		\$poiid & Identifikator des Interessenpunktes \\ \hline
	\end{tabular}
\end{table}
\paragraph{Beschreibung} Die Funktion ermittelt alle Kinosaalanzahlen eines Interessenpunktes. Die Funktion nutzt folgende Quellen:
\begin{itemize}
	\item Tabelle mit Kinosaalanzahlen
\end{itemize}
Es findet bei dieser Funktion kein Abruf von Daten aus {\glqq COSP\grqq} statt. Die Antwort wird als strukturiertes Array an den Aufrufer zurückgegeben.
\subsubsection{getCreatorByCinemasID}
\paragraph{Parameter} Die Funktion besitzt folgende Parameter:
\begin{table}[H]
	\begin{tabular}{|c|p{11cm}|}
		\hline
		\textbf{Parametername} & \textbf{Parameterbeschreibung} \\ \hline
		\$cinemas\_id & Identifikator der Saalanzahl \\ \hline
	\end{tabular}
\end{table}
\paragraph{Beschreibung} Die Funktion ermittelt den Ersteller einer Kinosaalanzahl. Die Funktion nutzt folgende Quellen:
\begin{itemize}
	\item Tabelle mit Kinosaalanzahlen
	\item Nutzer-Tabelle
\end{itemize}
Es findet bei dieser Funktion kein Abruf von Daten aus {\glqq COSP\grqq} statt. Die Antwort ist eine Zeichenkette.
\subsubsection{deleteCinemas}
\paragraph{Parameter} Die Funktion besitzt folgende Parameter:
\begin{table}[H]
	\begin{tabular}{|c|p{11cm}|}
		\hline
		\textbf{Parametername} & \textbf{Parameterbeschreibung} \\ \hline
		\$cinemas\_id & Identifikator der Saalanzahl \\ \hline
	\end{tabular}
\end{table}
\paragraph{Beschreibung} Die Funktion löscht eine Kinosaalanzahl. Die Funktion hat Auswirkungen auf folgende Quellen:
\begin{itemize}
	\item Tabelle mit Kinosaalanzahlen
\end{itemize}
Es findet bei dieser Funktion kein Abruf von Daten aus {\glqq COSP\grqq} statt. Es wird eine Antwort zurück gegeben.
\subsubsection{updateCinemas}
\paragraph{Parameter} Die Funktion besitzt folgende Parameter:
\begin{table}[H]
	\begin{tabular}{|c|p{11cm}|}
		\hline
		\textbf{Parametername} & \textbf{Parameterbeschreibung} \\ \hline
		\$id      & Identifikator der Saalanzahl \\ \hline
		\$cinemas & Anzahl der Kinosäle \\ \hline
		\$start   & Startjahr \\ \hline
		\$end     & Endjahr \\ \hline
	\end{tabular}
\end{table}
\paragraph{Beschreibung} Die Funktion aktualisiert eine Saalanzahl. Die Funktion hat Auswirkungen auf folgende Quellen:
\begin{itemize}
	\item Tabelle mit Kinosaalanzahlen
\end{itemize}
Es findet bei dieser Funktion kein Abruf von Daten aus {\glqq COSP\grqq} statt. Es wird eine Antwort zurück gegeben.
\subsubsection{updateDeletionStateCinemasById}
\paragraph{Parameter} Die Funktion besitzt folgende Parameter:
\begin{table}[H]
	\begin{tabular}{|c|p{11cm}|}
		\hline
		\textbf{Parametername} & \textbf{Parameterbeschreibung} \\ \hline
		\$id    & Identifikator der Saalanzahl \\ \hline
		\$state & Status der Löschung \\ \hline
	\end{tabular}
\end{table}
\paragraph{Beschreibung} Die Funktion markiert eine Kinosaalanzahl als gelöscht. Die Funktion hat Auswirkungen auf folgende Quellen:
\begin{itemize}
	\item Tabelle mit Kinosaalanzahlen
\end{itemize}
Es findet bei dieser Funktion kein Abruf von Daten aus {\glqq COSP\grqq} statt. Es wird eine Antwort zurück gegeben.