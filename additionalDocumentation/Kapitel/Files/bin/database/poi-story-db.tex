\subsection{Allgemeines} Diese Datei enthält alle Funktionen, welche die Tabelle mit Links zwischen Geschichten und Interessenpunkten benutzen.
\begin{table}[H]
	\begin{tabular}{|c|p{11cm}|}
		\hline
		\textbf{Einbindungspunkt} & inc-db.php \\ \hline
		\textbf{Einbindungspunkt} & inc-db-sub.php \\ \hline
	\end{tabular}
\end{table}
Die Datei ist nicht direkt durch den Nutzer aufrufbar, dies wird durch folgenden Code-Ausschnitt sichergestellt:
\begin{lstlisting}[language=php]
	if (!defined('NICE_PROJECT')) {
		die('Permission denied.');
	}
\end{lstlisting}
Der Globale Wert {\glqq NICE\_PROJECT\grqq} wird durch für den Nutzer valide Aufrufpunkte festgelegt, z.B. {\glqq api.php\grqq}.
\newpage
\subsection{Funktionen}
\subsubsection{insertPoiStory}
\paragraph{Parameter} Die Funktion besitzt folgende Parameter:
\begin{table}[H]
	\begin{tabular}{|c|p{11cm}|}
		\hline
		\textbf{Parametername} & \textbf{Parameterbeschreibung} \\ \hline
		\$poi\_id    & Identifikator eines Interessenpunktes \\ \hline
		\$storytoken & alphanumerischer Identifikator einer Geschichte \\ \hline
	\end{tabular}
\end{table}
\paragraph{Beschreibung} Die Funktion fügt einen neuen Link zwischen einem Interessenpunkt und einer Geschichte ein. Die Funktion hat Auswirkung auf folgende Quellen:
\begin{itemize}
	\item Tabelle mit Links zwischen Geschichten und Interessenpunkten
\end{itemize}
Es findet bei dieser Funktion kein Abruf von Daten aus {\glqq COSP\grqq} statt. Es gibt einen Rückgabewert.
\subsubsection{getPoiForStory}
\paragraph{Parameter} Die Funktion besitzt folgende Parameter:
\begin{table}[H]
	\begin{tabular}{|c|p{11cm}|}
		\hline
		\textbf{Parametername} & \textbf{Parameterbeschreibung} \\ \hline
		\$token    & alphanumerische Identifikator einer Geschichte \\ \hline
		\$override & Ignoriert Löschmarkierungen \\ \hline
		\$api      & Legt fest ob Aufrufer API ist \\ \hline
	\end{tabular}
\end{table}
\paragraph{Beschreibung} Die Funktion fragt alle Links zwischen einer Geschichte und Interessenpunkten ab. Die Funktion nutzt folgende Quellen:
\begin{itemize}
	\item Tabelle mit Links zwischen Geschichten und Interessenpunkten
\end{itemize}
Es findet bei dieser Funktion kein Abruf von Daten aus {\glqq COSP\grqq} statt. Die Antwort wird als strukturiertes Array an den Aufrufer zurückgegeben.
\subsubsection{getPoiForStoryByPoiId}
\paragraph{Parameter} Die Funktion besitzt folgende Parameter:
\begin{table}[H]
	\begin{tabular}{|c|p{11cm}|}
		\hline
		\textbf{Parametername} & \textbf{Parameterbeschreibung} \\ \hline
		\$poi\_id  & Identifikator eines Interessenpunktes \\ \hline
		\$override & Ignoriert Löschmarkierungen \\ \hline
	\end{tabular}
\end{table}
\paragraph{Beschreibung} Die Funktion fragt alle Links zwischen Geschichten und einem Interessenpunkt ab. Die Funktion nutzt folgende Quellen:
\begin{itemize}
	\item Tabelle mit Links zwischen Geschichten und Interessenpunkten
\end{itemize}
Es findet bei dieser Funktion kein Abruf von Daten aus {\glqq COSP\grqq} statt. Die Antwort wird als strukturiertes Array an den Aufrufer zurückgegeben.
\subsubsection{getPoiForStoryByPoiId}
\paragraph{Parameter} Die Funktion besitzt folgende Parameter:
\begin{table}[H]
	\begin{tabular}{|c|p{11cm}|}
		\hline
		\textbf{Parametername} & \textbf{Parameterbeschreibung} \\ \hline
		\$story\_poi\_id  & Identifikator eines Links zwischen einem Interessenpunkt und einer Geschichte \\ \hline
	\end{tabular}
\end{table}
\paragraph{Beschreibung} Die Funktion den Ersteller eines Links zwischen einer Geschichte und einem Interessenpunkt ab. Die Funktion nutzt folgende Quellen:
\begin{itemize}
	\item Tabelle mit Links zwischen Geschichten und Interessenpunkten
\end{itemize}
Es findet bei dieser Funktion kein Abruf von Daten aus {\glqq COSP\grqq} statt. Der Rückgabewert ist eine Zeichenkette.
\subsubsection{deletePoiStory}
\paragraph{Parameter} Die Funktion besitzt folgende Parameter:
\begin{table}[H]
	\begin{tabular}{|c|p{11cm}|}
		\hline
		\textbf{Parametername} & \textbf{Parameterbeschreibung} \\ \hline
		\$story\_poi\_id  & Identifikator eines Links zwischen einem Interessenpunkt und einer Geschichte \\ \hline
	\end{tabular}
\end{table}
\paragraph{Beschreibung} Die Funktion löscht einen Link zwischen einer Geschichte und einem Interessenpunkt ab. Die Funktion hat Auswirkungen auf folgende Quellen:
\begin{itemize}
	\item Tabelle mit Links zwischen Geschichten und Interessenpunkten
\end{itemize}
Es findet bei dieser Funktion kein Abruf von Daten aus {\glqq COSP\grqq} statt. Es gibt einen Rückgabewert.
\subsubsection{updateDeletionStateLinkPoiStoryByID}
\paragraph{Parameter} Die Funktion besitzt folgende Parameter:
\begin{table}[H]
	\begin{tabular}{|c|p{11cm}|}
		\hline
		\textbf{Parametername} & \textbf{Parameterbeschreibung} \\ \hline
		\$id    & Identifikator eines Links zwischen einem Interessenpunkt und einer Geschichte \\ \hline
		\$state & Status der Löschung \\ \hline
	\end{tabular}
\end{table}
\paragraph{Beschreibung} Die Funktion markiert einen Link zwischen einer Geschichte und einem Interessenpunkt als gelöscht. Die Funktion hat Auswirkungen auf folgende Quellen:
\begin{itemize}
	\item Tabelle mit Links zwischen Geschichten und Interessenpunkten
\end{itemize}
Es findet bei dieser Funktion kein Abruf von Daten aus {\glqq COSP\grqq} statt. Es gibt einen Rückgabewert.
\subsubsection{updateDeletionPoiStateLinkPoiStoryByID}
\paragraph{Parameter} Die Funktion besitzt folgende Parameter:
\begin{table}[H]
	\begin{tabular}{|c|p{11cm}|}
		\hline
		\textbf{Parametername} & \textbf{Parameterbeschreibung} \\ \hline
		\$id    & Identifikator eines Links zwischen einem Interessenpunkt und einer Geschichte \\ \hline
		\$state & Status der Löschung \\ \hline
	\end{tabular}
\end{table}
\paragraph{Beschreibung} Die Funktion markiert einen Link zwischen einer Geschichte und einem Interessenpunkt als gelöscht, aufgrund eines gelöschten Interessenpunktes. Die Funktion hat Auswirkungen auf folgende Quellen:
\begin{itemize}
	\item Tabelle mit Links zwischen Geschichten und Interessenpunkten
\end{itemize}
Es findet bei dieser Funktion kein Abruf von Daten aus {\glqq COSP\grqq} statt. Es gibt einen Rückgabewert.
\subsubsection{updateDeletionStoryStateLinkPoiStoryByID}
\paragraph{Parameter} Die Funktion besitzt folgende Parameter:
\begin{table}[H]
	\begin{tabular}{|c|p{11cm}|}
		\hline
		\textbf{Parametername} & \textbf{Parameterbeschreibung} \\ \hline
		\$id    & Identifikator eines Links zwischen einem Interessenpunkt und einer Geschichte \\ \hline
		\$state & Status der Löschung \\ \hline
	\end{tabular}
\end{table}
\paragraph{Beschreibung} Die Funktion markiert einen Link zwischen einer Geschichte und einem Interessenpunkt als gelöscht, aufgrund einer gelöschten Geschichte. Die Funktion hat Auswirkungen auf folgende Quellen:
\begin{itemize}
	\item Tabelle mit Links zwischen Geschichten und Interessenpunkten
\end{itemize}
Es findet bei dieser Funktion kein Abruf von Daten aus {\glqq COSP\grqq} statt. Es gibt einen Rückgabewert.
\subsubsection{LinkPoiStoryRestictedPOI}
\paragraph{Parameter} Die Funktion besitzt folgende Parameter:
\begin{table}[H]
	\begin{tabular}{|c|p{11cm}|}
		\hline
		\textbf{Parametername} & \textbf{Parameterbeschreibung} \\ \hline
		\$id    & Identifikator eines Links zwischen einer Geschichte und einem Interessenpunkt \\ \hline
	\end{tabular}
\end{table}
\paragraph{Beschreibung} Die Funktion fragt ab, ob eine Restriktion der Wiederherstellung aufgrund eines gelöschten Interessenpunktes besteht. Die Funktion nutzt folgende Quellen:
\begin{itemize}
	\item Tabelle mit Links zwischen Geschichten und Interessenpunkten
\end{itemize}
Es findet bei dieser Funktion kein Abruf von Daten aus {\glqq COSP\grqq} statt. Der Rückgabewert ist ein Boolean.
\subsubsection{LinkPoiStoryRestictedStory}
\paragraph{Parameter} Die Funktion besitzt folgende Parameter:
\begin{table}[H]
	\begin{tabular}{|c|p{11cm}|}
		\hline
		\textbf{Parametername} & \textbf{Parameterbeschreibung} \\ \hline
		\$id    & Identifikator eines Links zwischen einer Geschichte und einem Interessenpunkt \\ \hline
	\end{tabular}
\end{table}
\paragraph{Beschreibung} Die Funktion fragt ab, ob eine Restriktion der Wiederherstellung aufgrund eines gelöschten Bildes besteht. Die Funktion nutzt folgende Quellen:
\begin{itemize}
	\item Tabelle mit Links zwischen Geschichten und Interessenpunkten
\end{itemize}
Es findet bei dieser Funktion kein Abruf von Daten aus {\glqq COSP\grqq} statt. Der Rückgabewert ist ein Boolean.