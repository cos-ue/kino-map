\subsection{Allgemeines} Diese Datei enthält alle Funktionen, welche die Sitzplatzanzahl-Tabelle benutzen.
\begin{table}[H]
	\begin{tabular}{|c|p{11cm}|}
		\hline
		\textbf{Einbindungspunkt} & inc-db.php \\ \hline
		\textbf{Einbindungspunkt} & inc-db-sub.php \\ \hline
	\end{tabular}
\end{table}
Die Datei ist nicht direkt durch den Nutzer aufrufbar, dies wird durch folgenden Code-Ausschnitt sichergestellt:
\begin{lstlisting}[language=php]
	if (!defined('NICE_PROJECT')) {
		die('Permission denied.');
	}
\end{lstlisting}
Der Globale Wert {\glqq NICE\_PROJECT\grqq} wird durch für den Nutzer valide Aufrufpunkte festgelegt, z.B. {\glqq api.php\grqq}.
\newpage
\subsection{Funktionen}
\subsubsection{insertSeatsOfPOI}
\paragraph{Parameter} Die Funktion besitzt folgende Parameter:
\begin{table}[H]
	\begin{tabular}{|c|p{11cm}|}
		\hline
		\textbf{Parametername} & \textbf{Parameterbeschreibung} \\ \hline
		\$poi\_id   & Identifikator eines Interessenpunktes \\ \hline
		\$start     & Startjahr \\ \hline
		\$end       & Endjahr \\ \hline
		\$seatcount & Sitzplatzanzahl \\ \hline
	\end{tabular}
\end{table}
\paragraph{Beschreibung} Die Funktion fügt eine neue Sitzplatzanzahl einem Interessenpunkt hinzu. Die Funktion hat Auswirkungen auf folgende Quellen:
\begin{itemize}
	\item Sitzplatzanzahl-Tabelle
\end{itemize}
Es findet bei dieser Funktion kein Abruf von Daten aus {\glqq COSP\grqq} statt. Es gibt einen Rückgabewert.
\subsubsection{getSeatsByPoiId}
\paragraph{Parameter} Die Funktion besitzt folgende Parameter:
\begin{table}[H]
	\begin{tabular}{|c|p{11cm}|}
		\hline
		\textbf{Parametername} & \textbf{Parameterbeschreibung} \\ \hline
		\$poiid & Identifikator eines Interessenpunktes \\ \hline
	\end{tabular}
\end{table}
\paragraph{Beschreibung} Die Funktion fragt alle Sitzplatzanzahlen eines Interessenpunktes ab. Die Funktion nutzt folgende Quellen:
\begin{itemize}
	\item Sitzplatzanzahl-Tabelle
\end{itemize}
Es findet bei dieser Funktion kein Abruf von Daten aus {\glqq COSP\grqq} statt. Die Antwort wird als strukturiertes Array an den Aufrufer zurückgegeben.
\subsubsection{getCreatorBySeatsID}
\paragraph{Parameter} Die Funktion besitzt folgende Parameter:
\begin{table}[H]
	\begin{tabular}{|c|p{11cm}|}
		\hline
		\textbf{Parametername} & \textbf{Parameterbeschreibung} \\ \hline
		\$seat\_id & Identifikator der Sitzplatzanzahl \\ \hline
	\end{tabular}
\end{table}
\paragraph{Beschreibung} Die Funktion fragt den Ersteller einer Sitzplatzanzahl ab. Die Funktion nutzt folgende Quellen:
\begin{itemize}
	\item Sitzplatzanzahl-Tabelle
\end{itemize}
Es findet bei dieser Funktion kein Abruf von Daten aus {\glqq COSP\grqq} statt. Der Rückgabewert ist eine Zeichenkette.
\subsubsection{deleteSeats}
\paragraph{Parameter} Die Funktion besitzt folgende Parameter:
\begin{table}[H]
	\begin{tabular}{|c|p{11cm}|}
		\hline
		\textbf{Parametername} & \textbf{Parameterbeschreibung} \\ \hline
		\$seat\_id & Identifikator der Sitzplatzanzahl \\ \hline
	\end{tabular}
\end{table}
\paragraph{Beschreibung} Die Funktion löscht eine Sitzplatzanzahl. Die Funktion hat Auswirkungen auf folgende Quellen:
\begin{itemize}
	\item Sitzplatzanzahl-Tabelle
\end{itemize}
Es findet bei dieser Funktion kein Abruf von Daten aus {\glqq COSP\grqq} statt. Es gibt einen Rückgabewert.
\subsubsection{updateSeats}
\paragraph{Parameter} Die Funktion besitzt folgende Parameter:
\begin{table}[H]
	\begin{tabular}{|c|p{11cm}|}
		\hline
		\textbf{Parametername} & \textbf{Parameterbeschreibung} \\ \hline
		\$id        & Identifikator der Sitzplatzanzahl \\ \hline
		\$seatCount & Sitzplatzanzahl \\ \hline
		\$start     & Startjahr \\ \hline
		\$end       & Endjahr \\ \hline
	\end{tabular}
\end{table}
\paragraph{Beschreibung} Die Funktion aktualisiert eine Sitzplatzanzahl. Die Funktion hat Auswirkungen auf folgende Quellen:
\begin{itemize}
	\item Sitzplatzanzahl-Tabelle
\end{itemize}
Es findet bei dieser Funktion kein Abruf von Daten aus {\glqq COSP\grqq} statt. Es gibt einen Rückgabewert.
\subsubsection{updateDeletionStateSeatsById}
\paragraph{Parameter} Die Funktion besitzt folgende Parameter:
\begin{table}[H]
	\begin{tabular}{|c|p{11cm}|}
		\hline
		\textbf{Parametername} & \textbf{Parameterbeschreibung} \\ \hline
		\$id        & Identifikator der Sitzplatzanzahl \\ \hline
		\$state & Status der Löschung \\ \hline
		\$start     & Startjahr \\ \hline
		\$end       & Endjahr \\ \hline
	\end{tabular}
\end{table}
\paragraph{Beschreibung} Die Funktion markiert eine Sitzplatzanzahl als gelöscht. Die Funktion hat Auswirkungen auf folgende Quellen:
\begin{itemize}
	\item Sitzplatzanzahl-Tabelle
\end{itemize}
Es findet bei dieser Funktion kein Abruf von Daten aus {\glqq COSP\grqq} statt. Es gibt einen Rückgabewert