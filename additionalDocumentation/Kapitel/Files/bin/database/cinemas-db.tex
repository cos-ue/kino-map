\subsection{Allgemeines} Diese Datei enthält alle Funktionen, welche die Saalanzahl-Tabelle nutzen.
\begin{table}[H]
	\begin{tabular}{|c|p{11cm}|}
		\hline
		\textbf{Einbindungspunkt} & inc-db.php \\ \hline
		\textbf{Einbindungspunkt} & inc-db-sub.php \\ \hline
	\end{tabular}
\end{table}
Die Datei ist nicht direkt durch den Nutzer aufrufbar, dies wird durch folgenden Code-Ausschnitt sichergestellt:
\begin{lstlisting}[language=php]
if (!defined('NICE_PROJECT')) {
	die('Permission denied.');
}
\end{lstlisting}
Der Globale Wert {\glqq NICE\_PROJECT\grqq} wird durch für den Nutzer valide Aufrufpunkte festgelegt, z.B. {\glqq api.php\grqq}.
\newpage
\subsection{Funktionen}
\subsubsection{insertCinemasOfPOI}
\paragraph{Parameter} Die Funktion besitzt folgende Parameter:
\begin{table}[H]
	\begin{tabular}{|c|p{11cm}|}
		\hline
		\textbf{Parametername} & \textbf{Parameterbeschreibung} \\ \hline
		\$poi\_id       & Identifikator eines Interessenpunktes \\ \hline
		\$start         & Startjahr \\ \hline
		\$end           & Endjahr \\ \hline
		\$cinema\_count & Saalanzahl \\ \hline
	\end{tabular}
\end{table}
\paragraph{Beschreibung} Die Funktion fügt eine neue Sallanzahl hinzu. Die Funktion hat Auswirkungen auf folgende Quellen:
\begin{itemize}
	\item Saalanzahl-Tabelle
\end{itemize}
Es findet bei dieser Funktion kein Abruf von Daten aus {\glqq COSP\grqq} statt. Es wird eine Antwort zurück gegeben.
\subsubsection{getCinemasByPoiId}
\paragraph{Parameter} Die Funktion besitzt folgende Parameter:
\begin{table}[H]
	\begin{tabular}{|c|p{11cm}|}
		\hline
		\textbf{Parametername} & \textbf{Parameterbeschreibung} \\ \hline
		\$poiid       & Identifikator eines Interessenpunktes \\ \hline
	\end{tabular}
\end{table}
\paragraph{Beschreibung} Die Funktion fragt alle zu einem Interessenpunkt zugehörigen Saalanzahlen ab. Die Funktion nutzt folgende Quellen:
\begin{itemize}
	\item Saalanzahl-Tabelle
\end{itemize}
Es findet bei dieser Funktion kein Abruf von Daten aus {\glqq COSP\grqq} statt. Die Antwort wird als strukturiertes Array an den Aufrufer zurückgegeben.
\subsubsection{getCreatorByCinemasID}
\paragraph{Parameter} Die Funktion besitzt folgende Parameter:
\begin{table}[H]
	\begin{tabular}{|c|p{11cm}|}
		\hline
		\textbf{Parametername} & \textbf{Parameterbeschreibung} \\ \hline
		\$cinemas\_id       & Identifikator einer Saalanzahl \\ \hline
	\end{tabular}
\end{table}
\paragraph{Beschreibung} Die Funktion fragt den Ersteller einer Saalanzahl ab. Die Funktion nutzt folgende Quellen:
\begin{itemize}
	\item Saalanzahl-Tabelle
\end{itemize}
Es findet bei dieser Funktion kein Abruf von Daten aus {\glqq COSP\grqq} statt. Die Antwort wird als strukturiertes Array an den Aufrufer zurückgegeben.
\subsubsection{deleteCinemas}
\paragraph{Parameter} Die Funktion besitzt folgende Parameter:
\begin{table}[H]
	\begin{tabular}{|c|p{11cm}|}
		\hline
		\textbf{Parametername} & \textbf{Parameterbeschreibung} \\ \hline
		\$cinemas\_id       & Identifikator einer Saalanzahl \\ \hline
	\end{tabular}
\end{table}
\paragraph{Beschreibung} Die Funktion löscht eine Saalanzahl. Die Funktion hat Auswirkung auf folgende Quellen:
\begin{itemize}
	\item Saalanzahl-Tabelle
\end{itemize}
Es findet bei dieser Funktion kein Abruf von Daten aus {\glqq COSP\grqq} statt. Die Antwort wird als strukturiertes Array an den Aufrufer zurückgegeben.
\subsubsection{updateCinemas}
\paragraph{Parameter} Die Funktion besitzt folgende Parameter:
\begin{table}[H]
	\begin{tabular}{|c|p{11cm}|}
		\hline
		\textbf{Parametername} & \textbf{Parameterbeschreibung} \\ \hline
		\$id       & Identifikator einer Saalanzahl \\ \hline
		\$cinemas  & Saalanzahl \\ \hline
		\$start    & Startjahr \\ \hline
		\$end      & Endjahr \\ \hline
	\end{tabular}
\end{table}
\paragraph{Beschreibung} Die Funktion aktualisiert eine Saalanzahl. Die Funktion hat Auswirkung auf folgende Quellen:
\begin{itemize}
	\item Saalanzahl-Tabelle
\end{itemize}
Es findet bei dieser Funktion kein Abruf von Daten aus {\glqq COSP\grqq} statt. Es wird eine Antwort zurück gegeben.
\subsubsection{updateDeletionStateCinemasById}
\paragraph{Parameter} Die Funktion besitzt folgende Parameter:
\begin{table}[H]
	\begin{tabular}{|c|p{11cm}|}
		\hline
		\textbf{Parametername} & \textbf{Parameterbeschreibung} \\ \hline
		\$id       & Identifikator einer Saalanzahl \\ \hline
		\$state    & Status der Löschung \\ \hline
	\end{tabular}
\end{table}
\paragraph{Beschreibung} Die Funktion setzt den Löschungsstatus einer Saalanzahl. Die Funktion hat Auswirkung auf folgende Quellen:
\begin{itemize}
	\item Saalanzahl-Tabelle
\end{itemize}
Es findet bei dieser Funktion kein Abruf von Daten aus {\glqq COSP\grqq} statt. Es wird eine Antwort zurück gegeben.