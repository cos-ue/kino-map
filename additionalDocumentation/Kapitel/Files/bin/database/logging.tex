\subsection{Allgemeines} Diese Datei enthält Funktionen zum Abfragen der Tabelle zum erfassen statistischer Nutzer.
\begin{table}[H]
	\begin{tabular}{|c|p{11cm}|}
		\hline
		\textbf{Einbindungspunkt} & inc-db.php \\ \hline
		\textbf{Einbindungspunkt} & inc-db-sub.php \\ \hline
	\end{tabular}
\end{table}
Die Datei ist nicht direkt durch den Nutzer aufrufbar, dies wird durch folgenden Code-Ausschnitt sichergestellt:
\begin{lstlisting}[language=php]
if (!defined('NICE_PROJECT')) {
	die('Permission denied.');
}
\end{lstlisting}
Der Globale Wert {\glqq NICE\_PROJECT\grqq} wird durch für den Nutzer valide Aufrufpunkte festgelegt, z.B. {\glqq api.php\grqq}.
\newpage
\subsection{Funktionen}
\subsubsection{insertLogUniqueVisitors}
\paragraph{Parameter} Die Funktion besitzt folgende Parameter:
\begin{table}[H]
	\begin{tabular}{|c|p{11cm}|}
		\hline
		\textbf{Parametername} & \textbf{Parameterbeschreibung} \\ \hline
		\$ip   & Ip-Adresse des Aufrufers \\ \hline
		\$type & Typ der Nutzung (Gast oder Nutzer) \\ \hline
	\end{tabular}
\end{table}
\paragraph{Beschreibung} Die Funktion trägt eine neue Nutzung in die Tabelle ein. Die Funktion hat Auswirkung auf folgende Quellen:
\begin{itemize}
	\item Tabelle mit statistischen Nutzerdaten
\end{itemize}
Es findet bei dieser Funktion kein Abruf von Daten aus {\glqq COSP\grqq} statt. Die Funktion gibt eine Antwort zurück.
\subsubsection{getStatisticalDataLastWeeks}
\paragraph{Parameter} Die Funktion besitzt folgende Parameter:
\begin{table}[H]
	\begin{tabular}{|c|p{11cm}|}
		\hline
		\textbf{Parametername} & \textbf{Parameterbeschreibung} \\ \hline
		\$number & Anzahl der Wochen \\ \hline
	\end{tabular}
\end{table}
\paragraph{Beschreibung} Die Funktion fragt statistische Daten zu Nutzern für den in Wochen angegeben Zeitraum ab. Die Funktion nutzt folgende Quellen:
\begin{itemize}
	\item Tabelle mit statistischen Nutzerdaten
\end{itemize}
Es findet bei dieser Funktion kein Abruf von Daten aus {\glqq COSP\grqq} statt. Die Antwort wird als strukturiertes Array an den Aufrufer zurückgegeben.
\subsubsection{getStatisticalDataLastMonth}
\paragraph{Parameter} Die Funktion besitzt folgende Parameter:
\begin{table}[H]
	\begin{tabular}{|c|p{11cm}|}
		\hline
		\textbf{Parametername} & \textbf{Parameterbeschreibung} \\ \hline
		\$number & Anzahl der Monate \\ \hline
	\end{tabular}
\end{table}
\paragraph{Beschreibung} Die Funktion fragt statistische Daten zu Nutzern für den in Monaten angegeben Zeitraum ab. Die Funktion nutzt folgende Quellen:
\begin{itemize}
	\item Tabelle mit statistischen Nutzerdaten
\end{itemize}
Es findet bei dieser Funktion kein Abruf von Daten aus {\glqq COSP\grqq} statt. Die Antwort wird als strukturiertes Array an den Aufrufer zurückgegeben.
\subsubsection{getStatisticalDataLastYear}
\paragraph{Parameter} Die Funktion besitzt folgende Parameter:
\begin{table}[H]
	\begin{tabular}{|c|p{11cm}|}
		\hline
		\textbf{Parametername} & \textbf{Parameterbeschreibung} \\ \hline
		\$number & Anzahl der Jahre \\ \hline
	\end{tabular}
\end{table}
\paragraph{Beschreibung} Die Funktion fragt statistische Daten zu Nutzern für den in Jahren angegeben Zeitraum ab. Die Funktion nutzt folgende Quellen:
\begin{itemize}
	\item Tabelle mit statistischen Nutzerdaten
\end{itemize}
Es findet bei dieser Funktion kein Abruf von Daten aus {\glqq COSP\grqq} statt. Die Antwort wird als strukturiertes Array an den Aufrufer zurückgegeben.
\subsubsection{getStatisticalDataLastDays}
\paragraph{Parameter} Die Funktion besitzt folgende Parameter:
\begin{table}[H]
	\begin{tabular}{|c|p{11cm}|}
		\hline
		\textbf{Parametername} & \textbf{Parameterbeschreibung} \\ \hline
		\$number & Anzahl der Tage \\ \hline
	\end{tabular}
\end{table}
\paragraph{Beschreibung} Die Funktion fragt statistische Daten zu Nutzern für den in Tagen angegeben Zeitraum ab. Die Funktion nutzt folgende Quellen:
\begin{itemize}
	\item Tabelle mit statistischen Nutzerdaten
\end{itemize}
Es findet bei dieser Funktion kein Abruf von Daten aus {\glqq COSP\grqq} statt. Die Antwort wird als strukturiertes Array an den Aufrufer zurückgegeben.
