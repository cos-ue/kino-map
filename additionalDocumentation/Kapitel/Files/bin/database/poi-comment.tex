\subsection{Allgemeines} Diese Datei enthält alle Funktionen, welche die Kommentar-Tabelle benutzen.
\begin{table}[H]
	\begin{tabular}{|c|p{11cm}|}
		\hline
		\textbf{Einbindungspunkt} & inc-db.php \\ \hline
		\textbf{Einbindungspunkt} & inc-db-sub.php \\ \hline
	\end{tabular}
\end{table}
Die Datei ist nicht direkt durch den Nutzer aufrufbar, dies wird durch folgenden Code-Ausschnitt sichergestellt:
\begin{lstlisting}[language=php]
	if (!defined('NICE_PROJECT')) {
		die('Permission denied.');
	}
\end{lstlisting}
Der Globale Wert {\glqq NICE\_PROJECT\grqq} wird durch für den Nutzer valide Aufrufpunkte festgelegt, z.B. {\glqq api.php\grqq}.
\newpage
\subsection{Funktionen}
\subsubsection{insertComment}
\paragraph{Parameter} Die Funktion besitzt folgende Parameter:
\begin{table}[H]
	\begin{tabular}{|c|p{11cm}|}
		\hline
		\textbf{Parametername} & \textbf{Parameterbeschreibung} \\ \hline
		\$data & Array mit Eingabewerten \\ \hline
	\end{tabular}
\end{table}
\subparagraph{\$data}Das Array enthält folgende Elemente:
\begin{table}[H]
	\begin{tabular}{|c|p{11cm}|}
		\hline
		\textbf{Parametername} & \textbf{Parameterbeschreibung} \\ \hline
		poi\_id & Identifikator eines Interessenpunktes  \\ \hline
		comment & Inhalt des Kommentars \\ \hline
	\end{tabular}
\end{table}
\paragraph{Beschreibung} Die Funktion fügt einem Interessenpunkt einen Kommentar hinzu. Die Funktion hat Auswirkungen auf folgende Quellen:
\begin{itemize}
	\item Kommentar-Tabelle
\end{itemize}
Es findet bei dieser Funktion kein Abruf von Daten aus {\glqq COSP\grqq} statt. Es gibt einen Rückgabewert.
\subsubsection{selectComments}
\paragraph{Parameter} Die Funktion besitzt folgende Parameter:
\begin{table}[H]
	\begin{tabular}{|c|p{11cm}|}
		\hline
		\textbf{Parametername} & \textbf{Parameterbeschreibung} \\ \hline
		\$poi\_id & Identifikator eines Interessenpunktes \\ \hline
	\end{tabular}
\end{table}
\paragraph{Beschreibung} Die Funktion fragt alle Kommentare zu einem gegebenen Interessenpunkt ab. Die Funktion nutzt folgende Quellen:
\begin{itemize}
	\item Kommentar-Tabelle
\end{itemize}
Es findet bei dieser Funktion kein Abruf von Daten aus {\glqq COSP\grqq} statt. Die Antwort wird als strukturiertes Array an den Aufrufer zurückgegeben.
\subsubsection{deleteComment}
\paragraph{Parameter} Die Funktion besitzt folgende Parameter:
\begin{table}[H]
	\begin{tabular}{|c|p{11cm}|}
		\hline
		\textbf{Parametername} & \textbf{Parameterbeschreibung} \\ \hline
		\$id & Identifikator des Kommentars \\ \hline
	\end{tabular}
\end{table}
\paragraph{Beschreibung} Die Funktion löscht einen Kommentar. Die Funktion hat Auswirkungen auf folgende Quellen:
\begin{itemize}
	\item Kommentar-Tabelle
\end{itemize}
Es findet bei dieser Funktion kein Abruf von Daten aus {\glqq COSP\grqq} statt. Es gibt einen Rückgabewert.
\subsubsection{deleteCommentByPOI}
\paragraph{Parameter} Die Funktion besitzt folgende Parameter:
\begin{table}[H]
	\begin{tabular}{|c|p{11cm}|}
		\hline
		\textbf{Parametername} & \textbf{Parameterbeschreibung} \\ \hline
		\$poiid & Identifikator des Interessenpunktes \\ \hline
	\end{tabular}
\end{table}
\paragraph{Beschreibung} Die Funktion löscht alle Kommentare eines Interessenpunktes. Die Funktion hat Auswirkungen auf folgende Quellen:
\begin{itemize}
	\item Kommentar-Tabelle
\end{itemize}
Es findet bei dieser Funktion kein Abruf von Daten aus {\glqq COSP\grqq} statt. Es gibt einen Rückgabewert.
\subsubsection{selectCommentsByCommentID}
\paragraph{Parameter} Die Funktion besitzt folgende Parameter:
\begin{table}[H]
	\begin{tabular}{|c|p{11cm}|}
		\hline
		\textbf{Parametername} & \textbf{Parameterbeschreibung} \\ \hline
		\$commentID & Identifikator eines Kommentars \\ \hline
	\end{tabular}
\end{table}
\paragraph{Beschreibung} Die Funktion fragt einen bestimmten Kommentar ab. Die Funktion nutzt folgende Quellen:
\begin{itemize}
	\item Kommentar-Tabelle
\end{itemize}
Es findet bei dieser Funktion kein Abruf von Daten aus {\glqq COSP\grqq} statt. Die Antwort wird als strukturiertes Array an den Aufrufer zurückgegeben.
\subsubsection{selectCommentsByCommentID}
\paragraph{Parameter} Die Funktion besitzt folgende Parameter:
\begin{table}[H]
	\begin{tabular}{|c|p{11cm}|}
		\hline
		\textbf{Parametername} & \textbf{Parameterbeschreibung} \\ \hline
		\$commentID      & Identifikator eines Kommentars \\ \hline
		\$commentContent & Inhalt des Kommentars \\ \hline
	\end{tabular}
\end{table}
\paragraph{Beschreibung} Die Funktion aktualisiert einen bestimmten Kommentar. Die Funktion hat Auswirkungen folgende Quellen:
\begin{itemize}
	\item Kommentar-Tabelle
\end{itemize}
Es findet bei dieser Funktion kein Abruf von Daten aus {\glqq COSP\grqq} statt. Es gibt einen Rückgabewert.
\subsubsection{getAllCommentsOfUser}
\paragraph{Parameter} Die Funktion besitzt folgende Parameter:
\begin{table}[H]
	\begin{tabular}{|c|p{11cm}|}
		\hline
		\textbf{Parametername} & \textbf{Parameterbeschreibung} \\ \hline
		\$username & Nutzername \\ \hline
	\end{tabular}
\end{table}
\paragraph{Beschreibung} Die Funktion fragt alle Kommentare eines Nutzers ab. Die Funktion nutzt folgende Quellen:
\begin{itemize}
	\item Kommentar-Tabelle
\end{itemize}
Es findet bei dieser Funktion kein Abruf von Daten aus {\glqq COSP\grqq} statt. Die Antwort wird als strukturiertes Array an den Aufrufer zurückgegeben.
\subsubsection{updateDeletionStateCommentByPoiid}
\paragraph{Parameter} Die Funktion besitzt folgende Parameter:
\begin{table}[H]
	\begin{tabular}{|c|p{11cm}|}
		\hline
		\textbf{Parametername} & \textbf{Parameterbeschreibung} \\ \hline
		\$poiid & Identifikator eines Interessenpunktes \\ \hline
		\$state & Status der Löschung \\ \hline
	\end{tabular}
\end{table}
\paragraph{Beschreibung} Die Funktion markiert alle Kommentare eines Interessenpunktes als gelöscht. Die Funktion hat Auswirkungen auf folgende Quellen:
\begin{itemize}
	\item Kommentar-Tabelle
\end{itemize}
Es findet bei dieser Funktion kein Abruf von Daten aus {\glqq COSP\grqq} statt. Es gibt einen Rückgabewert.
\subsubsection{updateDeletionStateCommentById}
\paragraph{Parameter} Die Funktion besitzt folgende Parameter:
\begin{table}[H]
	\begin{tabular}{|c|p{11cm}|}
		\hline
		\textbf{Parametername} & \textbf{Parameterbeschreibung} \\ \hline
		\$id    & Identifikator eines Kommentars \\ \hline
		\$state & Status der Löschung \\ \hline
	\end{tabular}
\end{table}
\paragraph{Beschreibung} Die Funktion markiert einen Kommentare als gelöscht. Die Funktion hat Auswirkungen auf folgende Quellen:
\begin{itemize}
	\item Kommentar-Tabelle
\end{itemize}
Es findet bei dieser Funktion kein Abruf von Daten aus {\glqq COSP\grqq} statt. Es gibt einen Rückgabewert.