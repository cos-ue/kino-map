\subsection{Allgemeines} Diese Datei enthält alle Funktionen, welche die Tabelle mit Quellen von Interessenpunkten benutzen.
\begin{table}[H]
	\begin{tabular}{|c|p{11cm}|}
		\hline
		\textbf{Einbindungspunkt} & inc-db.php \\ \hline
		\textbf{Einbindungspunkt} & inc-db-sub.php \\ \hline
	\end{tabular}
\end{table}
Die Datei ist nicht direkt durch den Nutzer aufrufbar, dies wird durch folgenden Code-Ausschnitt sichergestellt:
\begin{lstlisting}[language=php]
	if (!defined('NICE_PROJECT')) {
		die('Permission denied.');
	}
\end{lstlisting}
Der Globale Wert {\glqq NICE\_PROJECT\grqq} wird durch für den Nutzer valide Aufrufpunkte festgelegt, z.B. {\glqq api.php\grqq}.
\newpage
\subsection{Funktionen}
\subsubsection{insertSourceOfPOI}
\paragraph{Parameter} Die Funktion besitzt folgende Parameter:
\begin{table}[H]
	\begin{tabular}{|c|p{11cm}|}
		\hline
		\textbf{Parametername} & \textbf{Parameterbeschreibung} \\ \hline
		\$type     & Identifikator des Typs der Quelle \\ \hline
		\$source   & Quellenangabe \\ \hline
		\$relation & Identifikator des Bezugs der Quelle \\ \hline
		\$poiid    & Identifikator eines Interessenpunktes \\ \hline
	\end{tabular}
\end{table}
\paragraph{Beschreibung} Die Funktion fügt eine neue Quelle einem Interessenpunkt hinzu. Die Funktion hat Auswirkungen auf folgende Quellen:
\begin{itemize}
	\item Tabelle mit Quellenangaben
\end{itemize}
Es findet bei dieser Funktion kein Abruf von Daten aus {\glqq COSP\grqq} statt. Es wird eine Antwort zurück gegeben.
\subsubsection{getSourcePoiAPI}
\paragraph{Parameter} Die Funktion besitzt folgende Parameter:
\begin{table}[H]
	\begin{tabular}{|c|p{11cm}|}
		\hline
		\textbf{Parametername} & \textbf{Parameterbeschreibung} \\ \hline
		\$poiid    & Identifikator eines Interessenpunktes \\ \hline
	\end{tabular}
\end{table}
\paragraph{Beschreibung} Die Funktion ruft alle Quellen eines Interessenpunktes ab. Die Funktion nutzt folgende Quellen:
\begin{itemize}
	\item Tabelle mit Quellenangaben
\end{itemize}
Es findet bei dieser Funktion kein Abruf von Daten aus {\glqq COSP\grqq} statt. Es wird eine Antwort zurück gegeben.
\subsubsection{updateSource}
\paragraph{Parameter} Die Funktion besitzt folgende Parameter:
\begin{table}[H]
	\begin{tabular}{|c|p{11cm}|}
		\hline
		\textbf{Parametername} & \textbf{Parameterbeschreibung} \\ \hline
		\$id       & Identifikator einer Quelle \\ \hline
		\$relation & Identifikator des Bezugs der Quelle \\ \hline
		\$source   & Quellenangabe \\ \hline
		\$type     & Identifikator des Typs der Quelle \\ \hline
	\end{tabular}
\end{table}
\paragraph{Beschreibung} Die Funktion ändert eine Quelle eines Interessenpunktes. Die Funktion hat Auswirkungen auf folgende Quellen:
\begin{itemize}
	\item Tabelle mit Quellenangaben
\end{itemize}
Es findet bei dieser Funktion kein Abruf von Daten aus {\glqq COSP\grqq} statt. Es wird eine Antwort zurück gegeben.
\subsubsection{getSource}
\paragraph{Parameter} Die Funktion besitzt folgende Parameter:
\begin{table}[H]
	\begin{tabular}{|c|p{11cm}|}
		\hline
		\textbf{Parametername} & \textbf{Parameterbeschreibung} \\ \hline
		\$id & Identifikator einer Quelle \\ \hline
	\end{tabular}
\end{table}
\paragraph{Beschreibung} Die Funktion ruft eine Quelle ab. Die Funktion nutzt folgende Quellen:
\begin{itemize}
	\item Tabelle mit Quellenangaben
\end{itemize}
Es findet bei dieser Funktion kein Abruf von Daten aus {\glqq COSP\grqq} statt. Es wird eine Antwort zurück gegeben.
\subsubsection{updateSourceDeletionState}
\paragraph{Parameter} Die Funktion besitzt folgende Parameter:
\begin{table}[H]
	\begin{tabular}{|c|p{11cm}|}
		\hline
		\textbf{Parametername} & \textbf{Parameterbeschreibung} \\ \hline
		\$id    & Identifikator einer Quelle \\ \hline
		\$state & Identifikator des Bezugs der Quelle \\ \hline
	\end{tabular}
\end{table}
\paragraph{Beschreibung} Die Funktion aktualisiert den Löschungszustand einer Quelle eines Interessenpunktes. Die Funktion hat Auswirkungen auf folgende Quellen:
\begin{itemize}
	\item Tabelle mit Quellenangaben
\end{itemize}
Es findet bei dieser Funktion kein Abruf von Daten aus {\glqq COSP\grqq} statt. Es wird eine Antwort zurück gegeben.
\subsubsection{deleteSource}
\paragraph{Parameter} Die Funktion besitzt folgende Parameter:
\begin{table}[H]
	\begin{tabular}{|c|p{11cm}|}
		\hline
		\textbf{Parametername} & \textbf{Parameterbeschreibung} \\ \hline
		\$id & Identifikator einer Quelle \\ \hline
	\end{tabular}
\end{table}
\paragraph{Beschreibung} Die Funktion löscht eine Quelle eines Interessenpunktes. Die Funktion hat Auswirkungen auf folgende Quellen:
\begin{itemize}
	\item Tabelle mit Quellenangaben
\end{itemize}
Es findet bei dieser Funktion kein Abruf von Daten aus {\glqq COSP\grqq} statt. Es wird eine Antwort zurück gegeben.