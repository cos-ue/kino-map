\subsection{Allgemeines} Diese Datei enthält alle Funktionen zum Löschen von Daten beziehungsweise um Daten als gelöscht zu markieren.
\begin{table}[H]
	\begin{tabular}{|c|p{11cm}|}
		\hline
		\textbf{Einbindungspunkt} & inc.php \\ \hline
		\textbf{Einbindungspunkt} & inc-sub.php \\ \hline
	\end{tabular}
\end{table}
Die Datei ist nicht direkt durch den Nutzer aufrufbar, dies wird durch folgenden Code-Ausschnitt sichergestellt:
\begin{lstlisting}[language=php]
if (!defined('NICE_PROJECT')) {
	die('Permission denied.');
}
\end{lstlisting}
Der Globale Wert {\glqq NICE\_PROJECT\grqq} wird durch für den Nutzer valide Aufrufpunkte festgelegt, z.B. {\glqq api.php\grqq}.
\newpage
\subsection{Funktionen}
\subsubsection{deleteSeatsDBWrap}
\paragraph{Parameter} Die Funktion besitzt folgende Parameter:
\begin{table}[H]
	\begin{tabular}{|c|p{11cm}|}
		\hline
		\textbf{Parametername} & \textbf{Parameterbeschreibung} \\ \hline
		\$seatID    & Identifikator einer Sitzplatzanzahl \\ \hline
		\$overwrite & ermöglicht direktes Löschen \\ \hline
	\end{tabular}
\end{table}
\paragraph{Beschreibung} Die Funktion löscht oder markiert eine Sitzalplatzanzahl als gelöscht.
\begin{itemize}
	\item Sitzplatzanzahl-Tabelle
	\item Tabelle mit Validierungsinformationen zu Sitzplatzanzahlen
\end{itemize}
Es findet bei dieser Funktion kein Abruf von Daten aus {\glqq COSP\grqq} statt. Die Antwort wird als strukturiertes Array an den Aufrufer zurückgegeben.
\subsubsection{deleteNamesDBWrap}
\paragraph{Parameter} Die Funktion besitzt folgende Parameter:
\begin{table}[H]
	\begin{tabular}{|c|p{11cm}|}
		\hline
		\textbf{Parametername} & \textbf{Parameterbeschreibung} \\ \hline
		\$nameID    & Identifikator eines Namens \\ \hline
		\$overwrite & ermöglicht direktes Löschen \\ \hline
	\end{tabular}
\end{table}
\paragraph{Beschreibung} Die Funktion löscht oder markiert einen Namen als gelöscht.
\begin{itemize}
	\item Namen-Tabelle
	\item Tabelle mit Validierungsinformationen zu Namen
\end{itemize}
Es findet bei dieser Funktion kein Abruf von Daten aus {\glqq COSP\grqq} statt. Die Antwort wird als strukturiertes Array an den Aufrufer zurückgegeben.
\subsubsection{deleteCinemasDBWrap}
\paragraph{Parameter} Die Funktion besitzt folgende Parameter:
\begin{table}[H]
	\begin{tabular}{|c|p{11cm}|}
		\hline
		\textbf{Parametername} & \textbf{Parameterbeschreibung} \\ \hline
		\$cinemaID  & Identifikator einer Saalanzahl \\ \hline
		\$overwrite & ermöglicht direktes Löschen \\ \hline
	\end{tabular}
\end{table}
\paragraph{Beschreibung} Die Funktion löscht oder markiert eine Saalanzahl als gelöscht.
\begin{itemize}
	\item Saalanzahl-Tabelle
	\item Tabelle mit Validierungsinformationen zu Saalanzahlen
\end{itemize}
Es findet bei dieser Funktion kein Abruf von Daten aus {\glqq COSP\grqq} statt. Die Antwort wird als strukturiertes Array an den Aufrufer zurückgegeben.
\subsubsection{deleteOperatorsDBWrap}
\paragraph{Parameter} Die Funktion besitzt folgende Parameter:
\begin{table}[H]
	\begin{tabular}{|c|p{11cm}|}
		\hline
		\textbf{Parametername} & \textbf{Parameterbeschreibung} \\ \hline
		\$operatorID & Identifikator eines Betreibers \\ \hline
		\$overwrite  & ermöglicht direktes Löschen \\ \hline
	\end{tabular}
\end{table}
\paragraph{Beschreibung} Die Funktion löscht oder markiert einen Betreiber als gelöscht.
\begin{itemize}
	\item Betreiber-Tabelle
	\item Tabelle mit Validierungsinformationen zu Betreibern
\end{itemize}
Es findet bei dieser Funktion kein Abruf von Daten aus {\glqq COSP\grqq} statt. Die Antwort wird als strukturiertes Array an den Aufrufer zurückgegeben.
\subsubsection{deleteHistAddressDBWrap}
\paragraph{Parameter} Die Funktion besitzt folgende Parameter:
\begin{table}[H]
	\begin{tabular}{|c|p{11cm}|}
		\hline
		\textbf{Parametername} & \textbf{Parameterbeschreibung} \\ \hline
		\$histAddressID & Identifikator einer historischen Adresse \\ \hline
		\$overwrite     & ermöglicht direktes Löschen \\ \hline
	\end{tabular}
\end{table}
\paragraph{Beschreibung} Die Funktion löscht oder markiert eine historische Adresse als gelöscht.
\begin{itemize}
	\item Tabelle mit historischen Adressen
	\item Tabelle mit Validierungsinformationen zu historischen Adressen
\end{itemize}
Es findet bei dieser Funktion kein Abruf von Daten aus {\glqq COSP\grqq} statt. Die Antwort wird als strukturiertes Array an den Aufrufer zurückgegeben.
\subsubsection{deleteCommentsDBWrap}
\paragraph{Parameter} Die Funktion besitzt folgende Parameter:
\begin{table}[H]
	\begin{tabular}{|c|p{11cm}|}
		\hline
		\textbf{Parametername} & \textbf{Parameterbeschreibung} \\ \hline
		\$commentID & Identifikator eines Kommentars \\ \hline
		\$overwrite & ermöglicht direktes Löschen \\ \hline
	\end{tabular}
\end{table}
\paragraph{Beschreibung} Die Funktion löscht oder markiert einen Kommentar als gelöscht.
\begin{itemize}
	\item Kommentar-Tabelle
\end{itemize}
Es findet bei dieser Funktion kein Abruf von Daten aus {\glqq COSP\grqq} statt. Die Antwort wird als strukturiertes Array an den Aufrufer zurückgegeben.
\subsubsection{deleteCommentsByPoiidDBWrap}
\paragraph{Parameter} Die Funktion besitzt folgende Parameter:
\begin{table}[H]
	\begin{tabular}{|c|p{11cm}|}
		\hline
		\textbf{Parametername} & \textbf{Parameterbeschreibung} \\ \hline
		\$POIID     & Identifikator eines Interessenpunktes \\ \hline
		\$overwrite & ermöglicht direktes Löschen \\ \hline
	\end{tabular}
\end{table}
\paragraph{Beschreibung} Die Funktion löscht oder markiert alle Kommentare eines Interessenpunktes als gelöscht.
\begin{itemize}
	\item Kommentar-Tabelle
\end{itemize}
Es findet bei dieser Funktion kein Abruf von Daten aus {\glqq COSP\grqq} statt. Die Antwort wird als strukturiertes Array an den Aufrufer zurückgegeben.
\subsubsection{deletePoiDBWrap}
\paragraph{Parameter} Die Funktion besitzt folgende Parameter:
\begin{table}[H]
	\begin{tabular}{|c|p{11cm}|}
		\hline
		\textbf{Parametername} & \textbf{Parameterbeschreibung} \\ \hline
		\$POIID     & Identifikator eines Interessenpunktes \\ \hline
		\$overwrite & ermöglicht direktes Löschen \\ \hline
	\end{tabular}
\end{table}
\paragraph{Beschreibung} Die Funktion löscht oder markiert einen Betreiber als gelöscht.
\begin{itemize}
	\item Interessenpunkt-Tabelle
	\item Kommentar-Tabelle
	\item Saalanzahl-Tabelle
	\item Sitzplatzanzahl-Tabelle
	\item Namen-Tabelle
	\item Betreiber-Tabelle
	\item Tabelle mit historischen Adressen
	\item Tabelle mit Links zwischen Interessenpunkten und Bildern
	\item Tabelle mit Links zwischen Interessenpunkten und Geschichten
	\item Tabelle mit Validierungsinformationen zu Interessenpunkten
	\item Tabelle mit Validierungsinformationen zu Saalanzahlen
	\item Tabelle mit Validierungsinformationen zu Sitzplatzanzahlen
	\item Tabelle mit Validierungsinformationen zu Namen
	\item Tabelle mit Validierungsinformationen zu historischen Adressen
	\item Tabelle mit Validierungsinformationen zu Links zwischen Interessenpunkten und Bildern
	\item Tabelle mit Validierungsinformationen zu Links zwischen Interessenpunkten und Geschichten
\end{itemize}
Es findet bei dieser Funktion kein Abruf von Daten aus {\glqq COSP\grqq} statt. Die Antwort wird als strukturiertes Array an den Aufrufer zurückgegeben.
\subsubsection{deletePoiPicLinkByIDDBWrap}
\paragraph{Parameter} Die Funktion besitzt folgende Parameter:
\begin{table}[H]
	\begin{tabular}{|c|p{11cm}|}
		\hline
		\textbf{Parametername} & \textbf{Parameterbeschreibung} \\ \hline
		\$LinkID    & Identifikator eines Links zwischen einem Interessenpunkt und einem Bild \\ \hline
		\$overwrite & ermöglicht direktes Löschen \\ \hline
	\end{tabular}
\end{table}
\paragraph{Beschreibung} Die Funktion löscht oder markiert einen Link zwischen einem Interessenpunkt und einem Bild als gelöscht.
\begin{itemize}
	\item Tabelle mit Links zwischen Interessenpunkten und Bildern
	\item Tabelle mit Validierungsinformationen zu Links zwischen Interessenpunkten und Bildern
\end{itemize}
Es findet bei dieser Funktion kein Abruf von Daten aus {\glqq COSP\grqq} statt. Die Antwort wird als strukturiertes Array an den Aufrufer zurückgegeben.
\subsubsection{deletePoiStoryLinkByIDDBWrap}
\paragraph{Parameter} Die Funktion besitzt folgende Parameter:
\begin{table}[H]
	\begin{tabular}{|c|p{11cm}|}
		\hline
		\textbf{Parametername} & \textbf{Parameterbeschreibung} \\ \hline
		\$LinkID    & Identifikator eines Links zwischen einem Interessenpunkt und einem Geschichte \\ \hline
		\$overwrite & ermöglicht direktes Löschen \\ \hline
	\end{tabular}
\end{table}
\paragraph{Beschreibung} Die Funktion löscht oder markiert einen Link zwischen einem Interessenpunkt und einer Geschichte als gelöscht.
\begin{itemize}
	\item Tabelle mit Links zwischen Interessenpunkten und Geschichten
	\item Tabelle mit Validierungsinformationen zu Links zwischen Interessenpunkten und Geschichten
\end{itemize}
Es findet bei dieser Funktion kein Abruf von Daten aus {\glqq COSP\grqq} statt. Die Antwort wird als strukturiertes Array an den Aufrufer zurückgegeben.
\subsubsection{deletePoiMainPicDBWrap}
\paragraph{Parameter} Die Funktion besitzt folgende Parameter:
\begin{table}[H]
	\begin{tabular}{|c|p{11cm}|}
		\hline
		\textbf{Parametername} & \textbf{Parameterbeschreibung} \\ \hline
		\$POIID     & Identifikator eines Interessenpunktes \\ \hline
		\$overwrite & ermöglicht direktes Löschen \\ \hline
	\end{tabular}
\end{table}
\paragraph{Beschreibung} Die Funktion löscht oder markiert das Hauptbild eines Interessenpunktes als gelöscht.
\begin{itemize}
	\item Interessenpunkt-Tabelle
	\item Tabelle mit Validierungsinformationen zu Interessenpunkten
\end{itemize}
Es findet bei dieser Funktion kein Abruf von Daten aus {\glqq COSP\grqq} statt. Die Antwort wird als strukturiertes Array an den Aufrufer zurückgegeben.
\subsubsection{deleteSourceDBWrap}
\paragraph{Parameter} Die Funktion besitzt folgende Parameter:
\begin{table}[H]
	\begin{tabular}{|c|p{11cm}|}
		\hline
		\textbf{Parametername} & \textbf{Parameterbeschreibung} \\ \hline
		\$sid       & Identifikator einer Quelle \\ \hline
		\$overwrite & ermöglicht direktes Löschen \\ \hline
	\end{tabular}
\end{table}
\paragraph{Beschreibung} Die Funktion löscht oder markiert eine Quelle als gelöscht.
\begin{itemize}
	\item Interessenpunkt-Tabelle
	\item Tabelle mit Validierungsinformationen zu Interessenpunkten
\end{itemize}
Es findet bei dieser Funktion kein Abruf von Daten aus {\glqq COSP\grqq} statt. Die Antwort wird als strukturiertes Array an den Aufrufer zurückgegeben.