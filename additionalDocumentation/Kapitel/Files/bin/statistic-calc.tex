\subsection{Allgemeines} Diese Datei enthält Funktionen, welche zur Darstellung statistischer Daten verwendet werden.
\begin{table}[H]
	\begin{tabular}{|c|p{11cm}|}
		\hline
		\textbf{Einbindungspunkt} & inc.php \\ \hline
		\textbf{Einbindungspunkt} & inc-sub.php \\ \hline
	\end{tabular}
\end{table}
Die Datei ist nicht direkt durch den Nutzer aufrufbar, dies wird durch folgenden Code-Ausschnitt sichergestellt:
\begin{lstlisting}[language=php]
if (!defined('NICE_PROJECT')) {
	die('Permission denied.');
}
\end{lstlisting}
Der Globale Wert {\glqq NICE\_PROJECT\grqq} wird durch für den Nutzer valide Aufrufpunkte festgelegt, z.B. {\glqq api.php\grqq}.
\newpage
\subsection{Funktionen}
\subsubsection{loginStatistics}
\paragraph{Parameter} Die Funktion besitzt folgende Parameter:
\begin{table}[H]
	\begin{tabular}{|c|p{11cm}|}
		\hline
		\textbf{Parametername} & \textbf{Parameterbeschreibung} \\ \hline
		\$Input & Eingabedaten als Array \\ \hline
	\end{tabular}
\end{table}
\subparagraph{\$json}Das Array enthält Einträge mit folgenden Elemente:
\begin{table}[H]
	\begin{tabular}{|c|p{11cm}|}
		\hline
		\textbf{Parametername} & \textbf{Parameterbeschreibung} \\ \hline
		Amount & Anzahl an Zeiteinheiten \\ \hline
		type   & Typ der Zeiteinheit \\ \hline
	\end{tabular}
\end{table}
\paragraph{Beschreibung} Die Funktion erstellt ein Datenarray, welches zur Anzeige der Nutzungsstatistiken verwendet wird. Die Funktion nutzt folgende Quellen:
\begin{itemize}
	\item Tabelle mit statistischen Nutzungsdaten
\end{itemize}
Es findet bei dieser Funktion kein Abruf von Daten aus {\glqq COSP\grqq} statt. Die Antwort wird als strukturiertes Array an den Aufrufer zurückgegeben.
\subsubsection{fillUnknownData}
\paragraph{Parameter} Die Funktion besitzt folgende Parameter:
\begin{table}[H]
	\begin{tabular}{|c|p{11cm}|}
		\hline
		\textbf{Parametername} & \textbf{Parameterbeschreibung} \\ \hline
		\$statisticalData & Array mit statistischen Daten \\ \hline
		\$periodeAmount   & Anzahl an Zeiteinheiten \\ \hline
		\$type            & Typ der Zeiteinheit \\ \hline
	\end{tabular}
\end{table}
\paragraph{Beschreibung} Die Funktion füllt fehlende Einträge in den statistischen Daten mit dem Wert {\glqq 0\grqq} auf. Es findet bei dieser Funktion kein Abruf von Daten aus {\glqq COSP\grqq} statt. Die Antwort wird als strukturiertes Array an den Aufrufer zurückgegeben.
\subsubsection{createGraph}
\paragraph{Parameter} Die Funktion besitzt folgende Parameter:
\begin{table}[H]
	\begin{tabular}{|c|p{11cm}|}
		\hline
		\textbf{Parametername} & \textbf{Parameterbeschreibung} \\ \hline
		\$data      & statistische Daten in Array-Form \\ \hline
		\$colorBg   & Füllfarbe \\ \hline
		\$colorFont & Schriftfarbe \\ \hline
		\$label     & Name des Datensatzes \\ \hline
		\$fill      & Gibt an, ob Graph gefüllt werden soll \\ \hline
	\end{tabular}
\end{table}
\paragraph{Beschreibung} Die Funktion generiert einen durch das Chart.js anzeigbaren Datensatz. Es findet bei dieser Funktion kein Abruf von Daten aus {\glqq COSP\grqq} statt. Die Antwort wird als strukturiertes Array an den Aufrufer zurückgegeben.
\subsubsection{CreateStatistics}
\paragraph{Parameter} Die Funktion besitzt folgende Parameter:
\begin{table}[H]
	\begin{tabular}{|c|p{11cm}|}
		\hline
		\textbf{Parametername} & \textbf{Parameterbeschreibung} \\ \hline
		\$Input  & Eingabedaten als Array \\ \hline
		\$source & Quelle der statistischen Daten \\ \hline
	\end{tabular}
\end{table}
\subparagraph{\$json}Das Array enthält Einträge mit folgenden Elemente:
\begin{table}[H]
	\begin{tabular}{|c|p{11cm}|}
		\hline
		\textbf{Parametername} & \textbf{Parameterbeschreibung} \\ \hline
		Amount & Anzahl an Zeiteinheiten \\ \hline
		type   & Typ der Zeiteinheit \\ \hline
	\end{tabular}
\end{table}
\paragraph{Beschreibung} Die Funktion erstellt ein Datenarray, welches zur Anzeige der Statistik zu neuen/geänderten Interessenpunkten oder Kommentaren verwendet wird. Die Funktion nutzt folgende Quellen:
\begin{itemize}
	\item Interessenpunkt-Tabelle
	\item Kommentar-Tabelle
\end{itemize}
Es findet bei dieser Funktion kein Abruf von Daten aus {\glqq COSP\grqq} statt. Die Antwort wird als strukturiertes Array an den Aufrufer zurückgegeben.
\subsubsection{correctData}
\paragraph{Parameter} Die Funktion besitzt folgende Parameter:
\begin{table}[H]
	\begin{tabular}{|c|p{11cm}|}
		\hline
		\textbf{Parametername} & \textbf{Parameterbeschreibung} \\ \hline
		\$inputArray  & Eingabedaten als Array \\ \hline
	\end{tabular}
\end{table}
\paragraph{Beschreibung} Die Funktion korrigiert fehlerhafte statistische Daten. Die Antwort wird als strukturiertes Array an den Aufrufer zurückgegeben.