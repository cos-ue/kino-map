\subsection{Allgemeines} Diese Datei enthält alle grundlegenden JavaScript-Funktionen für Kartenmarker.
Die Ausführung des Codes findet im Browser statt. Auch werden hier einige in verschiedenen anderen Javascripts verwendeten Variablen festgelegt.
\begin{lstlisting}[language=JavaScript]
var anz = 0;
var anzEditMap = 0;
\end{lstlisting}
\newpage
\subsection{Funktionen}
\subsubsection{delMark}
\paragraph{Parameter} Die Funktion besitzt keine Parameter.
\paragraph{Beschreibung} Die Funktion löscht den gesetzten Marker auf der Karte. Es findet bei dieser Funktion kein Abruf von Daten aus {\glqq COSP\grqq} statt.
\subsubsection{onMove}
\paragraph{Parameter} Die Funktion besitzt folgende Parameter:
\begin{table}[H]
	\begin{tabular}{|c|p{11cm}|}
		\hline
		\textbf{Parametername} & \textbf{Parameterbeschreibung} \\ \hline
		e       & Marker \\ \hline
		editMap & Gibt an, ob das JavaScript auf der Interessenpunkt-Ändern-Seite ausgeführt wird \\ \hline
		Minimap & Gibt an, ob der Marker auf der Minimap ist \\ \hline
	\end{tabular}
\end{table}
\paragraph{Beschreibung} Die Funktion aktualisiert die Koordinaten in den verdeckten Formularfeldern, wenn der Marker bewegt wird. Es findet bei dieser Funktion kein Abruf von Daten aus {\glqq COSP\grqq} statt.
\subsubsection{onMoveMM}
\paragraph{Parameter} Die Funktion besitzt folgende Parameter:
\begin{table}[H]
	\begin{tabular}{|c|p{11cm}|}
		\hline
		\textbf{Parametername} & \textbf{Parameterbeschreibung} \\ \hline
		latLng & Array mit Längen- und Breitengrad \\ \hline
	\end{tabular}
\end{table}
\paragraph{Beschreibung} Die Funktion aktualisiert die Koordinaten in den verdeckten Formularfeldern, wenn die Minimap beim einfügen eines neuen Interessenpunktes aktiv ist. Es findet bei dieser Funktion kein Abruf von Daten aus {\glqq COSP\grqq} statt.
\subsubsection{placeMarker}
\paragraph{Parameter} Die Funktion besitzt folgende Parameter:
\begin{table}[H]
	\begin{tabular}{|c|p{11cm}|}
		\hline
		\textbf{Parametername} & \textbf{Parameterbeschreibung} \\ \hline
		coordinates & Array mit Längen- und Breitengrad \\ \hline
		editMap     & Gibt an, ob das JavaScript auf der Interessenpunkt-Ändern-Seite ausgeführt wird \\ \hline
	\end{tabular}
\end{table}
\paragraph{Beschreibung} Die Funktion setzt den Interessenpunkt-Hinzufügen-Marker an der gegebenen Stelle. Es findet bei dieser Funktion kein Abruf von Daten aus {\glqq COSP\grqq} statt.
\subsubsection{insertCoordinates}
\paragraph{Parameter} Die Funktion besitzt folgende Parameter:
\begin{table}[H]
	\begin{tabular}{|c|p{11cm}|}
		\hline
		\textbf{Parametername} & \textbf{Parameterbeschreibung} \\ \hline
		editMap     & Gibt an, ob das JavaScript auf der Interessenpunkt-Ändern-Seite ausgeführt wird \\ \hline
	\end{tabular}
\end{table}
\paragraph{Beschreibung} Die Funktion aktualisiert die verdeckten Formularfelder mit aktualisierten Koordinaten. Es findet bei dieser Funktion kein Abruf von Daten aus {\glqq COSP\grqq} statt.
\subsubsection{resizeMap}
\paragraph{Parameter} Die Funktion besitzt folgende Parameter:
\begin{table}[H]
	\begin{tabular}{|c|p{11cm}|}
		\hline
		\textbf{Parametername} & \textbf{Parameterbeschreibung} \\ \hline
		map         & Variable, welche Karte darstellt \\ \hline
		coordinates & Array mit Längen- und Breitengrad \\ \hline
	\end{tabular}
\end{table}
\paragraph{Beschreibung} Die Funktion aktualisiert den gezeigten Ausschnitt der angegeben Karte mit 300 Milisekunden verzögerung. Es findet bei dieser Funktion kein Abruf von Daten aus {\glqq COSP\grqq} statt.