\subsection{Allgemeines} Diese Datei enthält alle JavaScript-Funktionen, welche für den persönlichen Bereich benötigt werden.
Die Ausführung des Codes findet im Browser statt.
\subsection{Funktionen}
\subsubsection{loadPersonalArea}
\paragraph{Parameter} Die Funktion besitzt keine Parameter.
\paragraph{Beschreibung} Die Funktion lädt alle benötigten Daten für den persönlichen Bereich und zeigt diesen an. Die Funktion nutzt folgende Quellen:
\begin{itemize}
	\item Frontend-API
\end{itemize}
Es findet bei dieser Funktion kein Abruf von Daten aus {\glqq COSP\grqq} statt.
\subsubsection{deletePoiFinalPersonalArea}
\paragraph{Parameter} Die Funktion besitzt folgende Parameter:
\begin{table}[H]
	\begin{tabular}{|c|p{11cm}|}
		\hline
		\textbf{Parametername} & \textbf{Parameterbeschreibung} \\ \hline
		id & Identifikator eines Interessenpunktes \\ \hline
	\end{tabular}
\end{table}
\paragraph{Beschreibung} Die Funktion löscht einen Interessenpunkt endgültig und lädt den persönlichen Bereich neu. Die Funktion hat Auswirkungen auf folgende Quellen:
\begin{itemize}
	\item Frontend-API
\end{itemize}
Es findet bei dieser Funktion kein Abruf von Daten aus {\glqq COSP\grqq} statt.
\subsubsection{restorePoiPersonalArea}
\paragraph{Parameter} Die Funktion besitzt folgende Parameter:
\begin{table}[H]
	\begin{tabular}{|c|p{11cm}|}
		\hline
		\textbf{Parametername} & \textbf{Parameterbeschreibung} \\ \hline
		id & Identifikator eines Interessenpunktes \\ \hline
	\end{tabular}
\end{table}
\paragraph{Beschreibung} Die Funktion stellt einen Interessenpunkt wieder her und lädt den persönlichen Bereich neu. Die Funktion hat Auswirkungen auf folgende Quellen:
\begin{itemize}
	\item Frontend-API
\end{itemize}
Es findet bei dieser Funktion kein Abruf von Daten aus {\glqq COSP\grqq} statt.
\subsubsection{deleteCommentFinalPersonalArea}
\paragraph{Parameter} Die Funktion besitzt folgende Parameter:
\begin{table}[H]
	\begin{tabular}{|c|p{11cm}|}
		\hline
		\textbf{Parametername} & \textbf{Parameterbeschreibung} \\ \hline
		id & Identifikator eines Kommentars \\ \hline
	\end{tabular}
\end{table}
\paragraph{Beschreibung} Die Funktion löscht einen Kommentar endgültig und lädt den persönlichen Bereich neu. Die Funktion hat Auswirkungen auf folgende Quellen:
\begin{itemize}
	\item Frontend-API
\end{itemize}
Es findet bei dieser Funktion kein Abruf von Daten aus {\glqq COSP\grqq} statt.
\subsubsection{restoreCommentPersonalArea}
\paragraph{Parameter} Die Funktion besitzt folgende Parameter:
\begin{table}[H]
	\begin{tabular}{|c|p{11cm}|}
		\hline
		\textbf{Parametername} & \textbf{Parameterbeschreibung} \\ \hline
		id & Identifikator eines Kommentars \\ \hline
	\end{tabular}
\end{table}
\paragraph{Beschreibung} Die Funktion stellt einen Kommentar wieder her und lädt den persönlichen Bereich neu. Die Funktion hat Auswirkungen auf folgende Quellen:
\begin{itemize}
	\item Frontend-API
\end{itemize}
Es findet bei dieser Funktion kein Abruf von Daten aus {\glqq COSP\grqq} statt.
\subsubsection{editComment}
\paragraph{Parameter} Die Funktion besitzt folgende Parameter:
\begin{table}[H]
	\begin{tabular}{|c|p{11cm}|}
		\hline
		\textbf{Parametername} & \textbf{Parameterbeschreibung} \\ \hline
		commentID & Identifikator eines Kommentars \\ \hline
	\end{tabular}
\end{table}
\paragraph{Beschreibung} Die Funktion bereitet das Kommentar-Ändern-Modal vor und öffnet dieses. Es findet bei dieser Funktion kein Abruf von Daten aus {\glqq COSP\grqq} statt.
\subsubsection{saveEditedComment}
\paragraph{Parameter} Die Funktion besitzt keine Parameter.
\paragraph{Beschreibung} Die Funktion speichert einen geänderten Kommentar und schließt das entsprechende Modal. Die Funktion hat Auswirkungen auf folgende Quellen:
\begin{itemize}
	\item Frontend-API
\end{itemize}
Es findet bei dieser Funktion kein Abruf von Daten aus {\glqq COSP\grqq} statt.