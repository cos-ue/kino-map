\subsection{Allgemeines} Diese Datei dient enthält alle Funktionen, welche für die Statistik-Seite zusätzlich benötigt werden.
Die Ausführung des Codes findet im Browser statt. Hier wird eine Variable für statistische Daten gesetzt:
\begin{lstlisting}[language=JavaScript]
var StatData = {};
\end{lstlisting} 
\subsection{Funktionen}
\subsubsection{generateConfig}
\paragraph{Parameter} Die Funktion besitzt folgende Parameter:
\begin{table}[H]
	\begin{tabular}{|c|p{11cm}|}
		\hline
		\textbf{Parametername} & \textbf{Parameterbeschreibung} \\ \hline
		ID   & Position der Daten im {\glqq StatData\grqq}-Array \\ \hline
		type & Typ der Zeiteinheit (Tag: D, Woche: W, Monat: M, Jahr; Y) \\ \hline
	\end{tabular}
\end{table}
\paragraph{Beschreibung} Die Funktion erzeugt die Konfiguration, welche für eine Anzeige der Daten mittels {\glqq Chart.js\grqq} benötigt wird. Es findet bei dieser Funktion kein Abruf von Daten aus {\glqq COSP\grqq} statt.
\subsubsection{loadStatisticalData}
\paragraph{Parameter} Die Funktion besitzt keine Parameter.
\paragraph{Beschreibung} Die Funktion lädt alle benötigten Daten mittels API. Die Funktion nutzt folgende Quellen:
\begin{itemize}
	\item Frontend-API
\end{itemize}
Es findet bei dieser Funktion kein Abruf von Daten aus {\glqq COSP\grqq} statt.
