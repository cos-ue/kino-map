\subsection{Allgemeines} Diese Datei enthält diverse Hilfsfunktionen und ermöglicht das Suchen.
Die Ausführung des Codes findet im Browser statt. Hier wird der Trigger zum Ausführen der Suche bei Enter gesetzt, sowie zur Auswertung gesetzter Cookies:
\begin{lstlisting}[language=JavaScript]
$(document).ready(function () {
	// Trigger search button on enter at search input
	$("#search").keyup(function (e) {
		if(e.which === 13) {
			$("#poiSearchButton").click();
		}
	});
	
	//On pressing a key on "Search box" in "search.php" file. This function will be called.
	$("#poiSearchButton").click(function () {
		var name = $('#search').val();
		if (name === "") {
			$("#display").html("");
			pois = null;
		} else {
			if (pois == null) {
				$.ajax({
					type: "POST",
					contentType: "application/json",
					url: "Formular/api.php",
					data: JSON.stringify({
						type: "gpu"
					}),
					success: create
				});
			} else {
				create();
			}
		}
	});
	var OpenPersonalArea = getCookie("personalArea");
	if (OpenPersonalArea === "1") {
		var delayInMilliseconds = 300;
		setTimeout(function () {
			deleteCookie("personalArea");
			loadPersonalArea();
		}, delayInMilliseconds);
	}
});
\end{lstlisting} 
Auch werden hier einige in verschiedenen anderen Javascripts verwendeten Variablen festgelegt.
\begin{lstlisting}[language=JavaScript]
var pois = null;
var focusComment = -1;
\end{lstlisting} 
\newpage
\subsection{Funktionen}
\subsubsection{openSearchModal}
\paragraph{Parameter} Die Funktion besitzt folgende Parameter:
\begin{table}[H]
	\begin{tabular}{|c|p{11cm}|}
		\hline
		\textbf{Parametername} & \textbf{Parameterbeschreibung} \\ \hline
		json & Array mit Daten \\ \hline
	\end{tabular}
\end{table}
\subparagraph{\$json}Das Array enthält Einträge mit folgenden Elementen:
\begin{table}[H]
	\begin{tabular}{|c|p{11cm}|}
		\hline
		\textbf{Parametername} & \textbf{Parameterbeschreibung} \\ \hline
		name             & Namen eines Interessenpunktes  \\ \hline
		current\_address & Aktuelle Adresse eines Interessenpunktes  \\ \hline
		hist\_address    & historische Adresse eines Interessenpunktes (veraltet) \\ \hline
		operator         & Betreiber eines Interessenpunktes (veraltet) \\ \hline
		history          & Geschichte eines Interessenpunktes \\ \hline
		lng              & Längengrad eines Interessenpunktes \\ \hline
		lat              & Breitengrad eines Interessenpunktes \\ \hline
	\end{tabular}
\end{table}
\paragraph{Beschreibung} Die Funktion öffnet das Such-Modal und sucht in den gegebenen Daten. Die FUnktion nutzt folgende Quellen:
\begin{itemize}
	\item Frontend-API
\end{itemize}
Es findet bei dieser Funktion kein Abruf von Daten aus {\glqq COSP\grqq} statt.
\subsubsection{focusPOI}
\paragraph{Parameter} Die Funktion besitzt folgende Parameter:
\begin{table}[H]
	\begin{tabular}{|c|p{11cm}|}
		\hline
		\textbf{Parametername} & \textbf{Parameterbeschreibung} \\ \hline
		lng & Längengrad \\ \hline
		lat & Breitengrad \\ \hline
	\end{tabular}
\end{table}
\paragraph{Beschreibung} Die Funktion setzt den Focus auf die gegebenen Koordinaten und schließt das Such-Modal. Es findet bei dieser Funktion kein Abruf von Daten aus {\glqq COSP\grqq} statt.
